\chapter{Conclusions and Future Work}

\section{Conclusions}

In this dissertation, the effect of alloying elements on Ti-based alloys are systematically studied. The work begins by using first-principles based DFT calculations and the CALPHAD method to study the effect that the alloying elements Mo, Nb, Sn, Ta and Zr have on the equilibrium phase stability, thermodynamics, and elastic properites. The work uses the equation of states fitting of the energy vs. volume curves to get the ground state equilibrium properties. The Debye-Gr\"uneisen model and phonon quasiharmonic approach are used to study the effect of temperature on the phase stability. A new theoretic framework is proposed to study the formation of the metastable phases. The accuracy of the theoretic framework and the transformation that occurs when these metastable phases form is studied using neutron scattering experiments. The compilation of the work develops a knowledge base for Ti-based alloys and will help to guide the future design of biocompatible implants. The main conclusions from this work are included below:

\begin{enumerate}
	\item The thermodynamic descriptions were all incorporated into a complete database that accurately predicts the phase stability of the Ti-Mo-Nb-Sn-Ta-Zr systems. The thermodynamic descriptions of the pure elements are adopted from the SGTE database. All of the binary systems had previous thermodynamic descriptions available in literature except the Mo-Sn and Ta-Sn systems. All of the binary systems had previous thermodynamic descriptions available in literature except Mo-Sn and Ta-Sn. A previous model was evaluated for accuarcy when avaliable for the binary systems. The Sn-Ta system was modeled in chapter 4. After evaluation the thermodynamic descriptions were incorporated into the present database. The binary interpolations of the Ti-containing ternary systems were plotted and compared with the available experimental data as well as the enthalpy of formation of the bcc phase calculated from first-principles based on DFT. The Ti-Sn-X systems (X = Mo, Nb, Ta, Zr) will be modeled in the future work with a thermodynamci description of the Mo-Sn system. The binary interpolations of the Ti-Nb-Zr and Ti-Ta-Zr systems had previously been plotted but no interaction parameters had been introduced. The present evaluation agreed with the previous evaluations and no ternary interaction parameters were introduced. The Ti-Mo-Zr system had previously been modeled and the present work agreed with the evaluation. The Ti-Mo-Nb, Ti-Mo-Ta and Ti-Nb-Ta systems had never previously been modeled. The present work evaluated interaction parameters for the Ti-Mo-Ta and Ti-Nb-Ta systems but didn't introduce any interaction parameters for the Ti-Mo-Nb system. The completed database is in appendix b. 
	\item Sn-Ta modeling was completed using data from DFT-based first-principles calculations and the available experimental data in the literature to model the Gibbs energies for the bcc and liquid solution phases and the stoichiometric Ta$_3$Sn and TaSn$_2$ phases of the Sn-Ta system. First-principles calculations were used to predict the enthalpy of formation of the bcc phase for the evaluation of interaction parameters in the phase. The decomposition temperature of Ta$_3$Sn was predicted to be 2884 $^\circ$K. The completed thermodynamic description was complied into a tdb file in appendix b.
	\item The effects of five alloying elements on the elastic properties of bcc Ti-X (X = Mo, Nb, Sn, Ta, Zr) alloys, including the elastic stiffness constants, bulk modulus, shear modulus, and Young's modulus, were systematically studied using first-principles calculations. The CALPAHD methodology was used to evaluate interaction parameters to predict the elastic properties as a function of composition. The calculations showed that 5.5, 11.5, 51.5, 9.5, and 4.0 at. \% of Mo, Nb, Sn, Ta and Zr, respectively, were required to stabilize the bcc phase according to the Born criteria. The trends observed were summarized for each Ti-X (X= Mo, Nb, Sn, Ta, Zr) binary system. Alloying with Mo, Nb, and Ta results in similar trends, which is probably because Mo, Nb, and Ta are strong bcc stabilizers and stable in the bcc structure at room temperature. The interaction parameters determined in the current work were used to predict the elastic properties of higher order alloys. The accuracy of database predictions of the Young’s modulus was evaluated by comparing the calculated and experimental Young's moduli. Overall, the database provides good predictions of the elastic properties of Ti-alloys in the bcc phase as a function of composition.
	\item The elastic properties of the bcc Ti-X-Y ternary alloys (X $\neq$ Y = Mo, Nb, Sn, Ta, Zr), including the elastic stiffness constants, bulk modulus, shear modulus, and Young's modulus. The general CALPHAD modeling approach was used to fit ternary interaction parameters. From the elastic stiffness constant data, the Ti-X-Y (X $\neq$ Y = Mo, Nb, Ta) show the same trends in the data. This is to be expected because Mo, Nb, and Ta are similar elements that are strong $\beta$-stabilizers and stable in the bcc phase at low temperatures. It was also seen that the Ti-X-Sn (X = Mo, Nb, Ta) alloys showed similar trends in the data for most of the elastic stiffness constants, so do the Ti-X-Zr (X = Mo, Nb, Ta) alloys. The present calculations showed that the bcc Ti-alloy was mechanically stabilized at compositions less than 91, 92, 95, 93, 91, 94, 87, 77, 89, and 80 at \% Ti for the Ti-Mo-Nb, Ti-Mo-Ta, Ti-Mo-Zr, Ti-Nb-Zr, Ti-Sn-Zr, Ti-Ta-Zr, Ti-Mo-Sn, Ti-Nb-Sn, Ti-Nb-Ta and Ti-Sn-Ta alloys, respectively. As discussed above, Mo, Nb and Ta are strong $\beta$-stabilizers and thus the Ti-Mo-Nb, Ti-Mo-Ta, and Ti-Nb-Ta systems stabilize the bcc phase similarly. Also, discussed previously, Zr is a weak $\beta$-stabilizer alone but when alloyed with other elements it acts a strong $\beta$-stabilizer. This is observed with these results with the Ti-Mo-Zr, Ti-Nb-Zr, Ti-Ta-Zr systems all stabilizing the bcc phase at high Ti concentrations (95, 93, and 94 at.\% respectively). Zr is even able to stabilize the Ti-Sn-Zr system at a high Ti concentration of 91 at.\% Ti, even with Sn. Sn is not stable in the bcc phase and is not a $\beta$-stabilizer. So, when alloyed with Sn, a higher concentration of other alloying elements is needed to stabilize the bcc phase. The ternary interaction parameters were combined with the previously determined pure elements and binary interaction parameters to map some of the possible alloy compositions to find potential materials with a Young's modulus in the target range for biomedical load-bearing implants. Overall, the introduction of the ternary interaction parameters improved the database's ability to predict the $E$ of higher order alloys by a small amount. The complete database, however, satisfactorily predicts the elastic properties of higher order Ti-alloys.
	\item PUT IN CONCLUSIONS FROM CHAPTER 7
\end{enumerate}

\section{Future Work}

The following are presented as future work to be able to improve this thesis work:
\begin{enumerate}
	\item Focusing on the modeling of the Sn binary and Ti-Sn-X (X = Mo, Nb, Ta, Zr) and incorporating them into this database would further improve the knowledge base of Ti-alloys that this thesis presents. As discussed, Sn is being added due to its low cost and the fact that at low concentrations it doesn not effect the alloys biocompatiblity. But even at low compositions having a better understanding of how Sn effects the phase stability would be helpful.
	\item As discussed, introducting interaction parameters to describe the elastic properites for the non Ti-containing binary and ternary systems would improve the databases accuracy
	\item The work on using the partition function approach will be continued. The next steps will be to calculate the helmholtz energies of all of the pure elements and to extend this to calculate better helmholtz eneriges for the metastable  and unstable phases. The ability of the partition function approach to calculate more accurate entropies would be a great advancement in the field. 