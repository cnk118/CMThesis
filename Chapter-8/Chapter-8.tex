\chapter{Conclusions and Future Work}

\section{Conclusions}

In this dissertation, the effect of alloying elements on Ti-based alloys was systematically studied. The work began by using first-principles based DFT calculations and the CALPHAD method to study the effect that the alloying elements Mo, Nb, Sn, Ta and Zr have on the equilibrium phase stability, thermodynamics, and elastic properties. The work uses the equation of states fitting of the energy vs. volume curves to get the ground state equilibrium properties. The Debye-Gr\"uneisen model and phonon quasiharmonic approach were used to study the effect of temperature on the phase stability. A new theoretic framework was proposed to study the formation of the metastable phases. The accuracy of the theoretic framework and the transformation that occurs when these metastable phases form was studied using neutron scattering experiments. The compilation of the work develops a knowledge base for Ti-based alloys and will help to guide the future design of biocompatible implants. The main conclusions from this work are included below:

\begin{enumerate}
	\item A compatible thermodynamic database for the Ti-Mo-Nb-Ta-Zr system was built using descriptions of five pure elements (Ti, Mo, Nb, Ta, Zr), ten binary systems (Ti-Mo, Ti-Nb, Ti-Ta, Ti-Zr, Mo-Nb, Mo-Ta, Mo-Zr, Nb-Ta, Nb-Zr, Ta-Zr), and six Ti-containing ternary systems (Ti-Mo-Nb, Ti-Mo-Ta, Ti-Mo-Zr, Ti-Nb-Ta, Ti-Nb-Zr, Ti-Ta-Zr). Sn was excluded from the database due to the lack of modeling for the Sn binary systems. The Sn-Ta and Mo-Sn system lacked a thermodynamic description and the thermodynamic modeling of the Sn-Zr system is incompatible with the current database. The present work began modeling the Sn binaries with the Sn-Ta system discussed in chapter 4. Until the binaries are properly modeled Sn was not included in the database which shouldn't affect the use of the database for biomedical applications since Sn will only be added to biomedical alloys in small percentages. The thermodynamic descriptions of the pure elements were adopted from the SGTE database \cite{Dinsdale1991}. All of the binary systems had previous thermodynamic descriptions available and the compatible descriptions were evaluated for accuracy and incorporated into the database. The binary interpolations of the Ti-containing ternary systems were plotted and compared with the available experimental data as well as the enthalpy of formation of the bcc phase calculated from first-principles based on DFT. The Ti-Sn-X systems (X = Mo, Nb, Ta, Zr) will be modeled in the future work once the Sn binaries are modeled. The binary interpolations of the Ti-Nb-Zr and Ti-Ta-Zr systems had previously been plotted but no interaction parameters had been introduced. The present evaluation agreed with the previous evaluations and no ternary interaction parameters were introduced. The Ti-Mo-Zr system had previously been modeled and the present work agreed with the evaluation. The Ti-Mo-Nb, Ti-Mo-Ta and Ti-Nb-Ta systems had never previously been modeled. The present work evaluated interaction parameters for the Ti-Mo-Ta and Ti-Nb-Ta systems but didn't introduce any interaction parameters for the Ti-Mo-Nb system. The thermodynamic descriptions were all incorporated into a complete database that accurately predicts the phase stability of the Ti-Mo-Nb-Ta-Zr systems. The completed database is in appendix A.
	\item Sn-Ta modeling was completed using data from DFT-based first-principles calculations and the available experimental data in the literature to model the Gibbs energies for the bcc and liquid solution phases and the stoichiometric Ta$_3$Sn and TaSn$_2$ phases of the Sn-Ta system. First-principles calculations were used to predict the enthalpy of formation of the bcc phase for the evaluation of interaction parameters in the phase. The decomposition temperature of Ta$_3$Sn was predicted to be 2884 $^\circ$K. The completed thermodynamic description was compiled into a tdb file in appendix B.
	\item The effects of five alloying elements on the elastic properties of bcc Ti-X (X = Mo, Nb, Sn, Ta, Zr) alloys, including the elastic stiffness coefficients, bulk modulus, shear modulus, and Young's modulus, were systematically studied using first-principles calculations. The CALPAHD methodology was used to evaluate interaction parameters to predict the elastic properties as a function of composition. The calculations showed that 4.0, 5.5, 9.5, 11.5 and 51.5 at. \% of Zr, Mo, Ta, Nb and Sn, respectively, were required to stabilize the bcc phase according to the Born criteria. While the Mo, Nb and Ta elements are considered strong $\beta$-stabilizers and Zr is considered a weak $\beta$-stabilizer, this work shows that Zr stabilizes the bcc phase at the lowest concentration. The trends observed were summarized for each Ti-X (X= Mo, Nb, Sn, Ta, Zr) binary system. Alloying with Mo, Nb, and Ta results in similar trends, which is probably because Mo, Nb, and Ta are strong bcc stabilizers and stable in the bcc structure at room temperature. The interaction parameters determined in the current work were used to predict the elastic properties of higher order alloys. The accuracy of database predictions of the Young's modulus was evaluated by comparing the calculated and experimental Young's moduli. Overall, the database provides good predictions of the elastic properties of Ti-alloys in the bcc phase as a function of composition.
	\item The elastic properties of the bcc Ti-X-Y ternary alloys (X $\neq$ Y = Mo, Nb, Sn, Ta, Zr), including the elastic stiffness coefficients, bulk modulus, shear modulus, and Young's modulus were systematically studied using first-principles based on DFT calculations. The general CALPHAD modeling approach was used to fit ternary interaction parameters. From the elastic stiffness constant data, the Ti-X-Y (X $\neq$ Y = Mo, Nb, Ta) show the same trends in the data. This is to be expected because Mo, Nb, and Ta are similar elements that are strong $\beta$-stabilizers and stable in the bcc phase at low temperatures. It was also seen that the Ti-X-Sn (X = Mo, Nb, Ta) alloys showed similar trends in the data for most of the elastic stiffness coefficients, so do the Ti-X-Zr (X = Mo, Nb, Ta) alloys. The present calculations showed that the bcc Ti-alloy was mechanically stabilized at compositions less than 95, 94, 93, 92, 91, 91, 89, 87, 80, and 77 at \% Ti for the Ti-Mo-Zr, Ti-Ta-Zr, Ti-Nb-Zr, Ti-Mo-Ta, Ti-Mo-Nb, Ti-Sn-Zr, Ti-Nb-Ta, Ti-Mo-Sn, Ti-Sn-Ta and Ti-Nb-Sn alloys, respectively. As discussed above, Mo, Nb and Ta are strong $\beta$-stabilizers and thus the Ti-Mo-Nb, Ti-Mo-Ta, and Ti-Nb-Ta systems stabilize the bcc phase similarly. Also, discussed previously, Zr is known as a weak $\beta$-stabilizer alone but when alloyed with other elements it acts a strong $\beta$-stabilizer. However, Zr was able to stabilize the bcc phase in the Ti-X alloys at a lower concentration than any other element. This is observed with the Ti-Mo-Zr, Ti-Nb-Zr, Ti-Ta-Zr systems all stabilizing the bcc phase at the highest Ti concentrations (95, 93, and 94 at.\% respectively). Zr was even able to stabilize the Ti-Sn-Zr system at a high Ti concentration of 91 at.\% Ti, even with Sn not being a $\beta$-stabilizer or stable in the bcc phase. The ternary interaction parameters were combined with the previously determined pure elements and binary interaction parameters to map some of the possible alloy compositions to find potential materials with a Young's modulus in the target range for biomedical load-bearing implants. Overall, the introduction of the ternary interaction parameters improved the database's ability to predict the $E$ of higher order alloys by a small amount. The complete database is in appendix D and satisfactorily predicts the elastic properties of higher order Ti-alloys.
	\item The elastic stiffness coefficients and Young's moduli of the Ti-Nb system in the bcc, hcp, $\omega$, and $\alpha"$ phases were systematically calculated using first-principles based on DFT. The general CALPHAD modeling approach was used to fit binary interaction parameters. The $E$ values were similar for the hcp and $\omega$ phase, which is reasonable since they both have hexagonal symmetry. The $\alpha"$ phase has $E$ values that were higher than the other three phases which explains why the $E$ increases when the $\alpha"$ phase forms. Experiments showed that up to 10 at. \% Nb the samples formed solely the hcp phase and the database predicted the $E$ values by an average variance of 3 GPa from the experimental $E$. The samples from 10 at. \% Nb to 30 at. \% Nb formed the bcc and $\alpha"$ or $\omega$ phases. If the samples were slow cooled they form the bcc and $\omega$ phases. If the samples were quenched they form the bcc and $\alpha"$ phases. Using experimentally determined phase fractions and the rule of mixtures, the database accurately predicted the $E$ values by an average variance of 0.52 GPa when compared with the experimental $E$ values. At Nb concentrations greater than 30 at. \% Nb samples form solely the bcc phase and the database predicted the $E$ values by an average variance of 7 GPa from the experimental $E$ values. The phonon DOS of the slow cooled samples and the phonon DOS of the quenched samples were plotted together for the same compositions and showed differences that were expected for the samples have different phases. This difference was also seen when looking at the entropy difference between the two samples. The entropy difference between the quenched and slow cooled samples increased from 10 at. \% Nb to 20 at. \% Nb. This increase in entropy difference must be investigated further in order to understand this observation. Using the diffraction patterns, the phase fractions of each sample were approximated. The implementation of the partition function approach is in progress but the results presented here show that the elastic database can accurately assess the Young's moduli and elastic stiffness coefficients of Ti-Nb alloys if the phase fractions of the metastable phases can be predicted.
\end{enumerate}

\section{Future Work}

The following are presented as future work to be able to improve this thesis work:
\begin{enumerate}
	\item Focusing on the modeling of the Sn binary and Ti-Sn-X (X = Mo, Nb, Ta, Zr) systems and incorporating them into this database would further improve the knowledge base of Ti-alloys that this thesis presents. As discussed, Sn is being added due to its low cost and the fact that at low concentrations it does not affect the alloys biocompatibility. But even at low compositions having a better understanding of how Sn effects the phase stability would be helpful.
	\item As discussed, introducing interaction parameters to describe the elastic properties for the non Ti-containing binary and ternary systems and the hcp would improve the databases accuracy and extend its uses to other applications
	\item The work on using the partition function approach will be continued. The next steps will be to calculate the Helmholtz energies of all of the pure elements and to extend this to calculate better Helmholtz energies for the metastable  and unstable phases. The ability of the partition function approach to calculate more accurate entropies would be a great advancement in the field. 
	\item Study the temperature dependence of the phonon DOS to gain insight into the type of transformation occurring when $\alpha"$ and $\omega$ form
\end{enumerate}