\Appendix{Pycalphad script}

\noindent This code is used in pycalphad to plot the elastic moduli and elastic properites as a function of composition. A tdb file must be loaded into the script. In the present work the tdb file in Appendix D is used.
{\lstset{language=Python}
\footnotesize
\begin{lstlisting}
import matplotlib
from matplotlib.axes import Axes
from matplotlib.patches import Polygon
from matplotlib.path import Path
from matplotlib.ticker import NullLocator, Formatter, FixedLocator
from matplotlib.transforms import Affine2D, BboxTransformTo, IdentityTransform
from matplotlib.projections import register_projection
import matplotlib.spines as mspines
import matplotlib.axis as maxis
import matplotlib.pyplot as plt

import numpy as np

class TriangularAxes(Axes):
    """
    A custom class for triangular projections.
    """

    name = 'triangular'

    def __init__(self, *args, **kwargs):
         Axes.__init__(self, *args, **kwargs)
         self.set_aspect(1, adjustable='box', anchor='SW')
         self.cla()

    def _init_axis(self):
         self.xaxis = maxis.XAxis(self)
         self.yaxis = maxis.YAxis(self)
         self._update_transScale()

    def cla(self):
         """
         Override to set up some reasonable defaults.
         """
         # Don't forget to call the base class
         Axes.cla(self)

         x_min = 0
         y_min = 0
         x_max = 1
         y_max = 1
         x_spacing = 0.1
         y_spacing = 0.1
         self.xaxis.set_minor_locator(NullLocator())
         self.yaxis.set_minor_locator(NullLocator())
         self.xaxis.set_ticks_position('bottom')
         self.yaxis.set_ticks_position('left')
         Axes.set_xlim(self, x_min, x_max)
         Axes.set_ylim(self, y_min, y_max)
         self.xaxis.set\_ticks(np.arange(x\_min, x\_max+x\_spacing, x\_spacing))
         self.yaxis.set\_ticks(np.arange(y\_min, y\_max+y\_spacing, y\_spacing))

    def \_set\_lim\_and\_transforms(self):
         """
         This is called once when the plot is created to set up all the
         transforms for the data, text and grids.
         """
         # There are three important coordinate spaces going on here:
         #
         #    1. Data space: The space of the data itself
         #
         #    2. Axes space: The unit rectangle (0, 0) to (1, 1)
         #       covering the entire plot area.
         #
         #    3. Display space: The coordinates of the resulting image,
         #       often in pixels or dpi/inch.

         # This function makes heavy use of the Transform classes in
         # ``lib/matplotlib/transforms.py.`` For more information, see
         # the inline documentation there.

         # The goal of the first two transformations is to get from the
         # data space (in this case longitude and latitude) to axes
         # space.  It is separated into a non-affine and affine part so
         # that the non-affine part does not have to be recomputed when
         # a simple affine change to the figure has been made (such as
         # resizing the window or changing the dpi).

         # 1) The core transformation from data space into
         # rectilinear space defined in the HammerTransform class.
         self.transProjection = IdentityTransform()
         # 2) The above has an output range that is not in the unit
         # rectangle, so scale and translate it so it fits correctly
         # within the axes.  The peculiar calculations of xscale and
         # yscale are specific to a Aitoff-Hammer projection, so don't
         # worry about them too much.
         self.transAffine = Affine2D.from_values(
              1., 0, 0.5, np.sqrt(3)/2., 0, 0)
         self.transAffinedep = Affine2D.from_values(
              1., 0, -0.5, np.sqrt(3)/2., 0, 0)
         #self.transAffine = IdentityTransform()

         # 3) This is the transformation from axes space to display
         # space.
         self.transAxes = BboxTransformTo(self.bbox)

         # Now put these 3 transforms together -- from data all the way
         # to display coordinates.  Using the '+' operator, these
         # transforms will be applied "in order".  The transforms are
         # automatically simplified, if possible, by the underlying
         # transformation framework.
         self.transData = \
              self.transProjection + \
              self.transAffine + \
              self.transAxes

         # The main data transformation is set up.  Now deal with
         # gridlines and tick labels.

         # Longitude gridlines and ticklabels.  The input to these
         # transforms are in display space in x and axes space in y.
         # Therefore, the input values will be in range (-xmin, 0),
         # (xmax, 1).  The goal of these transforms is to go from that
         # space to display space.  The tick labels will be offset 4
         # pixels from the equator.

         self.\_xaxis\_pretransform = IdentityTransform()
         self.\_xaxis\_transform = \
              self.\_xaxis\_pretransform + \
              self.transData
         self.\_xaxis\_text1\_transform = \
              Affine2D().scale(1.0, 0.0) + \
              self.transData + \
              Affine2D().translate(0.0, -20.0)
         self.\_xaxis\_text2\_transform = \
              Affine2D().scale(1.0, 0.0) + \
              self.transData + \
              Affine2D().translate(0.0, -4.0)

         # Now set up the transforms for the latitude ticks.  The input to
         # these transforms are in axes space in x and display space in
         # y.  Therefore, the input values will be in range (0, -ymin),
         # (1, ymax).  The goal of these transforms is to go from that
         # space to display space.  The tick labels will be offset 4
         # pixels from the edge of the axes ellipse.

         self._yaxis_transform = self.transData
         yaxis_text_base = \
              self.transProjection + \
              (self.transAffine + \
              self.transAxes)
         self._yaxis_text1_transform = \
              yaxis_text_base + \
              Affine2D().translate(-8.0, 0.0)
         self._yaxis_text2_transform = \
              yaxis_text_base + \
              Affine2D().translate(8.0, 0.0)

    def get_xaxis_transform(self,which='grid'):
         assert which in ['tick1','tick2','grid']
         return self._xaxis_transform

    def get_xaxis_text1_transform(self, pad):
         return self._xaxis_text1_transform, 'bottom', 'center'

    def get_xaxis_text2_transform(self, pad):
         return self._xaxis_text2_transform, 'top', 'center'

    def get_yaxis_transform(self,which='grid'):
         assert which in ['tick1','tick2','grid']
         return self._yaxis_transform

    def get\_yaxis_text1_transform(self, pad):
         return self._yaxis_text1_transform, 'center', 'right'

    def get\_yaxis_text2_transform(self, pad):
         return self._yaxis_text2_transform, 'center', 'left'

    def _gen_axes_spines(self):
         dep_spine = mspines.Spine.linear_spine(self,
                                                    'right')
         # Fix dependent axis to be transformed the correct way
         dep_spine.set_transform(self.transAffinedep + self.transAxes)
         return {'left':mspines.Spine.linear_spine(self,
                                                        'left'),
                     'bottom':mspines.Spine.linear_spine(self,
                                                            'bottom'),
            'right':dep_spine}

    def _gen_axes_patch(self):
         """
         Override this method to define the shape that is used for the
         background of the plot.  It should be a subclass of Patch.
         Any data and gridlines will be clipped to this shape.
         """

         return Polygon([[0,0], [0.5,np.sqrt(3)/2], [1,0]], closed=True)

    # Interactive panning and zooming is not supported with this projection,
    # so we override all of the following methods to disable it.
    def can_zoom(self):
         """
         Return True if this axes support the zoom box
         """
         return False
    def start_pan(self, x, y, button):
         pass
    def end_pan(self):
         pass
    def drag_pan(self, button, key, x, y):
         pass


# Now register the projection with matplotlib so the user can select
# it.
register_projection(TriangularAxes)





import pycalphad.io.tdb_keywords
pycalphad.io.tdb_keywords.TDB_PARAM_TYPES.extend\
(['EM', 'BULK', 'SHEAR', 'C11', 'C12', 'C44'])
from pycalphad import Database, Model, equilibrium, calculate
import numpy as np
import pycalphad.variables as v
import sympy
from tinydb import where

class ElasticModel(Model):
    def build_phase(self, dbe):
         phase = dbe.phases[self.phase_name]
         param_search = dbe.search
         # EM, BULK, SHEAR, C11, C12, C44
         for prop in ['EM', 'BULK', 'SHEAR', 'C11', 'C12', 'C44']:
              prop_param_query = (
              (where('phase_name') == phase.name) & \
              (where('parameter_type') == prop) & \
              (where('constituent_array').test(self._array_validity))
              )
              prop_val = self.redlich_kister_sum \
              (phase, param_search, prop_param_query).subs(dbe.symbols)
              setattr(self, prop, prop_val)





dbf = Database('ElasticTi.tdb')
mod = ElasticModel(dbf, ['TI', 'SN', 'ZR', 'VA'], 'BCC_A2')
symbols = dict([(sympy.Symbol(s), val) for s, val in dbf.symbols.items()])
mod.C11 = mod.C11.xreplace(symbols)
mod.C12 = mod.C12.xreplace(symbols)
mod.C44 = mod.C44.xreplace(symbols)
x1 = np.linspace(0,1, num=100)
x2 = np.linspace(0,1, num=100)
mesh = np.meshgrid(x1, x2)
X = mesh[0]
Y = mesh[1]
mesh_arr = np.array(mesh)
mesh_arr = np.moveaxis(mesh_arr, 0, 2)
dep_col = 1 - np.sum(mesh_arr, axis=-1, keepdims=True)
mesh_arr = np.concatenate((mesh_arr, dep_col), axis=-1)
mesh_arr = np.concatenate((mesh_arr, np.ones(mesh_arr.shape[:-1] + (1,))), axis=-1)
orig_shape = tuple(mesh_arr.shape[:-1])
mesh_arr = mesh_arr.reshape(-1, mesh_arr.shape[-1])
mesh_arr[np.any(mesh_arr < 0, axis=-1), :] = np.nan
res_c11 = calculate(dbf, ['TI', 'SN', 'TA', 'VA'], 'BCC_A2', T=300, P=101325,
      model=mod, output='C11', points=mesh_arr)
res_c11 = res_c11.C11.values.reshape(orig_shape)
res_c12 = calculate(dbf, ['TI', 'SN', 'TA', 'VA'], 'BCC_A2', T=300, P=101325,
      model=mod, output='C12', points=mesh_arr)
res_c12 = res_c12.C12.values.reshape(orig_shape)
res_c44 = calculate(dbf, ['TI', 'SN', 'TA', 'VA'], 'BCC_A2', T=300, P=101325,
      model=mod, output='C44', points=mesh_arr)
res_c44 = res_c44.C44.values.reshape(orig_shape)





import numpy as np
def compute_moduli(c11, c12, c44):
    """Consume elastic stiffness constants and, under symmetry assumptions, compute
    bulk modulus, shear modulus, and Young's modulus.

    Parameters
    ----------
    c11: float64 array-like
    c12: float64 array-like
    c44: float64 array-like

    Returns
    -------
    B, G, Y : tuple of float64 array-likes"""
    # Ported from a matlab code
    c11 = np.array(c11)
    c12 = np.array(c12)
    c44 = np.array(c44)
    cij = np.zeros(c11.shape + (6,6))
    cij[..., 0, 0] = cij[..., 1, 1] = cij[..., 2, 2] = c11
    cij[..., 0, 1] = cij[..., 1, 0] = cij[..., 0, 2] = \
    cij[..., 2, 0] = cij[..., 1, 2] = cij[..., 2, 1] = c12
    cij[..., 3, 3] = cij[..., 4, 4] = cij[..., 5, 5] = c44
    sij = np.linalg.inv(cij)
    A_c = (cij[..., 0, 0] + cij[..., 1, 1] + cij[..., 2, 2]) / 3.
    B_c = (cij[..., 0, 1] + cij[..., 0, 2] + cij[..., 1, 2]) / 3.
    C_c = (cij[..., 3, 3] + cij[..., 4, 4] + cij[..., 5, 5]) / 3.
    A_s = (sij[..., 0, 0] + sij[..., 1, 1] + sij[..., 2, 2]) / 3.
    B_s = (sij[..., 0, 1] + sij[..., 0, 2] + sij[..., 1, 2]) / 3.
    C_s = (sij[..., 3, 3] + sij[..., 4, 4] + sij[..., 5, 5]) / 3.
    Bv = (A_c + 2*B_c) / 3.
    Gv = (A_c - B_c + 3*C_c) / 5.
    Br = 1. / (3*A_s + 6*B_s)
    Gr = 5. / (4*A_s - 4*B_s + 3*C_s)
    Bvrh = (Br + Bv) / 2.
    Gvrh = (Gr + Gv) / 2.
    Yvrh = (9*Bvrh*Gvrh) / (Gvrh + 3*Bvrh)
    return Bvrh, Gvrh, Yvrh

bulk_modulus, shear_modulus, young_modulus = \
compute_moduli(res_c11, res_c12, res_c44)





%matplotlib inline
import matplotlib.pyplot as plt

fig = plt.figure(figsize=(12,12))
ax = fig.gca(projection='triangular')
CS = ax.contour(X, Y, bulk_modulus, linewidths=4, \
levels=list(range(100, 300, 10)), cmap='cool')
ax.clabel(CS, inline=1, fontsize=13, fmt='%1.0f')
#PCM=ax.get_children()[0] #get the mappable, 
#the 1st and the 2nd are the x and y axes
#plt.colorbar(PCM, ax=ax)
ax.set_xlabel('x(Mo)', fontsize=18)
ax.set_ylabel('x(Nb)', fontsize=18, rotation=60, labelpad=-180)
ax.tick_params(axis='both', which='major', labelsize=18)
ax.tick_params(axis='both', which='minor', labelsize=18)
ax.set_title('Bulk modulus')
#fig.savefig('TiMoNb-Bulk.pdf')





%matplotlib inline
import matplotlib.pyplot as plt

fig = plt.figure(figsize=(12,12))
ax = fig.gca(projection='triangular')
CS = ax.contour(X, Y, shear_modulus, linewidths=4, \
levels=list(range(0, 150, 5)), cmap='cool')
ax.clabel(CS, inline=1, fontsize=13, fmt='%1.0f')
#PCM=ax.get_children()[0] #get the mappable, 
#the 1st and the 2nd are the x and y axes
#plt.colorbar(PCM, ax=ax)
ax.set_xlabel('x(Mo)', fontsize=18)
ax.set_ylabel('x(Nb)', fontsize=18, rotation=60, labelpad=-180)
ax.tick_params(axis='both', which='major', labelsize=18)
ax.tick_params(axis='both', which='minor', labelsize=18)
ax.set_title('Shear modulus')
#fig.savefig('TiMoNb-Shear.pdf')





%matplotlib inline
import matplotlib.pyplot as plt

fig = plt.figure(figsize=(12,12))
ax = fig.gca(projection='triangular')
CS = ax.contour(X, Y, young_modulus, linewidths=4, \
levels=list(range(0, 350, 10)), cmap='cool')
ax.clabel(CS, inline=1, fontsize=13, fmt='%1.0f')
#PCM=ax.get_children()[0] #get the mappable, 
#the 1st and the 2nd are the x and y axes
#plt.colorbar(PCM, ax=ax)
ax.set_xlabel('x(Mo)', fontsize=18)
ax.set_ylabel('x(Nb)', fontsize=18, rotation=60, labelpad=-180)
ax.tick_params(axis='both', which='major', labelsize=18)
ax.tick_params(axis='both', which='minor', labelsize=18)
ax.set_title('Young\'s modulus')
#fig.savefig('TiMoNb-Young.pdf')

\end{lstlisting}
}
