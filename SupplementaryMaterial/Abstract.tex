\paragraph*{} An aging population with an active lifestyle is demanding better load-bearing implants, which have biocompatibility and a low elastic modulus. Titanium alloys, in the body centered cubic phase, are great implant candidates due to their mechanical properties and biocompatibility. The present work aims at investigating the thermodynamic and elastic properties of bcc Ti-alloys using the integrated first-principles based on Density Functional Theory and the CALculation of PHAse Diagrams (CALPHAD) method. This use of integrated first-principles based on DFT and CALPHAD modeling has reduced the need for trial and error metallurgy which is ineffective and costly. The phase stability of Ti-alloys has seen to greatly affect the elastic properties. Traditionally, CALPHAD modeling is used to predict the equilibrium phase formation, in the case of Ti-alloys, predicting the formation of two metastable phases $\omega$ and $\alpha"$ is of great importance as these phases also drastically effect the elastic properties. To build a knowledge base of Ti-alloys for biomedical load-bearing implants the Ti-Mo-Nb-Sn-Ta-Zr system was chosen based on the biocompatibility of all elements as well as their bcc stabilizing effect.
\paragraph*{} With the focus being bcc Ti-rich alloys, the investigation began with building a database of the thermodynamic descriptions of each phase for the pure elements, binary and Ti-containing ternary alloys. Previous thermodynamic descriptions for the pure elements were adopted from the widely used SGTE database for global compatibility. The previous binary models from the literature were evaluated for accuracy and new thermodynamic descriptions were used for the Ti-containing ternary systems. The models were evaluated using available experimental data as well as the enthalpy of formation of the bcc phase calculated from first-principles based on DFT. The thermodynamic descriptions were combined into a database ensuring that the sublattice models were compatible with each other.
\paragraph*{} For subsystems, such as the Sn-Ta system, where no thermodynamic description had been evaluated and little to no experimental data was available, first-principles based on DFT was used. The Sn-Ta has two intermetallics, TaSn$_{2}$ and Ta$_{3}$Sn and five solution phases: liquid, bcc, hexagonal close packed (hcp), body centered tetragonal (bct) and diamond. First-principles calculations based on DFT were done on the intermetallics and solution phases. Special quasirandom structures (SQS) were used to obtain information about the solution phases across the entire composition range. The Debye-Gr$\ddot{u}$neisen approach as well as the quasiharmonic phonon method were used to obtain the finite-temperature data. The first-principles results as well as the experimental data was used to complete the thermodynamic description and the resulting phase diagram compares well with the data. To determine the effect of alloying on the elastic properties, first-principles based on DFT calculations were systematically done on the pure elements, five binary systems Ti-X (X = Mo, Nb, Ta, Zr, Sn) and Ti-containing ternary alloys Ti-X-Y (X $\neq $ Y = Mo, Nb, Sn, Ta Zr) in the bcc phase. The first-principles calculations predicted the single crystal elastic stiffness constants $c_{ij}$'s. Correspondingly, the polycrystalline aggregate properties are also estimated from the cij's, including bulk modulus, shear modulus and Young's modulus. The results showed good agreement when compared with experimental results. The CALPHAD method was then adapted to be able to build a database of the elastic properties as function of composition. On average, the database predicted the elastic properties of higher ordered Ti-alloys within 5 GPa of the experimental results.
\paragraph*{} Finally, the metastable phase formation, $\omega$ and $\alpha"$ was studied in the Ti-Ta and Ti-Nb systems. The formation of energy of these phases, calculated from first-principles, at 0 $^\circ$K showed that these phases must be stabilized by entropy. The partition function approach was adapted to be able to predict increase in entropy due to the competition between the metastable and stable phase. Using this approach, the formation and phase fraction of the phases were predicted for the Ti-Nb system. The predicted phase fractions were used to calculate the mixed force constants to obtain the phonon density of states. Then the predicted phase fractions were used to calculate the elastic properties using the rule of mixtures. Inelastic neutron scattering experiments were completed to compare the predicted phase fractions and phonon density of states for accuracy. The predicted elastic properties were compared with available experimental data.
\paragraph*{} Overall, this thesis provides a knowledge base of the thermodynamic and elastic properties of Ti-alloys from computational thermodynamics. The databases created will impact research focused on Ti-alloys and especially on research focused on Ti-alloys for biomedical applications.




