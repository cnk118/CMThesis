\paragraph*{} An aging population with an active lifestyle requires the development of better load-bearing implants, which have high levels of biocompatibility and a low elastic modulus. Titanium alloys, in the body centered cubic phase, are great implant candidates, due to their mechanical properties and biocompatibility. The present work aims at investigating the thermodynamic and elastic properties of bcc Ti-alloys, using the integrated first-principles based on Density Functional Theory (DFT) and the CALculation of PHAse Diagrams (CALPHAD) method. The use of integrated first-principles calculations based on DFT and CALPHAD modeling has greatly reduced the need for trial and error metallurgy, which is ineffective and costly. The phase stability of Ti-alloys has been shown to greatly affect their elastic properties. Traditionally, CALPHAD modeling has been used to predict the equilibrium phase formation, but in the case of Ti-alloys, predicting the formation of two metastable phases $\omega$ and $\alpha"$ is of great importance as these phases also drastically effect the elastic properties. To build a knowledge base of Ti-alloys, for biomedical load-bearing implants, the Ti-Mo-Nb-Sn-Ta-Zr system was studied because of the biocompatibility and the bcc stabilizing effects of some of the elements.
\paragraph*{} With the focus on bcc Ti-rich alloys, a database of thermodynamic descriptions of each phase for the pure elements, binary and Ti-rich ternary alloys was developed in the present work. Previous thermodynamic descriptions for the pure elements were adopted from the widely used SGTE database for global compatibility. The previous binary and ternary models from the literature were evaluated for accuracy and new thermodynamic descriptions were developed when necessary. The models were evaluated using available experimental data, as well as the enthalpy of formation of the bcc phase obtained from first-principles calculations based on DFT. The thermodynamic descriptions were combined into a database ensuring that the sublattice models are compatible with each other.
\paragraph*{} For subsystems, such as the Sn-Ta system, where no thermodynamic description had been evaluated and minimal experimental data was available, first-principles calculations based on DFT were used. The Sn-Ta system has two intermetallic phases, TaSn$_{2}$ and Ta$_{3}$Sn, with three solution phases: bcc, body centered tetragonal (bct) and diamond. First-principles calculations were completed on the intermetallic and solution phases. Special quasirandom structures (SQS) were used to obtain information about the solution phases across the entire composition range. The Debye-Gr\"uneisen approach, as well as the quasiharmonic phonon method, were used to obtain the finite-temperature data. Results from the first-principles calculations and experiments were used to complete the thermodynamic description. The resulting phase diagram reproduced the first-principles calculations and experimental data accurately.
\paragraph*{} In order to determine the effect of alloying on the elastic properties, first-principles calculations based on DFT were systematically done on the pure elements, five Ti-X binary systems and Ti-X-Y ternary systems (X $\neq $ Y = Mo, Nb, Sn, Ta Zr) in the bcc phase. The first-principles calculations predicted the single crystal elastic stiffness constants $c_{ij}$'s. Correspondingly, the polycrystalline aggregate properties were also estimated from the $c_{ij}$'s, including bulk modulus $B$, shear modulus $G$ and Young's modulus $E$. The calculated results showed good agreement with experimental results. The CALPHAD method was then adapted to assist in the database development of the elastic properties as a function of composition. On average, the database predicted the elastic properties of higher order Ti-alloys within 5 GPa of the experimental results.
\paragraph*{} Finally, the formation of the metastable phases, $\omega$ and $\alpha"$ was studied in the Ti-Ta and Ti-Nb systems. The formation energy of these phases, calculated from first-principles at 0 $^\circ$K, shows that these phases must be stabilized by entropy. A new theoretic framework was introduced that allows the prediction a combined Helmholtz energy that takes into account the increase in entropy due to the competition between the metastable and stable phase. Using this approach, the combined Helmholtz energy predicted the phase fractions of the phases for the Ti-Nb system as well as plotting the mixed states phonon density of states. Results from inelastic neutron scattering experiments were compared to the predicted phase fractions and phonon density of states for accuracy. Then the predicted phase fractions were used to calculate the elastic properties using the rule of mixtures. The predicted elastic properties were compared with available experimental data.
\paragraph*{} This thesis provides a knowledge base of the thermodynamic and elastic properties of Ti-alloys from computational thermodynamics. The databases created will impact research activities on Ti-alloys and specifically efforts focused on Ti-alloys for biomedical applications.
