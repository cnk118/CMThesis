\chapter{Methodology}

\section{First-Principles Calcualtions}

In this disseration, the ground state energy structures, thermodynamic propertie and mechanical properties are calcualted using first-principles based on Density Functional Theory. The first-priniciples refers to the calculations orginating from first-principles, meaning that the inputs are the atomic coordinates and atomic numbers. This method computes the interactions between atoms in a periodic supercell. This is determined using quantum mechanical electronic theroy that is based on the electronic charge density. do not rely on any emprical data. This section provides a description of the DFT methodology.

\subsection{Density Functional Theory at 0 K}

Solving Schr$\ddot{o}$dinger's time-independent non-relativisitc equation is a solution to the many-body problem of calculating the interactions of postiviely charged nuclei and negatively charged electrons. The Schr$\ddot{o}$dinger equation:

%%
\begin{equation}
\label{eq: schrodinger}
\hat{H} \Psi = E \Psi
\end{equation}
%%

is based on $\hat{H}$ representing the Hamiltonian of the system, $\Psi$ describing the wave function of electrons and $E$ denoting the systems energy. The $\hat{H}$ of the system is described by:

%%
\begin{equation}
\label{eq: hamiltonian}
\hat{H} = \sum_{i=1}^{N} (-\frac{\hbar^{2}}{2m} \nabla_{i}^{2} - e^{2} \sum_{a} \frac{Z_{a}}{|r_{i} - R_{a}}) + \frac{1}{2} \sum_{i \neq j} \frac{e^{2}}{|r_{i}-r_{j}}
\end{equation}
%%

where the first-term represents the kinetic energy and $N$ is the total number of electons in the system, $\hbar$ is Planck's constant, $m$ is the mass of a electron. The second term gives the interaction with nuclei where $r_i$ represents the position of electron $i$, and $R_a$ represents the position of nuclei $a$ with a chage valance of $Z_a$. The third term describes the electron-electron interaction where $r_i$ represents the position of electron $i$ and $r_j$ represents the position of electron $j$. The nuclei-nuclei interactions have been neglected due to the Born-Oppenheimer approximation which allows us to assume the nuclei are stationary and ignore the motion of nuclei on the electronic timescale. This assumption is due to the mass difference, with nuceli being ~ 10$^3$ to 10$^5$ larger than electrons. 

Using Eq. \ref{eq: hamiltonian}, Eq. \ref{eq: schrodinger} can be solved for $\Psi$ with the lowest energy $E_0$ being the ground state energy. However, even with approximations solving Eq. \ref{eq: schrodinger} difficult to deal with due to the electron-electron Columb interactions making the electronic motion correlated and the fact that the many-body problem results in too many variables because of the 3N degrees of freedom. The Hohenberg-Kohn-Sham formulated DFT fixes this problem by introducing two thereoms \cite{Hohenberg1964,Kohn1965}. The first theorem is that the external potential as a unique functional of the electron density. The second thereom states that the density that minimizes the total energy is the exact ground state density and thus the ground state is obtained variationaly. WIth these thereoms the Hamiltonian fixes the ground state of the system and the electron density is the only variable of interest. Therefore, the $E$ can be expressed as function of the electron density $\rho(r)$:

\begin{equation}
\label{eq: hkenergy}
E [\rho(r)] = T_{KE}[\rho(r)] + \int V_{ext} (\rho(r))\rho(r) d^3 r + E_{ee} [\rho(r)]
\end{equation}
%%

 where $T_{KE}$ is the kinetic energy, $V_{ext}$ is the external potential and $E_{ee}$ is the electron-electron interaction energy. Kohn-Sham sovled the electon density $\rho{r}$ for $N$ electrons:
 
 \begin{equation}
 \label{eq: electrondensity}
 \rho(r) = \sum_{i}^{N} |\phi_{i}(r)|^{2} 
 \end{equation}
 %%
 
 where $\phi_{i}(r)$ is the Kohn-Sham orbital. The Kohn-Sham orbital is solved by the equation:
 
  \begin{equation}
 \label{eq: kohnsham}
[-\frac{\hbar^2}{2m} \nabla^{2} + V (r) + V_{Hartree}(r) + V_{XC} (r)] \phi_{1}(r) = \varepsilon_{i} \psi_{i}(r)
 \end{equation}
 %%
 
where the solution to the Kohn-Sham equations are single-electron wave functions that depend on three spatial variables. The $V (r)$ orbitabl energy of the correspoinding Kohn-Sham orbital is described by $\varepsilon_{i}$ unc  in the wave function and electron-electron 
DFT is a method to predict the Helmholtz energy, F(V,T) as a function of temperature T and volume V via the quasiharmonic approach \cite{Shang2010,Wang2004}:

%%
\begin{equation}
\label{eq: helmholtz}
F(V,T) = E_{0}(V) + F_{vib}(V,T) + F_{T-el}(V,T)
\end{equation}
%%

where $E_0$ is the static contribution at 0 $^\circ$K without the contribution of zero-point vibrational energy, $F_{vib}$ is the temperature-dependent vibrational contribution, and $F_{T-el}$ is the thermal electronic contribution. At ambient pressure, the Helmholtz energy of the system is equal to the Gibbs energy, which is used in the CALPHAD modeling. In the present work, the $E_0$ was calculated from the equation of state (EOS) fitted to the first-principles data points using the four-parameter Birch-Murnaghan (BM4) EOS \cite{Shang2010}:

%%
\begin{equation}
\label{eq: zeroenergy}
E_{0}(V) = a + bV^{\frac{-2}{3}} + cV^{{-4}{3}} + dV^{-2}
\end{equation}
%%

The EOS fitting is achieved through an energy-volume ($E_{0}-V$) curve of 7 different relaxed volumes based on the methodology of Shang et al. \cite{Shang2010}. The valance configuration for each element was selected based on the VASP recommendations. The p electrons were treated as valance for the Mo and Ta, the d electrons were treated as valance for Sn and the s electrons were treated as valance for Ti, Nb, and Zr \cite{Kresse1996,Kresse1999}. 

\subsection{Finite-temperature thermodynamics}

 The vibrational contribution is obtained through the phonon quasiharmonic supercell (phonon approach) or the Debye-Grüneisen method (Debye). The phonon approach is a more accurate approach compared to the Debye model but it is also more computationally expensive. In the present work both the phonon and Debye models are used in different sections. The vibrational contribution is obtained through phonon calculations of at least five different volumes [18]: 

%%
\begin{equation}
\label{eq: phonon}
F_{vib}(V,T) = k_{b}T \int_{0}^{infty} ln[2sinh\frac{\hbar \omega}{2k_BT}] g(\omega) d\omega
\end{equation}
%%

where $g(\omega)$ is the phonon density of states as a function of phonon frequency $\omega$ at volume $V$.  In addition, the Debye model is used to estimate the vibrational contribution \cite{Shang2010}: 

%%
\begin{equation}
\label{eq: debye}
F_{vib}(V,T) = \frac{9}{8} k_{b} \theta_{D}(V) - k_{B}T[D (\frac{\theta_D(V){T}}) + 3ln(1-e^{\frac{-\theta_{D}(V)}{T}})] 
\end{equation}
%%

where $\theta_{D}$ is the Debye temperature, $T$ is the temperature, and $D[\frac{\theta{D}(V)}{T}]$ is the Debye function. The Debye temperature is calculated through: 

%%
\begin{equation}
\label{eq: debyetemp}
\theta_{D} = s \frac{(6\pi^2)^{\frac{1}{3}}\hbar}{k_B} V_{0}^{\frac{1}{6}} (\frac{B}{M})^{\frac{1}{2}} (\frac{V_0}{V})^{\gamma} 
\end{equation}
%%

where $s$ is the Debye temperature scaling factor, $\gamma$ is the Grüneisen parameter determined by the pressure derivative of bulk modulus ($B'$), $B$ is the bulk modulus, $M$ is the atomic mass, and $V_0$ is the equilibrium volume. Here the equilibrium properties $V_0$, $B$, and $B'$ are estimated from the EOS of Eq. Y. The Debye temperature scaling factor was determined by Moruzzi et al. \cite{Moruzzi1988} to be 0.617 for nonmagnetic metals. However, this value has been shown to be less accurate for other materials. Liu et al. extensively looked at the Debye scaling factor and how to calculate the scaling factor based on a the Poisson’s ratio of a material [20]. The methodology by Liu et al. \cite{Liu2015} was used for the present work to calculate the scaling factor: 

%%
\begin{equation}
\label{eq: debyescaling}
s(\nu) = 3^{\frac{5}{6}} [4\sqrt{2} (\frac{1 + \nu}{1 - \nu})^{\frac{3}{2}} + (\frac{1 + \nu}{1 - \nu})^{\frac{-1}{3}}]
\end{equation}
%%

where $\nu$ is the Poisson’s ratio, which can be calculated from the elastic stiffness constants.

The thermal electronic contribution is based on the electronic density of states and calculated with the Fermi-Dirac statistics [17].


ADD-look at xuan's

\subsection{Elastic stiffness calculations}

The single crystal elastic stiffness constants $c_{ij}'s$ were calculated from the ground state energy structure using a stress-strain method developed by Shang et al. \cite{Shang2007c}. With this method, a set of independent strains $\epsilon = (\epsilon_{1}, \epsilon_{2}, \epsilon_{3}, \epsilon_{4}, \epsilon_{5}, \epsilon_{6})$  were imposed on the crystal lattice, where $\epsilon_{1}$, $\epsilon_{2}$, and $\epsilon_{3}$ are the normal strains, $\epsilon_{4}$, $\epsilon_{5}$, and $\epsilon_{6}$ the shear strains, and a set of stresses $\sigma = (\sigma_{1}, \sigma_{2}, \sigma_{3}, \sigma_{4}, \sigma_{5},\sigma_{6})$ are generated. Hooke's law was used to calculate the elastic stiffness constants: 

%%
\begin{equation}
\label{eq: hookes}
\begin{pmatrix}
	c_{11} & c_{12} & c_{13} & 0 & 0 & 0\\
	c_{12} & c_{22} & c_{23} & 0 & 0 & 0\\
	c_{13} & c_{23} & c_{33} & 0 & 0 & 0\\
	0 & 0 & 0 & c_{44} & 0 & 0\\
	0 & 0 & 0 & 0 &  c_{55} & 0\\
	0 & 0 & 0 & 0 & 0 & c_{66} \\    		
\end{pmatrix} =
\begin{pmatrix}
	\epsilon_{1,1} & & \epsilon_{1,n}\\
	\epsilon_{2,1} & & \epsilon_{2,n}\\
	\epsilon_{3,1} & ... & \epsilon_{3,n}\\
	\epsilon_{4,1} & & \epsilon_{4,n}\\
	\epsilon_{5,1} & & \epsilon_{5,n}\\
	\epsilon_{6,1} & & \epsilon_{6,n}\\					
\end{pmatrix}^{-1}
\begin{pmatrix}
	\sigma_{1,1} & & \sigma_{1,n}\\
	\sigma_{2,1} & & \sigma_{2,n}\\
	\sigma_{3,1} & ... & \sigma_{3,n}\\
	\sigma_{4,1} & & \sigma_{4,n}\\
	\sigma_{5,1} & & \sigma_{5,n}\\
	\sigma_{6,1} & & \sigma_{6,n}\\					
\end{pmatrix}
\end{equation}
%%

where “-1” represents the pseudo-inverse. Due to symmetry, the bcc structure only has three independent elastic stiffness constants. However, with a lack of bcc stability for some of the calculations, all of the elastic stiffness constants are calculated and the average $\overline{C}_{11}$, $\overline{C}_{12}$ and $\overline{C}_{44}$ values are calculated:

%%
\begin{equation}
\label{eq: averagec11}
\overline{C}_{11} = \frac{(c_{11} + c_{12} + c_{44})}{3}
\end{equation}
%%

%%
\begin{equation}
\label{eq: averagec12}
\overline{C}_{12} = \frac{(c_{12} + c_{13} + c_{23})}{3}
\end{equation}
%%

%%
\begin{equation}
\label{eq: averagec44}
\overline{C}_{44} = \frac{(c_{44} + c_{55} + c_{66})}{3}
\end{equation}
%%

This case is for the unstable bcc elastic calculations to mimic the behavior of a cubic structure. The largest variance between the similar elastic stiffness constants, when calculating the average, is used to show the deviation from the bcc symmetry in the calculations shown as error bars. The stable bcc structures shows no variance and thus no error bars. To examine the effects of different strain on the elastic properties, three groups of non-zero strain magnitudes of $\pm$0.01, $\pm$0.03, and $\pm$0.07 are tested in chapter 5 and it can be noted that the results have negligible changes with respect to the three groups of strains tested herein. Therefore, $\pm$0.01 are used for all the calculations. The calculated elastic stiffness constants are used to calculate the polycrystalline elastic properties including bulk (B), shear (G), and Young’s (E) modulus using the Voigt-Reuss-Hill (VRH) approach \cite{Simmons1971b}, and the average results from the Hill approach are reported herein.

Based on Born's criteria $\overline{C}_{11}-\overline{C}_12$ \cite{Born1998,Nye1985}:

%%
\begin{equation}
\label{eq: born1}
\overline{C}_{11} -|\overline{C}_{12}| > 0
\end{equation}
%%

%%
\begin{equation}
\label{eq: born2}
\overline{C}_{11} + 2\overline{C}_{12} > 0
\end{equation}
%%

%%
\begin{equation}
\label{eq: born3}
\overline{C}_{44} > 0
\end{equation}
%%

when $\overline{C}_{11} - \overline{C}_{12}$  becomes negative then the phase loses mechanical stability. 

\subsection{Special quasirandom structures (SQS)}

To calculate the energies, enthalpies of formation and elastic properties across the entire binary and ternary composition range, varying compositions of special quasirandom structures (SQS) are used. The SQS are small supercells used to mimic randomly substituted structures in terms of correlation functions. The binary and ternary bcc SQS used in the present work were previously generated by Jiang et al.\cite{Jiang2004,Jiang2009}. The relaxation of these structures is complicated because the local atomic relaxations can cause the structure to lose the bcc lattice symmetry. To preserve structural symmetry, methodology discussed by Liu et al. \cite{Liu2013} and Zacherl et al. \cite{Zacherl2012} was used and the calculations were carried out with different relaxation schemes only alloying the cell shape and cell volume to be relaxed and to determine the lowest energy structure. In order to ensure the bcc symmetry was preserved the energies were plotted as a function of composition. Then the symmetry was verified. There are two ways to verify that whether the SQS is still bcc or not after the relaxation. The first is to merge different elements into one element for the SQS structure, and then, use codes available to check the symmetry or space group (such as VASP and phonopy codes). The second is to visualize the structure directly using a visualization software such as VESTA and compare the symmetry to the unrelaxed bcc structure. For the present work, the relaxed structures were plotted in the visualization software and compared to the unrelaxed structure. In some cases, even without comparing it was very obvious that the structure had lost bcc symmetry. For the cases where it wasn't obvious the comparison was done using symmetry codes (VASP, phonopy, etc). After the relaxation, at least five different volume structures are generated and the ions are allowed to relax. This yields the different volumes needed for the EOS fitting described above, which allows a better prediction of the different properties as a function of composition. 

\subsection{High-throughput partition function}

ADD

\subsection{First-principles calcualtion error}

The error between the previous results and present results was calculated using:

%%
\begin{equation}
\label{eq: error}
\sqrt{\frac{\Sigma[(A_{calc}-A{ref})]^{2}}{k}} = Error
\end{equation}
%%

where $A_{calc}$ is from the present calculation and $A_{ref}$ is from the previous calculation, and $k$ is the total number of data points. 

\section{CALPHAD method}

CALPHAD modeling is used to evaluate the parameters of the Gibbs energy function for each individual phase \cite{Liu2009}. The Gibbs energy functions of pure elements are adopted from the SGTE (SSUB) database \cite{Dinsdale1991}. The solution phases are modeled as: 

%%
\begin{equation}
\label{eq: gibbssolution}
G_m^{\phi} = x_{i} ^{0}G_{j}^{\phi} + 
\end{equation}
%%

where xi, xj and xk are the mole fractions of elements i, j and k, respectively,  is the molar Gibbs energy of pure elements in the specific phase being modeled, taken from the SGTE database [24].  The last term represents the excess mixing energy. The excess mixing energy can be expressed as [25]: 

Eq. 10

where δi,j,k is defined as δi,j,k=(1-xi-xj-xk)/3, and  and   are the binary and ternary interaction parameters defined as: 

Eq. 11

where a and b are parameters being modeled. 
For the stoichiometric compounds ApBq, the Gibbs energy in per mole unit formula used in the present work is of the form [26]: 

Eq. 12

where a and b are model parameters determined from enthalpy and entropy of formation,  is the Gibbs energy of pure element A in the stable element reference (SER) state,   is the Gibbs energy of pure element B in the SER state, p is the number of atoms per unit formula of A, and q is the number of atoms per unit formula B. The data from the DFT calculations and experiments are accounted for in the parameter modeling using the PARROT module in the Thermo-Calc software [27].
To obtain the elastic properties as a function of composition, the CALPHAD modeling approach fits the Redlich-Kister polynomial to find the binary and ternary interaction parameters [12], [25]. The binary fitting is done by extrapolating a linear relationship between the pure elements Ti and alloying element X of a specific system and estimating the difference between the DFT calculations and the linear extrapolation. The ternary fitting is done similarly but the linear extrapolation is between pure Ti and the X0.5Y0.5 mixture. The differences are then used to find the binary and ternary fitting parameter L0: 

Eq. 13

where xTi, xX, xY, are the mole fractions of Ti and the alloying elements X and Y, respectively, and ETi, EX and EY are the elastic properties of Ti, X, and Y, respectively, in the specific phase being studied. 



\subsection{Order-disorder model}

ADD

\subsection{Stoichiometric compounds}

ADD

\subsection{Elastic Properties}

ADD

\subsection{Optimization of thermodynamic parameters}

\subsubsection{Thermochemical data}

ADD

\subsubsection{Phase equilibria data}

ADD

\section{Experimental}

ADD

\subsection{Ti-Nb sample preparation}

ADD

\subsection{Neutron Scattering}

ADD

\subsubsection{ARCS}

ADD

\subsubsection{Data Analysis}

ADD

%%
\begin{equation}
\label{eq: label}
x = y
\end{equation}
%%
