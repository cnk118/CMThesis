\Appendix{Calculation and Fitting Details}

\section*{Calculation Details for Chapter 3, 5, and 6}
As discussed in each chapter the Monkhorst-Pack scheme is used for Brillouin zone sampling \cite{Kresse1996,Monkhorst1976a}. The k-points grid used for each calculation done in chapter 3, 5 and 6 are listed in the table. The k-points grids are listed as AxAxA. In some cases the automated k-point mesh generator in VASP was used and the length of the subdivisions specified are specified as number such as 50. 

\begin{longtable}[H]{ c c c }
	\hline
	Structure & Type of Calc & relaxation k-points\\
	\hline
	\endhead
	\hline
	\endfoot
	hcp-Ti & Elemental & 30 \\
	bcc-Ti & Elemental & 50 \\
	bcc-Mo & Elemental & 50 \\
	bcc-Nb & Elemental & 50 \\
	bcc-Ta & Elemental & 50 \\
	hcp-Zr & Elemental & 30 \\
	bcc-Zr & Elemental & 4x4x4 \\
	bcc-Ti$_{15}$Mo & Dilute & 5x5x5 \\
	bcc-Ti$_{7}$Mo & Dilute & 4x4x4\\
	bcc-Ti$_{75}$Mo$_{25}$ & SQS & 4x4x4\\
	bcc-Ti$_{50}$Mo$_{50}$ & SQS & 4x4x4\\
	bcc-Ti$_{25}$Mo$_{75}$ & SQS & 4x4x4\\
	bcc-TiMo$_{15}$ & Dilute & 80\\
	bcc-TiMo$_{53}$ & Dilute & 50\\
	bcc-Ti$_{53}$Nb & Dilute & 4x4x4\\
	bcc-Ti$_{7}$Nb & Dilute & 4x4x4\\ 
	bcc-Ti$_{75}$Nb$_{25}$ & SQS & 4x4x4\\
	bcc-Ti$_{50}$Nb$_{50}$ & SQS & 4x4x4\\
	bcc-Ti$_{25}$Nb$_{75}$ & SQS & 4x4x4\\
	bcc-TiNb$_{15}$ & Dilute & 80\\
	bcc-TiNb$_{53}$ & Dilute & 80\\
	bcc-Ti$_{15}$Sn & Dilute & 80\\
	bcc-Ti$_{75}$Sn$_{25}$ & SQS & 4x4x4\\
	bcc-Ti$_{50}$Sn$_{50}$ & SQS & 80\\
	bcc-Ti$_{25}$Sn$_{75}$ & SQS & 4x4x4\\
	bcc-Ti$_{53}$Ta & Dilute & 50\\
	bcc-Ti$_{15}$Ta & Dilute & 80\\
	bcc-Ti$_{7}$Ta & Dilute & 80\\
	bcc-Ti$_{75}$Ta$_{25}$ & SQS & 80\\
	bcc-Ti$_{50}$Ta$_{50}$ & SQS & 80\\
	bcc-Ti$_{25}$Ta$_{75}$ & SQS & 4x4x4\\
	bcc-TiTa$_{15}$ & Dilute & 80\\
	bcc-TiTa$_{53}$ & Dilute & 80\\
	bcc-Ti$_{53}$Zr & Dilute & 3x3x3\\
	bcc-Ti$_{75}$Zr$_{25}$ & SQS & 4x4x4\\
	bcc-Ti$_{50}$Zr$_{50}$ & SQS & 4x4x4\\
	bcc-Ti$_{25}$Zr$_{75}$ & SQS & 4x4x4\\
	bcc-TiZr$_{15}$ & Dilute & 80\\
	bcc-Mo$_{50}$Nb$_{50}$ & SQS & 3x3x3\\
	bcc-Mo$_{50}$Sn$_{50}$ & SQS & 80\\
	bcc-Mo$_{50}$Ta$_{50}$ & SQS & 80\\
	bcc-Mo$_{50}$Zr$_{50}$ & SQS & 80\\
	bcc-Nb$_{50}$Sn$_{50}$ & SQS & 80\\
	bcc-Nb$_{50}$Ta$_{50}$ & SQS & 80\\
	bcc-Nb$_{50}$Zr$_{50}$ & SQS & 80\\
	bcc-Sn$_{50}$Ta$_{50}$ & SQS & 80\\
	bcc-Sn$_{50}$Zr$_{50}$ & SQS & 80\\
	bcc-Ta$_{50}$Zr$_{50}$ & SQS & 80\\
	bcc-TiMoNb & SQS & 4x4x4\\
	bcc-Ti$_{2}$MoNb & SQS & 4x4x4\\
	bcc-Ti$_{6}$MoNb & SQS & 4x4x4\\
	bcc-TiMoSn & SQS & 4x4x4\\
	bcc-Ti$_{2}$MoSn & SQS & 4x4x4\\
	bcc-Ti$_{6}$MoSn & SQS & 4x4x4\\
	bcc-TiMoTa & SQS & 4x4x4\\
	bcc-Ti$_{2}$MoTa & SQS & 4x4x4\\
	bcc-Ti$_{6}$MoTa & SQS & 4x4x4\\
	bcc-TiMoZr & SQS & 4x4x4\\
	bcc-Ti$_{2}$MoZr & SQS & 4x4x4\\
	bcc-Ti$_{6}$MoZr & SQS & 4x4x4\\
	bcc-TiNbSn & SQS & 4x4x4\\
	bcc-Ti$_{2}$NbSn & SQS & 4x4x4\\
	bcc-Ti$_{6}$NbSn & SQS & 4x4x4\\
	bcc-TiNbSn & SQS & 4x4x4\\
	bcc-Ti$_{2}$NbSn & SQS & 4x4x4\\
	bcc-Ti$_{6}$NbSn & SQS & 4x4x4\\
	bcc-TiNbTa & SQS & 4x4x4\\
	bcc-Ti$_{2}$NbTa & SQS & 4x4x4\\
	bcc-Ti$_{6}$NbTa & SQS & 4x4x4\\
	bcc-TiNbZr & SQS & 4x4x4\\
	bcc-Ti$_{2}$NbZr & SQS & 4x4x4\\
	bcc-Ti$_{6}$NbZr & SQS & 4x4x4\\
	bcc-TiNbZr & SQS & 4x4x4\\
	bcc-Ti$_{2}$NbZr & SQS & 4x4x4\\
	bcc-Ti$_{6}$NbZr & SQS & 4x4x4\\
	bcc-TiSnTa & SQS & 4x4x4\\
	bcc-Ti$_{2}$SnTa & SQS & 4x4x4\\
	bcc-Ti$_{6}$SnTa & SQS & 4x4x4\\
	bcc-TiSnZr & SQS & 4x4x4\\
	bcc-Ti$_{2}$SnZr & SQS & 4x4x4\\
	bcc-Ti$_{6}$SnZr & SQS & 4x4x4\\
	bcc-TiTaZr & SQS & 4x4x4\\
	bcc-Ti$_{2}$TaZr & SQS & 4x4x4\\
	bcc-Ti$_{6}$TaZr & SQS & 4x4x4\\
	\hline
\end{longtable}
%%%
\clearpage
%%%

\section*{Fitting Code for Chapter 5 and 6}
The code used in Mathematica to fit the binary and ternary interaction parameters is listed below.

input: n = {{0, 0}, {0.019, -2.81}, {0.019, -2.81}, {0.019, -2.81}, {0.019, \
		-2.81}, {0.019, -2.81}, {0.019, -2.81}, {0.125, 4.00}, {0.125, 
		4.00}, {0.125, 4.00}, {0.125, 4.00}, {0.125, 4.00}, {0.125, 
		4.00}, {0.25, 8.67}, {0.25, 8.67}, {0.25, 8.67}, {0.25, 
		8.67}, {0.25, 8.67}, {0.25, 8.67}, {0.50, 12.00}, {0.75, 
		1.00}, {0.938, 6.42}, {0.981, -0.11}, {1, 0}}
\\
\noindent output: {{0, 0}, {0.019, -2.81}, {0.019, -2.81}, {0.019, -2.81}, {0.019, \
		-2.81}, {0.019, -2.81}, {0.019, -2.81}, {0.125, 4.}, {0.125, 
		4.}, {0.125, 4.}, {0.125, 4.}, {0.125, 4.}, {0.125, 4.}, {0.25, 
		8.67}, {0.25, 8.67}, {0.25, 8.67}, {0.25, 8.67}, {0.25, 
		8.67}, {0.25, 8.67}, {0.5, 12.}, {0.75, 1.}, {0.938, 
		6.42}, {0.981, -0.11}, {1, 0}}
\\
\noindent input: gp = ListPlot[n, PlotMarkers -> {$\square$, 10}, PlotStyle -> {Blue}]
\noindent input: Fit[n, {(x*(1 - x)), ((x*(1 - x))*(x - (1 - x)))}, x] (For a two parameter fit)
\noindent input: Fit[n, {(x*(1 - x))}, x] (For a one parameter fit)
\noindent input: Plot[\%, {x, 0, 1}, PlotRange -> {-100, 100}]
\noindent input: Show[\%, gp]

\begin{figure}[H]
	\centering
	\includegraphics[width=\textwidth]{Appendix-C/Figures/fitting.png}
\end{figure}