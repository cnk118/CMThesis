%%%%%%%%%%%%%%%%%%%%%%%%%%%%%%%%%%%%%%%%%%%%%%%%%%%%%%%%%%%%%%%
%
%     filename  = "Cassie-Dissertation.tex",
%     version   = "1",
%     date      = "2017/02/10",
%     authors   = "Cassie N. Marker,
%     copyright = "Cassie N. Marker",
%     address   = "Materials Science and Engineering,
%                  319 Steidle Building,
%                  Penn State University,
%                  University Park, PA 16802,
%                  USA",
%     email     = "cnk118@psu.edu",
%
%%%%%%%%%%%%%%%%%%%%%%%%%%%%%%%%%%%%%%%%%%%%%%%%%%%%%%%%%%%%%%%
%
% Change History:
%
% 1.3.0	**	Added the cite package.
%
%		**	Give that the Graduate School now allows essentially
%			any line spacing, I have moved the line space setting
%			from psuthesis.cls to this driver file. Go ahead and
%			make it ugly if you want. :-)
%%%%%%%%%%%%%%%%%%%%%%%%%%%%%%%%%%%%%%%%%%%%%%%%%%%%%%%%%%%%%%%
%
% This is a template file to help get you started using the
% psuthesis.cls for theses and dissertations at Penn State
% University. You will, of course, need to put the
% psuthesis.cls file someplace that LaTeX will find it.
%
% We have set up a directory structure that we find to be clean
% and convenient. You can readjust it to suit your tastes. In
% fact, the structure used by our students is even a little
% more involved and commands are defined to point to the
% various directories.
%
% This document has been set up to be typeset using pdflatex.
% About the only thing you will need to change if typesetting
% using latex is the \DeclareGraphicsExtensions command.
%
% The psuthesis document class uses the same options as the
% book class. In addition, it requires that you have the
% ifthen, calc, setspace, and tocloft packages.
%
% The first additional option specifies the degree type. You
% can choose from:
%	Ph.D. using class option <phd>
%	M.S. using class option <ms>
%	M.Eng. using class option <meng>
%	M.A. using class option <ma>
%	B.S. using class option <bs>
%	B.A. using class option <ba>
%	Honors Baccalaureate using the option <honors>
%
% If you specify either ba or bs in addition to honors, it will
% just use the honors option and ignore the ba or bs option.
%
% The second additional option <inlinechaptertoc> determines
% the formatting of the Chapter entries in the Table of
% Contents. The default sets them as two-line entries (try it).
% If you want them as one-line entries, issue the
% inlinechaptertoc option.
%
% The class option ``honors'' should be used for theses
% submitted to the Schreyer Honors College. This option
% changes the formatting on the Title page so that the
% signatures appear on the Title page.
%
% The class option ``honorsdepthead'' adds the signature of the
% department head on the Title page for those baccalaureate
% theses that require this.
%
% The class option ``secondthesissupervisor'' should be used
% for baccalaureate honors degrees if you have a second
% Thesis Supervisor.
%
% The vita is only included with the phd option and it is
% placed at the end of the thesis. The permissions page is only
% included with the ms, meng, and ma options.
%%%%%%%%%%%%%%%%%%%%%%%%%%%%%%%%%%%%%%%%%%%%%%%%%%%%%%%%%%%%%%%
\documentclass[phd,12pt]{psuthesis}

\usepackage[T1]{fontenc}
\usepackage{lmodern}
\usepackage{textcomp}
\usepackage{microtype}

%%%%%%%%%%%%%%%%%%%%%%%%%%%%
% Packages we like to use. %
%%%%%%%%%%%%%%%%%%%%%%%%%%%%
\usepackage{amsmath}
\usepackage{amssymb}
\usepackage{amsthm}
\usepackage{exscale}
\usepackage[mathscr]{eucal}
\usepackage{bm}
\usepackage{eqlist} % Makes for a nice list of symbols.
\usepackage[final]{graphicx}
\usepackage[dvipsnames]{color}
\DeclareGraphicsExtensions{.pdf, .jpg}
\usepackage{float}
\usepackage{gensymb}
\usepackage{longtable}
\usepackage{wasysym}
% http://www.tex.ac.uk/cgi-bin/texfaq2html?label=citesort
\usepackage{cite}

\usepackage{titlesec}

%%%%%%%%%%%%%%%%%%%%%%%%%%%%%%%
% Use of the hyperref package %
%%%%%%%%%%%%%%%%%%%%%%%%%%%%%%%
%
% This is optional and is included only for those students
% who want to use it.
%
% To the hyperref package, uncomment the following line:
\usepackage{hyperref}
%
% Note that you should also uncomment the following line:
\renewcommand{\theHchapter}{\thepart.\thechapter}
%
% to work around some a problem hyperref has with the fact
% the psuthesis class has unnumbered pages after which page
% counters are reset.

% Set the baselinestretch using the setspace package.
% The LaTeX Companion claims that a \baselinestretch of
% 1.24 gives one-and-a-half line spacing, which is allowed
% by the PSU thesis office. As of October 18, 2013, the Graduate
% School states ``The text of an eTD may be single-, double- or
% one- and-a-half-spaced.'' Go nuts!
\setstretch{1.24}


%%%%%%%%%%%%%%%%%%%%%%%%%%%%%%%%%%%%
% SPECIAL SYMBOLS AND NEW COMMANDS %
%%%%%%%%%%%%%%%%%%%%%%%%%%%%%%%%%%%%
\input{SupplementaryMaterial/UserDefinedCommands}


%%%%%%%%%%%%%%%%%%%%%%%%%%%%%%%%%%%%%%%%%
% Renewed Float Parameters              %
% (Makes floats fit better on the page) %
%%%%%%%%%%%%%%%%%%%%%%%%%%%%%%%%%%%%%%%%%
\renewcommand{\floatpagefraction}{0.85}
\renewcommand{\topfraction}      {0.85}
\renewcommand{\bottomfraction}   {0.85}
\renewcommand{\textfraction}     {0.15}

% ----------------------------------------------------------- %

%%%%%%%%%%%%%%%%
% FRONT-MATTER %
%%%%%%%%%%%%%%%%
% Title
\title{DEVELOPMENT OF A KNOWLEDGE BASE OF TI-ALLOYS FROM FIRST-PRINCIPLES AND COMPUTATIONAL THERMODYNAMICS}

% Author and Department
\author{Cassie Marker}
\dept{Materials Science and Engineering}
% the degree will be conferred on this date
\degreedate{August 2017}
% year of your copyright
\copyrightyear{2017}


% This is the document type. For example, this could also be:
%	Comprehensive Document
%	Thesis Proposal
%\documenttype{Thesis}
\documenttype{Dissertation}
%\documenttype{Comprehensive Document}


% This will generally be The Graduate School, though you can
% put anything in here to suit your needs.
\submittedto{The Graduate School}

% This is the college to in which you are submitting the
% thesis/dissertation.
\collegesubmittedto{College of Earth and Mineral Sciences}


%%%%%%%%%%%%%%%%%%
% Signatory Page %
%%%%%%%%%%%%%%%%%%
% You can have up to 7 committee members, i.e., one advisor
% and up to 6 readers.
%
% Begin by specifying the number of readers.
\numberofreaders{4}

% Input reader information below. The optional argument, which
% comes first, goes on the second line before the name.
\advisor[Thesis Advisor, Chair of Committee]
		{Zi-Kui Liu}
		{Professor of Materials Science and Engineering}

\readerone
			{Allison Beese}
			{Assistant Professor of Materials Science and Engineering}

\readertwo
			{Kristen Fichthorn}
			{Merrell Fenskey Professor of Chemical Engineering and Professor of Physics}

\readerthree[Special Member]
			{Michael Manley}
			{Senior Researcher, Oak Ridge National Laboratory, Materials Science \& Technology Division}

\readerfour
			{Susan Sinnott}
			{Department Head and Professor of Materials Science and Engineering}

% Format the Chapter headings using the titlesec package.
% You can format section headings and the like here too.
\definecolor{gray75}{gray}{0.75}
\newcommand{\hsp}{\hspace{15pt}}
\titleformat{\chapter}[display]{\fontsize{30}{30}\selectfont\bfseries\sffamily}{Chapter \thechapter\hsp\textcolor{gray75}{\raisebox{3pt}{|}}}{0pt}{}{}

\titleformat{\section}[block]{\Large\bfseries\sffamily}{\thesection}{12pt}{}{}
\titleformat{\subsection}[block]{\large\bfseries\sffamily}{\thesubsection}{12pt}{}{}


% Makes use of LaTeX's include facility. Add as many chapters
% and appendices as you like.
\includeonly{%
Chapter-1/Chapter-1,%
Chapter-2/Chapter-2,%
Chapter-3/Chapter-3,%
Chapter-4/Chapter-4,%
Chapter-5/Chapter-5,%
Chapter-6/Chapter-6,%
Chapter-7/Chapter-7,%
Chapter-8/Chapter-8,%
Appendix-A/Appendix-A,%
Appendix-B/Appendix-B,%
Appendix-C/Appendix-C,%
Appendix-D/Appendix-D,%
Appendix-E/Appendix-E,%
Appendix-F/Appendix-F%
}

\usepackage{listings}
%%%%%%%%%%%%%%%%%
% THE BEGINNING %
%%%%%%%%%%%%%%%%%
\begin{document}
%%%%%%%%%%%%%%%%%%%%%%%%
% Preliminary Material %
%%%%%%%%%%%%%%%%%%%%%%%%
% This command is needed to properly set up the frontmatter.
\frontmatter

%%%%%%%%%%%%%%%%%%%%%%%%%%%%%%%%%%%%%%%%%%%%%%%%%%%%%%%%%%%%%%
% IMPORTANT
%
% The following commands allow you to include all the
% frontmatter in your thesis. If you don't need one or more of
% these items, you can comment it out. Most of these items are
% actually required by the Grad School -- see the Thesis Guide
% for details regarding what is and what is not required for
% your particular degree.
%%%%%%%%%%%%%%%%%%%%%%%%%%%%%%%%%%%%%%%%%%%%%%%%%%%%%%%%%%%%%%
% !!! DO NOT CHANGE THE SEQUENCE OF THESE ITEMS !!!
%%%%%%%%%%%%%%%%%%%%%%%%%%%%%%%%%%%%%%%%%%%%%%%%%%%%%%%%%%%%%%

% Generates the title page based on info you have provided
% above.
\psutitlepage

% Generates the committee page -- this is bound with your
% thesis. If this is an baccalaureate honors thesis, then
% comment out this line.
\psucommitteepage

% Generates the abstract. The argument should point to the
% file containing your abstract. 
\thesisabstract{SupplementaryMaterial/Abstract}

% Generates the Table of Contents
\thesistableofcontents

% Generates the List of Figures
\thesislistoffigures

% Generates the List of Tables
\thesislistoftables

% Generates the List of Symbols. The argument should point to
% the file containing your List of Symbols. 
%\thesislistofsymbols{SupplementaryMaterial/ListOfSymbols}

% Generates the Acknowledgments. The argument should point to
% the file containing your Acknowledgments. 
\thesisacknowledgments{SupplementaryMaterial/Acknowledgments}

% Generates the Epigraph/Dedication. The first argument should
% point to the file containing your Epigraph/Dedication and
% the second argument should be the title of this page. 
\thesisdedication{SupplementaryMaterial/Dedication}{Dedication}



%%%%%%%%%%%%%%%%%%%%%%%%%%%%%%%%%%%%%%%%%%%%%%%%%%%%%%
% This command is needed to get the main part of the %
% document going.                                    %
%%%%%%%%%%%%%%%%%%%%%%%%%%%%%%%%%%%%%%%%%%%%%%%%%%%%%%
\thesismainmatter

%%%%%%%%%%%%%%%%%%%%%%%%%%%%%%%%%%%%%%%%%%%%%%%%%%
% This is an AMS-LaTeX command to allow breaking %
% of displayed equations across pages. Note the  %
% closing the "}" just before the bibliography.  %
%%%%%%%%%%%%%%%%%%%%%%%%%%%%%%%%%%%%%%%%%%%%%%%%%%
\allowdisplaybreaks{
%\pagestyle{fancy}
%\fancyhead{}
%
%%%%%%%%%%%%%%%%%%%%%%
% THE ACTUAL CONTENT %
%%%%%%%%%%%%%%%%%%%%%%
% Chapters
%SourceDoc ../YourName-Dissertation.tex
%\vspace*{-80mm}
\chapter{Introduction} \label{chapter1:introduction}

\section{\sloppy Motivation}
Titanium (Ti) and its alloys have been used in biomedical applications for many years because of their biocompatibility and corrosion resistance properties \cite{Long1998a}. In recent years, there has been an increasing interest in developing better materials for load-bearing implants, due to the increase in total knee and hip replacements. Krutz et al. predicted that the total number of hip and knee replacements would increase by 174\% and 673\%, respectively, from 2005 to 2030, leading to 572,000 hip and 3.48 million knee procedures in the United States in 2030 \cite{Kurtz2007}. Two of the driving factors for this situation involve the increasing number of younger individuals requiring replacements and the fact that the average life of these implants is only about 7-12 years \cite{Krishna2007a}. These factors contribute significantly to the necessity for better implant materials. The primary considerations for biomedical implants, such as load-bearing knee and hip implants, are biocompatibility, corrosion resistance, fatigue strength, and Young's modulus ($E$) \cite{Long1998a}. In previous years, the most common implants for these applications have been Ti-6Al-4V, stainless steels, and MoCoCr alloys \cite{Niinomi2003,Niinomi2012}. However, there have been issues with these materials, such as cytotoxicity that has been observed with alloys containing aluminum and vanadium \cite{Ito1995a}. Another important impediment concerning the common implant materials is stress shielding, which can lead to implant failure. Stress shielding occurs when the $E$ of the implant is higher than that of bone. Due to the difference in $E$, load applications to the joint result in the implant material absorbing all of the stress and causing the bone surrounding the implant to atrophy, which leads to a loss in bone density, and can result in implant loosening and failure \cite{Long1998a}.  Table \ref{table:commonEM} summarizes the comparison of the $E$ of common implant materials (> 100 GPa) to bone (10-40 GPa) \cite{Long1998a} and shows the extreme elasticity mismatch between the various materials. Using computational thermodynamics to develop a knowledge base of Ti and its alloys is an extremely useful tool in overcoming these challenges. 

This work focused on investigating the thermodynamic and elastic properties of the biocompatible Ti-Mo-Nb-Sn-Ta-Zr system. The thermodynamic and elastic properties were calculated using first-principles based on Density Functional Theory (DFT). The parametrization of the properties was completed using the CALPHAD modeling approach. The combination of these two methodologies has been shown to eliminate the need for trial-and-error metallurgy, thus saving time, money and resources. A new computational methodology to predict the metastable phase formation was presented and verified by neutron scattering experiments. The culmination of this work provides a fundamental understanding of the thermodynamics and elastic properties for the Ti-Mo-Nb-Sn-Ta-Zr system. 


\section{Overview}

\subsection{Equilibrium Phases}

The phase stability of Ti alloys has been shown to greatly affect the mechanical properties of these materials, so predicting and understanding this aspect of a Ti alloy will greatly impact its effectiveness as a biomedical implant. Titanium is stable in the $\alpha$ (hexagonal close packed, hcp) phase (space group P$6_{3}/$mmc) under standard temperature and pressure. However, at temperatures above 1155 $^{\circ}$K, Ti is stable in the $\beta$ (body centered cubic, bcc) phase (space group Im$\overline{3}$m). The bcc phase can also be stabilized by alloying and such bcc Ti alloys have received much attention because of their low $E$ values. Ti in the hcp phase has a $E$ of 105 GPa while, Ti-6Al-4V which is a two phase mixture of hcp and bcc has an $E$ of 110 and the Ti-35Nb-5Ta-7Zr alloy in the bcc phase has a $E$ of 55 GPa which is more comparable to that of bone (10-40 GPa) \cite{Long1998a,Jain2013,Antolin2012,Mei2011,Brailovski2011b}. Bcc phase alloys having lower Young's moduli has been seen for many other alloys as well, such as Ti-13Nb-13Zr, Ti-35Nb-5Ta-7Zr-0.4O, Ti-29Nb-Ta-Zr, and Ti-25Nb-Ta-Zr which are all bcc alloys that have a $E$ of between 71 and 57 GPa \cite{Long1998a,Tane2008a,Tane2010a}. With phase stability playing an important role in the alloy selection for load-bearing implants, it is important to study how to stabilize the bcc phase at low temperature (<1155 $^{\circ}$K). Mo, Nb and Ta were all chosen to be studied because they are biocompatible elements and strong $\beta$-stabilizers, while Zr is a bio-compatible weak $\beta$-stabilizer individually but strong stabilizer when in combination with other elements, such as Mo, Nb, and Ta \cite{Long1998a}. In conjunction with their biocompatiblity, studies have shown excellent Mo, Nb, Ta and Zr have excellent corrosion resistance and no allergy problems \cite{Tane2008a}. Recently, Sn has also been studied for use in Ti-alloys, due to its low cost \cite{Niinomi2012} and in small concentrations, Sn does not effect the biocompatiblity of the alloy. 

When many bcc Ti-alloys have been shown to have a $E$ more closely matching that of bone, in some cases a bcc Ti-alloy can miss the mark and not have a $E$ that comes close to matching bone. Such an example is Ti-16Nb-13Ta-4Mo which has $E$ of 110 GPa \cite{Geetha2009}. So, while understanding the thermodynamics of the system will help to target the bcc phase, also being able to predict the elastic properites before attempting to develop alloys for biomedical implants will reduce the need for trial and error and narrow the sope of alloys being selected.Therefore, the present study focused on determining the effects of alloying Ti with Mo, Nb, Sn, Ta and Zr on the thermodynamic amd elastic properties. The combined DFT and CALPHAD approach was used to evaluate previous models and build new models for the binary and ternary alloys in the Ti-Mo-Nb-Sn-Ta-Zr system to build a completed database describing the thermodynamic and elastic properties of the system.

\subsection{Metastable Phases}

The completed thermodynamic database predicts the formation of the equilibrium phases, hcp and bcc. Based on the predictions, alloys that are in the bcc phase can be targeted and their elastic properties predicted. However, Ti and its alloys can form two metastable phases, $\alpha"$ and $\omega$. $\alpha"$ is an orthorhombic martensitic phase (space group Cmcm). The martensitic transformation is displacive \cite{Khachaturyan1985,Salje1990}. Thermodynamically the martensitic transformation is first-order and initiated by supercooling defined by $T_{0}-M_{S}$, where $T_{0}$ is the temperature where bcc transforms to hcp and $M_{S}$ is the temperature where the martensitic transformation begins. An applied stress can also contribute to the driving force for a martensitic transformation. Kinetically the martensitic transformation propagates in an athermal manner which suggests that the velocity of the interface and the force causing its movement contain an instability. The $\omega$ phase is a metastable hexagonal phase (space group P6/mmm) of Ti that has lattice parameters closely matching that of bcc Ti. The $\omega$ phase has been seen to form athermally when Ti is alloyed with $\beta$ stabilizing elements such as Mo, Nb and Ta. It has been shown that different cooling techniques of alloys at certain compositions in the Ti-Mo, Ti-Nb and Ti-Ta systems cause either the $\alpha"$ phase or the $\omega$ phase to form with a matrix of untransformed bcc phase. Quenching the samples leads to the formation of $\alpha"$, while slow cooling the samples leads to the formation of $\omega$ phase. The formation of theses phases causes variations to the predicted elastic properties as seen in Figure \ref{Ch1-figure:titaelastic} and \ref{Ch1-figure:tinbelasitc}, where the closed symbols represent the calculated $E$ and the open symbols represent the experimentally determined $E$ from the literature. The calculations and experiments agree well on the Ti-rich and Ta-rich sides but in the region marked by the purple box, the experiments show a higher $E$ than predicted by calculations. This is due to the formation of $\alpha"$ and $\omega$. While the formation of the metastable phases greatly effects the properties of the Ti-alloys, there is no current way to predict their formation. In this dissertation, a new theoretical framework was proposed to predict the formation. To introduce the theoretical framework and ensure the accuracy, the phase stability and effect on the elastic properties of the $\alpha"$ and $\omega$ phases were studied for the Ti-Nb and Ti-Ta systems. Initially, the ground state energy and elastic properties of multiple structures in the $\alpha"$, $\omega$, bcc and hcp were calculated, across all compositions, for the Ti-Ta and Ti-Nb systems. The new theoretic framework predicts the concentrations and phases fractions where the metastable phases form. The predicted phase fractions were then used to predict the $E$ using the rule of mixtures and mixed force constants are used to obtain the phonon density of states (DOS). The reults from inelastic neutron scattering were then used to determine diffraction patterns and phonon density of states. By looking at the diffraction patterns obtained at 300 $^\circ$K, the phase fractions were obtained and compared to the predicted phase fractions. The phonon DOS obtained from neutron scattering experiments at 300 $^\circ$K was compared with the mixed force constant first-principles phonon DOS. In order to study what type of transformation takes place for these metastable phases to form, the temperature dependence of the phonon DOS was analyzed. The predicted $E$ using the rule of mixtures was compared with experimental $E$ from literature.


\pagebreak
\section*{This completed thesis consists of the following main tasks:}

\begin{enumerate}
	\item The thermodynamics of the Ti-Mo-Nb-Sn-Ta-Zr system was investigated using first-principles calculations based on DFT and the CALPHAD method. 
	\begin{enumerate}
		\item Previous binary models were evaluated with the available experimental data as well as calculated enthalpy of formation of the bcc phase 
		\item The thermodynamic description of the Sn-Ta binary alloy was modeled
		\item The thermodynamic descriptions of the Ti-containing ternary alloys were modeled
	\end{enumerate}
	\item The elastic properties of the Ti-Mo-Nb-Sn-Ta-Zr system in the bcc phase were systematically calculated using first-principles based on DFT. The results were used to obtain interaction parameters, following the CALPHAD method, to predict the elastic properties as a function of composition. 
	\item The metastable phase formation in Ti alloys was investigated by first-principles calculations and experiments done on the Ti-Nb and Ti-Ta systems
	\begin{enumerate}
		\item The ground state energies and elastic properties of the Ti-Nb and Ti-Ta systems in the hcp, bcc, $\omega$ and $\alpha"$ were predicted using first-principles calculations
		\item From first-principles calculations:
			\begin{enumerate}
			\item The new theoretic framework was used to predict the phase fractions
			\item The phase fractions were used in a rule of mixtures to plot the phonon density of states and elastic properties
		\end{enumerate}
		\item From neutron scattering:
		\begin{enumerate}
			\item The phase fractions were determined and compared to the calculated predictions
			\item The phonon density of states, at 300 $^\circ$K, was compared to the mixed force constant predicted phonon density of states 
			\item The temperature dependent phonon density of states was used to study the transformation that occurs when these metastable phases form
		\end{enumerate}
	\end{enumerate} 
\end{enumerate}
				
\pagebreak
\begin{table}[H]
	\caption{Young's moduli of common implant materials compared with the Young's modulus of bone \cite{Long1998a}.}
	\centering
	\begin{tabular}{ c c }
		\hline
		Alloy & Young's Modulus (GPa) \\
		\hline
		Bone & 10-40\\
		cp-Ti* & 105\\
		Ti-6Al-4V & 110\\
		Stainless Steel & 200\\
		CoCrMo & 200-230\\
		\hline
		*cp-commercially pure titanium 
	\end{tabular}
\label{table:commonEM}
\end{table}
%%%
\clearpage

\newpage
%%%
\begin{figure}[H]
	\centering
	\includegraphics{Chapter-1/Figures/TiTaElastic.png}
	\caption{Comparison of first-principles calculations \cite{Wu2010a,Ikehata2004} and experimental measurements of the Young's modulus of Ti-Ta alloys \cite{Zhou2004a,Zhou2009a,Fedotov1985}. The purple box refers to the composition range when the calculations and experimentally determined $E$ do not match up due to the formation of two metastale phases $\omega$ and $\alpha"$}
	\label{Ch1-figure:titaelastic}
\end{figure}
%%%

\newpage
%%%
\begin{figure}[H]
	\centering
	\includegraphics{Chapter-1/Figures/TiNbElastic.png}
	\caption{Comparison of the present first-principles calculations and experimental measurements of the Young's modulus of Ti-Nb alloys \cite{Timoshevskii2011,Ozaki2004,Friak2012,Karre2015,Matsumoto2006}. The purple box refers to the composition range when the calculations and experimentally determined $E$ do not match up due to the formation of two metastale phases $\omega$ and $\alpha"$}
	\label{Ch1-figure:tinbelasitc}
\end{figure}
%%%

\chapter{Methodology}

\section{First-Principles Calculations}

In this dissertation, the ground state energy structures, thermodynamic properties and mechanical properties were calculated using first-principles based on Density Functional Theory. The first-principles refers to the calculations originating from "first-principles", meaning that the inputs were the atomic coordinates and atomic numbers. This method computes the interactions between atoms in a periodic supercell. This was determined using quantum mechanical electronic theory that is based on the electronic charge density and does not rely on any empirical data. This section provides a description of the DFT methodology.

Schr$\ddot{o}$dinger's time-independent non-relativistic equation is a solution to the many-body problem of calculating the interactions of positively charged nuclei and negatively charged electrons. The Schr$\ddot{o}$dinger equation is:

%%
\begin{equation}
\label{eq: schrodinger}
\left[ \sum_{i=1}^{N} \left( - \frac{\hbar^2}{2m} \nabla_{i}^2 + V_{ext} (r_{i}) \right) + \sum_{i<j} U (r_{i}, r_{j}) \right] \Psi = E_{s} \Psi
\end{equation}
%%

\noindent where the first bracketed part represents the Hamiltonian ($\hat{H}$), $\Psi$ describes the wave function of electrons, $E_{s}$ describes the systems energy. The $\hat{H}$ of the system is described by three parts, the first part represents the kinetic energy with $N$ being the total number of electrons in the system, $\hbar$ being Planck's constant and $m$ the mass of an electron. The second term $V_{ext}$ gives the external potential and $U$ is the potential of the electron-electron repulsion.

Eq. \ref{eq: schrodinger} can be solved for $\Psi$ with the lowest energy $E_s$ being the ground state energy, assuming the nuclei-nuclei interactions are neglected due to the Born-Oppenheimer approximation. The Born-Oppenheimer approximation allows us to assume the nuclei are stationary and ignore the motion of the nuclei on the electronic timescale, due to the mass difference, with nuclei being $\sim$ 10$^3$ to 10$^5$ larger than electrons. However, even with this approximation solving Eq. \ref{eq: schrodinger} is difficult to deal with due to the electron-electron Columb interactions making the electronic motion correlated and the fact that the many-body problem results in too many variables because there are 3N degrees of freedom. 

Hohenberg-Kohn formulated two theorems to simplify this problem \cite{Hohenberg1964}. The first theorem states that the external potential is a unique functional of the electron density. The second theorem states that the density that minimizes the total energy is the exact ground state density and thus the ground state is obtained variationaly. With these theorems, Kohn-Sham proved that the problem can be solved as if the electrons are not interacting and still obtain the density as if they were, by \cite{Kohn1965}:
 
  \begin{equation}
 \label{eq: kohnsham}
\left[ -\frac{\hbar^2}{2m} \nabla^{2} + V_{ext} (r) + V_{Hartree}(r) + V_{XC} (r) \right] \phi_{1}(r) = \epsilon_{i} \psi_{i}(r)
 \end{equation}
 %%
 
 \noindent where $V_{ext}(r)$ describes the electron-nuclei interaction similar as in Eq. \ref{eq: schrodinger}:

\begin{equation}
\label{eq: vext}
V_{ext} = - e^2 \sum_{a} \frac {Z_{a}}{|r_i - R_a|}
\end{equation}

\noindent where $r_i$ represents the position of electron $i$ and $R_a$ represents the position of nucleus $a$ with a charge valance of $Z_a$. The electron-electron interactions are represented by $V_{Hartree}$:

\begin{equation}
\label{eq: vhartree}
V_{Hartree} (r) = e^{2} \int \frac {\rho(r)}{|r - r_{j}|} d^3r
\end{equation}
%%

\noindent where $r$ and $r_j$ represent the electrons and $\rho (r)$ is described by :

\begin{equation}
\label{eq: rhop}
\rho (r) = \sum_{i}^N | \psi_{i} (r) |^2
\end{equation}
%%

\noindent The final term of $V_{xc}$ is the exchange correlation potential that is described in terms of an exchange-correlation energy. While there is no exact solution to the exchange-correlation (X-C) energy available, there are multiple different approximations. Each approximation is done to account for different things. In the present work, the generalized gradient approximation by Perdew and Wang (PW91) \cite{Perdew1992} and the generalized gradient approximation by Perdew, Burke and Ernzerhoff (PBE) \cite{Perdew1996a} where used. The generalized gradient approach improves the total energies, atomization energies as opposed to other methods such as the local density approximation \cite{Ceperley1980} but can over-correct for the expansion and softening of bonds. The generalized gradient approximation (GGA) is favored for densities that are inhomogeneous. Based on previous research done by Perdew et al. \cite{Perdew1996a}, GGA's are considered to be adequate approximations for calculating metals \cite{Perdew1996a}. The use of PW91 vs PBE was compared when looking at the elastic properties of the Ti-Ta system in Figure \ref{Ch2-figure:PBEvsPW91}. The results showed little difference in calculated elastic properties. The PW91 X-C functional was designed to satisfy as many exact conditions as possible and thus has some issues. Perdew introduced the PBE X-C functional as an improvement to PW91 which satisfied less exact conditions and only looked at the ones that were energetically significant for metals. Due to this and the fact the results vary so little, the PBE X-C functional was chosen to be used for the present thesis work. By implementing the theorems and the Kohn-Sham equation, the energy of the system can thus be calculated.

\subsection{Density Functional Theory at 0 $^\circ$K}

 The ground state energy at 0 $^\circ$K without the contribution of zero-point vibrational energy was calculated by using the equation of states (EOS) fitting for the relationship between the energy and volume of the structure. The EOS fitting was achieved through an energy-volume ($E_{0}-V$) curve of 5 or more relaxed volumes and using the four-parameter Birch-Murnaghan (BM4) EOS \cite{Shang2010}:

%%
\begin{equation}
\label{eq: zeroenergy}
E_{0}(V) = a + bV^{\frac{-2}{3}} + cV^{{-4}{3}} + dV^{-2}
\end{equation}
%%

\noindent where $a$, $b$, $c$ and $d$ are fitting parameters. From this, the volume $V_0$, ground state energy $E_{0}$, bulk modulus $B_{EOS}$, and first derivative with respect to pressure $B'_{EOS}$ can be calculated. 

From the ground state energies, the energy of formation and enthalpy of formation at 0 $^\circ$K was calculated by finding the difference between the structures energy and a reference state, normally the standard element reference (SER) state at standard pressure and temperature. The valance configuration for each element was selected based on the VASP recommendations. The p electrons were treated as valance for the Mo and Ta, the d electrons were treated as valance for Sn and the s electrons were treated as valance for Ti, Nb, and Zr \cite{Kresse1996,Kresse1999}.

\subsection{Finite-temperature thermodynamics}

The Helmholtz energy, $F(V,T)$ was calulated, with DFT, as a function of temperature $T$ and volume $V$:
 %%
 \begin{equation}
 \label{eq: helmholtz}
 F(V,T) = E_{0}(V) + F_{vib}(V,T) + F_{T-el}(V,T)
 \end{equation}
 %%
 
\noindent where $E_0$ is the static contribution at 0 $^\circ$K calculated from Eq. \ref{eq: zeroenergy}, $F_{vib}$ is the temperature-dependent vibrational contribution, and $F_{T-el}$ is the thermal electronic contribution. At ambient pressure, the Helmholtz energy of the system is equal to the Gibbs energy, which is used in the CALPHAD modeling. The vibrational contribution was obtained through the phonon quasiharmonic supercell (phonon approach) or the Debye-Gr\"uneisen method (Debye). The phonon approach is a more accurate approach compared to the Debye model but it is also more computationally expensive. In the present work, both the phonon and Debye models are used in different sections. The vibrational contribution obtained through phonon calculations of at least five different volumes is expressed by \cite{Wang2012}:

%%
\begin{equation}
\label{eq: phonon}
F_{vib}(V,T) = k_{b}T \int_{0}^{\infty} ln \left[ 2sinh \frac{\hbar \varrho}{2k_BT} \right] g(\varrho) d\varrho
\end{equation}
%%

\noindent where $g(\varrho)$ is the phonon density of states as a function of phonon frequency $\varrho$ at volume $V$. $\varrho$ is normally expressed in literature as $\omega$, however, due to the extensive discussion of the $\omega$ phase in this work the phonon frequency is expressed as $\varrho$ to avoid confusion. In addition, the Debye model is used to estimate the vibrational contribution \cite{Shang2010}: 

%%
\begin{equation}
\label{eq: debye}
F_{vib}(V,T) = \frac{9}{8} k_{b} \theta_{D}(V) + k_{B}T \left[ 3 ln \left( 1 - e^{\frac{-\theta_{D}}{T}} \right) - D \left( \frac{\theta_D}{T} \right) \right] 
\end{equation}
%%

\noindent where $\theta_{D}$ is the Debye temperature, $T$ is the temperature, and $D \left[ \frac{\theta_{D}}{T} \right] $ is the Debye function. $\theta_{D}$ is calculated through: 

%%
\begin{equation}
\label{eq: debyetemp}
\theta_{D} = s \frac{(6\pi^2)^{\frac{1}{3}}\hbar}{k_B} V_{0}^{\frac{1}{6}} \left( \frac{B}{M} \right)^{\frac{1}{2}} \left( \frac{V_0}{V} \right)^{\gamma} 
\end{equation}
%%

\noindent where $s$ is the Debye temperature scaling factor, $\gamma$ is the Gr$\ddot{u}$neisen parameter determined by the pressure derivative of bulk modulus ($B'_{EOS}$), $B_{EOS}$ is the bulk modulus, $M$ is the atomic mass, and $V_0$ is the equilibrium volume. Here the equilibrium properties $V_0$, $B_{EOS}$, and $B'_{EOS}$ are estimated from the EOS of Eq. \ref{eq: zeroenergy}. The Debye temperature scaling factor was determined by Moruzzi et al. \cite{Moruzzi1988} to be 0.617 for nonmagnetic metals. However, this value has been shown to be less accurate for other materials. Liu et al. extensively looked at the Debye scaling factor and how to calculate the scaling factor based on the Poisson's ratio of a material \cite{Liu2015}. The methodology by Liu et al. \cite{Liu2015} was used for the present work to calculate the scaling factor: 

%%
\begin{equation}
\label{eq: debyescaling}
s(\nu) = 3^{\frac{5}{6}} \left[ 4\sqrt{2} \left( \frac{1 + \nu}{1 - \nu} \right)^{\frac{3}{2}} + \left( \frac{1 + \nu}{1 - \nu} \right)^{\frac{-1}{3}} \right]
\end{equation}
%%

\noindent where $\nu$ is the Poisson's ratio, which can be calculated from the elastic stiffness constants.

The thermal electronic contribution is based on the electronic density of states and calculated with the Fermi-Dirac statistics \cite{Shang2010,Wang2004}:

%%
\begin{equation}
\label{eq:thermalelectronic}
F_{T-el} = E_{T-el} - T S_{T-el}
\end{equation}
%%

\noindent The $E_{T-el}$ and $S_{T-el}$ represent the energy and entropy of the thermal electron excitations, respectively. The $E_{T-el}$ is expressed by:

%%
\begin{equation}
\label{eq:etel}
E_{T-el} (V,T) = \int n\left(\epsilon, V\right) f \left(\epsilon, T\right) \epsilon d \epsilon - \int^{\epsilon_{f}} n (\epsilon) \epsilon d \epsilon
\end{equation}
%%

\noindent and the entropy $S_{T-el}$ is expressed by:

%%
\begin{equation}
\label{eq:sel}
S_{T-el} (V,T) = -k_{B} \int n(\epsilon, V) \left[ ln f \left(\epsilon,T\right) + \left( 1 - f(\epsilon, T) \right) ln \left( 1 - f \left(\epsilon, T \right) \right) \right] d\epsilon 
\end{equation}
%%

\noindent where $n(\epsilon, V)$ is the electronic density of states (DOS) at energy $\epsilon$,  $f (\epsilon,T)$ is the Fermi-Dirac distribution, $\epsilon_{f}$ is the Fermi energy level and $k_{B}$ is Boltzmann's constant. The Fermi-Dirac distribution $f (\epsilon, T)$ is expressed by:

%%
\begin{equation}
\label{eq:fermidirac}
f (\epsilon,T) = \left[ exp \left( \frac{\epsilon - \mu}{k_{B} T} \right) + 1 \right]^{-1}
\end{equation}
%%

\noindent and $\mu$ is the chemical potential of the electrons. 


\subsection{Elastic stiffness calculations}

The single crystal elastic stiffness constants ($c_{ij}$$'s$) were calculated from the ground state energy structure using a stress-strain method developed by Shang et al. \cite{Shang2007c}. With this method, a set of independent strains $\varepsilon = (\varepsilon_{1}, \varepsilon_{2}, \varepsilon_{3}, \varepsilon_{4}, \varepsilon_{5}, \varepsilon_{6})$ were imposed on the crystal lattice, where $\varepsilon_{1}$, $\varepsilon_{2}$, and $\varepsilon_{3}$ are the normal strains, $\varepsilon_{4}$, $\varepsilon_{5}$, and $\varepsilon_{6}$ the shear strains, and a set of stresses $\sigma = (\sigma_{1}, \sigma_{2}, \sigma_{3}, \sigma_{4}, \sigma_{5},\sigma_{6})$ were generated. Hooke's law is then used to calculate the elastic stiffness constants: 

%%
\begin{equation}
\label{eq: hookes}
\begin{pmatrix}
	c_{11} & c_{12} & c_{13} & 0 & 0 & 0\\
	c_{12} & c_{22} & c_{23} & 0 & 0 & 0\\
	c_{13} & c_{23} & c_{33} & 0 & 0 & 0\\
	0 & 0 & 0 & c_{44} & 0 & 0\\
	0 & 0 & 0 & 0 &  c_{55} & 0\\
	0 & 0 & 0 & 0 & 0 & c_{66} \\    		
\end{pmatrix} =
\begin{pmatrix}
	\varepsilon_{1,1} & & \varepsilon_{1,n}\\
	\varepsilon_{2,1} & & \varepsilon_{2,n}\\
	\varepsilon_{3,1} & ... & \varepsilon_{3,n}\\
	\varepsilon_{4,1} & & \varepsilon_{4,n}\\
	\varepsilon_{5,1} & & \varepsilon_{5,n}\\
	\varepsilon_{6,1} & & \varepsilon_{6,n}\\					
\end{pmatrix}^{-1}
\begin{pmatrix}
	\sigma_{1,1} & & \sigma_{1,n}\\
	\sigma_{2,1} & & \sigma_{2,n}\\
	\sigma_{3,1} & ... & \sigma_{3,n}\\
	\sigma_{4,1} & & \sigma_{4,n}\\
	\sigma_{5,1} & & \sigma_{5,n}\\
	\sigma_{6,1} & & \sigma_{6,n}\\					
\end{pmatrix}
\end{equation}
%%

\noindent where "-1" represents the pseudo-inverse. Due to symmetry, the bcc structure only has three independent elastic stiffness constants. However, with a lack of bcc stability for some of the calculations, all of the elastic stiffness constants were calculated and the average $\overline{C}_{11}$, $\overline{C}_{12}$ and $\overline{C}_{44}$ values were used:

%%
\begin{equation}
\label{eq: averagec11}
\overline{C}_{11} = \frac{(c_{11} + c_{12} + c_{44})}{3}
\end{equation}
%%

%%
\begin{equation}
\label{eq: averagec12}
\overline{C}_{12} = \frac{(c_{12} + c_{13} + c_{23})}{3}
\end{equation}
%%

%%
\begin{equation}
\label{eq: averagec44}
\overline{C}_{44} = \frac{(c_{44} + c_{55} + c_{66})}{3}
\end{equation}
%%

\noindent This case is for the unstable bcc elastic calculations to mimic the behavior of a cubic structure. The largest variance between the similar elastic stiffness constants, when calculating the average, was used to show the deviation from the bcc symmetry in the calculations, shown as error bars. The stable bcc structures show no variance and thus no error bars. To examine the effects of different strain on the elastic properties, three groups of non-zero strain magnitudes, $\pm$0.01, $\pm$0.03, and $\pm$0.07, were tested in chapter 5 and it can be noted that the results have negligible changes with respect to the three groups of strains tested herein. Therefore, $\pm$0.01 was used for all the calculations. The polycrystalline elastic properties including bulk ($B$), shear ($G$), and $E$ modulus were calculated from the elastic stiffness constants, based on the Voigt-Reuss-Hill approach \cite{Simmons1971b}. The Voigt gives the upper elastic bound due to the assumption of constant strain in all grains, the Reuss gives the lower elastic bound due to the assumption of constant stress in all grains, and the Hill approach is the average of Voigt and Reuss and is closer to real case \cite{Zuo1993,Chung1967}. The Hill approach is hence what is listed in the results section. The Voigt and Reuss bounds were also plotted in the figures to give the bounds of the moduli values.

In order to fully investigate the effects of the alloying elements on the Ti-alloys, the mechanical stability of the bcc phase was studied. The mechanical stability was given by on Born's criteria for a cubic crystal  \cite{Born1998,Nye1985}:


\begin{equation}
\label{eq: born1}
\overline{C}_{11} -|\overline{C}_{12}| > 0
\end{equation}
\begin{equation}
\label{eq: born2}
\overline{C}_{11} + 2\overline{C}_{12} > 0
\end{equation}
\begin{equation}
\label{eq: born3}
\overline{C}_{44} > 0
\end{equation}

\noindent Based on Born's criteria, when$\overline{C}_{11} - \overline{C}_{12}$ becomes negative then the phase, bcc in this case, loses mechanical stability and thus $\overline{C}_{11} - \overline{C}_{12}$ is plotted in the results section.

\subsection{Special quasirandom structures (SQS)}

To calculate the energies, enthalpies of formation and elastic properties across the entire binary and ternary composition range, varying compositions of special quasirandom structures (SQS) were used. The SQS are small supercells used to mimic randomly substituted structures in terms of correlation functions. The binary and ternary bcc SQS, used in the present work, were previously generated by Jiang et al.\cite{Jiang2004,Jiang2009}. The relaxation of these structures is complicated because local atomic relaxations can cause the structure to lose the bcc lattice symmetry. To preserve structural symmetry, the calculations were carried out with different relaxation schemes only alloying the cell shape and cell volume to be relaxed to determine the lowest energy structure. In order to ensure the bcc symmetry was preserved the energies were plotted as a function of composition. Then the symmetry was verified. There are two ways to verify whether the SQS is still bcc or not after the relaxation. The first is to merge different elements into one element for the SQS structure, and then, use codes available to check the symmetry or space group (such as VASP and phonopy codes). The second is to visualize the structure directly using a visualization software such as VESTA and compare the symmetry to the unrelaxed bcc structure. For the present work, the relaxed structures were plotted in the visualization software and compared to the unrelaxed structure. In some cases, even without comparing, it was very obvious that the structure had lost bcc symmetry. For the cases where it wasn't obvious the comparison was done using symmetry codes (VASP, phonopy, etc). After the relaxation, at least five different volume structures are generated and the ions were allowed to relax. This yields the different volumes needed for the EOS fitting described above, which allows a better prediction of the different properties as a function of composition. 

\subsection{High-throughput partition function}

As discussed, the potential energy of a structure can be calculated using first-principles calculations based on DFT and the ground state energy will have the lowest potential energy and the metastable or unstable structures will have higher potential energies. In order to predict the formation of the metastable phases the combined potential energy can be calculated REF:

%%
\begin{equation}
\label{eq: combined_energy}
E_{c} = E_{g} + \sum p_{i} \left( E_{i} - E_{g} \right) = \frac{A_{g} + \sum p_{i} \left( A_{i} - A_{g} \right)}{r^{12}} - \frac{B_{g} + \sum p_{i} \left( B_{i} - B_{g} \right)}{r^{6}} = \frac{A_{c}}{r^{12}} - \frac{B_{c}}{r^{6}}
\end{equation}
%%

\subsection{First-principles calculation error}

The error between the previous results (experimental or calculation) and present results was calculated using:

%%
\begin{equation}
\label{eq: error}
\sqrt{\frac{\Sigma[(A_{calc}-A{ref})]^{2}}{\kappa}} = Error
\end{equation}
%%

\noindent where $A_{calc}$ is from the present calculation and $A_{ref}$ is from the previous experiment or calculation, and $\kappa$ is the total number of data points. 

\section{CALPHAD method}

The CALPHAD method evaluates parameters to represent the Gibbs energy of individual phases as a function of temperature, pressure and composition. Thermochemical and phase boundary data obtained from experiments and first-principles calculations were used in the PARROT module of Thermo-Calc to evaluate the thermodynamic interaction parameters \cite{Andersson2002}. The Gibbs free energy is described by enthalpy $H$, temperature $T$ and entropy $S$ as follows:

%%
\begin{equation}
\label{eq: gibbs}
G = H - T S 
\end{equation}
%%

\noindent The Gibbs energy is then parameterized and expressed by:

%%
\begin{equation}
\label{eq: parameterizaiton}
G - H^{SER} = a + bT + cT ln T + d T^2 + \sum_{2}^{n} e_{n} T^{n}
\end{equation}
%%

\noindent where $H^{SER}$ refers to the elemental enthalpy in the SER state and $a$, $b$, $c$, $d$, $e$ are coefficients. Other thermodynamic properties such as, enthalpy, entropy and heat capacity can be derived from this equation. The parameterized equations for the pure elements have been determined and widely adopted from the SGTE to ensure global compatibility between different databases \cite{Dinsdale1991}.

\subsection{Solution phases}

The databases were built upon the pure elements and then the effects of the binary and ternary interactions are modeled. Normally, the effects of alloying are modeled as solution phases or stoichiometric phases. The solution phases with one sublattice were described by: 

%%
\begin{equation}
\label{eq: gibbssolution}
G_m^{\phi} = \sum x_{A} ^{0}G_{A}^{\phi} + R T \sum x_{A} ln x_{A} + ^{XS}G_{m}^{\phi}
\end{equation}
%%

\noindent where $x_{A}$ is the mole fraction of element $A$ and $^{0}G_{A}^{\phi}$ is the molar Gibbs energy of pure element $A$ in the specific phase ($\phi$) being modeled, the second term describes the ideal interaction between elements. The last term represents the excess mixing energy, representing the non-ideal interactions between species $A$ and $B$. The excess mixing energy can be expressed by the Redlich-Kister polynomial as \cite{Redlich1948b}: 

%%
\begin{equation}
\label{eq: gibbexsol}
^{XS}G_m^{\phi} = \sum x_{A} x_{B} \sum_{k=0} ^{k}L_{A,B}^{\phi} (x_{A} - x_{B})^k
\end{equation}
%%

\noindent where $^kL_{A,B}^{\phi}$ represents the interaction parameter for elements $A$ and $B$ in phase $\phi$ described by:

%%
\begin{equation}
\label{eq: binip}
L_{A,B}^{\phi} = ^{k}a + ^{k}bT
\end{equation}
%%

\noindent in which $^{k}a$ and $^{k}b$ are evaluated model parameters. Eq. \ref{eq: gibbexsol} can be extended to multi-component systems as:

%%
\begin{multline}
\label{eq: gibbexsolmulti}
^{XS}G_m^{\phi} = \sum x_{A} x_{B} \sum_{k=0} ^{k}L_{A,B}^{\phi} (x_{A} - x_{B})^k + \sum x_{A} x_{B} x_{C} \\ \left[ ^{0}L_{A, B, C}^{\phi} (x_{A} + \delta_{A, B, C} + ^{1}L_{A, B, C}^{\phi} (x_{B} + \delta_{A, B, C} + ^{2}L_{A, B, C}^{\phi} (x_{c} + \delta_{A, B, C} ) \right]
\end{multline}
%%

\noindent where the ternary interaction parameters $L_{A, B, C}^{\phi}$ are described the same as the binary interaction parameters in Eq. \ref{eq: binip} and $\delta_{A, B, C}$ is defined as $\delta_{A, B, C} = ( 1 - x_{A} - x_{B} - x_{C})/3$

In the present thesis, while no order-disorder modeling was done, the order-disorder model was used in many of the previous models evaluated. At low temperatures in many of the Ti-containing binary alloys, the randomly substituted bcc-A2 phase goes through a second order transition to become the ordered simple cubic B2 phase (CsCl-type). This modeling was discussed extensively in the COST 507 by Ansara \cite{Ansara1998}. 

\subsection{Stoichiometric compounds}

The Gibbs energy of stoichiometric compounds was modeled in per mole unit formula. For the stoichiometric compound, $A_{p}B_{q}$, the Gibbs energy is expressed by \cite{Zacherl2012}: 

%%
\begin{equation}
\label{eq: stoichiometric}
^{0}G_{m}^{A_{p}B_{q}} = a + bT + p * ^{0}G_{A}^{SER} + q * ^{0}G_{B}^{SER}
\end{equation}
%%

\noindent where $a$ and $b$ are the evaluated parameters, $^{0}G_{A}^{SER}$ is the Gibbs energy of pure element $A$ in the SER phase, $^{0}G_{B}^{SER}$ is the Gibbs energy of pure element $B$ in the SER phase, $p$ is the number of atoms per unit formula of element $A$ and $q$ is the number of atoms per unit formula of element $B$.

\subsection{Elastic Properties}

To obtain the elastic properties as a function of composition, the CALPHAD modeling approach was adopted and the Redlich-Kister polynomial was used to describe the elastic stiffness constants by \cite{Redlich1948b,Shang2010c}: 

%%
\begin{equation}
\label{eq: elastic}
E(x) = \sum x_{A} E_{A}^{\phi} + \sum x_{A} x_{B} ^{k}L_{A,B}^{\phi} + \sum x_{A} x_{B} x_{C} ^{0}L_{A, B, C}^{\phi}
\end{equation}
%%

\noindent where similar to Eq. \ref{eq: gibbexsol} and \ref{eq: gibbexsolmulti} the $x_A$, $x_B$, and $x_C$ refer to the mole fraction of element $A$, $B$, and $C$ respectively, $E_{A}$ is the elastic property of element $A$, $^{k}L_{A,B}^{\phi}$ and $^{k}L_{A,B,C}^{\phi}$ are the binary and ternary interaction parameters, respectively. The binary and ternary interaction parameters were described in terms of Eq. \ref{eq: binip} but with solely an $a$ coefficient. For the binary modeling, the addition of one and two interaction parameters were studied to ensure the best fit. For the ternary systems, only one interaction parameter was introduced. This is due to the fact that only one ternary interaction parameter is needed to describe the interaction between the ternary systems on the Ti-rich side. The fittings of the binary and ternary interaction parameters were completed using the Mathematica code in Appendix C. The binary interaction parameters were fit using the difference between the pure elements extrapolation and the first-principles results for each Ti-X (X= Mo, Nb, Sn, Ta, Zr). The ternary interaction were fit using the difference between the interpolaiton calculated from the pure elements and binary interaction parameters and the results. The interaction parameters determined to describe the elastic stiffness constants were incorporated into a database. Pycalphd \cite{Otis2017} was then used to calculate the moduli values as a function of compostion based on the Voigt-Reuss-Hill approach.


\section{Experimental}

To study the effect of the metastable phase formation, Ti-Nb alloys at different compositions were made by arc melting. Two samples were made at each composition. One set of samples was quenched to form the $\alpha"$ and the second set of samples was slow cooled to form the $\omega$ phase. The samples were then studied using inelastic neutron scattering. 

\subsection{Ti-Nb sample preparation}

The first set of samples, which hereafter be referred to as set 1, included four Ti-Nb samples made from pieces of titanium (Alfa Aesar, Stock No. 14004) and niobium (Alfa Aesar, Stock No. 00241) at 0.1, 0.12, 0.18, 0.2 mole fraction of Nb. The second set of samples, which hereafter will be reffered to as set 2, included four Ti-Nb samples made from pieces of titanium (Sigma Aldrich, Stock No. 305812) and niobium (Alfa Aesar, Stock No. 00241) at 0.1, 0.12, 0.18, 0.2 mole fraction of Nb. Two different titanium pieces were used because the Alfa Aesar titnaium pieces were out of stock but they both had the same purity and thus should not lead to any issues with the data analysis. The alloyed samples were made using the MAM1, Edmund Buhler GmbH arc melter under argon atmosphere. The alloys were machined into a cylindrical shape (0.7 inches in diameter and 0.7 inches in thickenss). The samples were then heat treated using a Lindberg 59544 tube furance. The tube was made of Al$_{2}$O$_{3}$ and was under vaccuum. The samples were annealed at 1273 $^\circ$K for 24 hours. The samples from set 1 were quenched in a water solution. The samples in set 2 were slow cooled. 

\subsection{Neutron Scattering}

\subsubsection{ARCS}

The inelastic neutron scattering measurements were carried out using the Wide Angular-Range Chopper Spectrometer (ARCS) at the Spallation Neutron Source (SNS) at Oak Ridge National Laboratory. ARCS is a time-of-flight spectrometer so the neutrons original position and energy are fixed and neutrons final position and the time elapse are recorded. From this information, the momentum and energy of the neutrons was calculated and plotted. The samples were loaded into a customized vanadium sample holder. Vanadium was chosen due to the high temperatures measured. The empty vanadium sample holder was measured at the same conditions at each temperature. The holder was mounted into the furnace and kept under vacuum throughout all measurements. Two incident neutron energies, $E_{i} = 25 meV$ and $E_{i} = 50 meV$, were used at each temperature (300, 500, 700, 900, 1110 K). The data was corrected for the empty can scattering as well as the ARCS background. The diffraction patterns and phonon density of states were then obtained.

\subsubsection{Data Analysis}

After the empty sample holder and background were subtracted from the data, the diffraction patterns and phonon density of states were investigated. Diffraction is prominently an elastic scattering process. The intensity of the diffraction patterns obtained were compared with intensities of Ti and Nb single-phase diffraction patterns in the $\alpha"$, $\omega$, bcc and hcp from literature. The single-phase diffraction patterns from literature are not always at the right compositions, but by comparing the intensities the phase fractions can be predicted fairly accurately. First the distance between scattering planes is taken into account for each phase. The intensity at each momentum was calculated and modeled as a Gaussian function for each phase and then the data was fitted with the literature data \cite{Young1998,Toraya1986}. 

The phonon DOS contains the one-phonon and multi-phonon contributions; however, the one-phonon density of states is what is needed. An iterative process was done starting with a trial phonon DOS to separate the multi-phonon and one phonon contributions. This is accomplished by summing over the entire detector angle range (2 $\theta$) and calculating the total incoherent dynamic structure factor expressed by \cite{Manley2001}:

%%
\begin{equation}
\label{eq: td_dynstrufac}
\overline{S}_{calc}^{inc} (\varrho) = \sum_{\theta} \frac{1}{2 \pi \hbar} e^{- Q^{2} (\theta, \varrho) \left<u^{2} \right>} \int_{-\infty}^{\infty} dte^{-i \varrho t} e^{\hbar^{2}Q^{2}(\theta, \varrho) G(t)/2M} \left[ e^{\frac{-t^{2}}{2} \left(\frac{\bigtriangleup E (\varrho)}{2\hbar} \right)^{2}} \right]
\end{equation}
%%

\noindent where $\left< u^{2} \right>$  is the Debye-Waller factor describing the mean square atomic displacement. The anisotropy in the Debye-Waller factor was neglected because the resulting errors were negligible \cite{Manley2001}. The $\bigtriangleup E$ is the Gaussian instrument energy resolution of variable width. $M$ is the mass of a neutron, $G(t)$ is the time dependent self-correlation function which is expressed by \cite{Manley2001}:

%%
\begin{equation}
\label{eq: td_selfcorrelation}
G (t) = \int_{- \infty}^{\infty} d \varrho \frac{Z(\varrho)}{\varrho} n(\varrho) e^{- i \varrho t}
\end{equation}
%%

\noindent where $g(\varrho)$ is the phonon density of states as a function of phonon frequency $\varrho$ and $n(\varrho)$ is the thermal occupancy factor. Finally, $Q$ is defined by \cite{Manley2001}:

%%
\begin{equation}
\label{eq: Q}
Q(\theta, \varrho) = \sqrt{\frac{2M}{\hbar^{2}} \left( 2E - \hbar \varrho - 2E \sqrt{1-\frac{\hbar \varrho}{E}}cos(2\theta) \right)}
\end{equation}
%%

\noindent The calculated $\overline{S}_{0,calc}^{inc}$ and the elastic scattering was then subtracted from the total scattering which leaves us with the angle-averaged multiphonon structure factor which is a good starting approximation for the multiphonon coherent scattering \cite{Manley2001}. 

The elastic peak was then subtracted from the measured total dynamical structure factor.  Then the total dynamical structure factor was averaged over $2\theta$ and scaled so the multiphonon coherent scattering could be subtracted. After the subtractions, the new phonon DOS was determined and was then used to recalculate the multiphonon contribution and the procedure was repeated. The procedure was repeated three times to converge the phonon DOS based on previous recommendations that showed three iterations were enough to converge within statistical errors \cite{Manley2001}.

\pagebreak
\begin{figure}[H]
	\centering
	\includegraphics[width=\textwidth]{Chapter-2/Figures/PBEvsPW91.png}
	\caption{Elastic stiffness constants of the bcc Ti-Ta binary system calculated with the GGA and PBE exchange correction functions, respectively.}
	\label{Ch2-figure:PBEvsPW91}
\end{figure}

\chapter{Ti-Mo-Nb-Sn-Ta-Zr Thermodynamic Database}

\section{Introduction}

The design of Ti-alloys for biomedical applications necessitates a completed thermodynamic database that will facilitate the prediction of phase compositions and fractions as a function of composition and temperature. However, there is no completed thermodynamic database for the Ti-Mo-Nb-Sn-Ta-Zr system and thus the present work aims at building a complete database with special focus on the Ti-rich alloys and bcc phase models. With this in mind the pure elements have been extensively studied and are widely adopted from the SGTE database \cite{Dinsdale1991}. The modeling of the binary systems has been widely documented with the exception of the Ta-Sn and Mo-Sn systems, while experimental phase boundary data is available for the ternary systems but little to no modeling has been completed. The Mo-Sn and Sn-Ta subsystems have high melting temperatures and little to no experimental data. In these cases, first-principles calculations based on DFT can be used to aid in modeling and supplement the lack of experimental data. The complete modeling of the Ta-Sn system is discussed in Ch3. In the present chapter, the thermodynamic descriptions of the Ti-Mo-Nb-Ta-Zr system are described. 

While many of the alloys in this Ti system have been studied experimentally, yielding phase equilibrium data, only limited calorimetry data is available. With the present work focuses on bcc Ti-rich alloys, first-principles calculations based on DFT of the enthalpy of formation of the bcc phase were calculated. The thermodynamic descriptions were built or evaluated using available experimental phase boundary data and present calculated thermochemical data. This works looks at evaluating new and previous models for the binary and Ti-containing ternary systems. 

\section{Computational details}

First-principles results based on Density Functional Theory (DFT) are used to predict the enthalpy of formation of specific phases. In the present work, the enthalpy of formation of the bcc phase was calculated for the Ti-X and Ti-X-Y (X$\neq$Y= Mo, Nb, Ta, Zr) using the calculated energy of the pure elements in their SER states. For each Ti-containing binary system at least 5 composition energy structures were calculated and three compositions for the Ti-containing ternary system. For each system, three special quasirandom structures (SQS) of varying compositions were relaxed to get an accurate representation of the enthalpy of formation across the composition range. For the binary phases the compositions are each 16-atom unit cells at Ti$_4$X$_{12}$, Ti$_8$X$_8$, and Ti$_{12}$X$_4$ and the ternary phase compositions of Ti$_{12}$X$_{12}$Y$_{12}$ (36-atom), Ti$_{16}$X$_8$Y$_8$ (32-atom), Ti$_{48}$X$_8$Y$_8$ (64-atom) structures are used with one X$_8$Y$_8$ (16-atom), where X and Y are the alloying elements. These SQS in the bcc phase that were implemented were developed by Jiang et al. \cite{Jiang2004,Jiang2009}. Relaxation of these structures is complicated and explained in the methodology section. The DFT calculations are completed using VASP (Vienna ab-initio Simulation Package) \cite{Kresse1996}. The ion-electron interactions were described using the projector augmented wave (PAW) \cite{Kresse1999,Blochl1994} method with the exchange-correlation (X-C) functional implemented by Perdew, Burke, and Ernzerhof (PBE) \cite{Perdew1996a}. For consistency, a 310 eV energy cutoff was adopted for all calculations, which is roughly 1.3 times higher than the default value. The energy convergence criterion was 10-6 eV/atom, and the Monkhorst-Pack scheme is used for Brillouin zone sampling \cite{Kresse1996,Monkhorst1976a}. 

\section{Binary systems}

In order to build the database, as discussed above, the pure elements were adapted from the SGTE database \cite{Dinsdale1991}. After the incorporation of the pure elements, all the binary systems were then evaluated. When applicable, previous models of the binary systems were evaluated for accuracy and model compatibility and incorporated into the database. After incorporation of all binary systems, the Ti-containing ternary systems were evaluated, with the exception of Mo-Nb-Ta which had a previous model available by Xiong et al. \cite{Xiong2004} and was incorporated into this work. To do the binary and ternary evaluations, the enthalpy of formation of the bcc phase for the Ti-containing alloys was calculated across the composition ranges. To calculate the enthalpy of formation of the bcc phase, the enthalpy of the pure elements had to be calculated in their standard element reference states (SER). The phases and results are listed in Table \ref{Ch3-table:pspureele}. In order to ensure the accuracy of the calculations, the bulk modulus \textit{B} calculated in the present work was compared with the experimental data. The experimental \textit{B} varied on average by 1 GPa from the calculated \textit{B}. This discrepancy is not large and is attributed with the temperature difference thus these calculations prove to be accurate. 

The previous thermodynamic descriptions for the Ti-containing binaries were evaluated with experimental data as well as looking at the enthalpy of formation of the bcc phase using present first-principles calculations. The enthalpy of formation of the bcc phase calculated for each Ti-containing binary from first-principles is listed in Table \ref{Ch3-table:hof}. 



\subsection{Ti-Mo}

The Ti-Mo binary evaluation was taken from the COST 507 database and the thermodynamic description was evaluated by Saunders \cite{Ansara1998}. This model was chosen because it is the modeled incorporated into the Ti-Mo-Zr modeling done by Kar et al. \cite{Kar2008}. Interaction parameters were evaluated for the liquid, fcc (face centered cubic), hcp, bcc$\#$1, bcc$\#$2 ordered and disordered phases, AlM-D019, AlM-D022, and the AlTi-L10 phases. Figure \ref{Ch3-figure:TiMo}a shows the comparison of the available phase boundary data from Murray et al. \cite{Murray1981} with the Saunders model. The phase boundary data is reproduced in the with the Saunders model. Figure \ref{Ch3-figure:TiMo}b shows the enthalpy of formation of the bcc phase predicted from the model at 300 K compared with the present first-principles results at 0 $^\circ$K. The enthalpy of formation of the bcc phase predicted varies from the first-principles calculations drastically between 20 and 80 at $\%$ Mo. This discrepancy is seen due to the disagreement on the existence of a bcc miscibility gap. Previous research has shown the need for a bcc miscibility gap which would fit what is seen in the first-principles calculations \cite{Predel1997,Hoffman1967}. Kar et al. showed that the experimental data from higher-component systems fit better with the description containing no miscibility gap. Based on this, the model by Saunders was adopted for the database with no changes. 

\subsection{Ti-Nb}

The Ti-Nb binary evaluation was taken from Zhang et al. \cite{Zhang2001} and plotted in Figure \ref{Ch3-figure:TiNb}a. Originally the binary evaluation was taken from Kumar et al. \cite{Kumar1994}. The figure shows solidus data ($\square$) and hcp and bcc solvus data ($\circ$) used in the Kumar et al. evaluation. The model from Kumar was used in the modeling of the Ti-Nb-Zr. The phase diagram was evaluated using phase boundary data on the Ti-rich side and liquidus data. Not only was the binary evaluation adapted due to the use in the Ti-Nb-Zr ternary but it also reproduced the experimental phase boundary data reasonably well. However, the model was changed to the new model by Zhang et al. ($\triangle$) due to the new phase boundary data on the Nb rich side. The present first-principles calculations of the enthalpy of formation of the bcc phase are plotted in Figure \ref{Ch3-figure:TiNb}b. The present DFT results (circles) are at 0 $^\circ$K while the CALPHAD prediction (solid line) is at 300 $^\circ$K which explains the average variance of 0.17 kJ/mol-atom between the DFT and CALPHAD predictions. However, even with the variance the CALPHAD prediction compares well with the DFT results and the sublattice models are similar to the database being built. With the reproduction of experimental data and the first-principles results no alterations were made to the thermodynamic evaluation.

\subsection{Ti-Ta}

The Ti-Ta binary was taken from the COST 507 database and evaluated by Saunders \cite{Ansara1998}. The phase diagram is shown in Figure \ref{Ch3-figure:TiTa}a. The figure is plotted with liquidus and solidus experimental data ($\lozenge$, Y) as well as bcc and hcp solvus data ($\triangle$, $\square$, $\circ$). The evaluation includes interaction parameters for the fcc, hcp, liquid, AlM-D019, AlM-D022, AlTi-L10 and the bcc$\#$1 and bcc$\#$2 ordered and disordered phases. The thermodynamic description reproduces the phase boundary experimental data from Murray et al. \cite{Murray1987}. The first-principles results (circles) and the enthalpy of formation of the bcc phase predicted by the CALPHAD modeling is shown in Figure \ref{Ch3-figure:TiTa}b. The CALPHAD prediction of the enthalpy of formation reproduces the results from first-principles reasonably well on the Ti-rich and Ta-rich sides. The first-principles results between 10 and 60 at $\%$ Ta vary on an average by 0.17 kJ/mol-atom. However, the CALPHAD prediction is at 300 $^\circ$K and the first-principles are at 0 $^\circ$K which explains the variance and the CALPHAD prediction follows the same trend that is seen in the first-principle results. The modeling and sublattices used were compatible with the database being built and thus the thermodynamic description by Saunders was not altered. 

\subsection{Ti-Zr}

The thermodynamic description of the Ti-Zr alloy system evaluated by Kumar et al. \cite{Kumar1994a} is used in the present work. The model by Kumar et al. was used in the ternary modeling of the Ti-Mo-Zr and Ti-Nb-Zr ternary alloys. The sublattice modeling was consistent with the previous binary systems. The evaluation introduced interaction parameters for the liquid, bcc, and hcp solution phases. Figure \ref{Ch3-figure:TiZr}a plots the evaluation compared with phase boundary data for the bcc to hcp phase ($\circ$) transformation and solidus ($\lozenge$). The thermodynamic description accurately reproduces the phase boundary data. The evaluation also included heat of transformation data shown in the paper by Kumar et al. \cite{Kumar1994a}. Figure \ref{Ch3-figure:TiZr}b plots the first-principles results (circles) versus the CALPHAD prediction (solid line) for the enthalpy of formation of the bcc phase. The DFT results and CALPHAD modeling vary on average by 1.2 kJ/mol-atom. The variance is larger than the other binary alloys due to the instability of the bcc phase at both 0 $^\circ$K and 300 $^\circ$K for the Ti-Zr alloy but the calculations and CALPHAD prediction follow the same trend. Based on the agreement between the experimental data, first-principles and CALPHAD prediction no alterations were made to the thermodynamic description. 

\subsection{Non Ti-containing binaries}

Figure \ref{Ch3-figure:binary1} shows the evaluation of the Mo-Nb, Mo-Ta, and Nb-Ta alloys. The Mo-Nb, Mo-Ta and Nb-Ta binary descriptions were adapted from the Mo-Nb-Ta ternary paper by Xiong et al. \cite{Xiong2004}. Xiong et al. introduced binary interaction parameters for the liquid and bcc solution phases. The Mo-Nb evaluation was completed using differential thermal analysis experiments that measured both the liquidus and solidus temperature. Multiple authors measured experiments and Xiong et al. picked the experiments (shown in the figure as X, +) that estimated the pure elements reasonably well to build the evaluation on. Xiong et al. saw that the experiments shown by $\circ$, $\triangle$, $\square$ agreed well with the thermodynamic description. The remaining work ($\lozenge$, *) was too low and thus wasn’t included in the evaluation of the binary alloy. The Mo-Ta alloy evaluation was done using two sets of experimental data ($\triangle$,* , $\lozenge$, $\square$) for the evaluation that agree well with the accepted melting temperatures of Mo and Ta. The remaining experimental data ($\circ$) was higher than the other experimental work by 70 $^\circ$C and thus was not used for the evaluation. The Nb-Ta binary evaluation was done using the melting temperature experimental data ($\circ$, $\triangle$, $\square$) because it predicted the melting temperatures of Nb and Ta accurately. The remaining experimental work which measured the solidus temperature (*) was not used because the melting temperatures of Nb and Ta showed discrepancy. The remaining experimental work ($\lozenge$) was not included in the evaluation because similarly to the Mo-Ta alloy the experimental work was 70 $^\circ$C higher than the other work. These three evaluations were chosen because these binaries were evaluated to make up the Mo-Nb-Ta ternary alloy and the modeling was done consistently using the same sublattices models. The evaluations fit well with available experimental data and thus they can be added to the database without any changes.

Figure \ref{Ch3-figure:binary2} shows the Mo-Zr, Nb-Zr, and Ta-Zr alloys. For the Mo-Zr binary alloy system, there were multiple previous thermodynamic descriptions and experimental results available. In the present work, the evaluation by Perez et al. \cite{Perez2003} was chosen. The experimental data plotted was determined for the single-phase region, two-phase region, phase boundaries and peritectic and eutectiod reactions. Perez et al. goes into more detail on the available experimental data and was included in the evaluation of the thermodynamic description. The thermodynamic description generally reproduces all experimental data and is compatible with the sublattice modeling used in the database. The Perez et al. model was also incorporated into the Ti-Mo-Zr ternary modeling. Perez et al. introduced interaction parameters for the liquid, bcc, hcp and MZr$_2$ (laves$\_$c15) phases. The Nb-Zr alloy thermodynamic description evaluated by Guillermet \cite{Guillermet1991} was chosen for the present work. The figure shows the solidus experimental data ($\lozenge$, *) as well as the hcp solvus (Y, $\circ$)and bcc miscibility gap ($\square$, +, $\triangle$)data. The description includes interaction parameters for the liquid, bcc and hcp solution phases. The thermodynamic description was chosen because it was the description that Kumar et al. \cite{Kumar1994a} included in his modeling of the Ti-Nb-Zr alloy system. The experimental phase boundary and liquidus data from Abriata et al. \cite{Abriata1982} fits well with the model. The Ta-Zr binary evaluated by Guillermet \cite{Guillermet1995} was chosen for the present database. As Guillermet discusses there was quite a lot of experimental data, shown in the figure. Guillermet discusses the experimental data more in his paper. The figure plots single-phase, two-phase, phase boundary, and solidus experimental data. Interaction parameters were introduced for the bcc, hcp and liquid phases. Guillermet had phase boundary results from at least 5 different authors and thermodynamic results from three different authors. The evaluation reproduced the data fairly well. The sublattice modeling fit with the database and thus this binary was selected.The interaction parameters for all the binary systems are listed in Table \ref{table:ip}.

\section{Ti-containing ternary sections}

\subsection{Ti-Mo-Nb}

A thermodynamic description for the Ti-Mo-Nb ternary system has never been evaluated. Two experimental investigations were done on the Ti-Mo-Nb system at 873 $^\circ$K and 1373 $^\circ$K \cite{English1961,Prokoshkin1967}. While both investigations agree that the isothermal section at 1373 $^\circ$K is solely the bcc phase, the investigations differed on the phase boundary data at 873 $^\circ$K. It is suspected that at such a low temperature the samples did not reach equilibrium which accounts for the discrepancy. Based on this, the binary interpolation of the isothermal sections at 1373 $^\circ$K and 873 $^\circ$K were plotted. The phase diagram at 1373 $^\circ$K agreed with the experimentally determined phase diagram to be solely the bcc phase. The phase diagram at 873 $^\circ$K is plotted in Figure \ref{Ch3-figure:TiMoNb}a. The discrepancy at 873 $^\circ$K, is the existence of the bcc miscibility gaps as well as what compositions the phase boundary lines lie at. First-principles calculations of the enthalpy of formation of the bcc phase are plotted against the binary interpolation in Figure \ref{Ch3-figure:TiMoNb}b. While the first-principles calculations are at 0 $^\circ$K and the binary interpolation is at 300 $^\circ$K the calculation results are reproduced with the CALPHAD prediction. The prediction varies by less than 1.5 kJ/mol-atom for all the calculations except at Mo$_{0.5}$Nb$_{0.5}$. While the calculation varies substantially from the prediction at Mo$_{0.5}$Nb$_{0.5}$, in order to improve this, the Mo-Nb binary system would have to be adjusted. In the present work, no thermochemical data was used to ensure the accuracy of the non Ti-containing binary systems but the previous binary models were able to reproduce the phase boundary data as discussed above. Based on the discrepancy between the experimental data and the fact that the thermochemical first-principles calculations are reproduced well by the binary interpolation, no ternary interaction parameters were evaluated. 

\subsection{Ti-Mo-Ta}

The thermodynamic description of the Ti-Mo-Ta alloys system had not been previously modeled. The binary interpolation of the Ti-Mo-Ta alloy is plotted in Figure \ref{Ch3-figure:TiMoTa1}a at 873 $^\circ$K and compared with experimental data from Nikitin et al. \cite{Nikitin1971}. At 873 $^\circ$K, the Ti-Mo-Ta alloy has the bcc and hcp solution phases. The experimental data showed a two-phase bcc-hcp region in the Ti-rich corner, while the binary interpolation shows a bcc miscibility gap which forms a tie triangle with the hcp phase. Figure \ref{Ch3-figure:TiMoTa1}b shows the first-principles calculations (circles) of the enthalpy of formation of the bcc phase compared with the binary interpolation from the CALPHAD prediction. The first-principles calculations line up fairly well with the CALPHAD prediction. However, due to the discrepancy of the experimental data, ternary interaction parameters were evaluated using the experimental and first-principles results. Interaction parameters for the hcp and bcc phases were investigated. The evaluated interaction parameters are listed in Table \ref{Ch3-table:ip}. After assessing the ternary interaction parameters, the isothermal section was again plotted and compared with experimental data in Figure \ref{Ch3-figure:TiMoTa2}a and the enthalpy of formation of the newly assessed bcc phase is plotted as a dashed line in Figure \ref{Ch3-figure:TiMoTa1}b. The assessment reproduces the first-principles results. With the introduction of the interaction parameters the isothermal section fits with the experimental data. Figure \ref{Ch3-figure:TiMoTa2}b is zoomed in on the Ti-rich corner. The work by Nikitin determined hcp phase boundary data plotted as ($\circ$) and two phase experimental data as $\LEFTcircle$. The two phase experimental data is reproduced by the current model. The hcp phase boundary data is not reproduced. However, reliable solid phase boundary data is difficult to obtain at such a low temperature and if the evaluation is altered to fit the data it then over fits and stabilizes non-equilibrium phases. 

\subsection{Ti-Mo-Zr}

The thermodynamic description of the Ti-Mo-Zr alloy system was previously modeled by Kar et al. \cite{Kar2008}. The same binary phases used in the modeling by Kar et al. were included in the database. The phases in this system are liquid, bcc, hcp and laves$\_$c15. After interpolating the ternary system from the binary models and comparing to two sets of available experimental data, Kar et al. introduced interaction parameters for the laves$\_$c15 phase. Using the model by Kar et al., the present work plotted the ternary isothermal section at 1273 $^\circ$K comparing to phase boundary data plotted in Figure \ref{Ch3-figure:TiMoZr}a. As discussed by Kar et al. there is phase boundary data from two authors. The sets of phase boundary data conflict on how far out the two-phase region should extend toward the Ti-rich corner and whether there is a bcc miscibility gap. After plotting the data, Kar et al. decided not to introduce any bcc, liquid or hcp interaction parameters. Only one set of phase boundary data is plotted \cite{Prokoshkin1967}. The phase boundary data fits well on the Zr-Mo binary side. The predicted enthalpy of formation of the bcc phase is plotted with the first-principles results in Figure \ref{Ch3-figure:TiMoZr}b. The first-principles results vary on average by 0.025 kJ/mol-atom. Based on the available experimental data, first-principles calculations and the conclusions from Kar et al. the present work agrees with the introduction of the ternary laves$\_$c15 interaction parameters and lack of liquid, bcc and hcp ternary interaction parameters. The ternary laves$\_$c15 interaction parameters are listed in Table \ref{Ch3-table:ip}.

\subsection{Ti-Nb-Ta}

A thermodynamic description of the Ti-Nb-Ta had not been evaluated but different isothermal sections had been estimated by Na et al. \cite{Na2001} using phase boundary data. The phase boundary data was obtained through XRD. Na et al. looked at samples at 823 $^\circ$K and 673 $^\circ$K. The authors discussed that it is likely that the alloys at 673 $^\circ$K never reached equilibrium conditions. The experimental results were plotted on the binary interpolation in Figure \ref{Ch3-figure:TiNbTa1}a and Figure \ref{Ch3-figure:TiNbTa1}b. The bcc phase boundary data does not match with the binary interpolation. Figure \ref{Ch3-figure:TiNbTa1}c plots the enthalpy of formation of the bcc phase predicted by the CALPHAD modeling and compared with the first-principles results. The first-principles results vary from the CALPHAD prediction. The variance of the first-principles results and the variance of the experimental data lead to the evaluation of ternary interaction parameters for the bcc and hcp phases. The evaluation was done using the 823 $^\circ$K experimental data and first-principles calculations. Due to the conclusion by Na et al. that the 673 $^\circ$K samples did not reach equilibrium the data was neglected during the evaluation. The evaluation led to only one bcc ternary interaction parameter needed and it is listed in Table \ref{Ch3-table:ip}. After evaluation, the ternary isothermal sections are plotted with the phase boundary data in Figure \ref{Ch3-figure:TiNbTa2}a and Figure \ref{Ch3-figure:TiNbTa2}b. The isothermal sections at both 673 and 823 $^\circ$K reproduces the experimental data reasonably well. The prediction of the enthalpy of formation of the bcc phase also improved to accurately predict the first-principles results.

\subsection{Ti-Nb-Zr}

The Ti-Nb-Zr was previously evaluated by multiple authors. In the present work the ternary isothermal sections were compared with work by Kumar et al. \cite{Kumar1994a} and Tokunaga et al. \cite{Tokunaga2007}. The Ti-Zr and Nb-Zr binary alloys were the same modeled binaries as chosen by Kumar et al. The Ti-Nb binary however was changed to a newer evaluation by Zhang et al. \cite{Zhang2001} which was discussed above. In the evaluation done by Kumar et al. \cite{Kumar1994a}, the liquidus projection and various isothermal sections were interpolated from the binary alloys but not compared with experimental data. In the present work, the binary interpolation was used to plot the isothermal sections at the same temperature as Kumar et al. The ternary isothermal sections varied due to the change in the Ti-Nb binary. The binary interpolation of the isothermal section at 843 $^\circ$K is plotted in Figure \ref{Ch3-figure:TiNbZr}a and compared with experimental data for the tie triangle ($\triangle$) and two-phase region (\Leftcircle) shown in the paper by Tokunaga et al. \cite{Tokunaga2007}. The binary interpolation reproduced the hcp and bcc disordered and ordered tie triangle phase boundary data and the two-phase bcc miscibility gap region. The enthalpy of formation of the bcc phase is plotted in Figure \ref{Ch3-figure:TiTa}b. The CALPHAD prediction varies on an average by 1.34 kJ/mol-atom. While it the CALPHAD prediction varies more drastically than the other enthalpies of formation from the calculated first-principles results, it follows the same trend and fits well with the experimental data. So, the conclusion was reached to not introduce ternary interaction parameters.  

\subsection{Ti-Ta-Zr}

For the Ti-Ta-Zr alloys system, Lin et al. \cite{Lin1996} calculated the isothermal sections using binary interpolations and introduced no interaction parameters. The isothermal sections at 1273 and 1773 $^\circ$K are plotted in Figure \ref{Ch3-figure:TiTaZr}a and Figure \ref{Ch3-figure:TiTaZr}b. Experimental phase boundary data ($\circ$) along the bcc miscibility gap at 1273 $^\circ$K and single phase $\CIRCLE$ and two phase region $\LEFTcircle$ data using x-ray diffraction at 1773 $^\circ$K are plotted to compare with the binary interpolation \cite{Lin1996,Hoch1964}. The phase boundary data is reproduced excellently. The single-phase data fits well with the interpolation with the two-phase region falling on the phase boundary line. Figure \ref{Ch3-figure:TiTaZr}c is the enthalpy of formation of the bcc predicted from first-principles calculations and the binary interpolation. On average the first-principles varies by 3.69 kJ/mol-atom. While this variance is larger than other plots it can be attributed to the temperature difference and instability of the bcc phase at 300 $^\circ$K and 0 $^\circ$K. Since the experimental data is reproduced, it was concluded that no overfitting to the first-principles calculations was needed and thus none were evaluated. 
After evaluation, the thermodynamic descriptions of all the binary systems and Ti-containing ternary systems are listed in Table \ref{Ch3-table:ip} and combined into a single tdb database in the appendix.

\section{Conclusion}

The present work builds a thermodynamic database for the Ti-Mo-Nb-Ta-Zr system using descriptions of the pure elements, binary systems and Ti-containing ternary systems. The thermodynamic descriptions of the pure elements was adapted from the SGTE database \cite{Dinsdale1991}. The previous models for the binary systems were evaluated for compatibility and accuracy. With the present application focusing on Ti-alloys, first-principles results for the enthalpy of formation of the bcc phase was used to ensure the accuracy of the thermodynamic descriptions in predicting thermochemical data. Binary interpolations were predicted for the Ti-containing ternary systems and compared with isothermal experimental data as well as first-principles of enthalpy of formation of the bcc phase. When needed the ternary interaction, parameters were evaluated. The compatible thermodynamic descriptions were complied into a single tdb database. The raw first-principles data as well as the tdb database are in appendix A. 


\newpage
\begin{table}[H]
	\caption{Bulk modulus \textit{B} and enthalpy are listed for each pure element in their SER phase listed. The sv and pv refer to the electrons chosen as valance according to the VASP recommendations. The results are compared with available experimental data.}
	\centering
	\begin{tabular}{ c c c c c }
		\hline
		Pure Elements & Reference & Phase & Enthalpy (kJ/mol-atom) & \textit{B} (GPa) \\
		\hline
		Ti$\_$sv & Shang \cite{Shang2010b} & hcp & -7.89 & 113\\
                 & Expt 300 K \cite{WolframResearch} & & & 110\\
        Mo$\_$pv & This work 0 K & bcc & -10.84 & 262\\
                        & Expt 300 K \cite{Dickinson1967a} & & & 261\\
        Nb$\_$sv & This work & bcc & -10.22 & 171\\
                       & Expt 300 K \cite{Bolef1961} & & & 172\\
         Ta$\_$pv & This work 0 K & bcc & -11.85 & 196\\
                       & Expt 300 K \cite{Bolef1961} & & & 196\\
          Zr$\_$sv & Shang 0 K \cite{Shang2010b} & hcp & -8.51 & 94\\
                        & Calc 0 K \cite{Treco1953_964,Bergerhoff1983,Karlsruhe,MaterialsProject} & & & 94\\
		\hline
	\end{tabular}
	\label{Ch3-table:pspureele}
\end{table}
\clearpage
%%%

\newpage
\begin{longtable}[H]{ c c c c }
	\caption{First-principles calculations of the enthalpy of formation of the bcc phase in kJ/mol-atom for different atomic percent compositions of the Ti-X binary systems at 0 K.} 	\label{Ch3-table:hof} \\
		\hline
		BCC Phase 0 K & x(Ti) & x(X) & HForm (kJ/mol-atom)\\
		\hline
        \endhead
        \hline
        \endfoot
		Ti & 1.000 & 0.000 & 7.29\\
		Ti15Mo & 0.937 & 0.063 & 3.08\\
		Ti7Mo & 0.875 & 0.125 & 2.82\\
		Ti75Mo25 & 0.750 & 0.250 & 1.12\\
		Ti50Mo50 & 0.500 & 0.500 & -3.67\\
		Ti25Mo75 & 0.250 & 0.750 & -5.18\\
		TiMo15 & 0.063 & 0.937 & 1.79\\
		TiMo53 & 0.019 & 0.981 & 5.82\\
		Mo & 0.000 & 1.000 & 0.00\\
		Ti53Nb & 0.981 & 0.019 & 6.92\\
		Ti7Nb & 0.875 & 0.125 & 5.88\\ 
		Ti75Nb25 & 0.750 & 0.250 & 7.57\\
		Ti50Nb50 & 0.500 & 0.500 & 8.54\\
		Ti25Nb75 & 0.250 & 0.750 & 1.15\\
		TiNb15 & 0.063 & 0.938 & 0.59\\
		TiNb53 & 0.019 & 0.981 & 0.20\\
		Nb & 0.000 & 1.000 & 0.00\\
		Ti53Ta & 0.981 & 0.019 & 7.21\\
		Ti15Ta & 0.938 & 0.063 & 7.04\\
		Ti7Ta & 0.875 & 0.125 & 9.28\\
		Ti75Ta25 & 0.750 & 0.250 & 4.89\\
		Ti50Ta50 & 0.500 & 0.500 & 3.94\\
		Ti25Ta75 & 0.250 & 0.750 & 3.10\\
		TiTa15 & 0.063 & 0.938 & 0.94\\
		TiTa53 & 0.019 & 0.981 & 0.28\\
		Ta & 0.000 & 1.000 & 0.00\\
		Ti53Zr & 0.981 & 0.019 & 5.49\\
		Ti75Zr25 & 0.750 & 0.250 & 4.59\\
		Ti50Zr50 & 0.500 & 0.500 & 1.94\\
		Ti25Zr75 & 0.250 & 0.750 & 3.50\\
		TiZr15 & 0.063 & 0.938 & 5.72\\
		Zr & 0.000 & 1.000 & 8.19\\
		\hline
\end{longtable}
%%%

\newpage
\begin{longtable}[H]{ c c c }
	\caption{Thermodynamic parameters for the Ti-Mo-Nb-Ta-Zr system. The thermodynamic description of the pure elements is not included. The pure elements were adopted from the SGTE database and are listed in the supplementary tdb file \cite{Long1998a}.}	\label{Ch3-table:ip}\\
		\hline
		Phase & Reference & Interaction Parameter\\
		\hline
		\endhead
		\hline
		\endfoot
		Liquid & \cite{Ansara1998} & $0^\textit{L}_{Ti,Mo} = -9000.0+2.00*T$\\
		          & \cite{Zhang2001} & $0^\textit{L}_{Ti,Nb} = 7406.1$\\
		          & \cite{Ansara1998} & $0^\textit{L}_{Ti,Ta} = 1000.0$\\
		          & \cite{Ansara1998} & $0^\textit{L}_{Ti,Ta} = -7000.0$\\
		          & \cite{Kumar1994a} & $0^\textit{L}_{Ti,Zr} = -967.7$\\
		          & \cite{Xiong2004} & $0^\textit{L}_{Mo,Nb} = 15253.7$\\
		          & \cite{Xiong2004} & $1^\textit{L}_{Mo,Nb} = 10594.2$\\
		          & \cite{Xiong2004} & $0^\textit{L}_{Mo,Ta} = 13978.9$\\
		          & \cite{Perez2003} & $0^\textit{L}_{Mo,Zr} = -24055.1+8.146*T$\\
		          & \cite{Perez2003} & $1^\textit{L}_{Mo,Zr} = -5132.17+4.804*T$\\
		          & \cite{Guillermet1991} & $0^\textit{L}_{Nb,Zr} = 10311.0$\\
		          & \cite{Guillermet1991} & $1^\textit{L}_{Nb,Zr} = 6709.0$\\
		          & \cite{Guillermet1995} & $0^\textit{L}_{Ta,Zr} = 13832.1$\\
		          & \cite{Guillermet1995} & $1^\textit{L}_{Ta,Zr} = -7150$\\
          BCC & \cite{Ansara1998} & $0^\textit{L}_{Ti,Mo} = 2000.0$\\
                  & \cite{Ansara1998} & $1^\textit{L}_{Ti,Mo} = -2000.0$\\
                  & \cite{Zhang2001} & $0^\textit{L}_{Ti,Nb} = 13045.3$\\
                  & \cite{Ansara1998} & $0^\textit{L}_{Ti,Ta} = 12000.0$\\
                  & \cite{Ansara1998} & $1^\textit{L}_{Ti,Ta} = -2500.0$\\
                  & \cite{Kumar1994a} & $0^\textit{L}_{Ti,Zr} = -4346.2+5.49*T$\\
                  & \cite{Xiong2004} & $0^\textit{L}_{Mo,Nb} = -68202.6+29.86*T$\\
                  & \cite{Xiong2004} & $1^\textit{L}_{Mo,Nb} = 8201.3$\\
                  & \cite{Xiong2004} & $0^\textit{L}_{Mo,Ta} = -75129.2+30.00*T$\\
                  & \cite{Xiong2004} & $1^\textit{L}_{Mo,Ta} = 6039.2$\\
                  & \cite{Perez2003} & $0^\textit{L}_{Mo,Zr} = 17936.0+3.10*T$\\
                  & \cite{Perez2003} & $1^\textit{L}_{Mo,Zr} = -991.0+4.30*T$\\
                  & \cite{Xiong2004} & $0^\textit{L}_{Nb,Ta} = 1298.0$\\
                  & \cite{Guillermet1991} & $0^\textit{L}_{Nb,Zr} = 15911.0+3.35*T$\\
                  & \cite{Guillermet1991} & $1^\textit{L}_{Nb,Zr} = 3919.0-1.09*T$\\
                  & \cite{Guillermet1995} & $0^\textit{L}_{Ta,Zr} = 29499.6+2.67*T$\\
                  & \cite{Guillermet1995} & $1^\textit{L}_{Ta,Zr} = -4396.2+4.43*T$\\
                  & \cite{Guillermet1995} & $2^\textit{L}_{Ta,Zr} = -6353.3+4.91*T$\\
                  & This work & $0^\textit{L}_{Ti,Mo,Ta} = -154731.2$\\
                  & This work & $0^\textit{L}_{Nb,Ta,Ti} = -136603.3$\\
                  & This work & $1^\textit{L}_{Nb,Ta,Ti} = -136602.7$\\
          HCP & \cite{Ansara1998} & $0^\textit{L}_{Ti,Mo} = 22760.0-6.00*T$\\      
                  & \cite{Zhang2001} & $0^\textit{L}_{Ti,Nb} = 11742.4$\\
                  & \cite{Ansara1998} & $0^\textit{L}_{Ti,Ta} = 8500.0$\\
                  & \cite{Kumar1994a} & $0^\textit{L}_{Ti,Zr} = 5133.0$\\
                  & \cite{Perez2003} & $0^\textit{L}_{Mo,Zr} = 26753.8+4.56*T$\\
                  & \cite{Guillermet1991} & $0^\textit{L}_{Nb,Zr} = 24411.0$\\
                  & \cite{Guillermet1995} & $0^\textit{L}_{Ta,Zr} = 30051.7$\\
           FCC & \cite{Ansara1998} & $0^\textit{L}_{Ti,Mo} = 16500.0$\\
                  & \cite{Ansara1998} & $0^\textit{L}_{Ti,Ta} = 8500.0$\\
    Al3M$\_$D022 & \cite{Ansara1998} & $0^\textit{L}_{Ti:Ti} = 4*GFCCTI$\\
                            & \cite{Ansara1998} & $0^\textit{L}_{Mo:Mo} = 4*GFCCMO$\\
                            & \cite{Ansara1998} & $0^\textit{L}_{Ti:Mo} = GFCCMO+3.0*GFCCTI$\\
                            & \cite{Ansara1998} & $0^\textit{L}_{Mo:Ti} = 3.0*GFCCMO+GFCCTI$\\
                            & \cite{Ansara1998} & $0^\textit{L}_{Ti:Ta} = GFCCTA+3.0*GFCCTI$\\
         AlM$\_$D019 & \cite{Ansara1998} & $0^\textit{L}_{Ti:Ti} = 4.0+4.0*GHSERTI$\\
                              & \cite{Ansara1998} & $0^\textit{L}_{Mo:Mo} = 4.0*GHCPMO$\\
                              & \cite{Ansara1998} & $0^\textit{L}_{Ta:Ta} = 4.0*GHCPTA$\\
                              & \cite{Ansara1998} & $0^\textit{L}_{Ti:Mo} = 17072.0-4.5*T+GHCPMO+3.0*GHSERTI$\\
                              & \cite{Ansara1998} & $0^\textit{L}_{Mo:Ti} = 17072.0-4.5*T+3.0*GHCPMO+GHSERTI$\\
                              & \cite{Ansara1998} & $0^\textit{L}_{Ti:Ta} = 6376.0+GHCPTA+3.0*GHSERTI$\\
                              & \cite{Ansara1998} & $0^\textit{L}_{Ta:Ti} = 6376.0+3.0*GHCPTA+GHSERTI$\\
                              & \cite{Ansara1998} & $0^\textit{L}_{Ti:Mo} = 51212.0-13.5*T$\\
                              & \cite{Ansara1998} & $0^\textit{L}_{Mo,Ti:Ti} = 51212.0-13.5*T$\\
                              & \cite{Ansara1998} & $0^\textit{L}_{Mo:Mo,Ti} = 5692.0-1.5*T$\\
                              & \cite{Ansara1998} & $0^\textit{L}_{Ti:Ti,Mo} = 5692.0-1.5*T$\\
                              & \cite{Ansara1998} & $0^\textit{L}_{Ta,Ti:Ta} = 19128.0$\\
                              & \cite{Ansara1998} & $0^\textit{L}_{Ta,Ti:Ti} = 19128.0$\\
                              & \cite{Ansara1998} & $0^\textit{L}_{Ta:Ta,Ti} = 2128.0$\\
                              & \cite{Ansara1998} & $0^\textit{L}_{TI:Ta,Ti} = 2128.0$\\
                       AlTi & \cite{Ansara1998} & $0^\textit{L}_{Ti:Ti} = 2.0*GFCCTI$\\
                              & \cite{Ansara1998} & $0^\textit{L}_{Mo:Mo} = 2.0*GFCCMO$\\
                              & \cite{Ansara1998} & $0^\textit{L}_{Ta:Ta} = 2.0*GFCCTA$\\
                              & \cite{Ansara1998} & $0^\textit{L}_{Ti:Mo} = 8250.0+GFCCMO+GFCCTI$\\
                              & \cite{Ansara1998} & $0^\textit{L}_{Mo:Ti} = 8250.0+GFCCMO+GFCCTI$\\
                              & \cite{Ansara1998} & $0^\textit{L}_{Ti:Ta} = 4250.0+GFCCTA+GFCCTI$\\
                              & \cite{Ansara1998} & $0^\textit{L}_{Ta:Ti} = 4250.0+GFCCTA+GFCCTI$\\
                              & \cite{Ansara1998} & $0^\textit{L}_{Mo,Ti:Mo} = 8250.0$\\
                              & \cite{Ansara1998} & $0^\textit{L}_{Mo,Ti:Ti} = 8250.0$\\
                        & \cite{Ansara1998} & $0^\textit{L}_{Mo:Mo,Ti} = 8250.0$\\
                        & \cite{Ansara1998} & $0^\textit{L}_{Ti:Mo,Ti} = 8250.0$\\
                        & \cite{Ansara1998} & $0^\textit{L}_{Ta,Ti:Ta} = 4250.0$\\
                        & \cite{Ansara1998} & $0^\textit{L}_{Ta,Ti:Ti} = 4250.0$\\
                        & \cite{Ansara1998} & $0^\textit{L}_{Ta:Ta,Ti} = 4250.0$\\
                        & \cite{Ansara1998} & $0^\textit{L}_{Ti:Ta,Ti} = 4250.0$\\
      BCC$\_$B2 & \cite{Ansara1998} & $0^\textit{L}_{Ti:Mo} = 10000.0$\\
  disordered phase & \cite{Ansara1998} & $0^\textit{L}_{Mo:Ti} = 10000.0$\\
                              & \cite{Ansara1998} & $0^\textit{L}_{Ti:Ta} = 5000.0$\\
                              & \cite{Ansara1998} & $0^\textit{L}_{Ta:Ti} = 5000.0$\\
       LAVES$\_$C15 & \cite{Kar2008} & $0^\textit{L}_{Ti:Ti} = 15000.0+3.0*GHSERTI$\\
                               & \cite{Perez2003} & $0^\textit{L}_{Mo:Mo} = 15000.0+3.0*GHSERMO$\\
                               & \cite{Perez2003}  & $0^\textit{L}_{Zr:Zr} = 15000.0+3.0*GHSERZR$\\
                               & \cite{Kar2008} & $0^\textit{L}_{Ti:Mo} = 15000.0+GHSERMO$\\
                               &                        & $+2.0*GHSERTI$\\
                               & \cite{Kar2008} & $0^\textit{L}_{Mo:Ti} = 15000.0$\\
                               &                        & $+2.0*GHSERMO+GHSERTI$\\
                               & \cite{Kar2008} & $0^\textit{L}_{Ti:Zr} = 9000.0$\\
                               &                        & $+GHSERZR+2.0*GHSERTI$\\
                               & \cite{Kar2008} & $0^\textit{L}_{Zr:Ti} = 15000.0+2.0*GHSERZR+GHSERTI$\\
                               & \cite{Perez2003}  & $0^\textit{L}_{Mo:Zr} = -21734.8+0.14*T$\\
                               &                             & $+GHSERZR+2.0*GHSERMO$\\
                               & \cite{Perez2003}  & $0^\textit{L}_{Zr:Mo} = 21734.8-0.14*T$\\
                               &                             & $+2.0*GHSERZR+GHSERMO$\\
                               & \cite{Perez2003} & $0^\textit{L}_{Mo:Mo,Zr} = 60000.0$\\
                               & \cite{Perez2003} & $0^\textit{L}_{Zr:Mo,Zr} = 60000.0$\\
                               & \cite{Perez2003}  & $0^\textit{L}_{Mo,Zr:Mo} = 100000.0$\\
                               & \cite{Perez2003}  & $0^\textit{L}_{Mo,Zr:Zr} = 100000.0$\\
                               & \cite{Kar2008} & $0^\textit{L}_{Ti:Mo,Zr} = 60000.0$\\
                               & \cite{Kar2008} & $0^\textit{L}_{Mo,Zr:Ti} = 100000.0$\\
                  OMEGA & \cite{Zhang2001} & $0^\textit{L}_{Ti} = 1886.7-0.15*T+GHSERTI$\\
                               & \cite{Zhang2001} &$0^\textit{L}_{Nb} = 15000.0+2.4*T+GHSERNB$\\
                               & \cite{Dinsdale1991} & $0^\textit{L}_{Zr} = -8878.082+144.432234*T$\\
                               &                               & $-26.8556*T*LN(T)-.002799446*T2+38376*T-1$\\                      
                               &      & $298.15 < T < 2128$\\
                               &      & $-29500.524+265.290858*T-42.144*T*LN(T)$\\
                               &      & $+7.17445E+31*T-9$\\          
                               &      &  $2128 < T< 6000$\\
                               & \cite{Zhang2001} & $0LTi,Nb = -3775.9$\\
		\hline
\end{longtable}
%%%



\newpage
%%%
\begin{figure}[H]
	\centering
	\includegraphics[width=\textwidth]{Chapter-3/Figures/TiMo.png}
	\caption{Figure a on the left plots the previously modeled thermodynamic description of the Ti-Mo system versus available experimental data to ensure accuracy \cite{Ansara1998,Murray1981}. Figure b on the right plots the enthalpy of formation of the bcc phase predicted by the previous thermodynamic modeling (solid line) at 300 K versus the present first-principles calculations (circles) at 0 K.}
	\label{Ch3-figure:TiMo}
\end{figure}
%%%

\newpage
%%%
\begin{figure}[H]
	\centering
	\includegraphics[width=\textwidth]{Chapter-3/Figures/TiNb.png}
	\caption{Figure a on the left plots the previously modeled thermodynamic description of the Ti-Nb system versus available experimental data to ensure accuracy \cite{Zhang2001,Kumar1994}. Figure b on the right plots the enthalpy of formation of the bcc phase predicted by the previous thermodynamic modeling (solid line) at 300 K versus the present first-principles calculations (circles) at 0 K.}
	\label{Ch3-figure:TiNb}
\end{figure}
%%%

\newpage
%%%
\begin{figure}[H]
	\centering
	\includegraphics[width=\textwidth]{Chapter-3/Figures/TiTa.png}
	\caption{Figure a on the left plots the previously modeled thermodynamic description of the Ti-Ta system versus available experimental data to ensure accuracy \cite{Ansara1998,Murray1987}. Figure b on the right plots the enthalpy of formation of the bcc phase predicted by the previous thermodynamic modeling (solid line) at 300 K versus the present first-principles calculations (circles) at 0 K.}
	\label{Ch3-figure:TiTa}
\end{figure}
%%%

\newpage
%%%
\begin{figure}[H]
	\centering
	\includegraphics[width=\textwidth]{Chapter-3/Figures/TiZr.png}
	\caption{Figure a on the left plots the previously modeled thermodynamic description of the Ti-Zr system versus available experimental data to ensure accuracy \cite{Kumar1994a}. Figure b on the right plots the enthalpy of formation of the bcc phase predicted by the previous thermodynamic modeling (solid line) at 300 K versus the present first-principles calculations (circles) at 0 K.
	}
	\label{Ch3-figure:TiZr}
\end{figure}
%%%

\newpage
%%%
\begin{figure}[H]
	\centering
	\includegraphics[width=\textwidth]{Chapter-3/Figures/binary1.png}
	\caption{The previously modeled thermodynamic descriptions of the Mo-Nb (a) \cite{Xiong2004}, Mo-Ta (b) \cite{Xiong2004} and Nb-Ta (c) \cite{Xiong2004} binary systems are evaluated by comparing with available phase boundary experimental data.}
	\label{Ch3-figure:binary1}
\end{figure}
%%%

\newpage
%%%
\begin{figure}[H]
	\centering
	\includegraphics[width=\textwidth]{Chapter-3/Figures/binary2.png}
	\caption{The previously modeled thermodynamic descriptions of the Mo-Zr \cite{Perez2003}, Nb-Zr \cite{Guillermet1991,Abriata1982} and Ta-Zr \cite{Guillermet1995} binary systems are evaluated for accuracy by comparing the available experimental phase boundary data.}
	\label{Ch3-figure:binary2}
\end{figure}
%%%

\newpage
%%%
\begin{figure}[H]
	\centering
	\includegraphics[width=\textwidth]{Chapter-3/Figures/TiMoNb.png}
	\caption{Figure a) on the left is a binary interpolation of the isothermal section of Ti-Mo-Nb plotted at 873 K. Figure b) plots the present calculations (circles) and binary interpolation (solid black line) of the enthalpy of formation of the bcc phase.}
	\label{Ch3-figure:TiMoNb}
\end{figure}
%%%

\newpage
%%%
\begin{figure}[H]
	\centering
	\includegraphics[width=\textwidth]{Chapter-3/Figures/TiMoTa1.png}
	\caption{Figure a) on the left is a binary interpolation of the isothermal section of Ti-Mo-Ta plotted at 873 K. Figure b) plots the present calculations (circles), binary interpolation (solid black line) and ternary assessed (red dotted line) enthalpy of formation of the bcc phase.}
	\label{Ch3-figure:TiMoTa1}
\end{figure}
%%%

\newpage
%%%
\begin{figure}[H]
	\centering
	\includegraphics[width=\textwidth]{Chapter-3/Figures/TiMoTa2.png}
	\caption{Figure a) on the left is the isothermal plot of Ti-Mo-Ta at 873 K after evaluation of the ternary interaction parameters. Figure b) on the right is zoomed in to show the comparison with the experimental data \cite{Nikitin1971}.}
	\label{Ch3-figure:TiMoTa2}
\end{figure}
%%%

\newpage
%%%
\begin{figure}[H]
	\centering
	\includegraphics[width=\textwidth]{Chapter-3/Figures/TiMoZr.png}
	\caption{Figure a) on the left is a binary interpolation of the isothermal section of Ti-Mo-Zr plotted at 1273 K with experimental phase data obtained at the same temperature \cite{Kar2008,Prokoshkin1967}. Figure b) plots the present calculations (circles) and binary interpolation (solid black line) of the enthalpy of formation of the bcc phase.}
	\label{Ch3-figure:TiMoZr}
\end{figure}
%%%

\newpage
%%%
\begin{figure}[H]
	\centering
	\includegraphics[width=\textwidth]{Chapter-3/Figures/TiNbTa1.png}
	\caption{Figure a) on the top left is a binary interpolation of the isothermal section of Ti-Nb-Ta plotted at 673 K. Figure b) is a binary interpolation of the isothermal section of Ti-Nb-Ta at 823 K. Both binary interpolations are compared with experimental phase boundary data \cite{Na2001}. Figure c) plots the present calculations (circles), binary interpolation (solid black line) and ternary assessed (red dotted line) enthalpy of formation of the bcc phase.}
	\label{Ch3-figure:TiNbTa1}
\end{figure}
%%%

\newpage
%%%
\begin{figure}[H]
	\centering
	\includegraphics[width=\textwidth]{Chapter-3/Figures/TiNbTa2.png}
	\caption{Figure a) on the left is the isothermal plot of Ti-Nb-Ta at 673 K after evaluation of the ternary interaction parameters. Figure b) is the evaluated isothermal plot of Ti-Nb-Ta at 823 K. Both plots have experimental phase boundary data \cite{Na2001}.}
	\label{Ch3-figure:TiNbTa2}
\end{figure}
%%%

\newpage
%%%
\begin{figure}[H]
	\centering
	\includegraphics[width=\textwidth]{Chapter-3/Figures/TiNbZr.png}
	\caption{Figure a) on the left is a binary interpolation of the isothermal section of Ti-Nb-Zr plotted at 843 K with experimental phase data \cite{Tokunaga2007}. Figure b) plots the present calculations (circles) and binary interpolation (solid black line) of the enthalpy of formation of the bcc phase.}
	\label{Ch3-figure:TiNbZr}
\end{figure}
%%%

\newpage
%%%
\begin{figure}[H]
	\centering
	\includegraphics[width=\textwidth]{Chapter-3/Figures/TiTaZr1.png}
	\caption{Figure a) on the top left is a binary interpolation of the isothermal section of Ti-Ta-Zr plotted at 1273 K. Figure b) is a binary interpolation of the isothermal section of Ti-Ta-Zr at 1773 K. Both binary interpolations are compared with experimental data \cite{Lin1996,Hoch1964}. Figure c) plots the present calculations (circles) and binary interpolation (solid black line) for the enthalpy of formation of the bcc phase.}
	\label{Ch3-figure:TiTaZr}
\end{figure}
%%%
\chapter{First-principles aided thermodynamic modeling of the Sn-Ta system}

\section{Introduction}

Currently, the biomaterial implant research of Ti alloys is focused on biocompatible elements that stabilize the body centered cubic (bcc, $\beta$) phase of Ti and help to lower its elastic modulus. Tantalum (Ta) is a biocompatible element and is considered to be a strong -stabilizers \cite{Brailovski2011b}. Recently, tin (Sn) has also been researched for use in Ti-alloys due to its biocompatibility and low cost \cite{Niinomi2012}. Kuroba et al. \cite{Kuroda1998} studied various Ti-alloys such as Ti-29-Nb-13Ta-2Sn (weight percentage, and similarly hereinafter unless specified otherwise), Ti-29Nb-13Ta-4Mo, and Ti-29Nb-13Ta-6Sn for use as biocompatible implant materials. Kuroba and Hagiwara \cite{He2004} also studied new Ti-Cu-Ni-Sn-Ta alloys for the artificial materials used in orthopedic surgeries. The Sn-Ta system is thus an important sub-system for this purpose \cite{He2006}.  A complete knowledge base of the thermodynamic description of Sn-Ta can be used to examine the effects of temperature and composition on phase stability for higher order systems and help to tailor experimental alloy selections to viable options. The CALPHAD technique, in combination with first-principles and phonon calculations based on the DFT, has been proven to provide valuable data to model the thermodynamic properties of binary such as Ta-Sn that lack sufficient experimental data \cite{Liu2009}. The Sn-Ta system has three solid solution phases and two intermetallic compounds, i.e. the bcc, body centered tetragonal (bct), and diamond solution phases, and the intermetallic compounds Ta$_{3}$Sn with space group $Pm\overline{3}n$ and TaSn$_{2}$ (Ta$_{1.2}$Sn$_{1.8}$) with space group $Fddd$ \cite{Okamoto2003}.

 In the present work, thermodynamic data was predicted using first-principles calculations for the two intermetallics and for the bcc, bct and diamond solution phases. The finite temperature properties of the phases were obtained using the Debye-Gr\"uneisen model \cite{Shang2010} and phonon calculations based on the supercell approach \cite{Wang2012}.  The DFT data was used to model the parameters of the Gibbs energy of each phase using the CALPHAD technique.

\section{Literature Review}

The Sn-Ta binary system was studied by Okamoto \cite{Okamoto2003}, Studnitzky and Schmid-Fetzer \cite{Studnitzky2002}, and Basile \cite{Basile1971}. Both of the intermetallic phases, Ta$_{3}$Sn and TaSn$_{2}$, were shown to have a very narrow homogeneity range. Basile \cite{Basile1971} observed that TaSn$_{2}$ is located around Ta$_{1.2}$Sn$_{1.8}$ which was then designated as Ta$_2$Sn$_3$ by Okamoto \cite{Okamoto2003}. It seems that TaSn$_2$ is a more compatible description of the stoichiometric compound based on the descriptions of similar systems (V-Sn, and Nb-Sn) \cite{Yue2009,Toffolon1998,Toffolon2002}, and thus will be used in the present work. Basile \cite{Basile1971} determined TaSn$_2$ has a peritectic reaction at 595 $^{\circ}$C and used X-ray diffraction (XRD) to elucidate the lattice parameters of TaSn$_2$.  

Studnitzky and Schmid-Fetzer \cite{Studnitzky2002} used powder samples to study the Ta$_3$Sn and TaSn$_2$ intermetallic phases and verified the results previously reported by Basile \cite{Basile1971}. They cold pressed the pure element powders at 600 MPa and then heated the pellets at 1000 $^{\circ}$C for up to 48 hours.  The resulting pellet was then cold pressed at 600 MPa again. Under these conditions TaSn$_2$ was observed at 400 $^{\circ}$C, but was not present as the temperature increased to 600 $^{\circ}$C.  In the work by Courtney et al. \cite{Courtney1965}, Ta$_3$Sn was studied to see how the temperature affects the long-range ordering parameter. In Courtney et al.'s work, Ta$_3$Sn powder samples were sintered at 600, 700, 950, 1200, and 1450 $^{\circ}$C for 2, 4, 7, and 16 days, respectively.  Each sample was then studied using x-ray diffraction at room temperature to examine the phases present and the long-range ordering. They concluded that the transition temperature of superconductivity for Ta$_3$Sn varied by a maximum of 4 $^{\circ}$K based on heat treatment and sintering times due to long-range ordering that occurred. Courtney et al. also measured the lattice parameter of each sample and reported the average value of this cubic phase being 5.285 $\AA$.


\section{Modeling and Calculations}

\subsection{First-principles details}

In the present work, the Vienna ab-initio Simulation Package (VASP) was used to perform the first-principles calculations \cite{Kresse1996}. The projector augmented-wave (PAW) \cite{Kresse1999,Blochl1994} method was used to describe the electron-ion interactions. Based on the work of comparing X-C functionals (Figure \ref{Ch5-figure:PBEvsPW91}) the exchange-correlation functional of the generalized gradient approach depicted by Perdew, Burke, and Ernzerhof (PBE) was employed \cite{Perdew1996a}. A sigma value of 0.2 eV and a plane wave energy cutoff of 1.3 times higher than the highest default cutoff was adopted. The Brillouin zone sampling was done with Bl\"ochl corrections \cite{Blochl1994} using a gamma centered Monkhorst-Pack (MP) scheme \cite{Monkhorst1976a}. The k-points grid for diamond-Sn, bcc-Ta, TaSn$_2$, and bcc-Sn were 4x4x4, 6x6x6, 10x10x5, and 6x6x6 respectively. The k-point grids for the bct-Sn, Ta$_3$Sn and bcc SQS calculations used an automated k-point mesh generator in VASP with the length of the subdivisions specified as 80. The energy convergence criterion of the electronic self-consistency is set as $10^{-4}$ eV/atom and $10^{-4}$ eV/A was set as the stopping criteria for the ionic relaxation loop for all of the calculations. 

To calculate the enthalpy of formation of the bcc phase across the entire composition range, the enthalpy of formation of Ta and Sn in the bcc phase were calculated with five different compositions of Ta$_{1-x}$Sn$_{x}$, where x=0.0185 (Ta$_{53}$Sn, 54 atoms), 0.25, 0.5, 0.75, and 0.9815 (TaSn$_{53}$, 54 atoms). For x=0.0185 and 0.9815, calculations were performed on a diluted 54 atom cell where all atoms but one was Sn or Ta (Ta$_{53}$Sn and TaSn$_{53}$). For x=0.25, 0.5, and 0.75, 16-atom special quasirandom structures (SQS) in the bcc phase developed by Jiang et al. \cite{Jiang2004} were used to mimic the behavior of random structures. The relaxation of these structures is complicated and discussed in the methodology section. The enthalpy of formation was plotted as a function of composition and then used for the modeling.  

\subsection{CALPHAD}

The Gibbs energy functions of the pure elements were adopted from the SGTE (SSUB) database \cite{Dinsdale1991}. In the present work, the bcc and liquid phases were modeled in conjunction with the two intermetallics Ta$_3$Sn and TaSn$_2$. Dilute first-principles calculations of Ta in Sn were done for the diamond and bct phases. However, there is little solubility of Ta in these phases and there is no description of pure Ta in these phases available in SGTE. So, no binary interaction parameters were introduced in the modeling similar to other Sn systems such as Nb-Sn and V-Sn \cite{Yue2009,Toffolon2002}. The interaction parameters of the liquid and bcc solution phases were modeled using Eq. \ref{eq: gibbssolution} and \ref{eq: gibbexsol}, while Ta$_3$Sn and TaSn$_2$ were modeled according to Eq. \ref{eq: stoichiometric}.

\section{Results and discussion}

\subsection{First-principles}

To evaluate the accuracy of phonon calculations for the present system, both the dispersion curves and the phonon DOS are plotted for bcc-Ta, bct-Sn, TaSn$_2$, and Ta$_3$Sn in Figure \ref{Ch4-figure:Taphonon}, \ref{Ch4-figure:Snphonon}, \ref{Ch4-figure:TaSn2phonon}, and \ref{Ch4-figure:Ta3Snphonon}, respectively.  The bcc-Ta phonon dispersion curve in  Figure \ref{Ch4-figure:Taphonon} is compared with values obtained by Taioli et al. \cite{Taioli2007a} using neutron scattering, showing good agreement. The longitudinal modes (LO) and the transverse modes (TO) measured by Raman spectroscopy \cite{Olijnyk1992} (open square) along with the previous theoretical predictions at the M point (filled square) for bct-Sn are compared with the calculated phonon dispersion curve in Figure \ref{Ch4-figure:Snphonon}. The substantial difference for the LO mode may be due to the temperature and pressure differences as pointed out by Olijnyk \cite{Olijnyk1992}. No imaginary phonon frequencies are obtained in the phonon DOS plots for bcc-Ta, bct-Sn, TaSn$_2$, Ta$_3$Sn, indicating that they are all mechanically and dynamically stable at 0 $^\circ$K. 

The calculated lattice parameters at 0 $^\circ$K from the EOS fitting and with the Debye and phonon models at 298 $^\circ$K are compared to available experimental and previous DFT results in Table \ref{Ch4-table:TaSnlattice}. The lattice parameters of Ta are compared with the experimental lattice parameters by Predmore and Arsenault \cite{Predmore1970} at room temperature and the previous 0 $^\circ$K DFT results by Shang et al. \cite{Shang2010b} who used the GGA-PW91 exchange correlation functional. The Sn lattice parameters are compared to experimental work by Allen et al. \cite{Allen1991} at 298 $^\circ$K and calculations by Arr\'oyave et al. \cite{Arroyave2006a}. The properties of the TaSn$_2$ and Ta$_3$Sn intermetallics are compared to experimental values by Calvert et al. \cite{Calvert1991} and Courtney et al. \cite{Courtney1965}, respectively. The results show a less than 0.5$\%$ difference when compared with other DFT results at 0 $^\circ$K. There is a less than 2$\%$ difference between the DFT 0 $^\circ$K results and the experiments, which are listed in Table \ref{Ch4-table:TaSnlattice}. The variance is due to the fact that the calculations are at 0 $^\circ$K and the experiments are at a higher temperature. When comparing the calculated lattice parameters at 298 $^\circ$K to the experiments, all of the predictions improve to show a less than 1$\%$ difference with the exception of Sn, which shows a less than 2$\%$ difference. 

Table \ref{Ch4-table:TaSnvolume} shows the equilibrium volume, $V_{0}$, bulk modulus, $B$, and the derivative of bulk modulus $B'$ obtained by the EOS E-V fitting of the first-principles data at 0 $^\circ$K.  The Sn and Ta calculations are compared with previous first-principles calculations and available experiments. The volume shows a less than 0.5$\%$ difference between the previous DFT results and current DFT results for both Sn and Ta \cite{Predmore1970,PeltzeryBlanca1993a}. The comparison of the DFT results at 0 $^\circ$K and the experimental results at 298 $^{\circ}$K for volume show a slightly higher variance of less than 5 $\%$ due to the difference in temperature \cite{Shang2010b,PeltzeryBlanca1993a}. The $B$ comparison of previous 0 $^\circ$K DFT results and the present 0 $^\circ$K DFT results show a less than 7 GPa difference and the DFT results at 0 $^\circ$K vary by less than 11 GPa from the experimental results at 298 $^\circ$K \cite{Predmore1970,Shang2010b,PeltzeryBlanca1993a}. The difference between the current calculations and the previous values may be due to many reasons; e.g. the different choices in input parameters used by Peltzer et al. \cite{PeltzeryBlanca1993a} and different exchange correlation functionals. Another reason is due to the temperature difference 0 $^\circ$K (calculations) versus 298 $^\circ$K (experiments). Figure \ref{Ch4-figure:Tafinitetemp} shows the enthalpy and entropy of Ta from the Debye and phonon approaches in comparison with the data from the SGTE pure element database \cite{Dinsdale1991}. Figure \ref{Ch4-figure:Snfinitetemp} shows the comparison of the enthalpy and entropy calculated for Sn from the phonon and Debye model to the SGTE pure element database \cite{Dinsdale1991}. Both show excellent agreement.

The elastic stiffness constants and polycrystalline elastic properties calculated by the Hill approach and the scaling factors for the Debye model are shown in Table \ref{Ch4-table:TaSnelastic}. To ensure the accuracy of the scaling factor, the elastic stiffness constants and moduli are compared with previous first-principles results \cite{Jouault1967_611,Bergerhoff1983,Geller1955_165,Karlsruhe,MaterialsProject}. The previous calculation results and present calculation results only vary slightly for the TaSn$_2$ structure. The present work calculated the elastic stiffness constants for the Ta$_3$Sn structure at 2 different atom sizes and compared the results with previous calculations by the Materials Project \cite{Jouault1967_611,Bergerhoff1983,Geller1955_165,Karlsruhe,MaterialsProject}. The present elastic stiffness results are quite similar. There is a larger variance between the present results and the Materials Project results. This can be attributed to the different input parameters and exchange correlation functional used (PBE in the present work and GGA-PW91 in Materials Project). $B$ calculated from the $c_{ij}$ methodology (designated as $B_{cij}$) is compared with the $B$ obtained from the EOS fitting (designated as $B_{EOS}$), showing a difference of less than 3$\%$. Since the $B_{EOS}$ from the EOS fitting is already compared to experiments, the elastic calculations and the scaling factor for the Debye model are thus deemed accurate.

\subsection{CALPHAD}

The PARROT module in the Thermo-Calc software \cite{Andersson2002} is used to optimize the parameters of the Gibbs energy function of the TaSn$_2$ and Ta$_3$Sn intermetallics as well as the binary interaction parameters for the bcc and liquid phases. The Gibbs energy parameters of the intermetallics are first estimated from the thermodynamic properties obtained by the phonon supercell method because the phonon calculations are regarded as more accurate than the Debye model. While the decomposition temperature of the TaSn$_2$intermetallic is known to be 868 $^\circ$K from experiments, the decomposition of the Ta$_3$Sn intermetallic has not been reported in the literature. It is noted that both the Nb-Sn and V-Sn systems, which are quite similar to the Ta-Sn system, have the X$_3$Sn phase forming through a peritectic reaction of bcc+Liquid$\rightarrow$X$_3$Sn \cite{Yue2009,Toffolon2002,Toffolon1998}. Based on the assumption from similar works that Ta$_3$Sn is also formed through a peritectic reaction, the Ta$_3$Sn parameters are adjusted and the parameters for the liquid phase are evaluated. The evaluation of the Gibbs parameters along with the results from the Debye model and the phonon quasiharmonic approach for TaSn$_2$ and Ta$_3$Sn are plotted in Figure \ref{Ch4-figure:TaSn2finitetemp} and Figure \ref{Ch4-figure:Ta3Snfinitetemp}, respectively. As seen in both figures, the data from the phonon method correlates well with the current CALPHAD modeling. This is to be expected since this data was used to evaluate the parameters. It is noted in Figure \ref{Ch4-figure:TaSn2finitetemp}, that the heat capacity and entropy of TaSn$_2$ from the current CALPHAD modeling is higher than those from the first-principles calculations. This is due to the fact that the enthalpy and entropy values from DFT were adjusted with the experimental data of the peritectic temperature.

The bct and diamond phases are treated as ideal due to the little solubility. As previously stated, the enthalpies of formation of the bcc phase for five different Sn-Ta compositions are calculated and plotted in Figure \ref{Ch4-figure:HofForm}, showing asymmetrical behavior. There is a discrepancy between the first-principles value and the CALPHAD modeling for the lattice stability of bcc-Sn. The first-principles predicts a value of 15.48 kJ/mol-atom and the CALPHAD model gives 4.42 kJ/mol-atom. This difference is expected to be due to the instability of Sn in the bcc phase. Wang et al. \cite{Wang2004a} concluded and discussed the same discrepancy when comparing first-principles DFT results to SGTE data for Os and Ru. Wang et al. calculated the lattice stability of bcc and fcc (face centered cubic) structure for Os and Ru, both stable in the hexagonal close packed phase at standard temperature and pressure, and concluded a difference of approximately 40 and 60 kJ/mol for Ru and Os, respectively. Wang et al. attributed this difference to the fact that when using first-principles calculations of unstable structures, frequencies of some of the phonon modes would become imaginary and thus the results would be less accurate. On the other hand, the CALPHAD technique can extrapolate lattice stabilities from binary solutions for which an alloying element has stabilized the otherwise unstable structure. These enthalpies of formation calculated from the SQS first-principles calculations are used to evaluate the bcc binary interaction parameters in the present CALPHAD modeling. The enthalpy of formation of the bcc phase is negative at the Ta rich side and becomes positive at the Sn rich side. This is common for X-Sn systems \cite{Yue2009,Toffolon2002}, such as the Nb-Sn system \cite{Toffolon2002} shown in Figure \ref{Ch4-figure:HofForm}. It should be noted that Toffolon et al. \cite{Toffolon1998,Toffolon2002} used experimental data on the Sn-rich bcc phase to evaluate the Nb-Sn system's bcc interaction parameters. Due to the asymmetry of enthalpy of formation for the bcc phase, a subregular $^1$L interaction parameters is introduced. 

The interaction parameters obtained in the present work are listed in Table \ref{Ch4-table:ip}. Based on these model parameters, the phase diagram is calculated and shown in Figure \ref{Ch4-figure:SnTaPD}. The melting temperature of Ta$_3$Sn is predicted to be 2884 $^\circ$K. Both the intermetallics decompose incongruently similar to those in the Nb-Sn and V-Sn systems. As seen in Table \ref{Ch4-table:ip}, both intermetallic phases have a negative enthalpy of formation and a negative entropy of formation. This goes along with previous predictions by Arroyave and Liu \cite{Arroyave2006} where they showed that the enthalpy and entropy of formation have the same sign. The calculated enthalpy of mixing of the liquid phase is plotted in Figure \ref{Ch4-figure:HofMix}. The interaction parameter for the liquid phase allows for an accurate representation of the phase stability in Figure \ref{Ch4-figure:SnTaPD} but may need to be slightly adjusted if experimental data would come available.

\section{Conclusion}

The present work incorporates the thermodynamic data from DFT-based first-principles calculations and the available experimental data in the literature to model the Gibbs energies for the bcc and liquid solution phases and the stoichiometric Ta$_3$Sn and TaSn$_2$ phases of the Sn-Ta system.  First-principles calculations are used to predict the enthalpy of formation of the bcc phase for the evaluation of interaction parameters in the phase. The decomposition temperature of Ta$_3$Sn is predicted to be 2884 $^\circ$K. The completed thermodynamic description is complied into a tdb file. The tdb file and raw data from the first-principles calculations are in appendix b.

\newpage
\begin{table}[H]
	\caption{Lattice parameters from first-principles calculations compared with experimental values.}
	\centering
	\begin{tabular}{ c c c c c c }
		\hline
		Phase & Space Group & $a$ ($\AA$) & $b$ ($\AA$)  & $c$ ($\AA$) & Reference\\
		\hline
		bcc-Ta & Im$\overline{3}$m & 3.316 & & &This work (0 $^\circ$K)\\
		            &                             & 3.328 & & & This work phonon (298 $^\circ$K)\\
		            &                             & 3.330 & & & This work Debye (298 $^\circ$K)\\
		            &                             & 3.30   & & & Expt. \cite{Predmore1970}\\
		            &                             & 3.32   & & & DFT (0 $^\circ$K) \cite{Shang2010b}\\
		bct-Sn & I4$_1$/amd & 5.939 & & 3.214 & This work (0 $^\circ$K)\\
		            &                   & 5.959 & & 3.236 & This work phonon (298 $^\circ$K)\\
		            &                   & 5.954 & & 3.222 & This work Debye (298 $^\circ$K)\\
		            &                   & 5.83   & & 3.18 & Expt. \cite{Allen1991}\\
		            &                   & 5.93   & & 3.23 & DFT (0 $^\circ$K) \cite{Arroyave2006a}\\
         TaSn$_2$ & Fddd & 5.641 & 9.766 & 19.200 & This work (0 $^\circ$K)\\
              &          & 5.652 & 9.786 & 19.238 & This work phonon (298 $^\circ$K)\\
              &          & 5.652 & 9.785 & 19.238 & This work Debye (298 $^\circ$K)\\
                         &          & 5.63 & 9.80 & 19.18 & Expt. \cite{Calvert1991}\\
          Ta$_3$Sn & Pm$\overline{3}$n & 5.304 & & &This work (0 $^\circ$K)\\
                          &                   & 5.319 & & & This  work phonon (298 $^\circ$K)\\
                          &                   & 5.319 & & & This work Debye (298 $^\circ$K)\\
                          &                   & 5.29 & & & Expt. \cite{Courtney1965}\\
		\hline
	\end{tabular}
	\label{Ch4-table:TaSnlattice}
\end{table}
\clearpage
%%%

\newpage
\begin{table}[H]
	\caption{Equilibrium volume $V_{0}$, bulk modulus $B$, and the first derivative of bulk modulus with respect to pressure $B'$, from fitted equilibrium properties from the EOS at 0 $^\circ$K compared to experimental work and previous DFT studies.}
	\centering
	\begin{tabular}{ c c c c c }
		\hline
		Phase & $V_{0}$ ($\AA^3$/atom) & $B$ (GPa) & $B'$ & Reference\\
		\hline
		bcc-Ta & 18.241 & 193.7 & 3.84 & This work\\
		            & 17.9685 & 200 &  & Expt. \cite{Predmore1970}\\
		            & 18.313 & 195.3 & 3.82 & DFT \cite{Shang2010b}\\
	    bct-Sn & 28.431 & 47.7 & 4.61 & This work\\
	               & 27.055 & 58.0 & & Expt. \cite{PeltzeryBlanca1993a}\\
	               & 28.396 & 54.0 & & DFT \cite{PeltzeryBlanca1993a}\\
	 TaSn$_2$ & 22.631 & 104.3 & 4.80 & This work\\
	 Ta$_3$Sn & 18.668 & 182.4 & 4.27 & This work\\
		\hline
	\end{tabular}
	\label{Ch4-table:TaSnvolume}
\end{table}
\clearpage
%%%

\newpage
\begin{table}[H]
	\caption{Elastic stiffness constants and elastic properties predicted using the Hill approach and the scaling factors used in the Debye model, calculated from the Poisson ratio, see Eq. \ref{eq: debyescaling}. To ensure the accuracy of the calculated scaling factor, the bulk modulus ($B$) calculated from the elastic constants was compared to the $B_{EOS}$ calculated from the EOS fitting Eq. \ref{eq: zeroenergy}.}
	\centering
	\begin{tabular}{ c c c c c c }
		\hline
		  & \multicolumn{2}{c}{TaSn$_2$} & \multicolumn{3}{c}{Ta$_3$Sn}\\
		  & This Work & FP & This Work & This Work  & FP\\
		  & & \cite{Jouault1967_611,Bergerhoff1983,Karlsruhe,MaterialsProject} & 8 atoms & 32 atoms & \cite{Bergerhoff1983,Geller1955_165,Karlsruhe,MaterialsProject}\\
		  \hline
		  C$_{11}$ (GPa) & 166 & 161 & 297 & 310 & 226\\
		  C$_{12}$ (GPa) & 79 & 78 & 127 & 131 & 155\\
		  C$_{13}$ (GPa) & 62 & 57 & & & \\
		  C$_{22}$ (GPa) & 189 & 182 & & & \\
		  C$_{23}$ (GPa) & 68 & 68 & & & \\
		  C$_{33}$ (GPa) & 187 & 183 & & & \\
		  C$_{44}$ (GPa) & 37 & 35 & 65 & 68 & 22\\
		  C$_{55}$ (GPa) & 59 & 55 & & & \\
		  C$_{66}$ (GPa) & 61 & 60 & & & \\
		  \textit{E} (GPa) & 135 & & 210 & 202 &  \\
		  \textit{G} (GPa) & 53 & 51 & 80 & 76 & 27\\
		  Poisson Ratio & 0.288 & 0.29 & 0.32 & 0.32 & 0.43\\
		  Scaling factor & 0.789 & & 0.71 & 0.71 & \\
		  $B_{cij}$ (GPa) & 107 & 107 & 184 & 190 & 179\\
		  $B_{EOS}$ (GPa) & 104 & & 182 & & \\
		\hline
	\end{tabular}
	\label{Ch4-table:TaSnelastic}
\end{table}
\clearpage
%%%

\newpage
\begin{table}[H]
 	\caption{Modeled parameters in SI units in the present work for the phases in the Sn-Ta binary system. These parameters were incorporated with the SGTE data for the pure elements \cite{Dinsdale1991}.}
 	\centering
 	\begin{tabular}{ c c }
 		\hline
 		Phase (model) & Modeled Paramters\\
 		\hline
 		bcc (Sn,Ta) & $^0$L$_{Ta,Sn}^bcc$ = + 70451\\
 		 & $^1$L$_{Ta,Sn}^bcc$ = + 112237\\
 		 Liquid (Sn,Ta) & $^0$L$_{Ta,Sn}^Liq$ = =17118\\
 		 TaSn$_2$ & $G^{TaSn_{2}} = 2^0G_{Sn}^{bct} + ^{0}G_{Ta}^{bcc} = -29678 - 4.202T$ \\
 		 Ta$_3$Sn & $G^{Ta_{3}Sn} = ^0G_{Sn}^{bct} + 3^{0}G_{Ta}^{bcc} = -68844 - 6.000T$ \\
 		\hline
 	\end{tabular}
 	\label{Ch4-table:ip}
 \end{table}
 \clearpage
 %%%

\pagebreak
\begin{figure}[H]
	\centering
	\includegraphics[width=\textwidth]{Chapter-4/Figures/Taphonondos.pdf}
	\caption{Calculated phonon dispersion curve of bcc-Ta, compared with neutron diffraction experiments ($\circ$) \cite{Taioli2007a} along with the phonon DOS. }
	\label{Ch4-figure:Taphonon}
\end{figure}

\pagebreak
\begin{figure}[H]
	\centering
	\includegraphics[width=\textwidth]{Chapter-4/Figures/Snphonondos.pdf}
	\caption{Calculated phonon dispersion curve of bct-Sn on the left and phonon DOS on the right. The open squares ($\square$) are the LO and TO modes from Raman \cite{Olijnyk1992} and the filled squares the theoretical prediction of the LO and TO modes at the M point \cite{Olijnyk1992}.}
	\label{Ch4-figure:Snphonon}
\end{figure}

\pagebreak
\begin{figure}[H]
	\centering
	\includegraphics[width=\textwidth]{Chapter-4/Figures/TaSn2phonondos.pdf}
	\caption{Calculated phonon dispersion curve for TaSn$_2$ at 0 $^\circ$K and the phonon DOS.}
	\label{Ch4-figure:TaSn2phonon}
\end{figure}

\pagebreak
\begin{figure}[H]
	\centering
	\includegraphics[width=\textwidth]{Chapter-4/Figures/Ta3Snphonondos.pdf}
	\caption{Calculated phonon dispersion curve of Ta$_3$Sn at 0 $^\circ$K on the left and the phonon DOS on the right.}
	\label{Ch4-figure:Ta3Snphonon}
\end{figure}

\pagebreak
\begin{figure}[H]
	\centering
	\includegraphics[scale=1.0]{Chapter-4/Figures/Tafinitetemp.pdf}
	\caption{Comparison of the enthalpy and entropy of bcc-Ta from the Debye model (solid line) and the quasiharmonic phonon calculations (red dotted line) to the SGTE data (blue dashed line) \cite{Dinsdale1991}.}
	\label{Ch4-figure:Tafinitetemp}
\end{figure}

\pagebreak
\begin{figure}[H]
	\centering
	\includegraphics[scale=1.0]{Chapter-4/Figures/Snfinitetemp.pdf}
	\caption{Comparison of the Gibbs energy of bct-Sn from the Debye model (solid line) and the quasiharmonic phonon calculations (red dotted line) to the SGTE data (blue dashed line) \cite{Dinsdale1991}.}
	\label{Ch4-figure:Snfinitetemp}
\end{figure}

\pagebreak
\begin{figure}[H]
	\centering
	\includegraphics[scale=1.0]{Chapter-4/Figures/TaSn2finitetemp.pdf}
	\caption{Heat capacity, enthalpy and entropy of TaSn$_2$ using the Debye model (solid line) and the quasiharmonic phonon calculation (red dotted line) from first-principles calculations, compared with those from the current CALPHAD modeling (blue dashed line).}
	\label{Ch4-figure:TaSn2finitetemp}
\end{figure}

\pagebreak
\begin{figure}[H]
	\centering
	\includegraphics[scale=1.0]{Chapter-4/Figures/Ta3Snfinitetemp.pdf}
	\caption{Heat capacity, enthalpy and entropy of Ta$_3$Sn using the Debye model (solid line) and the quasiharmonic phonon calculation (red dotted line) compared with those from the current CALPHAD modeling (blue dashed line).}
	\label{Ch4-figure:Ta3Snfinitetemp}
\end{figure}

\pagebreak
\begin{figure}[H]
	\centering
	\includegraphics[width=\textwidth]{Chapter-4/Figures/HofForm.pdf}
	\caption{Enthalpy of formation of the bcc phase of the Sn-Ta system as a function of composition at 298 $^{\circ}$K and ambient pressure from the current CALPHAD modeling (solid line) and from the first-principles calculations (dots), showing asymmetric behavior. This was compared with data of the Nb-Sn system from Toffolon et al. \cite{Toffolon1998} (dashed red line) which was modeled using experimental data, showing similar asymmetric behavior.}
	\label{Ch4-figure:HofForm}
\end{figure}

\pagebreak
\begin{figure}[H]
	\centering
	\includegraphics[width=\textwidth]{Chapter-4/Figures/SnTaPD.pdf}
	\caption{Calculated Sn-Ta phase diagram using the present thermodynamic description.}
	\label{Ch4-figure:SnTaPD}
\end{figure}

\pagebreak
\begin{figure}[H]
	\centering
	\includegraphics{Chapter-4/Figures/HofMix.pdf}
	\caption{Enthalpy of mixing of the liquid phase as a function of composition at 298 $^\circ$K and ambient pressure in the Sn-Ta system.}
	\label{Ch4-figure:HofMix}
\end{figure}
\chapter{Effects of alloying elements on the elastic properties of bcc Ti-X alloys}

\section{Introduction}

The present chapter is aimed at studying the effects of alloying elements on the mechanical properties of Ti-alloys as well as completing a database to calculate the elastic properties as a function of composition. This is accomplished by systematically studying the single crystal elastic stiffness coefficients (c$_{ij}$'s) and polycrystalline aggregate properties of bcc Ti-X (X = Mo, Nb, Ta, Sn, Zr) alloys. The elastic properties are calculated using first-principles calculations based on density functional theory (DFT). The composition dependence of the elastic properties of Ti-X alloys is explored through dilute solutions and special quasirandom structures (SQS) \cite{Jiang2004} for concentrated solutions using the methodologies outlined in chapter 2. The obtained elastic properties are then fit using the CALPHAD method and extrapolated to higher order Ti-alloys. 

\section{Modeling and Calculations}

\subsection{Calculation details}
In the present work, the Vienna ab-initio Simulation Package (VASP) \cite{Kresse1996} was employed to calculate the elastic properties of pure elements and Ti-containing binary systems in the bcc phase. The ion-electron interactions were described using the projector augmented wave (PAW) \cite{Kresse1999,Blochl1994} method. As discussed previously, the use of two X-C functionals were compared in this chapter and based on the results the generalized gradient approach depicted by Perdew, Burke, and Ernzerhof (PBE-GGA) was employed \cite{Perdew1996a}. For consistency, a 310 eV energy cutoff was adopted for all calculations, which is roughly 1.3 times higher than the default value. The energy convergence criterion of the electronic self-consistency was set as 10$^{-6}$ eV/atom, and the $\Gamma$-centered Monkhorst-Pack scheme was used for Brillouin zone sampling \cite{Kresse1996,Monkhorst1976a}. The k-points grid used for each calculation are listed in appendix C.

For the Ti-X binary systems, calculations for both dilute and SQS solutions were carried out. Three SQS cells with mole fractions of alloying element X atoms at 0.25, 0.5, and 0.75 were employed. Dilute solutions were calculated for each Ti-X binary alloy using different supercell sizes, i.e., Ti-Mo 4 dilute structures (Mo$_{0.98}$Ti$_{0.02}$ 54-atoms, Mo$_{0.94}$Ti$_{0.06}$ 16-atoms, Ti$_{0.88}$Mo$_{0.12}$ 8-atoms, Ti$_{0.94}$Mo$_{0.06}$ 16-atoms), Ti-Nb 4 dilute structures (Nb$_{0.98}$Ti$_{0.02}$ 54-atoms, Nb$_{0.94}$Ti$_{0.06}$ 16-atoms, Ti$_{0.88}$Nb$_{0.12}$ 8-atoms, Ti$_{0.98}$Nb$_{0.02}$ 54-atoms), Ti-Sn 1 dilute structure (Ti$_{0.94}$Sn$_{0.06}$ 16-atoms), Ti-Ta 5 dilute structures (Ta$_{0.98}$Ti$_{0.02}$ 54-atoms, Ta$_{0.94}$Ti$_{0.06}$ 16-atoms, Ti$_{0.88}$Ta$_{0.12}$ 8-atoms, Ti$_{0.94}$Ta$_{0.06}$ 16-atoms, Ti$_{0.98}$Ta$_{0.02}$ 54-atoms), Ti-Zr 2 dilute structures (Zr$_{0.94}$Ti$_{0.06}$ 16-atoms, Ti$_{0.98}$Zr$_{0.02}$ 54-atoms). The interaction parameters for the  elastic stiffness coefficients were then determined according to the methodology laid out in chapter 2.

\subsection{Modeling details}

The first-principles results were then used to model the elastic stiffness coefficients. The modeling was completed by calculating the difference between the first-principles calculations and a linear extrapolation between pure elements. The differences were then used to fit to the interaction parameters. Due to the limitations within the PARROT module, a Mathematica script was used to fit the interaction parameters. The Mathematica script is appended in appendix C. With the focus being Ti-rich alloys, the first-principles results with 70 at. \% Ti or higher were weighted heavier (x6, according to the authors' practices) than the other points for the fittings. The best fit was found by comparing the fittings obtained with one interaction parameter or with two interaction parameters. The moduli values were than calculated from the elastic stiffness coefficients according to the methodology in chapter 2.

\section{Results and discussion}

\subsection{Evaluation of calculation settings}

The X-C functionals of PW91 and PBE were tested on the Ti-Ta binary system. The results are plotted in Figure \ref{Ch5-figure:PBEvsPW91}, showing the $\overline{C}_{44}$ values differing by 10 GPa or less and the $\overline{C}_{11}$ and $\overline{C}_{12}$ values by less than 5 GPa for high Ta contents and by 26 GPa and 13 GPa with 25 at. \% Ta, respectively. Since overall the values vary by an error of less than 0.2 (calculated with Eq. \ref{eq: error}), and the PBE functional was developed as an improvement of the PW91 functional for metals, the PBE functional is chosen for the present work. 

Three magnitudes of strain are tested on the Ti-Mo binary system and plotted in Figure \ref{Ch5-figure:Strain}. The different strain magnitudes do not affect the results significantly. For example, the $\overline{C}_{11}$ values calculated using $\pm$0.01, $\pm$0.013, and $\pm$0.007 strains at Mo$_{0.94}$Ti$_{0.06}$ are 451 GPa, 450 GPa, and 450 GPa, respectively, varying within 1 GPa (< 0.01). The variances in the $\overline{C}_{12}$ and $\overline{C}_{44}$ results are similar. Overall, the variances in the $\overline{C}_{11}$ and $\overline{C}_{12}$ are less than 0.02 (Eq. \ref{eq: error}). The largest variance is in the Ti$_{0.50}$Mo$_{0.50}$ where the $\overline{C}_{44}$ values are 42 GPa, 42 GPa, and 65 GPa calculated with $\pm$0.01, $\pm$0.013, and $\pm$0.007 strains, respectively. Overall, the strain magnitude does not affect the calculated results significantly, and thus the $\pm$0.01 strain magnitude is used for all the calculations.

\subsection{Calculations of elastic coefficients in Ti-X binary systems}

The elastic stiffness coefficients and bulk moduli ($B$) are calculated for the pure elements in the bcc structure and reported in Table \ref{Ch5-table:pureeleelas}. The results in the present work are all obtained at 0 K. The results for Mo, Nb and Ta, which are stable in the bcc structure at low temperatures, are compared with experimental data at temperatures shown in Table \ref{Ch5-table:pureeleelas} \cite{Dickinson1967a,Bolef1961}. The error (Eq. \ref{eq: error}) between the present results and previous results for Mo, Nb and Ta are 0.0261, 0.075, and 0.0425, respectively \cite{Simmons1971b,Dickinson1967a,Bolef1961}. This discrepancy is due to the temperature difference between the experiments and calculations. The calculations are at 0 K while the experimental values were measured at higher temperatures. 

Ti and Zr are stable in the hcp structure at low temperatures, and Sn is stable in the body centered tetragonal and diamond structures at low temperatures. Due to the instability of Ti, Sn and Zr in the bcc structure at low temperatures, their elastic stiffness coefficients are compared to previous first-principles calculations at 0 K \cite{Shang2010b} that used the PW91 functional, and the errors/differences are 0.024, 0.528 and 0.051, respectively. The differences are related to the instability of the bcc structure, the different exchange correlation functionals and the different input parameters chosen. Due to the bcc instability, multiple relaxation schemes are compared in the present work to find the lowest energy structure retaining the bcc symmetry, making the results the most accurate representation of the bcc pure elements.  

Figure \ref{Ch5-figure:tixc11} to Figure \ref{Ch5-figure:tixc44} compare the calculated elastic stiffness coefficients, $\overline{C}_{11}$, $\overline{C}_{12}$, $\overline{C}_{44}$ (circles) with the currently modeled elastic stiffness coefficients (solid line) and linear combination from the pure elements for the Ti-Mo, Ti-Nb, Ti-Sn, Ti-Ta, and Ti-Zr systems. The model parameters are shown in Table \ref{Ch5-table:tixelasip}. The results calculated for the Ti-Mo, Ti-Nb, and Ti-Ta systems are compared with previous calculated results from Ikehata et al. \cite{Ikehata2004}, and the differences are due to the different input parameters and structures used at each composition. Ikehata et al. \cite{Ikehata2004} used the s orbital electrons as the valance electrons for Ti and used the B2 structure for their Ti$_{0.50}$X$_{0.50}$ compositions with Ti at the body centered site and X at the corner sites. For the Ti$_{0.25}$X$_{0.75}$ and Ti$_{0.75}$X$_{0.25}$ structures they used the DO$_{3}$ structure with space group $Fm\overline{3}m$, and not the bcc space group of $Im\overline{3}m$. The present work uses the p electrons as valance electrons for Ti based on updated recommendations by VASP and 16-atom SQS from Jiang et al. \cite{Jiang2004}. The SQS mimic the random substitution of elements with less error and represent the atomic structures of solution phases better than ordered structures. It can be seen that Mo, Nb, and Ta affect the elastic stiffness coefficients in a similar fashion. As shown in Figure \ref{Ch5-figure:tixc11}a, \ref{Ch5-figure:tixc11}b and \ref{Ch5-figure:tixc11}d the $\overline{C}_{11}$ increases in value from Ti to X (X = Mo, Nb, Ta). Figures \ref{Ch5-figure:tixc44}a, \ref{Ch5-figure:tixc44}b and \ref{Ch5-figure:tixc44}d show that the $\overline{C}_{44}$ values first decrease and then increase with the addition of the alloying element X (X = Mo, Nb and Ta). The $\overline{C}_{12}$ values increase by the addition of Mo or Ta (Figure \ref{Ch5-figure:tixc12}a and Figure \ref{Ch5-figure:tixc12}d, respectively), and the $\overline{C}_{12}$ values first decrease and then increase by the addition of Nb (Figure \ref{Ch5-figure:tixc12}b). A similar trend is shown in the $\overline{C}_{11}$ and $\overline{C}_{12}$ data for the Ti-Sn system (Figure \ref{Ch5-figure:tixc11} and Figure \ref{Ch5-figure:tixc12}c). The $\overline{C}_{11}$ and $\overline{C}_{12}$ values first increase and then decrease from pure Ti to pure Sn. As seen in Figure \ref{Ch5-figure:tixc44}c, $\overline{C}_{44}$ first increases in value, then decreases, and then increases again from pure Ti to pure Sn. In the Ti-Zr system, the $\overline{C}_{11}$ and $\overline{C}_{44}$ values first increase and then decrease with increasing Zr concentration (Figure \ref{Ch5-figure:tixc11}e and Figure \ref{Ch5-figure:tixc44}e). For the Ti-Zr system, the $\overline{C}_{12}$ values first decrease and then increase, as shown in Figure \ref{Ch5-figure:tixc12}e. The first-principles calculations based on DFT results for the elastic stiffness coefficients are listed in Table \ref{Ch5-table:tixelassc}.

Figure \ref{Ch5-figure:tixyoungs} summarizes the present Young's moduli results for each Ti-X binary system (circles). The $E$ results for the pure elements and Ti-X binary systems are listed in Table \ref{Ch5-table:tixelasmod}. The lines are the estimated $E$ using the Voigt-Reuss-Hill approach and predicted from the elastic stiffness coefficients that were modeled using Eq. \ref{eq: elastic}. The elastic stiffness coefficients model parameters are shown in Table \ref{Ch5-table:tixelasip}. The average from the Hill approach is plotted as a solid black line while those from the Voigt (high bound) and Reuss (low bound) approaches are plotted as dotted and dashed lines, respectively. For stable structures the Voigt and Reuss approaches do not vary drastically but when the structures are unstable the variance is larger with the average from the Hill approach being the value that the database will predict. The results for the Ti-Mo, Ti-Nb and Ti-Ta systems are compared to available experimental results \cite{Zhang2015,Boyer1994,Sung2015,Ozaki2004,Fedotov1985,Zhou2009a,Zhou2004a} also shown in Figure \ref{Ch5-figure:tixyoungs} and listed in Table \ref{Ch5-table:tixelasmod}. Figure \ref{Ch5-figure:tixyoungs}a compares the present $E$ results for the Ti-Mo alloy system with experimental data from Zhang et al. \cite{Zhang2015}, Collings et al. \cite{Boyer1994} and Sung et al. \cite{Sung2015}. It can be seen that the $E$ increases in value from pure Ti to pure Mo. The results from Sung et al. \cite{Sung2015} differ by about 60 GPa from the present work. However, during the XRD and TEM investigations by Sung et al. \cite{Sung2015}, one of the metastable phases, $\alpha"$ and $\omega$, was observed in the samples in addition to the bcc phase. The formation of the $\alpha"$ and $\omega$ phases causes variations in the elastic properties from those of the single bcc phase. Zhang et al. \cite{Zhang2015} and Collings et al. \cite{Boyer1994} did not observe the formation of either metastable phase. The Young's moduli determined by Zhang et al. \cite{Zhang2015} and Collings et al. \cite{Boyer1994} agree with the present Voigt-Reuss bounds and have an error of 0.39 (Eq. \ref{eq: error}) from the Hill average Young's modulus. 

The present $E$ of the Ti-Nb system are compared with data from Ozaki et al. \cite{Ozaki2004} and Collings et al. \cite{Boyer1994} in Figure \ref{Ch5-figure:tixyoungs}b, showing an increase in $E$ values with an increase in the Nb concentration. The analyses of the samples from the work by Ozaki et al. \cite{Ozaki2004} and Collings et al. \cite{Boyer1994} depict that all alloys contained the single bcc phase. The $E$ from the present first-principles calculations have an error of 0.09 (Eq. \ref{eq: error}) compared to the $E$ determined by Ozaki et al. \cite{Ozaki2004} and Collings et al. \cite{Boyer1994}. 

Figure \ref{Ch5-figure:tixyoungs}d shows the calculated $E$ for the Ti-Ta system in comparison with the experimental values determined by Fedotov et al. \cite{Fedotov1985} and Zhou et al. \cite{Zhou2004a,Zhou2009a}. The $E$ values, in the Ti-Ta system, increase from pure Ti to pure Ta, and the calculated Young's moduli have an error of 0.19 (Eq. \ref{eq: error}) compared to the experimental Young's moduli \cite{Fedotov1985,Zhou2004a,Zhou2009a}.  

The error between the experimentally determined Young's moduli and the calculated Young's moduli is expected due to the temperature difference (calculations at 0 K and experiments at 298 K). The experimentally determined Young's moduli agree well within the Voigt and Reuss bounds, and the present calculations provide good prediction of the elastic properties of the Ti-Mo, Ti-Nb and Ti-Ta alloys as a function of composition.

The calculated Young's moduli for the Ti-Sn (Figure \ref{Ch5-figure:tixyoungs}c) and Ti-Zr (Figure \ref{Ch5-figure:tixyoungs}e) systems cannot be compared to experimental data because the bcc phase is not stable in these systems at low temperatures. For the Ti-Sn alloy system the $E$ values increase from 0 to ~35 at. \% Sn and then decrease from ~35 to 100 at. \% Sn. The $E$, of the Ti-Zr system, increases in value from 0 to 40 at. \% Zr, and then decreases from 40 to 100 at. \% Zr. Figure \ref{Ch5-figure:tixmap} plots the Young's modulus as a function of composition from pure Ti in the bcc structure to the alloying element (X = Mo, Nb, Sn, Ta, Zr) and compares the effects of each alloying element on the Young's modulus. This mapping can be used to find alloy compositions that have the target $E$ values (10-40 GPa). From Figure \ref{Ch5-figure:tixmap}  there are multiple different compositions that have the target $E$.

As discussed in chapter 2, the instability of the bcc phase can be determined by Born's criteria (Eq. \ref{eq: born1}-\ref{eq: born3}). Figure \ref{Ch5-figure:tixc11-c12} shows the $\overline{C}_{11}$-$\overline{C}_{12}$ values from first-principles calculations and the present modeling, indicating the stability and instability regions of the bcc phase in different Ti-alloys. When the $\overline{C}_{11}$ - $\overline{C}_{12}$ is positive the bcc phase is mechanically stable and when the $\overline{C}_{11}$ - $\overline{C}_{12}$ is negative the bcc phase is mechanically unstable. The bcc phase is mechanically unstable at Mo, Nb, Sn, Ta, and Zr concentrations of less than 5.5, 11.5, 51.5, 9.5 and 4.0 at. \%, respectively. Alloying with only 4.0 at. \% Zr stabilizes the bcc phase, which is the lowest alloying concentration of any of the alloying elements even though it is known as a weak $\beta$-stabilizer. 5.5, 9.5 and 11.5 at. \% of Mo, Ta and Nb, respectively, are needed to alloy with Ti in order to stabilize the bcc phase. Mo, Nb, and Ta are similar elements that are all stable in the bcc phase and known to be $\beta$-stabilizers. 51.5 at. \% of Sn is needed to stabilize the bcc phase when alloyed with Ti. Sn is not stable in the bcc phase and is not a $\beta$-stabilizer. Based on the $E$ mapping in Figure \ref{Ch5-figure:tixmap}, the compositions that fall into the target $E$ range for the biomedical application (10-40 GPa) are targeted. However, the bcc phase of the targeted alloy compositions needs to be stable, which is determined by Figure \ref{Ch5-figure:tixc11-c12}. Using these results, the composition range for each Ti-X alloy that has the target $E$ and the bcc phase stable are listed in Table \ref{Ch5-table:targetalloys}. The Ti-Mo, Ti-Nb, Ti-Ta and Ti-Zr alloys show small compositions ranges where both the criteria are met while Ti-Sn has no composition range that stabilizes the bcc phase and has the target $E$. From these results, there are compositions that can be targeted and alloying with more than one element will show other compositions that have the properties desired. 

Figure \ref{Ch5-figure:tixbulk} and Figure \ref{Ch5-figure:tixshear} show the bulk ($B$) and shear ($G$) moduli of the Ti-X (X = Mo, Nb, Sn, Ta, Zr) systems, with the present results (circles), the Hill average (solid black line), and Voigt (purple dashed line) and Reuss (yellow dashed line) bounds plotted. Similar trends in the $B$ and $G$ data are seen for the Ti-Mo, Ti-Nb and Ti-Ta systems. The $B$ and $G$ increase in value with increasing Mo, Nb and Ta concentration, as shown in Figure \ref{Ch5-figure:tixbulk} and Figure \ref{Ch5-figure:tixshear}, respectively. The $B$ and $G$ values increase and then decrease from pure Ti to pure Sn in the Ti-Sn system (Figure 9c and Figure 10c). In the Ti-Zr system, the $B$ decreases in value from pure Ti to pure Zr (Figure \ref{Ch5-figure:tixbulk}e) and the $G$ first increases in value and then decreases from pure Ti to pure Zr (Figure \ref{Ch5-figure:tixshear}e). The predicted $B$ and $G$ values are listed in Table \ref{Ch5-table:tixelasmod}. 

\subsection{Extrapolation to ternary and higher ordered systems}

The interaction parameters in Table \ref{Ch5-table:tixelasip} can be used to predict the elastic stiffness coefficients of higher order Ti-alloys by summing the interaction parameters of each binary alloy contained in the multi-component alloy from Eq. \ref{eq: elastic}. The predicted elastic stiffness coefficients of the multi-component alloys can be used to calculate the Young's modulus as a function of composition. The accuracy of prediction of the elastic properties of higher order Ti alloys are evaluated by comparing the predicted results with previous experimental results \cite{Niinomi2012,Tane2010a,Geetha2009,Mohammed2014} as shown in Figure \ref{Ch5-figure:tixdatabase} and Table \ref{Ch5-table:tixdatacomp}. The black diagonal line represents a perfect correlation between the predicted and experimental Young's moduli. The grey region indicates the error (3 GPa) in the first-principles calculations, which is the average variance in $\overline{C}_{11}$, $\overline{C}_{12}$ and $\overline{C}_{44}$ from Eq. \ref{eq: averagec11}-\ref{eq: averagec44}. 

It can be seen that the difference between experimental Young's moduli at the same composition from Niinomi et al. \cite{Niinomi2012}, Geetha et al. \cite{Geetha2009}, Tane et al. \cite{Tane2010a} and Mohammad et al. \cite{Mohammed2014} varies from 2 GPa to 46 GPa with different heat treatments and measuring techniques. The scattering in the Young's moduli among experimental measurements is denoted by the vertical error bars in Figure \ref{Ch5-figure:tixdatabase}. The horizontal error bars show the Young's moduli ranges from the Reuss and Voigt bounds with the average from the Hill approach marked by the circle. The experimental Young's moduli deviate from the present predictions by 0.69 to 14 GPa. This difference can be contributed to the temperature difference between the first-principles data and the experimental results and uncertainties in calculations and experiments. Considering the fact that the experimental results from the literature at the same composition vary drastically, the present first-principles calculations give a good representation of the elastic properties of higher order Ti-alloys. It is hypothesized that introducing the binary interaction parameters of non-Ti containing alloys in the system and the ternary interaction parameters can further improve the database predictions. 

\section{Conclusion}

The elastic properties of five bcc Ti-X (X = Mo, Nb, Sn, Ta, Zr) systems, including the elastic stiffness coefficients, bulk modulus, shear modulus, and Young's modulus, were systematically studied using first-principles calculations at different compositions. The CALPAHD methodology was used to evaluate interaction parameters for the Ti-X elastic stiffness coefficients as a function of composition and the polycrystalline aggregate properties were predicted using the Voigt-Reuss-Hill approach. The present calculations showed that 5.5, 11.5, 51.5, 9.5, and 4.0 at. \% of Mo, Nb, Sn, Ta and Zr, respectively, are required to stabilize the bcc structure according to the Born criteria. The trends observed were summarized for each Ti-X (X= Mo, Nb, Sn, Ta, Zr) binary system. Alloying with Mo, Nb, and Ta resulted in similar trends, which is probably because Mo, Nb, and Ta are strong bcc stabilizers and stable in the bcc structure at room temperature. The interaction parameters determined in the current work were used to predict the elastic properties of higher order alloys. The accuracy of database predictions of the Young's modulus was evaluated by comparing the calculated and experimental Young's moduli. Overall, the database predicted the $E$ values on average within 7 GPa and provided good predictions of the elastic properties of Ti-alloys in the bcc phase as a function of composition. 

\newpage
\begin{table}[H]
	\caption{Calculated pure element elastic stiffness coefficients and the bulk modulus $B$ (in GPa) by X-C functional of PBE are compared with the previous first-principles calculations (FP) by X-C functional PW91 and experiments (Expt). Sv, pv and d referring to the s, p, and d states being treated as valance, respectively.}
	\centering
	\begin{tabular}{ c c c c c c }
		\hline
		Pure Elements & & $\bar{C}_{11}$ & $\bar{C}_{11}$  & $\bar{C}_{11}$ & \textit{B}\\
		\hline
		Ti$\_$sv & This work 0 K & 93 & 115 & 41 & 108\\
		& Calc 0 K \cite{Shang2010b} & 96 & 116 & 40 & 107\\
		Mo$\_$pv & This work 0 K & 475 & 164 & 108 & 268\\
		& Expt 73 K \cite{Simmons1971b} & 473 & 156 & 111 &\\
		& Expt 300 K \cite{Dickinson1967a} & 473 & 160 & 109 & 261\\
		Nb$\_$sv & This work 0 K & 245 & 144 & 27 & 178\\
		& Expt 4 K \cite{Simmons1971b} & 253 & 133 & 31 & \\
		& Expt 300 K \cite{Bolef1961} & 247 & 135 & 29 & 172\\
		Sn$\_$d & This work 0 K & 50 & 52 & 29 & 51\\
		& Calc 0 K \cite{Shang2010b} & 30 & 60 & 18 & 48\\
		Ta$\_$pv & This work 0 K & 278 & 164 & 81 & 202\\
		& Expt 0 K \cite{Simmons1971b} & 266 & 158 & 87 & \\
		& Expt 300 K \cite{Bolef1961} & 267 & 161 & 83 & 196\\
		Zr$\_$sv & This work 0 K & 86 & 91 & 32 & 89\\
		& Calc 0 K \cite{Shang2010b} & 82 & 94 & 30 & 90\\
		\hline
	\end{tabular}
	\label{Ch5-table:pureeleelas}
\end{table}
\clearpage
%%%

\newpage
\begin{table}[H]
	\caption{Evaluated interaction parameters $L_0$ and $L_1$ using the R-K polynomial Eq. \ref{eq: elastic} for the elastic stiffness coefficients for the Ti-X binary systems.}
	\centering
	\begin{tabular}{ c c c c c c c }
		\hline
		Alloy & Interaction Parameter & Ti-Mo & Ti-Nb & Ti-Sn & Ti-Ta & Ti-Zr\\
		\hline
		$\bar{C}_{11}$ & $L_0$ & -22.16 & 40.46 & 119.46 & 83.65 & 246.97\\
		& $L_1$ & 0 & 0 & 0 & -67.76 & -135.95\\
		$\bar{C}_{12}$ & $L_0$ & -36.40 & -32.39 & 15.90 & 38.05 & -110.53\\
		& $L_1$ & 0 & 0 & -146.80 & 0 & 78.00\\
		$\bar{C}_{44}$ & $L_0$ & -142.9 & -41.54 & 59.79 & -51.96 & 70.06\\
		& $L_1$ & 0 & -41.95 & -94.38 & 0 & 0\\	
		\hline
	\end{tabular}
	\label{Ch5-table:tixelasip}
\end{table}
\clearpage
%%%

\newpage
\begin{longtable}[H]{ c c c c c}
	\caption{First-principles calculations of the elastic stiffness coefficients $\overline{C}_{11}$, $\overline{C}_{12}$, and $\overline{C}_{44}$ for different atomic percent compositions of the bcc Ti-X binary systems at 0 K.}	\label{Ch5-table:tixelassc} \\
	\hline
	Reference & Ti$_{1-b}$X$_b$ & $\overline{C}_{11}$ & $\overline{C}_{12}$ & $\overline{C}_{44}$\\
	\hline
	\endhead
	\hline
	\endfoot
	This work & Ti & 93 & 115 & 41\\
	This work & Ti$_{0.94}$Mo$_{0.06}$ & 124 & 111 & 38\\
	This work & Ti$_{0.87}$Mo$_{0.13}$ & 146 & 113 & 29\\
	This work & Ti$_{0.75}$Mo$_{0.25}$ & 178 $\pm$3 & 123 $\pm$15 & 32 $\pm$11\\
	This work & Ti$_{0.50}$Mo$_{0.50}$ & 268 $\pm$9 & 136 $\pm$19 & 42 $\pm$9\\
	This work & Ti$_{0.25}$Mo$_{0.75}$ & 385 $\pm$9 & 146 $\pm$6 & 66 $\pm$6\\
	This work & Ti$_{0.06}$Mo$_{0.94}$ & 451 & 158 & 96\\
	This work & Ti$_{0.02}$Mo$_{0.98}$ & 464 & 163 & 100\\
	This work & Mo & 475 & 164 & 108\\
	This work & Ti$_{0.98}$Nb$_{0.02}$ & 93 & 115 & 35\\
	This work & Ti$_{0.87}$Nb$_{0.13}$ & 116 & 116 & 37\\
	This work & Ti$_{0.75}$Nb$_{0.25}$ & 140 $\pm$11 & 116 $\pm$13 & 34 $\pm$10\\
	This work & Ti$_{0.50}$Nb$_{0.50}$ & 181 $\pm$9 & 121 $\pm$2 & 31 $\pm$10\\
	This work & Ti$_{0.25}$Nb$_{0.75}$ & 208 $\pm$3 & 130 $\pm$4 & 15 $\pm$10\\
	This work & Ti$_{0.06}$Nb$_{0.94}$ & 242 & 134 & 18\\
	This work & Ti$_{0.02}$Nb$_{0.98}$ & 242 & 134 & 18\\
	This work & Nb & 245 & 144 & 27\\
	This work & Ti$_{0.94}$Sn$_{0.06}$ & 100 & 122 & 46 \\
	This work & Ti$_{0.75}$Sn$_{0.25}$ & 105 $\pm$5 & 114 $\pm$2 & 60 $\pm$4\\
	This work & Ti$_{0.50}$Sn$_{0.50}$ & 88 $\pm$9 & 93 $\pm$9 & 46 $\pm$4\\
	This work & Ti$_{0.25}$Sn$_{0.75}$ & 92 $\pm$9 & 55 $\pm$7 & 35 $\pm$8\\
	This work & Sn & 50 & 52 & 29\\
	This work & Ti$_{0.98}$Ta$_{0.02}$ & 100 & 115 & 39 \\
	This work & Ti$_{0.94}$Ta$_{0.06}$ & 116 & 113 & 30 \\
	This work & Ti$_{0.87}$Ta$_{0.13}$ & 120 & 121 & 39 \\
	This work & Ti$_{0.75}$Ta$_{0.25}$ & 167 $\pm$1 & 140 $\pm$3 & 45\\
	This work & Ti$_{0.50}$Ta$_{0.50}$ & 208 $\pm$1 & 159 & 51 $\pm$3\\
	This work & Ti$_{0.25}$Ta$_{0.75}$ & 239 $\pm$7 & 143 $\pm$5 & 62 $\pm$3\\
	This work & Ti$_{0.06}$Ta$_{0.94}$ & 257 & 158 & 72\\
	This work & Ti$_{0.02}$Ta$_{0.98}$ & 264 & 163 & 72\\
	This work & Ta & 278 & 164 & 81\\
	This work & Ti$_{0.98}$Zr$_{0.02}$ & 112 & 106 & 43\\
	This work & Ti$_{0.75}$Zr$_{0.25}$ & 148 $\pm$14 & 82 $\pm$7 & 54 $\pm$7\\
	This work & Ti$_{0.50}$Zr$_{0.50}$ & 152 $\pm$17 & 76 $\pm$12 & 48 $\pm$12\\
	This work & Ti$_{0.25}$Zr$_{0.75}$ & 126 $\pm$12 & 82 $\pm$3 & 45 $\pm$3\\
	This work & Ti$_{0.06}$Zr$_{0.94}$ & 89 & 90 & 34\\
	This work & Zr & 86 & 91 & 32\\
	\hline
\end{longtable}
%%%

\newpage
\begin{longtable}[H]{ c c c c c}
	\caption{First-principles calculations of the bulk modulus $B$, shear modulus $G$, and Young's modulus $E$ in GPa for different atomic percent compositions of the bcc Ti-X binary systems at 0 K. As well as experimental data for the Young's modulus obtained at 300 K by the reference stated.} 	\label{Ch5-table:tixelasmod} \\
	\hline
	Reference & Ti$_{1-b}$X$_b$ & \textit{B} & \textit{G} & \textit{E}\\
	\hline
	\endhead
	\hline
	\endfoot
	This work & Ti & 108 & -12.91 & -40.34\\
	This work & Ti$_{0.94}$Mo$_{0.06}$ & 115 & 20 & 54\\
	This work & Ti$_{0.87}$Mo$_{0.13}$ & 124 & 23 & 65\\
	This work & Ti$_{0.75}$Mo$_{0.25}$ & 141 $\pm$15 & 30 $\pm$15 & 84 $\pm$15\\
	This work & Ti$_{0.50}$Mo$_{0.50}$ & 180 $\pm$19 & 51 $\pm$19 & 138 $\pm$19\\
	This work & Ti$_{0.25}$Mo$_{0.75}$ & 226 $\pm$9 & 84 $\pm$9 & 224 $\pm$9\\
	This work & Ti$_{0.06}$Mo$_{0.94}$ & 256 & 114 & 397\\
	This work & Ti$_{0.02}$Mo$_{0.98}$ & 263 & 118 & 308\\
	This work & Mo & 268 & 125 & 325\\
	Expt 300 K \cite{Zhang2015} & Ti$_{0.92}$Mo$_{0.08}$ & & & 83\\
	Expt 300 K \cite{Zhang2015} & Ti$_{0.88}$Mo$_{0.12}$ & & & 90\\
	Expt 300 K \cite{Boyer1994} & Ti$_{0.92}$Mo$_{0.08}$ & & & 84\\
	Expt 300 K \cite{Boyer1994} & Ti$_{0.89}$Mo$_{0.11}$ & & & 89\\
	Expt 300 K \cite{Boyer1994} & Ti$_{0.82}$Mo$_{0.18}$ & & & 101\\
	This work & Ti$_{0.98}$Nb$_{0.02}$ & 108 & -18 & -56\\
	This work & Ti$_{0.87}$Nb$_{0.13}$ & 116 & 11 & 31\\
	This work & Ti$_{0.75}$Nb$_{0.25}$ & 124 $\pm$13 & 22 $\pm$13 & 63 $\pm$13\\
	This work & Ti$_{0.50}$Nb$_{0.50}$ & 141 $\pm$9 & 31 $\pm$10 & 86 $\pm$10\\
	This work & Ti$_{0.25}$Nb$_{0.75}$ & 156 $\pm$4 & 22 $\pm$10 & 64 $\pm$10\\
	This work & Ti$_{0.06}$Nb$_{0.94}$ & 170 & 28 & 81\\
	This work & Ti$_{0.02}$Nb$_{0.98}$ & 170 & 28 & 81\\
	This work & Nb & 178 & 35 & 98\\
	Expt 300 K \cite{Ozaki2004} & Ti$_{0.71}$Nb$_{0.29}$ & & & 67\\
	Expt 300 K \cite{Ozaki2004} & Ti$_{0.66}$Nb$_{0.34}$ & & & 74\\
	Expt 300 K \cite{Ozaki2004} & Ti$_{0.56}$Nb$_{0.44}$ & & & 84\\
	Expt 300 K \cite{Boyer1994} & Ti$_{0.74}$Nb$_{0.26}$ & & & 64\\
	Expt 300 K \cite{Boyer1994} & Ti$_{0.70}$Nb$_{0.30}$ & & & 65\\
	Expt 300 K \cite{Boyer1994} & Ti$_{0.66}$Nb$_{0.34}$ & & & 73\\
	Expt 300 K \cite{Boyer1994} & Ti$_{0.56}$Nb$_{0.44}$ & & & 83\\
	This work & Ti$_{0.94}$Sn$_{0.06}$ & 115 & -10 & -30\\
	This work & Ti$_{0.75}$Sn$_{0.25}$ & 111 $\pm$5 & 11 $\pm$5 & 31 $\pm$5\\
	This work & Ti$_{0.50}$Sn$_{0.50}$ & 91 $\pm$9 & 10 $\pm$9 & 29 $\pm$9\\
	This work & Ti$_{0.25}$Sn$_{0.75}$ & 67 $\pm$9 & 27 $\pm$9 & 72 $\pm$9\\
	This work & Sn & 51 & 7 & 21\\
	This work & Ti$_{0.98}$Ta$_{0.02}$ & 110 & -3 & -9\\
	This work & Ti$_{0.94}$Ta$_{0.06}$ & 114 & 11 & 32\\
	This work & Ti$_{0.87}$Ta$_{0.13}$ & 121 & 11 & 32\\
	This work & Ti$_{0.75}$Ta$_{0.25}$ & 149 $\pm$3 & 28 $\pm$3 & 78 $\pm$3\\
	This work & Ti$_{0.50}$Ta$_{0.50}$ & 175 $\pm$1 & 38 $\pm$3 & 106 $\pm$3\\
	This work & Ti$_{0.25}$Ta$_{0.75}$ & 175 $\pm$7 & 56 $\pm$7 & 152 $\pm$7\\
	This work & Ti$_{0.06}$Ta$_{0.94}$ & 191 & 62 & 168\\
	This work & Ti$_{0.02}$Ta$_{0.98}$ & 197 & 62 & 169\\
	This work & Ta & 202 & 70 & 189\\
	Expt 300 K \cite{Fedotov1985} & Ti$_{0.62}$Ta$_{0.38}$ & & & 62\\
	Expt 300 K \cite{Fedotov1985} & Ti$_{0.58}$Ta$_{0.42}$ & & & 79\\
	Expt 300 K \cite{Fedotov1985} & Ti$_{0.52}$Ta$_{0.48}$ & & & 95\\
	Expt 300 K \cite{Zhou2004a} & Ti$_{0.62}$Ta$_{0.38}$ & & & 67\\
	Expt 300 K \cite{Zhou2004a} & Ti$_{0.49}$Ta$_{0.51}$ & & & 105\\
	This work & Ti$_{0.98}$Zr$_{0.02}$ & 108 & 17 & 48\\
	This work & Ti$_{0.75}$Zr$_{0.25}$ & 104 $\pm$14 & 44 $\pm$14 & 116 $\pm$14\\
	This work & Ti$_{0.50}$Zr$_{0.50}$ & 101 $\pm$17 & 44 $\pm$17 & 115 $\pm$17\\
	This work & Ti$_{0.25}$Zr$_{0.75}$ & 97 $\pm$12 & 34 $\pm$12 & 91 $\pm$12\\
	This work & Ti$_{0.06}$Zr$_{0.94}$ & 90 & 9 & 27\\
	This work & Zr & 89 & 6 & 16\\
	\hline
\end{longtable}
%%%

\newpage
\begin{table}[H]
	\caption{Compositions of the binary alloys that fall in the target $E$ range compared to the bcc phase stability can predict the alloy compositions where the $E$ is close to bone and the bcc phase is stable and those compositions can be targeted.}
	\centering
	\begin{tabular}{ c c c c }
		\hline
		Alloy & at Target $E$ & bcc stabilized & Target Compositions\\
		\hline
		Ti-Mo & Pure Ti to 0.10 mol Mo & > 0.055 mol Mo & 0.055 to 0.10 mol Mo\\
		Ti-Nb & Pure Ti to 0.20 mol Nb & > 0.115 mol Nb & 0.115 to 0.20 mol Nb\\		
		Ti-Sn & Pure Ti to 0.40 mol Sn & > 0.515 mol Sn & N/A\\		
		Ti-Ta & Pure Ti to 0.15 mol Ta & > 0.095 mol Ta & 0.095 to 0.15 mol Ta\\
		Ti-Zr & Pure Ti to 0.05 mol Zr & > 0.040 mol Zr & 0.040 to 0.05 mol Zr\\		
		\hline
	\end{tabular}
	\label{Ch5-table:targetalloys}
\end{table}
\clearpage
%%%

\newpage
\begin{table}[H]
	\caption{Predicted Young's moduli (in GPa) of higher order alloys in the bcc phase compared to experimental values found with both the weight percent and atomic percent listed.}
	\centering
	\begin{tabular}{ c c c c }
		\hline
		Alloy Name (wt. \%) & at. \% & Calc $E$ & Expt $E$\\
		\hline
		Ti-35Nb-7Zr-5Ta \cite{Geetha2009} & Ti-24Nb-5Zr-2Ta & 81 & 80\\
		Ti-29Nb-13Ta-4.6Zr \cite{Geetha2009}  & Ti-20Nb-5Ta-3Zr & 76 & 75\\
		Ti-29Nb-13Ta-6Sn \cite{Geetha2009} & Ti-21Nb-5Ta-3Sn & 68 & 74\\
		Ti-29Nb-13Ta-4.6Sn \cite{Geetha2009} & Ti-20Nb-5Ta-3Sn & 67 & 66\\
		Ti-29Nb-13Ta-4.5Zr \cite{Geetha2009} & Ti-20Nb-5Ta-3Zr & 76 & 65\\
		Ti-29Nb-13Ta-4.6Zr \cite{Tane2010a} & Ti-21Nb-5Ta-3Zr & 76 & 64\\
		Ti-30Nb-10Ta-5Zr \cite{Tane2010a} & Ti-23Nb-4Ta-3Zr & 77 & 64\\
		Ti-35Nb-10Ta-5Zr \cite{Tane2010a} & Ti-25Nb-4Ta-4Zr & 80 & 65\\
		Ti-24Nb-4Zr-7.9Sn \cite{Mohammed2014} & Ti-15Nb-3Zr-4Sn & 65 & 54\\
		Ti-35Nb-2Ta-3Zr \cite{Mohammed2014} & Ti-23Nb-1Ta-2Zr & 69 & 61\\
		Ti-29Nb-11Ta-5Zr \cite{Mohammed2014} & Ti-20Nb-6Ta-2Zr & 74 & 60\\
		Ti-10Zr-5Ta-5Nb \cite{Mohammed2014} & Ti-6Zr-1Ta-3Nb & 64 & 52\\
		Ti-29Nb-13Ta-2Sn \cite{Mohammed2014} & Ti-20Nb-5Ta-1Sn & 66 & 62\\
		\hline
	\end{tabular}
	\label{Ch5-table:tixdatacomp}
\end{table}
\clearpage
%%%

\pagebreak
\begin{figure}[H]
	\centering
	\includegraphics[width=\textwidth]{Chapter-5/Figures/PBEvsPW91.png}
	\caption{Elastic stiffness coefficients of the bcc Ti-Ta binary system calculated with the PW91 and PBE exchange correction functions, respectively.}
	\label{Ch5-figure:PBEvsPW91}
\end{figure}

\pagebreak
\begin{figure}[H]
	\centering
	\includegraphics[width=\textwidth]{Chapter-5/Figures/Strain.png}
	\caption{Elastic stiffness coefficients for the bcc Ti-Mo binary system calculated with strains, $\pm$0.01, $\pm$0.013 and $\pm$0.07, respectively, showing comparable results.}
	\label{Ch5-figure:Strain}
\end{figure}

\pagebreak
\begin{figure}[H]
	\centering
	\includegraphics[width=\textwidth]{Chapter-5/Figures/tixc11.png}
	\caption{Calculated $\overline{C}_{11}$ values (circles) plotted with their errors as well as the linear combination of the pure element (red dashed line) and the present modeling (black dashed line) for five Ti-X binary systems (X = Mo, Nb, Ta, Sn, Zr). Ti-Mo, Ti-Nb, and Ti-Ta alloys are compared with previous calculations from Ikehata et al. \cite{Ikehata2004}.}
	\label{Ch5-figure:tixc11}
\end{figure}

\pagebreak
\begin{figure}[H]
	\centering
	\includegraphics[width=\textwidth]{Chapter-5/Figures/tixc12.png}
	\caption{Calculated $\overline{C}_{12}$ values (circles) plotted with their errors as well as the linear combination of the pure element (red dashed line) and the present modeling (black dashed line) for five Ti-X binary systems (X = Mo, Nb, Ta, Sn, Zr). Ti-Mo, Ti-Nb, and Ti-Ta alloys are compared with previous calculations from Ikehata et al. \cite{Ikehata2004}.}
	\label{Ch5-figure:tixc12}
\end{figure}

\pagebreak
\begin{figure}[H]
	\centering
	\includegraphics[width=\textwidth]{Chapter-5/Figures/tixc44.png}
	\caption{Calculated $\overline{C}_{44}$ values (circles) plotted with their errors as well as the linear combination of the pure element (red dashed line) and the present modeling (black dashed line) for five Ti-X binary systems (X = Mo, Nb, Ta, Sn, Zr). Ti-Mo, Ti-Nb, and Ti-Ta alloys are compared with previous calculations from Ikehata et al. \cite{Ikehata2004}.}
	\label{Ch5-figure:tixc44}
\end{figure}

\pagebreak
\begin{figure}[H]
	\centering
	\includegraphics[width=\textwidth]{Chapter-5/Figures/tixyoungs.png}
	\caption{Young's modulus $E$ of the Ti-X binary systems. The present calculations are plotted as the filled circles with the error bars. The dotted purple line is the Voigt upper Young's modulus bound, the gold dot dashed line is the lower Reuss Young's modulus bound and the black line is the Hill Young's modulus average. The experimental values \cite{Ikehata2004,Zhang2015,Boyer1994,Sung2015,Ozaki2004,Fedotov1985,Zhou2009a,Zhou2004a,Friak2012,Wu2010a} are also included for comparison. }
	\label{Ch5-figure:tixyoungs}
\end{figure}

\pagebreak
\begin{figure}[H]
	\centering
	\includegraphics{Chapter-5/Figures/emap.png}
	\caption{Young's moduli mapped as a function of composition from bcc Ti to bcc X (X=Mo, Nb, Sn, Ta, Zr).}
	\label{Ch5-figure:tixmap}
\end{figure}

\pagebreak
\begin{figure}[H]
	\centering
	\includegraphics{Chapter-5/Figures/tixc11-c12.png}
	\caption{Calculated $\overline{C}_{11}$-$\overline{C}_{12}$ values (circles) plotted with the present modeling (solid lines) for five Ti-X binary systems (X = Mo, Nb, Ta, Sn, Zr). The $\overline{C}_{11}$-$\overline{C}_{12}$ shows the stability of the bcc phase. When the $\overline{C}_{11}$-$\overline{C}_{12}$ value is negative the bcc phase is not stable in the corresponding composition ranges.}
	\label{Ch5-figure:tixc11-c12}
\end{figure}

\pagebreak
\begin{figure}[H]
	\centering
	\includegraphics[width=\textwidth]{Chapter-5/Figures/tixbulk.png}
	\caption{Bulk moduli $B$ of the Ti-X binary systems. The present calculations are plotted as the filled circles with the error bars. The dotted purple line is the Voigt upper bulk modulus bound, the gold dot dashed line is the lower Reuss bulk modulus bound and the black line is the bulk modulus from the Hill approach.}
	\label{Ch5-figure:tixbulk}
\end{figure}

\pagebreak
\begin{figure}[H]
	\centering
	\includegraphics[width=\textwidth]{Chapter-5/Figures/tixshear.png}
	\caption{Shear moduli $G$ of the Ti-X binary systems. The present calculations are plotted as the filled circles with the error bars. The dotted purple line is the Voigt upper shear modulus bound, the gold dot dashed line is the lower Reuss shear modulus bound and the black line is the shear modulus from the Hill approach.}
	\label{Ch5-figure:tixshear}
\end{figure}

\pagebreak
\begin{figure}[H]
	\centering
	\includegraphics{Chapter-5/Figures/edatabase.png}
	\caption{Young's moduli values of multicomponent bcc Ti alloys measured experimentally plotted against the predicted Young's moduli from the pure elements and binary interaction parameters with the black diagonal line showing the exact correlation between the experimental and calculated values. Error in the experiments and the bounds from Reuss and Voigt approximations are plotted as the vertical and horizontal error bars, respectively. The variance in the first-principles calculations from Eq.\ref{eq: averagec11}-Eq.\ref{eq: averagec44} was averaged and plotted as the grey region. More information on the alloys is in Table \ref{Ch5-table:tixdatacomp} \cite{Tane2010a,Geetha2009,Mohammed2014}}
	\label{Ch5-figure:tixdatabase}
\end{figure}
\chapter{Effects of alloying elements on the elastic properties of bcc ternary and higher ordered Ti-alloys}

\section{Introduction}

In order to develop a better understanding about the alloying effect on the elastic properties of Ti alloys, the present work is developing an elastic database for the Ti-Mo-Nb-Sn-Ta-Zr system. With the focus being on bcc Ti-alloys, the effects of alloying elements on the pure elements and Ti-X binary alloys in the bcc phase were calculated in chapter 5. After extrapolating to higher order systems, it was hypothesized that studying the effects of alloying on the elastic properties of ternary alloys would improve the database. The present work focuses on studying the elastic properties of the Ti-X-Y (X $\neq$Y = Mo, Nb, Ta, Sn, and Zr) ternary alloys in the bcc phase. The single crystal elastic stiffness constants (c$_{ij}$'s) and polycrystalline aggregate properties are predicted across the composition range using Density Functional Theory (DFT) at 0 $^\circ$K outlined in the methodology chapter. Based on the DFT results, the CALPHAD approach outlined in the methodology is used to evaluate ternary interaction parameters. The interaction parameters are then incorporated into the database and the database accuracy is again tested similarly to the testing in chapter 5. The completed database is used to map the elastic modulus as a function of composition.

\section{Modeling and Calculations}

\subsection{Calculation details}

To study the elastic properties of the ternary bcc Ti alloys in the Ti-Mo-Nb-Sn-Ta-Zr system, DFT-based first-principles calculations were employed using VASP (Vienna ab-initio simulation package) \cite{Kresse1996,Kresse1999}. Four kinds of calculations were performed for each ternary alloy Ti-X-Y, with the varying compositions of X$_{0.50}$Y$_{0.50}$ (16-atom supercell), Ti$_{0.33}$X$_{0.33}$Y$_{0.33}$ (36 atoms), Ti$_{0.50}$X$_{0.25}$Y$_{0.25}$ (32 atoms), Ti$_{0.74}$X$_{0.13}$Y$_{0.13}$ (64 atoms). The relaxation and use of SQS are discussed extensively in chapter 2. The SQS used in this chapter were generated by Jiang et al. \cite{Jiang2004,Jiang2009}. The projector augmented wave (PAW) method was used to describe the ion-electron interaction. Based on our previous work done in chapter 5 (Figure \ref{Ch5-figure:PBEvsPW91}), the X-C functional of the generalized gradient approximation depicted by Perdew, Burke, and Ernzerhof (PBE-GGA) \cite{Perdew1996a} was employed. An energy cutoff roughly 1.3 times higher than the default values among all elements (i.e., 310 eV) was used for all calculations. The Brillouin zone sampling was done using the $\Gamma$-centered Monkhorst-Pack scheme \cite{Monkhorst1976a}. The k-point grids used for the ternary SQS were 4x4x4 and the k-point grids used for the binary X$_{0.50}$Y$_{0.50}$ SQS structures were an automated k-point mesh generator in VASP with the length of the subdivision specified at 80. The elastic calculations were completed using a strain magnitude of $\pm$0.01 based on the study done in chapter 5 and the results seen in Figure \ref{Ch5-figure:Strain}.

\subsection{Modeling details}

The first-principles results were then used to model the ternary interaction parameters of the elastic stiffness constants. The modeling was completed by plotting the binary interpolation from the working database build in chapter 5. The plots started at a 50-50 mixture (X$_{0.50}$Y$_{0.50}$) of the alloying elements (X $\neq$ Y = Mo, Nb, Sn, Ta, and Zr) and plotted to pure Ti. The elastic stiffness constants of the pure elements and the binary interaction parameters from Table \ref{Ch5-table:tixelasip} were used to plot the binary interpolation. The differences between the ternary first-principles calculations and the binary interpolation were then used to obtain a single fitting parameter using the Mathematica code in appendix C. With the focus being Ti-rich alloys and wanting to follow the same modeling technique used on the binary alloys (chapter 5), the first-principles results with 70 at.\% Ti or higher were weighted heavier (x6, according to the authors' practices) than the other points for the fittings. The best fit was found and the ternary interaction parameters were incorporated into the database. The database was then used to predict the moduli values of the ternary and higher order alloys. 

\section{Results and discussion}

\subsection{Elastic calculation results}

The elastic stiffness coefficients $\overline{C}_{11}$, $\overline{C}_{12}$, and $\overline{C}_{44}$ are plotted in Figure \ref{Ch6-figure:tixyc11_1} to Figure \ref{Ch6-figure:tixyc44_2} for the Ti-X-Y alloys (X $\neq$ Y = Mo, Nb, Sn, Ta, Zr). The plots start from a 50-50 mixture of the alloying elements (X$_{0.50}$Y$_{0.50}$) to Ti. The calculations are plotted as circles, the interpolation from the pure elements and binary interaction parameters is plotted as a red dashed line. The difference between the calculations and binary interpolation was used to fit the ternary interaction parameters. The ternary fitting is plotted as a solid black line. The calculated elastic stiffness coefficients are listed in Table \ref{Ch6-table:tixyecijdata}.

The $\overline{C}_{11}$ values, for most of the Ti-X-Y systems, decrease from X$_{0.50}$Y$_{0.50}$ to Ti (see Figure \ref{Ch6-figure:tixyc11_1} and Figure \ref{Ch6-figure:tixyc11_2}). However, the Ti-Sn-Zr and Ti-Ta-Zr system differ. The $\overline{C}_{11}$ values, for the Ti-Sn-Zr (Figure \ref{Ch6-figure:tixyc11_2}d), first increase from Sn$_{0.50}$Zr$_{0.50}$ to 60 at. \% Ti and then decrease from 60 to 100 at. \% Ti. The $\overline{C}_{11}$ values, in the Ti-Ta-Zr system (Figure \ref{Ch6-figure:tixyc11_2}e), first increase from Ta$_{0.50}$Zr$_{0.50}$ to 35 at. \% Ti and then decrease from 35 to 100 at. \% Ti. The $\overline{C}_{12}$ results are plotted in Figure \ref{Ch6-figure:tixyc12_1} and Figure \ref{Ch6-figure:tixyc12_2}. The $\overline{C}_{12}$ values decrease from X$_{0.50}$Y$_{0.50}$ to Ti, for the Ti-Mo-Nb, Ti-Mo-Ta and Ti-Nb-Ta systems. The $\overline{C}_{12}$ values, for the Ti-Mo-Sn and Ti-Nb-Sn systems, first decrease from X$_{0.50}$Y$_{0.50}$ to 15 at. \% Ti then increase from 15 to 85 at. \% Ti and then decrease from 85 to 100 at. \% Ti. The Ti-Mo-Zr and Ti-Nb-Zr systems show a decrease in $\overline{C}_{12}$ value from X$_{0.50}$Y$_{0.50}$ to 60 at. \% Ti and then an increase from 60 to 100 at. \% Ti. The $\overline{C}_{12}$ values increase from X$_{0.50}$Y$_{0.50}$ to 70 at. \% Ti and then decrease from 70 to 100 at. \% Ti for the Ti-Sn-Ta system. For the Ti-Sn-Zr system, the values increase from 0 to 100 at. \% Ti and the $\overline{C}_{12}$ values, for the Ti-Ta-Zr system, first decrease from 0 to 70 at. \% Ti and then increase from 70 to 100 at. \% Ti. The $\overline{C}_{44}$ results are plotted in Figure \ref{Ch6-figure:tixyc44_1} and Figure \ref{Ch6-figure:tixyc44_2}. The $\overline{C}_{44}$ values decrease from X$_{0.50}$Y$_{0.50}$ to Ti, for the Ti-Mo-Sn and Ti-Ta-Zr systems. For the Ti-Mo-Zr and Ti-Mo-Ta systems, the $\overline{C}_{44}$ values first decrease from 0 to 80 at. \% Ti and then increase from 80 to 100 at. \% Ti. The $\overline{C}_{44}$ values, for the Ti-Mo-Nb and Ti-Nb-Ta systems, first decrease from X$_{0.50}$Y$_{0.50}$ to 65 at. \% Ti and then increase from 65 to 100 at. \% Ti. The $\overline{C}_{44}$ values first increase from 0 to 80 at. \% Ti and then decrease from 80 to 100 at. \% Ti for the Ti-Nb-Sn and Ti-Nb-Zr systems. For the Ti-Sn-Ta system (Figure \ref{Ch6-figure:tixyc44_2}c), the  values decrease from 0 until 20 at. \% Ti, then increase from 20 to 50 at. \% Ti and then decrease from 50 to 100 at. \% Ti. The $\overline{C}_{44}$ values, for the Ti-Sn-Zr system, first increase from X$_{0.50}$Y$_{0.50}$ to 60 at. \% Ti and then decrease from 60 to 100 at. \% Ti.

The trends in the ternary elastic stiffness coefficients can be summarized and explained by looking at the elastic stiffness calculations done on the pure elements and Ti-X (X = Mo, Nb, Sn, Ta, Zr) in chapter 5. The $\overline{C}_{11}$ and $\overline{C}_{12}$ of Mo, Nb and Ta are higher than Ti, while Sn and Zr are lower. The $\overline{C}_{44}$ for Mo and Ta are higher than Ti, while Nb, Sn, and Zr are lower. This can be explained because Mo, Nb and Ta are stable in the bcc structure at low temperatures while Ti, Sn and Zr are not, so the elastic stiffness coefficients are similar. The similarities of Mo, Nb and Ta are again noticed in the Ti-X data trends. The $\overline{C}_{11}$, $\overline{C}_{12}$, and $\overline{C}_{44}$ all follow the same trends for the Ti-Mo, Ti-Nb, and Ti-Ta systems with the $\overline{C}_{11}$ and $\overline{C}_{12}$ increasing in value from 100 to 0 at. \% Ti and the $\overline{C}_{44}$ decreasing and then increasing from 100 to 0 at. \% Ti. Based on this information, it is no surprise that the Ti-Mo-Nb, Ti-Mo-Ta and Ti-Nb-Ta systems have the same trends in their $c_{ij}$ data. When Ti-Mo, Ti-Nb, and Ti-Ta are alloyed with Sn in the ternary systems, they show similar trends for the most, not all, of the $c_{ij}$. The same is true for the Ti-Mo, Ti-Nb, and Ti-Ta systems alloyed with Zr.

Based on the discussion in the methodology, Born's criteria is used to look at the mechanical stability of the bcc phase. When $\overline{C}_{11}$-$\overline{C}_{12}$ becomes negative then the bcc phase loses mechanical stability and is thus plotted in Figure \ref{Ch6-figure:tixyc11-c12}. Based on the present results, the bcc phase loses mechanical stability in the Ti-Mo-Nb, Ti-Mo-Ta, Ti-Mo-Zr, Ti-Nb-Zr, Ti-Sn-Zr, and Ti-Ta-Zr systems when the Ti concentration is more than 90 at. \%, with the values being 91, 92, 95, 93, 91, and 94 at. \% Ti, respectively. The bcc phase loses mechanical stability at Ti concentrations above 87, 77, 89, and 80 at. \% Ti for the Ti-Mo-Sn, Ti-Nb-Sn, Ti-Nb-Ta and Ti-Sn-Ta systems, respectively. Close to where the bcc phase loses its mechanical stability, the Young's modulus is reduced and such compositions may be desirable low modulus bcc Ti alloys. As discussed above, Mo, Nb and Ta are strong $\beta$-stabilizers and thus the Ti-Mo-Nb, Ti-Mo-Ta, and Ti-Nb-Ta systems stabilize the bcc phase similarly. Also, discussed previously, Zr is known as a weak $\beta$-stabilizer alone but when alloyed with other elements it acts a strong $\beta$-stabilizer. However, when calculating the $\overline{C}_{11}$-$\overline{C}_{12}$ for the Ti-X binaries, it was observed that Zr stabilized the bcc phase at lower concentrations than the other elements. This is observed with these results with the Ti-Mo-Zr, Ti-Nb-Zr, Ti-Ta-Zr systems all stabilizing the bcc phase at high Ti concentrations (95, 93, and 94 at. \% respectively). Zr is even able to stabilize the Ti-Sn-Zr system at a high Ti concentration of 91 at. \% Ti. Sn is not stable in the bcc phase and is not a $\beta$-stabilizer. So, when alloyed with Sn, a higher concentration of other alloying elements is needed to stabilize the bcc phase. 

Figure \ref{Ch6-figure:tixyyoungs1} and Figure \ref{Ch6-figure:tixyyoungs2} plot the Young's moduli ($E$) calculations (circles) for each Ti-X-Y ternary system (X $\neq$ Y = Mo, Nb, Sn, Ta, Zr) starting from a 50-50 mixture of the two alloying elements (X$_{0.50}$Y$_{0.50}$) to Ti. The red dashed line is the average from the Hill approach interpolated from the binary interaction parameters shown in Table \ref{Ch6-table:tixyelasip}. The average from the Hill approach (black solid line), Voigt (purple dotted line), Reuss (gold dotted-dashed line) bounds are plotted using the binary and ternary interaction parameters. The Voigt and Reuss bounds vary more drastically when the bcc structure is unstable as opposed to when the bcc structure is stable. The average  from the Hill approach using the binary and ternary interaction parameters is what the database predicts because it has been shown to be a more accurate representation of the Young's modulus than the Voigt or Reuss approximations \cite{Yue2009,Chung1967}. Whenever possible experimentally determined Young's moduli \cite{Niinomi2012,Mohammed2014,Nozoe2007,Geetha2009} (data listed in Table \ref{Ch6-table:tixyelasdata}) are plotted for comparison. The difference/error between the previous results (both experimental and from calculations) and the present first-principles results are calculated by Eq. \ref{eq: error}.

The first-principles $E$, for the Ti-Mo-Nb (Figure \ref{Ch6-figure:tixyyoungs1}a) system are compared with experimentally obtained $E$ data from the review paper by Niinomi et al. \cite{Niinomi2012}. The experimental $E$ results were obtained using a nanoindenter after solution treatment. The experimental $E$ values are higher than the calculated values from the Hill approach (difference of 71 GPa or an error of 0.65 using Eq. \ref{eq: error}) and more closely match the Voigt bound (difference of 46 GPa). Niinomi \cite{Niinomi2012} pointed out that Young's moduli obtained from the microhardness testing are higher than the polycrystalline Young's moduli value, thus our calculation results should be close to the polycrystalline $E$ values. The present data show that the $E$ decreases in value from 0 to 100 at. \% Ti. 

In the literature, no bcc Ti-Mo-Sn (Figure \ref{Ch6-figure:tixyyoungs1}b) experimental $E$ results were found to be compared with the present work. The Voigt-Reuss bounds and the Hill average are quite similar until around 65 at. \% Ti when they begin to vary. The $E$ values decrease from 0 to 100 at. \% Ti. The calculated $E$ results for the Ti-Mo-Ta alloy system (Figure \ref{Ch6-figure:tixyyoungs2}c) are compared with experimental data reported by Niinomi et al. \cite{Niinomi2012} and Mohammed et al. \cite{Mohammed2014}. The $E$ values reported by Niinomi et al. \cite{Niinomi2012} were obtained using an ultrasonic measurement after the samples were solution treated. Niinomi et al. \cite{Niinomi2012} pointed out that $E$ values obtained using the ultrasonic method obtained normally fall in between the $E$ values determined using tensile testing or microhardness testing. Niinomi et al. \cite{Niinomi2012} also showed that when multiple authors test the same composition alloy using the ultrasonic technique their answers vary less drastically than when multiple authors test the same compositions using tensile or microhardness. The experimental $E$ fit well with the present Voigt bound (difference of 9 GPa) and had an error of 0.46 (Eq. \ref{eq: error}) from the Hill average (difference of 33 GPa). The $E$ values decrease from 0 to 100 at. \% Ti. 

The calculated $E$ of the Ti-Mo-Zr system (Figure \ref{Ch6-figure:tixyyoungs1}d) is compared with experimental values reported by Mohammed et al. \cite{Mohammed2014}. The error in the experiments and methodology was not discussed. The experimental $E$ \cite{Mohammed2014} and the present $E$ (Hill average) vary by less than 6 GPa and the $E$ values decrease from 0 to 100 at. \% Ti. Experimentally determined Young's moduli results from Niinomi et al. \cite{Niinomi2012}, Mohammed et al. \cite{Mohammed2014}, and Nozoe et al. \cite{Nozoe2007} are compared with the present $E$ calculations for the Ti-Nb-Sn system (Figure \ref{Ch6-figure:tixyyoungs1}e). The experimental $E$ results reported by Niinomi et al. \cite{Niinomi2012} were obtained through tensile testing of samples that were solution treated and cold rolled. Niinomi et al. \cite{Niinomi2012} showed that $E$ results at this specific composition varied by 10 GPa. The $E$ results reported by Niinomi et al. \cite{Niinomi2012} and Mohammed et al. \cite{Mohammed2014} differ from the Hill average by 9 and 15 GPa, respectively (an error of 0.28 using Eq. \ref{eq: error}). The $E$ results from Nozoe et al. \cite{Nozoe2007} differ by 10, 32 and 78 GPa from the Voigt, Hill and Reuss results, respectively, and have an error of 0.56 (Eq. \ref{eq: error}). However, Nozoe et al. reported that the samples formed the metastable $\omega$ phase. Nozoe et al. \cite{Nozoe2007} also discussed how the aging of the Ti-Mo-Sn samples greatly affected the $E$. The $E$ values, for the Ti-Nb-Sn system, increase from 0 to 35 at. \% Ti and then decrease from 35 to 100 at. \% Ti. So overall the calculations satisfactorily predicted the Young's moduli of the Ti-Mo-Nb, Ti-Mo-Sn, Ti-Mo-Ta, Ti-Mo-Zr and Ti-Nb-Sn systems. 

Figure \ref{Ch6-figure:tixyyoungs2} continues to plot the $E$ of the Ti-X-Y ternary alloy systems. The first-principles $E$, for the Ti-Nb-Ta system, are compared with experimentally determined $E$ values reported by Mohammed et al. \cite{Mohammed2014} (Figure \ref{Ch6-figure:tixyyoungs2}a). At this composition, the Voigt and Reuss bounds are very close to the Hill average. Thus, while the experimental $E$ result fits closer to the Reuss bound (difference of 8 GPa), the experimental $E$ result only varies by 15 and 21 GPa from the Hill and Voigt results, respectively and has an error of 0.28 (Eq. \ref{eq: error}). Niinomi et al. \cite{Niinomi2012} reported $E$ values for the Ti-Nb-Ta system. However, the experimental values at the same compositions varied by more than 40 GPa and thus his results were not plotted for comparison here. The $E$ decreases in value from 0 to 100 at. \% Ti in the Ti-Nb-Ta ternary system. 

The first-principles $E$ results for the Ti-Nb-Zr (Figure \ref{Ch6-figure:tixyyoungs2}b) system are compared with experimentally determined $E$ results from Geetha et al. \cite{Geetha2009},  Mohammed et al. \cite{Mohammed2014}, and Niinomi et al. \cite{Niinomi2012}. The experimental $E$ results reported by Niinomi et al. \cite{Niinomi2012} are for the Ti-13Nb-13Zr and Ti-27Nb-8Zr alloys. The results for the Ti-13Nb-13Zr were obtained using both the ultrasonic and 3-point bending tests and varied by 50 GPa, while the Ti-27Nb-8Zr results were obtained using tensile testing and varied by 50 GPa. The experimentally determined $E$ results varied from the present Voigt, Hill and Reuss results by an average of 9, 2, and 14 GPa, respectively and has an error of 0.08 (Eq. \ref{eq: error}) from the Hill average. The $E$ values first increase from 0 to 70 at. \% Ti and then decrease from 70 to 100 at. \% Ti, for the Ti-Nb-Zr system. The Ti-Sn-Ta, Ti-Sn-Zr and Ti-Ta-Zr alloy systems did not have experimental data to be compared with and our results are shown in Figure \ref{Ch6-figure:tixyyoungs2}c, Figure \ref{Ch6-figure:tixyyoungs2}d, and Figure \ref{Ch6-figure:tixyyoungs2}e, respectively. For the Ti-Sn-Ta system, the $E$ values decrease from 0 to 100 at. \% Ti. The $E$ values, for the Ti-Sn-Zr system, first increase from 0 to 60 at. \% Ti and then decrease from 60 to 100 at. \% Ti. The $E$ values for the Ti-Ta-Zr system first decrease from 0 to 15 at. \% Ti, where the values begin to increase from 15 to 30 at. \% Ti and then decrease from 30 to 100 at. \% Ti. 

The first-principles calculations and the CALPHAD fittings are done using data obtained at 0 $^\circ$K while the experiments are obtained using polycrystalline samples at 300 $^\circ$K. Considering this and the fact that the experimental measured $E$ values vary at the same composition, the experimental values fit well within the bounds set by Reuss and Voigt, and the Hill average generally reproduces the experimentally determined $E$ data for the Ti-X-Y ternary alloys. 

Using the complete database with interaction parameters listed in Table \ref{Ch5-table:tixelasip}, the elastic stiffness coefficients can be predicted and then the moduli values can be calculated and mapped. Figure \ref{Ch6-figure:tixymap1} and Figure \ref{Ch6-figure:tixymap2} uses the global minimization tools in pycalphad \cite{Otis2017} to map the $E$ based on composition for Ti-X-Y ternaries. The ternary maps all have regions in the Ti rich corner that are light blue indicating that the Young's moduli are low enough to be in the target $E$ range. With this fact and the fact that six of the Ti-X-Y ternaries all stabilize the bcc phase above 90 at. \% Ti, the database points to multiple composition ranges that would yield possible implant materials. The pycalphad code and completed database used are in appendix D and E.

The $B$ and $G$ moduli calculated (circles) are plotted in Figures \ref{Ch6-figure:tixybulk1}-\ref{Ch6-figure:tixyshear2} for each Ti-X-Y (X $\neq$ Y = Mo, Nb, Sn, Ta, Zr) system starting from a 50-50 mixture of X and Y (X$_{0.50}$Y$_{0.50}$) to Ti. The red dashed line is the Hill average interpolated from the binary interaction parameters shown in Table \ref{Ch6-table:tixyelasip}. The Hill average (black solid line), Voigt (purple dotted line), Reuss (gold dotted-dashed line) bounds are plotted using the binary and ternary interaction parameters and listed in Table \ref{Ch6-table:tixyelasdata}. The $B$ values, for all the Ti-X-Y ternaries except Ti-Nb-Sn, Ti-Sn-Ta and Ti-Sn-Zr decrease from 0 to 100 at. \% Ti, as shown in Figure \ref{Ch6-figure:tixybulk1} and Figure \ref{Ch6-figure:tixybulk2}. The $B$ values for the Ti-Nb-Sn and Ti-Sn-Ta systems first decrease from 0 to 10 at. \% Ti then increase from 10 to 55 at. \% Ti and then decrease from 55 to 100 at. \% Ti. For the Ti-Sn-Zr system, the $B$ values first increase from 0 to 85 at. \% Ti and then decrease from 85 to 100 at. \% Ti. The $G$ values decrease from 0 to 100 at. \% Ti for all the ternary systems except Ti-Nb-Sn, Ti-Nb-Zr, Ti-Sn-Zr and Ti-Ta-Zr (Figure \ref{Ch6-figure:tixyshear1} and Figure \ref{Ch6-figure:tixyshear2}). The $G$ values increase from 0 to 60 at. \% Ti and then decrease from 60 to 100 at. \% Ti for the Ti-Nb-Sn and Ti-Nb-Zr systems. For the Ti-Sn-Zr system, the $G$ values first increase from 0 to 55 at. \% Ti and then decrease from 55 to 100 at. \% Ti. The $G$ values increase from 0 to 40 at. \% Ti and then decrease from 40 to 100 at. \% Ti for the Ti-Ta-Zr system. 

Both the $G$ and $E$ are negative when they are close to Ti. This is due to the instability of the bcc phase close to Ti. As discussed by Born's criteria, when $\overline{C}_{11}$-$\overline{C}_{12}$ is negative the bcc phase loses mechanical stability. The Voigt bound of $G_{v}$ is expressed by:

%%
\begin{equation}
\label{eq: Gv}
G_v = \left( \overline{C}_{11}-\overline{C}_{12}+\overline{C}_{44} \right)/5
\end{equation}
%%

So, when $\overline{C}_{11}$-$\overline{C}_{12}$ is negative, it can cause $G$ to be negative. The $E$ is then calculated from the $B$ and $G$, so when $G$ is negative it can cause $E$ to be negative. This is one of the reasons that finding bcc Ti alloys close to the bcc stability will produce a low $E$. 

\subsection{Extrapolation to higher ordered systems}

The Young's moduli are predicted and compared with experimental results for higher order Ti alloys and the results are shown in Table \ref{Ch6-table:tixydatacomp} and Figure \ref{Ch6-figure:tixydatabase}. The same comparison was made in chapter 5, but those predictions were made without the ternary interaction parameters. Figure \ref{Ch6-figure:tixydatabase} plots the calculated Young's moduli (Hill average) versus the experimentally determined Young's moduli \cite{Tane2010a,Mohammed2014,Geetha2009}. The black diagonal line would be a perfect correlation between the predictions and experiments. The grey region is the average variance in the first-principles calculations when calculating the average elastic stiffness coefficients using Eq. \ref{eq: averagec11}-Eq. \ref{eq: averagec44} (3 GPa). The same higher order alloys were picked to compare the effect of introducing the ternary interaction parameters. As discussed previously, the error bars plotted for the experiments come from the variance that was seen when comparing the experimentally determined Young's moduli at the same composition from Niinomi et al. \cite{Niinomi2012}, Geetha et al. \cite{Geetha2009}, Tane et al. \cite{Tane2010a}, and Mohammed et al. \cite{Mohammed2014}. The horizontal error bars are the Voigt and Reuss bounds. Previously, without ternary interaction parameters, the predictions and experimental results varied anywhere between 0.69 and 14 GPa and on average by 7 GPa. The calculations are usually larger than the experimental values due to the temperature difference: the computed single crystal elastic stiffness coefficients are at 0 $^\circ$K, while the experiments on the polycrystalline samples are usually performed at 300 $^\circ$K. As expected, introducing ternary interaction parameters improves the database. The introduction of the ternary interaction parameters improved the predictions to vary anywhere from 0.39 to 13 GPa from the experimental values with an average variance of only 5 GPa. Thus, while the ternary interactions have small effects on the final results, the introduction of Ti-containing ternary interaction parameters still improves the predictions and the database is more accurate to predict the Young's moduli of higher order Ti alloys. 

\section{Conclusion}

The present study systematically calculated the elastic properties of the bcc Ti ternary alloys, including the elastic stiffness coefficients, bulk modulus, shear modulus, and Young's modulus. Five alloying elements, Mo, Nb, Sn, Ta and Zr were studied. The general CALPHAD modeling approach was used to fit ternary interaction parameters. From the elastic stiffness constant data, the Ti-X-Y (X $\neq$ Y = Mo, Nb, Ta) showed the same trends in the data. This is to be expected because Mo, Nb, and Ta are similar elements that are strong $\beta$-stabilizers and stable in the bcc phase at low temperatures. It was also seen that the Ti-X-Sn (X = Mo, Nb, Ta) alloys showed similar trends in the data for most of the elastic stiffness coefficients, so did the Ti-X-Zr (X = Mo, Nb, Ta) alloys. The present calculations showed that the bcc Ti-alloy was mechanically stabilized at compositions less than 95, 94, 93, 92, 91, 91, 89, 87, 80, and 77 at \% Ti for the Ti-Mo-Zr, Ti-Ta-Zr, Ti-Nb-Zr, Ti-Mo-Ta, Ti-Mo-Nb, Ti-Sn-Zr, Ti-Nb-Ta, Ti-Mo-Sn, Ti-Sn-Ta and Ti-Nb-Sn alloys, respectively. As discussed above, Mo, Nb and Ta are strong $\beta$-stabilizers and thus the Ti-Mo-Nb, Ti-Mo-Ta, and Ti-Nb-Ta systems stabilize the bcc phase similarly. Also, discussed previously, Zr is known as a weak $\beta$-stabilizer alone but when alloyed with other elements it acts a strong $\beta$-stabilizer. However, as discussed in chapter 5, Zr was able to stablize the bcc phase in the Ti-X systems at a lower concentration than the other elements. This was observed with the Ti-Mo-Zr, Ti-Nb-Zr, Ti-Ta-Zr systems all stabilizing the bcc phase at high Ti concentrations (95, 93, and 94 at.\% respectively). Zr was even able to stabilize the Ti-Sn-Zr system at a high Ti concentration of 91 at.\% Ti, even with Sn not being a $\beta$-stabilizer or stable in the bcc phase. 

The pure element elastic results along with the binary and ternary interaction parameters were combined into a database file in appendix D. Overall, the introduction of the ternary interaction parameters improved the database's ability to predict the $E$ of higher order alloys by a small amount. The complete database satisfactorily predicts the elastic properties of higher order Ti-alloys. The database was used to map possible alloy compositions to find potential materials with a Young's modulus in the target range for biomedical load-bearing implants using the pycalphd code in appendix E.  The database will help guide future research to develop low-modulus biocompatible Ti alloys.

\newpage
\begin{longtable}[H]{ c c c c c}
	\caption{First-principles calculations of the elastic stiffness constants for different atomic percent compositions of the bcc Ti-X-Y ternary systems at 0 $^{\circ}$K.} 	\label{Ch6-table:tixyecijdata} \\
	\hline
	Reference & Ti$_{1-bc}$X$_b$Y$_c$ & $\overline{C}_{11}$ & $\overline{C}_{11}$ & $\overline{C}_{11}$ \\
	\hline
	\endhead
	\hline
	\endfoot
	This work & Ti & 93 & 115 & 41 \\
	This work & Ti$_{0.74}$Mo$_{0.13}$Nb$_{0.13}$ & 155 & 121 $\pm$4 & 34 $\pm$4 \\
	This work & Ti$_{0.50}$Mo$_{0.25}$Nb$_{0.25}$ & 222 $\pm$3 & 129 $\pm$3 & 33 $\pm$3 \\
	This work & Ti$_{0.33}$Mo$_{0.33}$Nb$_{0.33}$ & 269 $\pm$5 & 134 $\pm$3 & 42 $\pm$4 \\
	This work & Mo$_{0.50}$Nb$_{0.50}$ & 414 $\pm$6 & 165 $\pm$3 & 68 \\
	This work & Ti$_{0.74}$Mo$_{0.13}$Sn$_{0.13}$ & 137 $\pm$15 & 121 $\pm$2 & 56 $\pm$13 \\
	This work & Ti$_{0.50}$Mo$_{0.25}$Sn$_{0.250}$ & 160 $\pm$3 & 130 $\pm$8 & 71 $\pm$2 \\
	This work & Ti$_{0.33}$Mo$_{0.33}$Sn$_{0.33}$ & 167 $\pm$8 & 133 $\pm$6 & 75 $\pm$2 \\
	This work & Mo$_{0.50}$Sn$_{0.50}$ & 192 $\pm$28 & 130 $\pm$36 & 40 $\pm$31 \\
	This work & Ti$_{0.74}$Mo$_{0.13}$Ta$_{0.13}$ & 153 $\pm$1 & 125 $\pm$4 & 38 $\pm$3 \\
	This work & Ti$_{0.50}$Mo$_{0.25}$Ta$_{0.25}$ & 222 $\pm$2 & 136 $\pm$1 & 45 $\pm$3 \\
	This work & Ti$_{0.33}$Mo$_{0.33}$Ta$_{0.33}$ & 263 $\pm$4 & 145 $\pm$6 & 49 $\pm$4 \\
	This work & Mo$_{0.50}$Ta$_{0.50}$ & 370 $\pm$13 & 163 $\pm$4 & 63 $\pm$4 \\
	This work & Ti$_{0.74}$Mo$_{0.13}$Zr$_{0.13}$ & 125 $\pm$1 & 109 $\pm$8 & 35 $\pm$1 \\
	This work & Ti$_{0.50}$Mo$_{0.25}$Zr$_{0.25}$ & 160 $\pm$1 & 116 $\pm$5 & 34 $\pm$2 \\
	This work & Ti$_{0.33}$Mo$_{0.33}$Zr$_{0.33}$ & 182 $\pm$1 & 116 $\pm$2 & 31 $\pm$8 \\
	This work & Mo$_{0.50}$Zr$_{0.50}$ & 231 $\pm$7 & 118 $\pm$5 & 33 $\pm$8 \\
	This work & Ti$_{0.74}$Nb$_{0.13}$Sn$_{0.13}$ & 115 $\pm$4 & 118 $\pm$3 & 55 \\
	This work & Ti$_{0.50}$Nb$_{0.25}$Sn$_{0.25}$ & 131 $\pm$9 & 121 $\pm$6 & 64 $\pm$3 \\
	This work & Ti$_{0.33}$Nb$_{0.33}$Sn$_{0.33}$ & 134 $\pm$2 & 122 $\pm$3 & 67 $\pm$6 \\
	This work & Nb$_{0.50}$Sn$_{0.50}$ & 132 $\pm$4 & 118 $\pm$8 & 56 $\pm$4 \\
	This work & Ti$_{0.74}$Nb$_{0.13}$Ta$_{0.13}$ & 130 $\pm$3 & 124 $\pm$4 & 37 $\pm$3 \\
	This work & Ti$_{0.50}$Nb$_{0.25}$Ta$_{0.25}$ & 182 $\pm$1 & 129 $\pm$4 & 43 $\pm$6 \\
	This work & Ti$_{0.33}$Nb$_{0.33}$Ta$_{0.33}$ & 208 & 135 $\pm$1 & 44 $\pm$1 \\
	This work & Nb$_{0.50}$Ta$_{0.50}$ & 260 $\pm$2 & 148 $\pm$3 & 47 $\pm$3 \\
	This work & Ti$_{0.74}$Nb$_{0.13}$Zr$_{0.13}$ & 101 $\pm$2 & 113 $\pm$4 & 32 $\pm$3\\
	This work & Ti$_{0.50}$Nb$_{0.25}$Zr$_{0.25}$ & 122 $\pm$1 & 113 $\pm$3 & 28 $\pm$3 \\
	This work & Ti$_{0.33}$Nb$_{0.33}$Zr$_{0.33}$ & 143 $\pm$2 & 107 $\pm$5 & 28 $\pm$3 \\
	This work & Nb$_{0.50}$Zr$_{0.50}$ & 154 $\pm$5 & 110 $\pm$3 & 15 $\pm$2 \\
	This work & Ti$_{0.74}$Sn$_{0.13}$Ta$_{0.13}$ & 115 $\pm$6 & 121 $\pm$4 & 60 $\pm$2 \\
	This work & Ti$_{0.50}$Sn$_{0.25}$Ta$_{0.25}$ & 138 $\pm$13 & 125 $\pm$4 & 75 $\pm$4 \\
	This work & Ti$_{0.33}$Sn$_{0.33}$Ta$_{0.33}$ & 138 $\pm$6 & 131 $\pm$8 & 78 $\pm$1 \\
	This work & Sn$_{0.50}$Ta$_{0.50}$ & 133 $\pm$8 & 130 $\pm$4 & 60 $\pm$4 \\
	This work & Ti$_{0.74}$Sn$_{0.13}$Zr$_{0.13}$ & 97 $\pm$5 & 111 $\pm$4 & 55 $\pm$2 \\
	This work & Ti$_{0.50}$Sn$_{0.25}$Zr$_{0.25}$ & 99 $\pm$12 & 103 $\pm$4 & 59 $\pm$7 \\
	This work & Ti$_{0.33}$Sn$_{0.33}$Zr$_{0.33}$ & 96 $\pm$7 & 98 $\pm$3 & 55 $\pm$3 \\
	This work & Sn$_{0.50}$Zr$_{0.50}$ & 85 $\pm$7 & 87 $\pm$9 & 42 $\pm$3 \\
	This work & Ti$_{0.74}$Ta$_{0.13}$Zr$_{0.13}$ & 136 $\pm$36 & 103 $\pm$21 & 44 $\pm$5 \\
	This work & Ti$_{0.50}$Ta$_{0.25}$Zr$_{0.25}$ & 130 $\pm$3 & 117 $\pm$4 & 42 $\pm$7 \\
	This work & Ti$_{0.33}$Ta$_{0.33}$Zr$_{0.33}$ & 148 $\pm$1 & 115 $\pm$2 & 44 $\pm$2 \\
	This work & Ta$_{0.50}$Zr$_{0.50}$ & 157 $\pm$2 & 123 $\pm$3 & 35 $\pm$3 \\
	\hline
\end{longtable}
%%%

\newpage
\begin{longtable}[H]{ c c c c c }
	\caption{First-principles calculations of the bulk modulus $B$, shear modulus $G$, and Young's modulus $E$ in GPa for different atomic percent compositions of the bcc Ti-X-Y ternary systems at 0 $^{\circ}$K as well as experimental data obtained for the Young's modulus at 300 $^{\circ}$K by the references stated.} 	\label{Ch6-table:tixyelasdata} \\
	\hline
	Reference & Ti$_{1-bc}$X$_b$Y$_b$ & $B$ &$G$ & $E$\\
	\hline
	\endhead
	\hline
	\endfoot
	This work & Ti & 108 & -12.91 & -40.34\\
	This work & Ti$_{0.74}$Mo$_{0.13}$Nb$_{0.13}$ & 132 $\pm$4 & 26 $\pm$4 & 73 $\pm$4\\
	This work & Ti$_{0.50}$Mo$_{0.25}$Nb$_{0.25}$ & 160 $\pm$3 & 38 $\pm$3 & 105 $\pm$3\\
	This work & Ti$_{0.33}$Mo$_{0.33}$Nb$_{0.33}$ & 179 $\pm$5 & 51 $\pm$5 & 139 $\pm$5\\
	This work & Mo$_{0.50}$Nb$_{0.50}$ & 248 $\pm$6 & 87 $\pm$6 & 233 $\pm$6\\
	Expt 300 K \cite{Niinomi2012} & Ti$_{0.92}$Mo$_{0.06}$Nb$_{0.02}$ & & & 110\\
	This work & Ti$_{0.74}$Mo$_{0.13}$Sn$_{0.13}$ & 126 $\pm$15 & 27 $\pm$15 & 75 +$\pm$15\\
	This work & Ti$_{0.50}$Mo$_{0.25}$Sn$_{0.25}$ & 140 $\pm$8 & 39 $\pm$8 & 106 $\pm$8\\
	This work & Ti$_{0.33}$Mo$_{0.33}$Sn$_{0.33}$ & 144 $\pm$8 & 42 $\pm$8 & 114 $\pm$8\\
	This work & Mo$_{0.50}$Sn$_{0.50}$ & 151 $\pm$36 & 36 $\pm$36 & 100 $\pm$36\\
	This work & Ti$_{0.74}$Mo$_{0.13}$Ta$_{0.13}$ & 134 $\pm$4 & 25 $\pm$4 & 72 $\pm$4\\
	This work & Ti$_{0.50}$Mo$_{0.25}$Ta$_{0.25}$ & 165 $\pm$2 & 44 $\pm$3 & 122 $\pm$3\\
	This work & Ti$_{0.33}$Mo$_{0.33}$Ta$_{0.33}$ & 184 $\pm$6 & 53 $\pm$6 & 145 $\pm$\\
	This work & Mo$_{0.50}$Ta$_{0.50}$ & 232 $\pm$13 & 77 $\pm$13 & 208 $\pm$13\\
	Expt 300 K \cite{Mohammed2014} & Ti$_{0.92}$Mo$_{0.07}$Ta$_{0.01}$ & & & 74\\
	Expt 300 K \cite{Niinomi2012}  & Ti$_{0.92}$Mo$_{0.07}$Ta$_{0.01}$ & & & 74\\
	This work & Ti$_{0.74}$Mo$_{0.13}$Zr$_{0.13}$ & 114 $\pm$8 & 20 $\pm$8 & 55 $\pm$8\\
	This work & Ti$_{0.50}$Mo$_{0.25}$Zr$_{0.25}$ & 131 $\pm$5 & 29 $\pm$5 & 80 $\pm$5\\
	This work & Ti$_{0.33}$Mo$_{0.33}$Zr$_{0.33}$ & 138 $\pm$2 & 32 $\pm$8 & 89 $\pm$8\\
	This work & Mo$_{0.50}$Zr$_{0.50}$ & 156 $\pm$7 & 41 $\pm$8 & 113 $\pm$8\\
	Expt 300 K \cite{Mohammed2014} & Ti$_{0.91}$Mo$_{0.07}$Zr$_{0.03}$ & & & 64\\
	This work & Ti$_{0.74}$Nb$_{0.13}$Sn$_{0.13}$ & 117 $\pm$4 & 14 $\pm$4 & 41 $\pm$4\\
	This work & Ti$_{0.50}$Nb$_{0.25}$Sn$_{0.25}$ & 124 $\pm$9 & 26 $\pm$9 & 72 $\pm$9\\
	This work & Ti$_{0.33}$Nb$_{0.33}$Sn$_{0.33}$ & 126 $\pm$6 & 28 $\pm$6 & 78 $\pm$6\\
	This work & Nb$_{0.50}$Sn$_{0.50}$ & 123 $\pm$8 & 26 $\pm$8 & 72 $\pm$8\\
	Expt 300 K \cite{Mohammed2014} & Ti$_{0.76}$Nb$_{0.22}$Sn$_{0.02}$ & & & 44\\
	Expt 300 K \cite{Niinomi2012} & Ti$_{0.76}$Nb$_{0.22}$Sn$_{0.02}$ & & & 50\\
	Expt 300 K \cite{Nozoe2007} & Ti$_{0.88}$Nb$_{0.09}$Sn$_{0.03}$ & & & 58\\
	This work & Ti$_{0.74}$Nb$_{0.13}$Ta$_{0.13}$ & 126 $\pm$4 & 15 $\pm$4 & 43 $\pm$4\\
	This work & Ti$_{0.50}$Nb$_{0.25}$Ta$_{0.25}$ & 147 $\pm$6 & 35 $\pm$6 & 98 $\pm$6\\
	This work & Ti$_{0.33}$Nb$_{0.33}$Ta$_{0.33}$ & 159 $\pm$1 & 41 $\pm$1 & 113 $\pm$1\\
	This work & Nb$_{0.50}$Ta$_{0.50}$ & 185 $\pm$3 & 50 $\pm$3 & 140 $\pm$3\\
	Expt 300 K \cite{Mohammed2014} & Ti$_{0.72}$Nb$_{0.10}$Ta$_{0.19}$ & & & 55\\
	This work & Ti$_{0.74}$Nb$_{0.13}$Zr$_{0.13}$ & 109 $\pm$4 & -2 $\pm$4 & -6 $\pm$4\\
	This work & Ti$_{0.50}$Nb$_{0.25}$Zr$_{0.25}$ & 116 $\pm$3 & 14 $\pm$3 & 40 $\pm$3\\
	This work & Ti$_{0.33}$Nb$_{0.33}$Zr$_{0.33}$ & 119 $\pm$5 & 23 $\pm$5 & 66 $\pm$5\\
	This work & Nb$_{0.50}$Zr$_{0.50}$ & 125 $\pm$5 & 17 $\pm$5 & 50 $\pm$5\\
	Expt 300 K \cite{Mohammed2014} & Ti$_{0.85}$Nb$_{0.08}$Zr$_{0.08}$ & & & 77\\
	Expt 300 K \cite{Mohammed2014} & Ti$_{0.86}$Nb$_{0.11}$Zr$_{0.03}$ & & & 50\\
	Expt 300 K \cite{Niinomi2012} & Ti$_{0.78}$Nb$_{0.17}$Zr$_{0.05}$ & & & 78\\
	Expt 300 K \cite{Geetha2009} & Ti$_{0.85}$Nb$_{0.08}$Zr$_{0.08}$ & & & 81\\
	This work & Ti$_{0.74}$Sn$_{0.13}$Ta$_{0.13}$ & 119 $\pm$6 & 13 $\pm$6 & 39 $\pm$6\\
	This work & Ti$_{0.50}$Sn$_{0.25}$Ta$_{0.25}$ & 129 $\pm$13 & 31 $\pm$13 & 86 $\pm$13\\
	This work & Ti$_{0.33}$Sn$_{0.33}$Ta$_{0.33}$ & 133 $\pm$8 & 28 $\pm$8 & 79 $\pm$8\\
	This work & Sn$_{0.50}$Ta$_{0.50}$ & 131 $\pm$8 & 20 $\pm$8 & 57 $\pm$8\\
	This work & Ti$_{0.74}$Sn$_{0.13}$Zr$_{0.13}$ & 106 $\pm$5 & 4 $\pm$5 & 13 $\pm$5\\
	This work & Ti$_{0.50}$Sn$_{0.25}$Zr$_{0.25}$ & 102 $\pm$12 & 15 $\pm$12 & 42 $\pm$12\\
	This work & Ti$_{0.33}$Sn$_{0.33}$Zr$_{0.33}$ & 97 $\pm$7 & 15 $\pm$7 & 43 $\pm$7\\
	This work & Sn$_{0.50}$Zr$_{0.50}$ & 86 $\pm$9 & 11 $\pm$9 & 32 $\pm$9\\
	This work & Ti$_{0.74}$Ta$_{0.13}$Zr$_{0.13}$ & 114 $\pm$21 & 30 $\pm$21 & 82 $\pm$21\\
	This work & Ti$_{0.50}$Ta$_{0.25}$Zr$_{0.25}$ & 121 $\pm$4 & 20 $\pm$7 & 58 $\pm$7\\
	This work & Ti$_{0.33}$Ta$_{0.33}$Zr$_{0.33}$ & 126 $\pm$2 & 30 $\pm$2 & 83 $\pm$2\\
	This work & Ta$_{0.50}$Zr$_{0.50}$ & 134 $\pm$3 & 26 $\pm$3 & 74 $\pm$3\\
	\hline
\end{longtable}
%%%

\newpage
\begin{table}[H]
	\caption{Evaluated interactions parameters ($L_2$, Eq. \ref{eq: error}) for the elastic stiffness coefficients of the Ti-containing ternary alloys.}
	\centering
	\begin{tabular}{ c c c c c }
		\hline
		Alloy & Interaction Parameter & $\overline{C}_{11}$ & $\overline{C}_{12}$ & $\overline{C}_{44}$\\
		\hline
		Ti-Mo-Nb & $L_2$ & -29.97 & 13.97 & 9.72\\
		Ti-Mo-Sn & $L_2$ & -83.85 & 31.80 & 74.73\\
		Ti-Mo-Ta & $L_2$ & -106.53 & -12.35 & 5.27\\
		Ti-Mo-Zr & $L_2$ & -245.27 & 50.43 & -44.96\\
		Ti-Nb-Sn & $L_2$ & -41.52 & 25.52 & 67.85\\
		Ti-Nb-Ta & $L_2$ & -93.77 & -15.80 & 4.25\\
		Ti-Nb-Zr & $L_2$ & -220.35 & 72.10 & -55.29\\
		Ti-Sn-Ta & $L_2$ & -95.39 & -10.94 & 67.85\\
		Ti-Sn-Zr & $L_2$ &  -155.34 & 68.86 & 3.85\\
		Ti-Ta-Zr & $L_2$ & -149.67 & -8.91 & -23.70\\		
		\hline
	\end{tabular}
	\label{Ch6-table:tixyelasip}
\end{table}
\clearpage
%%%


\newpage
\begin{table}[H]
	\caption{Predicted Young's moduli ($E$) (in GPa) of higher order alloys using the binary and ternary interaction parameters in the bcc phase compared to the predicted Young's moduli ($E_{BIN}$) using just the binary interaction parameters and the experimental values found with both the weight percent and atomic percent listed.}
	\centering
	\begin{tabular}{ c c c c c }
		\hline
		Alloy Name ($\%$wt) & at \% & Calc $E_{BIN}$ & Calc $E$ & Expt $E$\\
		\hline
		Ti-35Nb-7Zr-5Ta \cite{Geetha2009} & Ti-24Nb-5Zr-2Ta & 81 & 78 & 80\\
		Ti-29Nb-13Ta-4.6Zr \cite{Geetha2009}  & Ti-20Nb-5Ta-3Zr & 76 & 73 & 75\\
		Ti-29Nb-13Ta-6Sn \cite{Geetha2009} & Ti-21Nb-5Ta-3Sn & 68 & 68 & 74\\
		Ti-29Nb-13Ta-4.6Sn \cite{Geetha2009} & Ti-20Nb-5Ta-3Sn & 67 & 66 & 66\\
		Ti-29Nb-13Ta-4.5Zr \cite{Geetha2009} & Ti-20Nb-5Ta-3Zr & 76 & 73 & 65\\
		Ti-29Nb-13Ta-4.6Zr \cite{Tane2010a} & Ti-21Nb-5Ta-3Zr & 76 & 75 & 64\\
		Ti-30Nb-10Ta-5Zr \cite{Tane2010a} & Ti-23Nb-4Ta-3Zr & 77 & 74 & 64\\
		Ti-35Nb-10Ta-5Zr \cite{Tane2010a} & Ti-25Nb-4Ta-4Zr & 80 & 78 & 65\\
		Ti-24Nb-4Zr-7.9Sn \cite{Mohammed2014} & Ti-15Nb-3Zr-4Sn & 65 & 62 & 54\\
		Ti-35Nb-2Ta-3Zr \cite{Mohammed2014} & Ti-23Nb-1Ta-2Zr & 69 & 68 & 61\\
		Ti-29Nb-11Ta-5Zr \cite{Mohammed2014} & Ti-20Nb-6Ta-2Zr & 74 & 72 & 60\\
		Ti-10Zr-5Ta-5Nb \cite{Mohammed2014} & Ti-6Zr-1Ta-3Nb & 64 & 62 & 52\\
		Ti-29Nb-13Ta-2Sn \cite{Mohammed2014} & Ti-20Nb-5Ta-1Sn & 66 & 65 & 62\\
		\hline
	\end{tabular}
	\label{Ch6-table:tixydatacomp}
\end{table}
\clearpage
%%%

\pagebreak
\begin{figure}[H]
	\centering
	\includegraphics[width=\textwidth]{Chapter-6/Figures/tixyc111.png}
	\caption{Calculated $\overline{C}_{11}$ values (circles) plotted with their errors as well as the interpolation from the binary interaction parameters (red dashed line) and the ternary fitting (black dashed line) for five of the Ti-X-Y binary systems from a 50-50 mixture of the alloying elements X$_{0.5}$Y$_{0.5}$ to Ti (X $\neq$ Y = Mo, Nb, Ta, Sn, Zr).}
	\label{Ch6-figure:tixyc11_1}
\end{figure}

\pagebreak
\begin{figure}[H]
	\centering
	\includegraphics[width=\textwidth]{Chapter-6/Figures/tixyc112.png}
	\caption{Calculated $\overline{C}_{11}$ values (circles) plotted with their errors as well as the interpolation from the binary interaction parameters (red dashed line) and the ternary fitting (black dashed line) for five of the Ti-X-Y binary systems from a 50-50 mixture of the alloying elements X$_{0.5}$Y$_{0.5}$ to Ti (X $\neq$ Y = Mo, Nb, Ta, Sn, Zr).}
	\label{Ch6-figure:tixyc11_2}
\end{figure}


\pagebreak
\begin{figure}[H]
	\centering
	\includegraphics[width=\textwidth]{Chapter-6/Figures/tixyc121.png}
	\caption{Calculated $\overline{C}_{12}$ values (circles) plotted with their errors as well as the interpolation from the binary interaction parameters (red dashed line) and the ternary fitting (black dashed line) for five of the Ti-X-Y binary systems from a 50-50 mixture of the alloying elements X$_{0.5}$Y$_{0.5}$ to Ti (X $\neq$ Y = Mo, Nb, Ta, Sn, Zr).}
	\label{Ch6-figure:tixyc12_1}
\end{figure}

\pagebreak
\begin{figure}[H]
	\centering
	\includegraphics[width=\textwidth]{Chapter-6/Figures/tixyc122.png}
	\caption{Calculated $\overline{C}_{12}$ values (circles) plotted with their errors as well as the interpolation from the binary interaction parameters (red dashed line) and the ternary fitting (black dashed line) for five of the Ti-X-Y binary systems from a 50-50 mixture of the alloying elements X$_{0.5}$Y$_{0.5}$ to Ti (X $\neq$ Y = Mo, Nb, Ta, Sn, Zr).}
	\label{Ch6-figure:tixyc12_2}
\end{figure}

\pagebreak
\begin{figure}[H]
	\centering
	\includegraphics[width=\textwidth]{Chapter-6/Figures/tixyc441.png}
	\caption{Calculated $\overline{C}_{44}$ values (circles) plotted with their errors as well as the interpolation from the binary interaction parameters (red dashed line) and the ternary fitting (black dashed line) for five of the Ti-X-Y binary systems from a 50-50 mixture of the alloying elements X$_{0.5}$Y$_{0.5}$ to Ti (X $\neq$ Y = Mo, Nb, Ta, Sn, Zr).}
	\label{Ch6-figure:tixyc44_1}
\end{figure}

\pagebreak
\begin{figure}[H]
	\centering
	\includegraphics[width=\textwidth]{Chapter-6/Figures/tixyc442.png}
	\caption{Calculated $\overline{C}_{44}$ values (circles) plotted with their errors as well as the interpolation from the binary interaction parameters (red dashed line) and the ternary fitting (black dashed line) for five of the Ti-X-Y binary systems from a 50-50 mixture of the alloying elements X$_{0.5}$Y$_{0.5}$ to Ti (X $\neq$ Y = Mo, Nb, Ta, Sn, Zr).}
	\label{Ch6-figure:tixyc44_2}
\end{figure}

\pagebreak
\begin{figure}[H]
	\centering
	\includegraphics{Chapter-6/Figures/tixyc11-c12.png}
	\caption{Calculated $\overline{C}_{11}$-$\overline{C}_{12}$ values (circles) plotted with the present modeling (solid lines) for the Ti-X-Y ternary systems (X $\neq$ Y = Mo, Nb, Ta, Sn, Zr). The $\overline{C}_{11}$-$\overline{C}_{12}$ shows the stability of the bcc phase, when the value is negative the bcc phase is not stable in the corresponding composition ranges.}
	\label{Ch6-figure:tixyc11-c12}
\end{figure}

\pagebreak
\begin{figure}[H]
	\centering
	\includegraphics[width=\textwidth]{Chapter-6/Figures/tixyyoungs1.png}
	\caption{$E$ of five of the Ti-X-Y ternary systems are plotted from a 50-50 mixture of the alloying elements to Ti in the bcc phase. The present calculations are plotted as filled circles with the error bars. The red dotted line is the extrapolation for the pure elements and binary interaction parameters only. The dotted purple line is the Voigt upper $E$ bound, the gold dot dashed line is the lower Reuss $E$ bound and the black line is the $E$ from the Hill approach. Experimental values are included for comparison \cite{Niinomi2012,Mohammed2014,Nozoe2007,Geetha2009}.}
	\label{Ch6-figure:tixyyoungs1}
\end{figure}

\pagebreak
\begin{figure}[H]
	\centering
	\includegraphics[width=\textwidth]{Chapter-6/Figures/tixyyoungs2.png}
	\caption{$E$ of five of the Ti-X-Y ternary systems are plotted from a 50-50 mixture of the alloying elements to Ti in the bcc phase. The present calculations are plotted as filled circles with the error bars. The red dotted line is the extrapolation for the pure elements and binary interaction parameters only. The dotted purple line is the Voigt upper $E$ bound, the gold dot dashed line is the lower Reuss $E$ bound and the black line is the $E$ from the Hill approach. Experimental values are included for comparison \cite{Niinomi2012,Mohammed2014,Nozoe2007,Geetha2009}.}
	\label{Ch6-figure:tixyyoungs2}
\end{figure}

\pagebreak
\begin{figure}[H]
	\centering
	\includegraphics[width=\textwidth]{Chapter-6/Figures/tixymap1.png}
	\caption{The Young's modulus is mapped as a function of composition in GPa for the Ti-Mo-Nb, Ti-Mo-Sn, Ti-Mo-Ta, Ti-Mo-Zr and Ti-Nb-Sn alloy systems using pycalphad \cite{Otis2017}.}
	\label{Ch6-figure:tixymap1}
\end{figure}

\pagebreak
\begin{figure}[H]
	\centering
	\includegraphics[width=\textwidth]{Chapter-6/Figures/tixymap2.png}
	\caption{The Young's modulus is mapped as a function of composition in GPa for the Ti-Mo-Nb, Ti-Mo-Sn, Ti-Mo-Ta, Ti-Mo-Zr and Ti-Nb-Sn alloy systems using pycalphad \cite{Otis2017}.}
	\label{Ch6-figure:tixymap2}
\end{figure}

\pagebreak
\begin{figure}[H]
	\centering
	\includegraphics[width=\textwidth]{Chapter-6/Figures/tixybulk1.png}
	\caption{$B$ are plotted from a 50-50 mixture of alloying elements X$_{0.5}$Y$_{0.5}$ to Ti in the bcc phase of five of the Ti-X-Y ternary systems (X $\neq$ Y = Mo, Nb, Ta, Sn, Zr). The present calculations are plotted as the filled circles with error bars as well as the interpolation from the binary interaction parameters (red dashed line). The dotted purple line is the Voigt upper $B$ bound, the gold dashed line is the lower Reuss $B$ bound and the black line is the $B$ from the Hill approach.}
	\label{Ch6-figure:tixybulk1}
\end{figure}

\pagebreak
\begin{figure}[H]
	\centering
	\includegraphics[width=\textwidth]{Chapter-6/Figures/tixybulk2.png}
	\caption{$B$ are plotted from a 50-50 mixture of alloying elements X$_{0.5}$Y$_{0.5}$ to Ti in the bcc phase of five of the Ti-X-Y ternary systems (X $\neq$ Y = Mo, Nb, Ta, Sn, Zr). The present calculations are plotted as the filled circles with error bars as well as the interpolation from the binary interaction parameters (red dashed line). The dotted purple line is the Voigt upper $B$ bound, the gold dashed line is the lower Reuss $B$ bound and the black line is the $B$ from the Hill approach.}
	\label{Ch6-figure:tixybulk2}
\end{figure}

\pagebreak
\begin{figure}[H]
	\centering
	\includegraphics[width=\textwidth]{Chapter-6/Figures/tixyshear1.png}
	\caption{$G$ are plotted from a 50-50 mixture of alloying elements X$_{0.5}$Y$_{0.5}$ to Ti in the bcc phase of five of the Ti-X-Y ternary systems (X $\neq$ Y = Mo, Nb, Ta, Sn, Zr). The present calculations are plotted as the filled circles with error bars as well as the interpolation from the binary interaction parameters (red dashed line). The dotted purple line is the Voigt upper $G$ bound, the gold dashed line is the lower Reuss $G$ bound and the black line is the $G$ from the Hill approach.}
	\label{Ch6-figure:tixyshear1}
\end{figure}

\pagebreak
\begin{figure}[H]
	\centering
	\includegraphics[width=\textwidth]{Chapter-6/Figures/tixyshear2.png}
	\caption{$G$ are plotted from a 50-50 mixture of alloying elements X$_{0.5}$Y$_{0.5}$ to Ti in the bcc phase of five of the Ti-X-Y ternary systems (X $\neq$ Y = Mo, Nb, Ta, Sn, Zr). The present calculations are plotted as the filled circles with error bars as well as the interpolation from the binary interaction parameters (red dashed line). The dotted purple line is the Voigt upper $G$ bound, the gold dashed line is the lower Reuss $G$ bound and the black line is the $G$ from the Hill approach.}
	\label{Ch6-figure:tixyshear2}
\end{figure}

\pagebreak
\begin{figure}[H]
	\centering
	\includegraphics{Chapter-6/Figures/tixydatabase.png}
	\caption{$E$ of multicomponent bcc Ti alloys predicted from the database are compared with measured experimental results. The vertical and horizontal error bars plotted are from the variation in experimentally determined Young's moduli values and the variance in the Voigt-Reuss bounds from the calculations for the specific multi-component alloy, respectively. The grey region refers to the error in the first-principles calculations. More information on the alloys is Table \ref{Ch6-table:tixydatacomp} \cite{Mohammed2014,Geetha2009,Tane2010a}.}
	\label{Ch6-figure:tixydatabase}
\end{figure}
\chapter{Phase stability and elastic properties study of the Ti-Ta and Ti-Nb systems}

\section{Introduction}

PUT IN AN INTRODUCTION ON THE PREVIOUS STUDIES OF THE PARTITION FUNCTION APPROACH

The present chapter is aimed at studying the formation of the metastable phases $\omega$ and $\alpha"$. The phase stability of bcc, hcp, $omega$, $alpha"$ was calculated for the Ti-Ta and Ti-Nb alloys. Multiple structures were calculated across the entire Ti-Ta and Ti-Nb systems at 0 $^\circ$K. The elastic properties of the four phases were then calculated systematically. CALPHAD modeling was completed to be able to predict the elastic properties as a function of composition. The partition function approach was then used to calculate the phase fraction formed. The predicted phase fractions were used to calculate the mixed force constants and phonon density of states as well as the elastic properites using the rule of mixtures. Inelastic neutron scattering experiments were completed to compare to the predicted phase fractions and phonon density of states. The elastic predictions were compared to experimental data in the literature. 

\section{Modeling and Calculations}

\subsection{Computational details}

In the present work the Vienna ab-initio Simulation Package (VASP) \cite{Kresse1996} was employed to calculate the ground state energy and elastic properties of the pure elements and Ti-Nb and Ti-Ta systems in the bcc, hcp, $\omega$, and $\alpha"$ phases. The ion-electron interactions were described using the projector augmented wave (PAW) \cite{Kresse1999,Blochl1994} method and based on the previous work of comparing X-C functionals (Figure \ref{Ch2-figure:PBEvsPW91}) the exchange-correlation functional of the generalized gradient approach depicted by Perdew, Burke, and Ernzerhof (PBE-GGA) was employed \cite{Perdew1996a}. The energy convergence criterion was 10$^{-6}$ eV/atomThe Brillouin zone sampling was done using the $\gamma$-centered Monkhorst-Pack scheme \cite{Monkhorst1976a}. The ground state energy of 330 structures in the bcc phase, across the entire compostition range, were calculated using 8x8x8 k-point meshes. The ground state energy of 21 structures in the hcp phase, across the entire compostition range, were calculated using 10x10x13 k-point meshes. The ground state energy of 73 structures in the $\omega$ phase, across the entire compostition range, were calculated using 13x13x7 k-point meshes. The ground state energy of 33 structures in the $\alpha"$ phase, across the entire compostition range, were calculated using 12x11x10 k-point meshes.The elastic properties were then calculated using a $\pm$0.01 magntiude of strain.

\subsection{Modeling details}

The elastic stiffness constants were modeled using the first-principles based DFT results. The modeling was completed by calculating the difference between the first-principles calculations and a linear extrapolation between pure elements. The differences were then used to fit to the interaction parameters. Due to the limitations within the PARROT module, a mathmatica script was used to fit the interaction parameters. The mathematica script is appended in appendix C. The same modeling procedure used for the bcc phase, in Chapter 5 and 6, was used in the prsent work. The first-principles results with 70 at. \% Ti or higher were weighted heavier (x6, according to the authors' practices) than the other points for the fittings. The best fit was found by comparing the fittings obtained with one interaction parameter or with two interaction parameters. The moduli values were than calculated using pycalphad and the code in appendix D and E \cite{Otis2017}.

\section{Results and discussion}

\subsection{First-principles calculations at 0 K}

The phase stability is calculated as a function of composition for the Ti-Nb and Ti-Ta systems. Figure \ref{Ch7-figure:tinb0K} shows the relative energy of the bcc, hcp, $\omega$, and $\alpha"$ phases from 100 at. \% Ti to 100 at. \% Nb. To calculate the relative energy, the ground state energies of the pure elements in the SER state are multipled by the composition of the specific structure and then stubtracted from the ground state energy of the structure being studied. Figure \ref{Ch7-figure:titab0K} is the relative energy of the bcc, hcp, $\omega$, and $\alpha"$ phases from 100 at. \% Ti to 100 at. \% Ta. Figure \ref{Ch7-figure:tinb0K} and \ref{Ch7-figure:titab0K} are both at 0 $^\circ$K. The figures show that the bcc and hcp phases are the lowest phases in energy. This shows that the $\omega$ and $\alpha"$ phases are stabilized by entropy.

The Ti-Nb system is chosen to study more in depth due to the experimental work avaliable in the literature which mapped the martensitic transformation temperature for Ti-Nb alloys between the 20 and 30 at. \% Nb.

\subsection{Elastic properties}

For the Ti-Nb system, the elastic properties are calculated as a function of composition and are plotted in Figure \ref{Ch7-figure:tinbelastic}. The calculations are shown as symbols and the results are listed in Table \ref{Ch7-figure:tinbelastic}. The dotted lines are the fittings that use the interaction parameters in Table \ref{Ch7-table:intpara}. The calculations show that the hcp elastic properites go from being positive at 100 at. \% Ti to negative at 100 at. \% Nb and vice versa for the bcc phase. A negative Young's modulus can indicate that the phase is not stable at that composition. From Figure \ref{Ch7-figure:tinbelastic}, it can be seen that the Young's moduli of the $\omega$ and $\alpha"$ phases are higher than the Young's moduli of the bcc phase for all the compositions. The Young's moduli of the $\omega$ and $\alpha"$ phases are higher than the Young's moduli of the hcp phase at almost all compositions. This would explain why in Figure REF the experimental Young's moduli increase in value with the formation of the metastable. 

\subsection{Neutron scattering results}

\subsubsection{Phonon density of states at 300 K}

\subsubsection{Diffraction patterns at 300 K}

\subsubsection{Temperature dependent phonon density of states }

\subsubsection{Temperature dependent diffraction patterns}

\subsection{Partition function approach results}

\subsubsection{Phase fractions}

\subsubsection{Mixed force constants}

\subsection{Comparison of elastic properties}

\section{Conclusion}



\newpage
\begin{longtable}[H]{ c c c c c c c c c c }
	\caption{First-principles calculations of the elastic stiffness constants in GPa for different atomic percent compositions in the $\alpha"$, bcc, hcp, and $\omega$  phases in the Ti-Nb system at 0 $^\circ$K.} 	\label{Ch7-table:tinbdata} \\
	\hline
	Ti$_{1-b}$Nb$_b$ & c$_{11}$ & c$_{12}$ & c$_{13}$ & c$_{22}$ & c$_{23}$ & c$_{33}$ & c$_{44}$ & c$_{55}$ & c$_{66}$\\
	\hline
	\endhead
	\hline
	\endfoot
	\multicolumn{10}{c}{$\alpha"$}\\
	\hline
	Ti & 198 & 69 & 84 & 197 & 84 & 189 & 40 & 40 & 63 \\		
	TiNb$_{2}$ & A & B & C & D & E & F & G & H & I \\
	TiNb$_{3}$ & 106 & 112 & 123 & 152 & 45 & 138 & 25 & 17 & 38 \\
	TiNb$_{13}$ & A & B & C & D & E & F & G & H & I \\
	TiNb$_{94}$ & 307 & 94 & 119 & 248 & 143 & 214 & 31 & -24 & 13 \\
	TiNb$_{97}$ & 293 & 88 & 115 & 232 & 124 & 284 & 59 & -58 & 8 \\
	TiNb$_{98}$ & A & B & C & D & E & F & G & H & I \\
	Nb & 306 & 88 & 125 & 240 & 135 & 284 & 47 & -69 & 9 \\
	\hline
	\multicolumn{10}{c}{bcc}\\
	\hline
	Ti & 93 & 115 & - & - & - & - & 41 & - & - \\		
	TiNb$_{2}$ & 93 & 115 & - & - & - & - & 35 & - & - \\
	TiNb$_{13}$ & 116 & 116 & - & - & - & - & 37 & - & - \\
	TiNb$_{25}$ & 140 & 116 & - & - & - & - & 34 & - & - \\
	TiNb$_{50}$ & 181 & 121 & - & - & - & - & 31 & - & - \\
	TiNb$_{75}$ & 208 & 130 & - & - & - & - & 15 & - & - \\
	TiNb$_{94}$ & 242 & 134 & - & - & - & - & 18 & - & - \\
	TiNb$_{98}$ & 242 & 134 & - & - & - & - & 18 & - & - \\
	Nb & 245 & 144 & - & - & - & - & 27 & - & - \\
	\hline
	\multicolumn{10}{c}{hcp}\\
	\hline
	Ti & 175 & 88 & 80 & - & - & 190 & 41 & - & - \\		
	TiNb$_{2}$ & A & B & C & - & - & D & E & - & - \\
	TiNb$_{13}$ & A & B & C & - & - & D & E & - & - \\
	TiNb$_{25}$ & A & B & C & - & - & D & E & - & - \\
	TiNb$_{50}$ & A & B & C & - & - & D & E & - & - \\
	TiNb$_{75}$ & A & B & C & - & - & D & E & - & - \\
	TiNb$_{94}$ & A & B & C & - & - & D & E & - & - \\
	TiNb$_{98}$ & A & B & C & - & - & D & E & - & - \\
	Nb & 24 & 18 & 11 & - & - & 25 & -6 & - & - \\
	\hline
	\multicolumn{10}{c}{$\omega$}\\
	Ti & 194 & 87 & 61 & - & - & 246 & 54 & - & - \\
	TiNb$_{2}$ & 187 & B & C & - & - & D & E & - & - \\
	TiNb$_{13}$ & A & B & C & - & - & D & E & - & - \\
	TiNb$_{94}$ & A & B & C & - & - & D & E & - & - \\
	TiNb$_{98}$ & A & B & C & - & - & D & E & - & - \\
	Nb & 243 & 181 & 110 & - & - & 212 & -55 & - & - \\
	\hline
\end{longtable}
%%%

\newpage
\begin{table}[H]
	\caption{Evaluated interaction parameters $L_0$ and $L_1$, using Eq. \ref{eq: elastic}, for the elastic stiffness constants of the bcc, hcp, $\alpha"$ and $\omega$ phases in the Ti-Nb systems.}
	\centering
	\begin{tabular}{ c c c c c c }
		\hline
		Alloy & Interaction Parameter & $\alpha"$ & bcc & hcp & $\omega$\\
		\hline
		c$_{11}$ & $L_{0}$ & A & B & C & D \\
		& $L_{1}$ & A & B & C & D \\
		c$_{12}$ & $L_{0}$ & A & B & C & D \\
		& $L_{1}$ & A & B & C & D \\
		c$_{13}$ & $L_{0}$ & A & N/A & C & D \\
		& $L_{1}$ & A & N/A & C & D \\
		c$_{22}$ & $L_{0}$ & A & N/A & N/A & N/A \\
		& $L_{1}$ & A & N/A & N/A & N/A \\
		c$_{23}$ & $L_{0}$ & A & N/A & N/A & N/A \\
		& $L_{1}$ & A & N/A & N/A & N/A \\
		c$_{33}$ & $L_{0}$ & A & N/A & C & D \\
		& $L_{1}$ & A & N/A & C & D \\
		c$_{44}$ & $L_{0}$ & A & B & C & D \\
		& $L_{1}$ & A & B & C & D \\
		c$_{55}$ & $L_{0}$ & A & N/A & N/A & N/A \\
		& $L_{1}$ & A & N/A & N/A & N/A \\
		c$_{66}$ & $L_{0}$ & A & N/A & N/A & N/A \\
		& $L_{1}$ & A & N/A & N/A & N/A \\
		\hline
	\end{tabular}
	\label{Ch7-table:intpara}
\end{table}
\clearpage
%%%



\pagebreak
\begin{figure}[H]
	\centering
	\includegraphics[width=\textwidth]{Chapter-7/Figures/tinb0k.png}
	\caption{The relative energy of the bcc, hcp, $\omega$, $\alpha"$ phases in the Ti-Nb system are plotted from 100 at. \% Ti to 100 at. \% Nb.}
	\label{Ch7-figure:tinb0K}
\end{figure}

\pagebreak
\begin{figure}[H]
	\centering
	\includegraphics[width=\textwidth]{Chapter-7/Figures/tita0k.png}
	\caption{The relative energy of the bcc, hcp, $\omega$, $\alpha"$ phases in the Ti-Ta system are plotted from 100 at. \% Ti to 100 at. \% Ta.}
	\label{Ch7-figure:titab0K}
\end{figure}

\pagebreak
\begin{figure}[H]
	\centering
	\includegraphics[width=\textwidth]{Chapter-7/Figures/tinbelastic.png}
	\caption{The elastic properites of the bcc, hcp, $\omega$, $\alpha"$ phases in the Ti-Nb system calculated from first-principles based on DFT are plotted as symbols. The CALPHAD fitting are plotted as the dashed lines. The figure is plotted from 100 at. \% Ti to 100 at. \% Nb.}
	\label{Ch7-figure:tinbelastic}
\end{figure}


\chapter{Conclusions and Future Work}

\section{Conclusions}

In this dissertation, the effect of alloying elements on Ti-based alloys are systematically studied. The work begins by using first-principles based DFT calculations and the CALPHAD method to study the effect that the alloying elements Mo, Nb, Sn, Ta and Zr have on the equilibrium phase stability, thermodynamics, and elastic properites. The work uses the equation of states fitting of the energy vs. volume curves to get the ground state equilibrium properties. The Debye-Gr\"uneisen model and phonon quasiharmonic approach are used to study the effect of temperature on the phase stability. A new theoretic framework is proposed to study the formation of the metastable phases. The accuracy of the theoretic framework and the transformation that occurs when these metastable phases form is studied using neutron scattering experiments. The compilation of the work develops a knowledge base for Ti-based alloys and will help to guide the future design of biocompatible implants. The main conclusions from this work are included below:

\begin{enumerate}
	\item The thermodynamic descriptions were all incorporated into a complete database that accurately predicts the phase stability of the Ti-Mo-Nb-Sn-Ta-Zr systems. The thermodynamic descriptions of the pure elements are adopted from the SGTE database. All of the binary systems had previous thermodynamic descriptions available in literature except the Mo-Sn and Ta-Sn systems. All of the binary systems had previous thermodynamic descriptions available in literature except Mo-Sn and Ta-Sn. A previous model was evaluated for accuarcy when avaliable for the binary systems. The Sn-Ta system was modeled in chapter 4. After evaluation the thermodynamic descriptions were incorporated into the present database. The binary interpolations of the Ti-containing ternary systems were plotted and compared with the available experimental data as well as the enthalpy of formation of the bcc phase calculated from first-principles based on DFT. The Ti-Sn-X systems (X = Mo, Nb, Ta, Zr) will be modeled in the future work with a thermodynamci description of the Mo-Sn system. The binary interpolations of the Ti-Nb-Zr and Ti-Ta-Zr systems had previously been plotted but no interaction parameters had been introduced. The present evaluation agreed with the previous evaluations and no ternary interaction parameters were introduced. The Ti-Mo-Zr system had previously been modeled and the present work agreed with the evaluation. The Ti-Mo-Nb, Ti-Mo-Ta and Ti-Nb-Ta systems had never previously been modeled. The present work evaluated interaction parameters for the Ti-Mo-Ta and Ti-Nb-Ta systems but didn't introduce any interaction parameters for the Ti-Mo-Nb system. The completed database is in appendix b. 
	\item Sn-Ta modeling was completed using data from DFT-based first-principles calculations and the available experimental data in the literature to model the Gibbs energies for the bcc and liquid solution phases and the stoichiometric Ta$_3$Sn and TaSn$_2$ phases of the Sn-Ta system. First-principles calculations were used to predict the enthalpy of formation of the bcc phase for the evaluation of interaction parameters in the phase. The decomposition temperature of Ta$_3$Sn was predicted to be 2884 $^\circ$K. The completed thermodynamic description was complied into a tdb file in appendix b.
	\item The effects of five alloying elements on the elastic properties of bcc Ti-X (X = Mo, Nb, Sn, Ta, Zr) alloys, including the elastic stiffness constants, bulk modulus, shear modulus, and Young's modulus, were systematically studied using first-principles calculations. The CALPAHD methodology was used to evaluate interaction parameters to predict the elastic properties as a function of composition. The calculations showed that 5.5, 11.5, 51.5, 9.5, and 4.0 at. \% of Mo, Nb, Sn, Ta and Zr, respectively, were required to stabilize the bcc phase according to the Born criteria. The trends observed were summarized for each Ti-X (X= Mo, Nb, Sn, Ta, Zr) binary system. Alloying with Mo, Nb, and Ta results in similar trends, which is probably because Mo, Nb, and Ta are strong bcc stabilizers and stable in the bcc structure at room temperature. The interaction parameters determined in the current work were used to predict the elastic properties of higher order alloys. The accuracy of database predictions of the Young’s modulus was evaluated by comparing the calculated and experimental Young's moduli. Overall, the database provides good predictions of the elastic properties of Ti-alloys in the bcc phase as a function of composition.
	\item The elastic properties of the bcc Ti-X-Y ternary alloys (X $\neq$ Y = Mo, Nb, Sn, Ta, Zr), including the elastic stiffness constants, bulk modulus, shear modulus, and Young's modulus. The general CALPHAD modeling approach was used to fit ternary interaction parameters. From the elastic stiffness constant data, the Ti-X-Y (X $\neq$ Y = Mo, Nb, Ta) show the same trends in the data. This is to be expected because Mo, Nb, and Ta are similar elements that are strong $\beta$-stabilizers and stable in the bcc phase at low temperatures. It was also seen that the Ti-X-Sn (X = Mo, Nb, Ta) alloys showed similar trends in the data for most of the elastic stiffness constants, so do the Ti-X-Zr (X = Mo, Nb, Ta) alloys. The present calculations showed that the bcc Ti-alloy was mechanically stabilized at compositions less than 91, 92, 95, 93, 91, 94, 87, 77, 89, and 80 at \% Ti for the Ti-Mo-Nb, Ti-Mo-Ta, Ti-Mo-Zr, Ti-Nb-Zr, Ti-Sn-Zr, Ti-Ta-Zr, Ti-Mo-Sn, Ti-Nb-Sn, Ti-Nb-Ta and Ti-Sn-Ta alloys, respectively. As discussed above, Mo, Nb and Ta are strong $\beta$-stabilizers and thus the Ti-Mo-Nb, Ti-Mo-Ta, and Ti-Nb-Ta systems stabilize the bcc phase similarly. Also, discussed previously, Zr is a weak $\beta$-stabilizer alone but when alloyed with other elements it acts a strong $\beta$-stabilizer. This is observed with these results with the Ti-Mo-Zr, Ti-Nb-Zr, Ti-Ta-Zr systems all stabilizing the bcc phase at high Ti concentrations (95, 93, and 94 at.\% respectively). Zr is even able to stabilize the Ti-Sn-Zr system at a high Ti concentration of 91 at.\% Ti, even with Sn. Sn is not stable in the bcc phase and is not a $\beta$-stabilizer. So, when alloyed with Sn, a higher concentration of other alloying elements is needed to stabilize the bcc phase. The ternary interaction parameters were combined with the previously determined pure elements and binary interaction parameters to map some of the possible alloy compositions to find potential materials with a Young's modulus in the target range for biomedical load-bearing implants. Overall, the introduction of the ternary interaction parameters improved the database's ability to predict the $E$ of higher order alloys by a small amount. The complete database, however, satisfactorily predicts the elastic properties of higher order Ti-alloys.
	\item PUT IN CONCLUSIONS FROM CHAPTER 7
\end{enumerate}

\section{Future Work}

The following are presented as future work to be able to improve this thesis work:
\begin{enumerate}
	\item Focusing on the modeling of the Sn binary and Ti-Sn-X (X = Mo, Nb, Ta, Zr) and incorporating them into this database would further improve the knowledge base of Ti-alloys that this thesis presents. As discussed, Sn is being added due to its low cost and the fact that at low concentrations it doesn not effect the alloys biocompatiblity. But even at low compositions having a better understanding of how Sn effects the phase stability would be helpful.
	\item As discussed, introducting interaction parameters to describe the elastic properites for the non Ti-containing binary and ternary systems would improve the databases accuracy
	\item The work on using the partition function approach will be continued. The next steps will be to calculate the helmholtz energies of all of the pure elements and to extend this to calculate better helmholtz eneriges for the metastable  and unstable phases. The ability of the partition function approach to calculate more accurate entropies would be a great advancement in the field. 
%%%%%%%%%%%%%%%%%%%%%%%%%%%%%%%%%%%%%%%%%%%%%%%%%%%%%%%%%%%%%%%
% Appendices
%
% Because of a quirk in LaTeX (see p. 48 of The LaTeX
% Companion, 2e), you cannot use \include along with
% \addtocontents if you want things to appear the proper
% sequence.
%%%%%%%%%%%%%%%%%%%%%%%%%%%%%%%%%%%%%%%%%%%%%%%%%%%%%%%%%%%%%%%
\appendix
\titleformat{\chapter}[display]{\fontsize{30}{30}\selectfont\bfseries\sffamily}{Appendix \thechapter\textcolor{gray75}{\raisebox{3pt}{|}}}{0pt}{}{}
% If you have a single appendix, then to prevent LaTeX from
% calling it ``Appendix A'', you should uncomment the following two
% lines that redefine the \thechapter and \thesection:
%\renewcommand\thechapter{}
%\renewcommand\thesection{\arabic{section}}
\Appendix{Ti-Mo-Nb-Ta-Zr Database}

\begin{table}[H]
	\centering
	\begin{tabular}{ l l l }
	\hline
	\multicolumn{3}{l}{\$*************************************************************}\\
	\multicolumn{3}{l}{\$ The definition of the pure elements, vacancy and species}\\
	\multicolumn{3}{l}{\$-----------------------------------------------------------------------------------------------}\\
	\multicolumn{3}{l}{TEMPERATURE\_LIMIT 0 6000.00 !}\\
	ELEMENT /- &ELECTRON\_GAS & 0.0000E+00  0.0000E+00  0.0000E+00!\\
	ELEMENT MO & BCC\_A2 & 95.94       4589.0      28.56 !\\
	ELEMENT NB & BCC\_A2 & 92.9064     5220.0      36.27 !\\
	ELEMENT TA & BCC\_A2 & 180.9479    5681.872    41.4718 !\\
	ELEMENT TI & HCP\_A3 & 4.7880E+01  4.8100E+03  3.0648E+01!\\
	ELEMENT ZR & HCP\_A3 & 9.1224E+01  5.5663E+03  3.9181E+01!\\
	ELEMENT VA & VACUUM & 0.0         0.0         0.0 !\\
	\end{tabular}
\label{ab-table:timonbtazr}
\end{table}
\begin{longtable}[H]{ l l l }
	\label{ab-table:timonbtazr1} \\
	\hline
	\endhead
	\hline
	\endfoot
	\multicolumn{3}{l}{\$*************************************************************}\\
	\multicolumn{3}{l}{\$ The Gibbs energies of the elements}\\
	\multicolumn{3}{l}{\$ in the stable and metastable forms from SGTE}\\
	\multicolumn{3}{l}{\$-----------------------------------------------------------------------------------------------}\\
	\multicolumn{3}{l}{\$-----------------------------------------------------------------------------------------------}\\
	\multicolumn{3}{c}{* TI *}\\
	\multicolumn{3}{l}{\$-----------------------------------------------------------------------------------------------}\\
	\multicolumn{3}{l}{\$-----------------------------------------------------------------------------------------------}\\
	FUNCTION GBCCTI & & \\
	\multicolumn{3}{l}{2.98150E+02  -1272.064+134.71418*T-25.5768*T*LN(T)}\\
	\multicolumn{3}{l}{-6.63845E-04*T**2-2.78803E-07*T**3+7208*T**(-1);  1.15500E+03  Y}\\
	\multicolumn{3}{l}{+6667.385+105.366379*T-22.3771*T*LN(T)+.00121707*T**2-8.4534E-07*T**3}\\
	\multicolumn{3}{l}{-2002750*T**(-1);  1.94100E+03  Y}\\
	\multicolumn{3}{l}{+26483.26-182.426471*T+19.0900905*T*LN(T)-.02200832*T**2}\\
	\multicolumn{3}{l}{+1.228863E-06*T**3+1400501*T**(-1);  4.00000E+03  N REF:20 !}\\
	FUNCTION GHSERTI & & \\
	\multicolumn{3}{l}{2.98150E+02  -8059.921+133.615208*T-23.9933*T*LN(T)}\\
	\multicolumn{3}{l}{-.004777975*T**2+1.06716E-07*T**3+72636*T**(-1);  9.00000E+02  Y}\\
	\multicolumn{3}{l}{-7811.815+132.988068*T-23.9887*T*LN(T)-.0042033*T**2-9.0876E-08*T**3}\\
	\multicolumn{3}{l}{+42680*T**(-1);  1.15500E+03  Y}\\
	\multicolumn{3}{l}{+908.837+66.976538*T-14.9466*T*LN(T)-.0081465*T**2+2.02715E-07*T**3}\\
	\multicolumn{3}{l}{-1477660*T**(-1);  1.94100E+03  Y}\\
	\multicolumn{3}{l}{-124526.786+638.806871*T-87.2182461*T*LN(T)+.008204849*T**2}\\
	\multicolumn{3}{l}{-3.04747E-07*T**3+36699805*T**(-1);  4.00000E+03  N REF:20 !}\\
	FUNCTION GFCCTI & & \\
	\multicolumn{3}{l}{2.98150E+02  +6000-.1*T+GHSERTI;   6.00000E+03   N REF:20 !}\\
	FUNCTION GLIQTI & & \\
	\multicolumn{3}{l}{2.98150E+02  +12194.415-6.980938*T+GHSERTI; 1.30000E+03  Y}\\
	\multicolumn{3}{l}{+368610.36-2620.99904*T+357.005867*T*LN(T)-.155262855*T**2}\\
	\multicolumn{3}{l}{+1.2254402E-05*T**3-65556856*T**(-1)+GHSERTI;  1.94100E+03  Y}\\
	\multicolumn{3}{l}{+104639.72-340.070171*T+40.9282461*T*LN(T)-.008204849*T**2}\\
	\multicolumn{3}{l}{+3.04747E-07*T**3-36699805*T**(-1)+GHSERTI;  6.00000E+03  N REF:20 !}\\
	\multicolumn{3}{l}{\$-----------------------------------------------------------------------------------------------}\\
	\multicolumn{3}{l}{\$-----------------------------------------------------------------------------------------------}\\
	\multicolumn{3}{c}{* MO *}\\
	\multicolumn{3}{l}{\$-----------------------------------------------------------------------------------------------}\\
	\multicolumn{3}{l}{\$-----------------------------------------------------------------------------------------------}\\	
	FUNCTION GHSERMO & & \\
	\multicolumn{3}{l}{298.15 -7746.302+131.9197*T-23.56414*T*LN(T)}\\
	\multicolumn{3}{l}{-.003443396*T**2+5.662834E-07*T**3+65812.39*T**(-1)}\\
	\multicolumn{3}{l}{-1.309265E-10*T**4; 2896.00 Y}\\
	\multicolumn{3}{l}{-30556.41+283.559746*T-42.63829*T*LN(T)}\\
	\multicolumn{3}{l}{-4.849315E+33*T**(-9); 4000.00 N REF:20 !}\\
	FUNCTION GLIQMO & & \\
	\multicolumn{3}{l}{298.15 41831.347-14.694912*T+4.24519E-22*T**7}\\
	\multicolumn{3}{l}{+GHSERMO;   2896.00 Y}\\
	\multicolumn{3}{l}{34095.373-11.890046*T+4.849315E33*T**(-9)+GHSERMO;}\\
	\multicolumn{3}{l}{4000.00 N REF:20 !}\\
	FUNCTION GFCCMO & & \\
	\multicolumn{3}{l}{298.15 15200+0.63*T+GHSERMO; 4000.00 N REF:20 !}\\
	FUNCTION GHCPMO & & \\
	\multicolumn{3}{l}{298.15 11550+GHSERMO; 4000.00 N REF:20 !}\\
	\multicolumn{3}{l}{\$-----------------------------------------------------------------------------------------------}\\
	\multicolumn{3}{l}{\$-----------------------------------------------------------------------------------------------}\\
	\multicolumn{3}{c}{* NB *}\\
	\multicolumn{3}{l}{\$-----------------------------------------------------------------------------------------------}\\
	\multicolumn{3}{l}{\$-----------------------------------------------------------------------------------------------}\\
	FUNCTION GHSERNB & & \\
	\multicolumn{3}{l}{2.98140E+02  -8519.353+142.045475*T}\\
	\multicolumn{3}{l}{-26.4711*T*LN(T)+2.03475E-04*T**2-3.5012E-07*T**3}\\
	\multicolumn{3}{l}{+93399*T**(-1);  2.75000E+03  Y}\\
	\multicolumn{3}{l}{-37669.3+271.720843*T-41.77*T*LN(T)+1.528238E+32*T**(-9);}\\
	\multicolumn{3}{l}{6.00000E+03 N  REF:20 !}\\
	FUNCTION GHEXTNB & & \\
	\multicolumn{3}{l}{2.98150E+02  -8519.35+142.048*T}\\
	\multicolumn{3}{l}{-26.4711*T*LN(T)+2.03475E-04*T**2-3.50119E-07*T**3}\\
	\multicolumn{3}{l}{+93398.8*T**(-1);   6.00000E+03   N REF:23 !}\\
	FUNCTION GLIQNB & & \\
	\multicolumn{3}{l}{298.15   29781.555-10.816417*T}\\
	\multicolumn{3}{l}{-3.06098E-23*T**7+GHSERNB; 2750.00 Y}\\
	\multicolumn{3}{l}{+30169.901-10.964695*T-1.52824E32*T**(-9)+GHSERNB;}\\
	\multicolumn{3}{l}{6000.00 N REF:20 !}\\
	FUNCTION GFCCNB & & \\
	\multicolumn{3}{l}{298.15     +13500+1.7*T+GHSERNB; 6000.00 N REF:20 !}\\
	FUNCTION GHCPNB & & \\
	\multicolumn{3}{l}{298.15   +10000+2.4*T+GHSERNB; 6000.00 N REF:20 !}\\
	\multicolumn{3}{l}{\$-----------------------------------------------------------------------------------------------}\\
	\multicolumn{3}{l}{\$-----------------------------------------------------------------------------------------------}\\
	\multicolumn{3}{c}{* TA *}\\
	\multicolumn{3}{l}{\$-----------------------------------------------------------------------------------------------}\\
	\multicolumn{3}{l}{\$-----------------------------------------------------------------------------------------------}\\	
	FUNCTION GHSERTA & & \\
	\multicolumn{3}{l}{2.98150E+02  -7285.889+119.139858*T}\\
	\multicolumn{3}{l}{-23.7592624*T*LN(T)-.002623033*T**2+1.70109E-07*T**3}\\
	\multicolumn{3}{l}{-3293*T**(-1);  1.30000E+03  Y}\\
	\multicolumn{3}{l}{-22389.955+243.88676*T-41.137088*T*LN(T)+.006167572*T**2}\\
	\multicolumn{3}{l}{-6.55136E-07*T**3+2429586*T**(-1);  2.50000E+03  Y}\\
	\multicolumn{3}{l}{+229382.886-722.59722*T+78.5244752*T*LN(T)-.017983376*T**2}\\
	\multicolumn{3}{l}{+1.95033E-07*T**3-93813648*T**(-1);  3.25800E+03  Y}\\
	\multicolumn{3}{l}{-963392.734+2773.7774*T-337.227976*T*LN(T)+.039791303*T**2}\\
	\multicolumn{3}{l}{-9.74251E-07*T**3+5.09949511E+08*T**(-1); 6.00000E+03  N REF:20 !}\\
	FUNCTION GFCCTA & & \\
	\multicolumn{3}{l}{2.98150E+02  +16000+1.7*T+GHSERTA; 6.00000E+03  N REF:20 !}\\
	FUNCTION GHCPTA & & \\
	\multicolumn{3}{l}{2.98150E+02  +12000+2.4*T+GHSERTA;  6.00000E+03   N REF:20 !}\\
	FUNCTION GLIQTA & & \\
	\multicolumn{3}{l}{2.98150E+02  +29160.975-7.578729*T+GHSERTA;  1.00000E+03  Y}\\
	\multicolumn{3}{l}{+51170.228-181.121652*T+23.7872147*T*LN(T)-.009707033*T**2}\\
	\multicolumn{3}{l}{+4.4449E-07*T**3-3520045*T**(-1)+GHSERTA;  1.30000E+03  Y}\\
	\multicolumn{3}{l}{+66274.294-305.868555*T+41.1650403*T*LN(T)-.018497638*T**2}\\
	\multicolumn{3}{l}{+1.269735E-06*T**3-5952924*T**(-1)+GHSERTA;  2.50000E+03  Y}\\
	\multicolumn{3}{l}{-185498.547+660.615425*T-78.4965229*T*LN(T)+.00565331*T**2}\\
	\multicolumn{3}{l}{+4.19566E-07*T**3+90290310*T**(-1)+GHSERTA;  3.29000E+03  Y}\\
	\multicolumn{3}{l}{+1036069.47-2727.38037*T+320.319132*T*LN(T)-.043117795*T**2}\\
	\multicolumn{3}{l}{+1.055148E-06*T**3-5.54714342E+08*T**(-1)+GHSERTA;}\\
	\multicolumn{3}{l}{6.00000E+03  N REF:20 !}\\
	FUNCTION TATIB2 & & \\
	\multicolumn{3}{l}{2.98150E+02  2500;   6.00000E+03  N REF:25 !}\\
	\multicolumn{3}{l}{\$-----------------------------------------------------------------------------------------------}\\
	\multicolumn{3}{l}{\$-----------------------------------------------------------------------------------------------}\\
	\multicolumn{3}{c}{* ZR *}\\
	\multicolumn{3}{l}{\$-----------------------------------------------------------------------------------------------}\\
	\multicolumn{3}{l}{\$-----------------------------------------------------------------------------------------------}\\
	FUNCTION GLIQZR & & \\
	\multicolumn{3}{l}{2.98140E+02  +18147.69-9.080812*T}\\
	\multicolumn{3}{l}{+1.6275E-22*T**7+GHSERZR;  2.12800E+03  Y}\\
	\multicolumn{3}{l}{+17804.661-8.911574*T+1.342895E+31*T**(-9)+GHSERZR;}\\
	\multicolumn{3}{l}{6.00000E+03  N REF:20 !}\\
	FUNCTION GBCCZR & & \\
	\multicolumn{3}{l}{2.98140E+02  -525.539+124.9457*T}\\
	\multicolumn{3}{l}{-25.607406*T*LN(T)-3.40084E-04*T**2-9.729E-09*T**3}\\
	\multicolumn{3}{l}{+25233*T**(-1)-7.6143E-11*T**4;  2.12800E+03  Y}\\
	\multicolumn{3}{l}{-30705.955+264.284163*T-42.144*T*LN(T)+1.276058E+32*T**(-9);}\\
	\multicolumn{3}{l}{6.00000E+03  N REF:20 !}\\
	FUNCTION GHSERZR & & \\
	\multicolumn{3}{l}{1.30000E+02  -7827.595+125.64905*T}\\
	\multicolumn{3}{l}{-24.1618*T*LN(T)-.00437791*T**2+34971*T**(-1); 2.12800E+03  Y}\\
	\multicolumn{3}{l}{-26085.921+262.724183*T-42.144*T*LN(T)-1.342895E+31*T**(-9);}\\
	\multicolumn{3}{l}{6.00000E+03  N REF:20 !}\\
	\multicolumn{3}{l}{\$-----------------------------------------------------------------------------------------------}\\
	\multicolumn{3}{l}{FUNCTION UN\_ASS 298.15 0; 300 N !}\\
	\multicolumn{3}{l}{\$-----------------------------------------------------------------------------------------------}\\
	\multicolumn{3}{l}{\$*************************************************************}\\
	\multicolumn{3}{l}{TYPE\_DEFINITION \% SEQ * !}\\
	\multicolumn{3}{l}{TYPE\_DEFINITION G SEQ * !}\\
	\multicolumn{3}{l}{DEFINE\_SYSTEM\_DEFAULT SPECIE 5 !}\\
	\multicolumn{3}{l}{DEFAULT\_COMMAND DEF\_SYS\_ELEMENT VA !}\\
	\multicolumn{3}{l}{\$-----------------------------------------------------------------------------------------------}\\
	\multicolumn{3}{l}{\$*************************************************************}\\
	\multicolumn{3}{l}{\$*************************************************************}\\
	\multicolumn{3}{l}{\$-----------------------------------------------------------------------------------------------}\\
	\multicolumn{3}{l}{PHASE LIQUID \% 1 1.0 !}\\
	\multicolumn{3}{l}{CONSTITUENT LIQUID :MO,TA,NB,TI,ZR: !}\\
	& & \\
	PARAMETER G(LIQUID,TI;0) & &\\
	\multicolumn{3}{l}{298.15 GLIQTI; 6000.00 N REF:20 !}\\
	PARAMETER G(LIQUID,MO;0)& &\\
	\multicolumn{3}{l}{298.15 GLIQMO; 6000.00 N REF:20 !}\\
	PARAMETER G(LIQUID,NB;0) & & \\
	\multicolumn{3}{l}{298.15 GLIQNB; 6000.00 N REF:20 }\\
	PARAMETER G(LIQUID,TA;0) & &\\
	\multicolumn{3}{l}{298.15 GLIQTA; 6000.00 N REF:20 !}\\
	PARAMETER G(LIQUID,ZR;0) & &\\
	\multicolumn{3}{l}{298.15 +GLIQZR; 6000.00 N REF:20 !}\\
	PARAMETER G(LIQUID,MO,TI;0) & & \\
	\multicolumn{3}{l}{298.15 -9000.0+2*T; 6000.00 N REF:25 !}\\
	PARAMETER G(LIQUID,NB,TI;0) & & \\
	\multicolumn{3}{l}{298.15 +7406.1; 6000.00 N REF:23 !}\\	 
	PARAMETER G(LIQUID,TA,TI;0) & & \\
	\multicolumn{3}{l}{298.15 +1000; 6000.00 N REF:25 !}\\
	PARAMETER G(LIQUID,TA,TI;1) & & \\
	\multicolumn{3}{l}{298.15 -7000; 6000.00 N REF:25 !}\\	 
	PARAMETER G(LIQUID,TI,ZR;0) & & \\
	\multicolumn{3}{l}{298.15 -967.66; 6000.00 N REF:22 !}\\ 
	PARAMETER G(LIQUID,MO,NB;0) & *& \\
	\multicolumn{3}{l}{298.15 15253.7; 6000.00 N REF:21 !}\\
	PARAMETER G(LIQUID,MO,NB;1) & & \\
	\multicolumn{3}{l}{298.15 10594.2; 6000.00 N REF:21 !}\\	 	 
	PARAMETER G(LIQUID,MO,TA;0) & & \\
	\multicolumn{3}{l}{298.15 13978.9; 6000.00 N REF:21 !}\\
	PARAMETER G(LIQUID,MO,TA;1) & & \\
	\multicolumn{3}{l}{298.15 577.5; 6000.00 N REF:21 !}\\
	PARAMETER G(LIQUID,MO,ZR;0) & & \\
	\multicolumn{3}{l}{298.15  -24055.120+8.146158*T; 6000.00  N REF:26 !}\\
	PARAMETER G(LIQUID,MO,ZR;1) & & \\
	\multicolumn{3}{l}{298.15  -5132.1665+4.8041224*T; 6000.00  N REF:26 !}\\	 
	PARAMETER G(LIQUID,NB,TA;0) & & \\
	\multicolumn{3}{l}{298.15 0; 6000.00 N REF:21 !}\\
	PARAMETER G(LIQUID,NB,ZR;0) & & \\
	\multicolumn{3}{l}{298.15 10311; 6000.00 N REF:28 !}\\
	PARAMETER G(LIQUID,NB,ZR;1) & & \\
	\multicolumn{3}{l}{298.15  6709; 6000.00 N REF:28 !}\\  
	PARAMETER G(LIQUID,TA,ZR;0) & & \\
	\multicolumn{3}{l}{298.15 13832.1; 6000.00 N REF:27 !}\\
	PARAMETER G(LIQUID,TA,ZR;1) & & \\
	\multicolumn{3}{l}{298.15 -7150;   6000.00 N REF:27 !}\\
	\multicolumn{3}{l}{\$-----------------------------------------------------------------------------------------------}\\
	\multicolumn{3}{l}{PHASE BCC\_A2 \% 2 1 3 !}\\
	\multicolumn{3}{l}{CONSTITUENT BCC\_A2 :MO,TA,NB,TI,ZR:VA: !}\\
	& & \\
	PARAMETER G(BCC\_A2,TI:VA;0) & & \\
	\multicolumn{3}{l}{298.15 +GBCCTI; 6000.0  N REF:20 !}\\
	PARAMETER G(BCC\_A2,MO:VA;0) & & \\
	\multicolumn{3}{l}{298.15 +GHSERMO; 6000.00 N REF:20 !}\\
	PARAMETER G(BCC\_A2,NB:VA;0) & & \\
	\multicolumn{3}{l}{298.15 +GHSERNB; 6000.00 N REF:20 !}\\ 
	PARAMETER G(BCC\_A2,TA:VA;0) & & \\
	\multicolumn{3}{l}{298.15 +GHSERTA; 6000.00 N REF:20 !}\\
	PARAMETER G(BCC\_A2,ZR:VA;0) & & \\
	\multicolumn{3}{l}{298.15 +GBCCZR; 6000.00 N REF:20 !}\\
	PARAMETER G(BCC\_A2,MO,TI:VA;0) & & \\
	\multicolumn{3}{l}{298.15   2000.0; 6000.00 N REF:25 !}\\
	PARAMETER G(BCC\_A2,MO,TI:VA;1) & & \\
	\multicolumn{3}{l}{298.15  -2000.0; 6000.00 N REF:25 !}\\
	PARAMETER G(BCC\_A2,NB,TI:VA;0) & & \\
	\multicolumn{3}{l}{298.15  +13045.3; 6000.00 N REF:23 !}\\	 
	PARAMETER G(BCC\_A2,TA,TI:VA;0) & & \\
	\multicolumn{3}{l}{298.15  12000; 6000.00 N REF:25 !}\\
	PARAMETER G(BCC\_A2,TA,TI:VA;1) & & \\
	\multicolumn{3}{l}{298.15  -2500; 6000.00 N REF:25 !}\\	 
	PARAMETER G(BCC\_A2,TI,ZR:VA;0) & & \\
	\multicolumn{3}{l}{298.15  -4346.16+5.48903*T; 6000.00 N REF:22  !}\\	 
	PARAMETER G(BCC\_A2,MO,NB:VA;0) & & \\
	\multicolumn{3}{l}{298.15 -68202.6+29.85596*T; 6000.00 N REF:21 !}\\
	PARAMETER G(BCC\_A2,MO,NB:VA;1) & & \\
	\multicolumn{3}{l}{298.15 8201.3; 6000.00 N REF:21 !}\\
	PARAMETER G(BCC\_A2,MO,TA:VA;0) & & \\
	\multicolumn{3}{l}{298.15 -75129.2+30*T; 6000.00 N REF:21 !}\\
	PARAMETER G(BCC\_A2,MO,TA:VA;1) & & \\
	\multicolumn{3}{l}{298.15 6039.24; 6000.00 N REF:21 !}\\
	PARAMETER G(BCC\_A2,MO,ZR:VA;0) & & \\
	\multicolumn{3}{l}{298.15 +17935.985+3.102*T; 6000.00 N REF:26 !}\\
	PARAMETER G(BCC\_A2,MO,ZR:VA;1)  & & \\
	\multicolumn{3}{l}{298.15  -990.9911+4.299*T; 6000.00 N REF:26 !}\\
	PARAMETER G(BCC\_A2,NB,TA:VA;0) & & \\
	\multicolumn{3}{l}{298.15 1298.02870; 6000.00 N REF:21 !}\\
	PARAMETER G(BCC\_A2,NB,ZR:VA;0) & & \\
	\multicolumn{3}{l}{298.15  +15911+3.35*T; 6000.00 N REF:28 !}\\
	PARAMETER G(BCC\_A2,NB,ZR:VA;1) & & \\
	\multicolumn{3}{l}{298.15 +3919-1.091*T; 6000.00 N REF:28 !}\\
	PARAMETER G(BCC\_A2,ZR,TA:VA;0) & & \\
	\multicolumn{3}{l}{298.15 29499.6+2.6723*T; 6000.00 N REF:27 !}\\
	PARAMETER G(BCC\_A2,ZR,TA:VA;1) & & \\
	\multicolumn{3}{l}{298.15 -4396.2+4.4302*T; 6000.00 N REF:27 !}\\
	PARAMETER G(BCC\_A2,ZR,TA:VA;2) & &\\
	\multicolumn{3}{l}{298.15 -6353.3+4.9066*T; 6000.00 N REF:27 !}\\
	PARAMETER G(BCC\_A2,MO,TA,TI:VA;0) & \multicolumn{2}{l}{298.15   0; 6000.00  N !}\\
	PARAMETER G(BCC\_A2,MO,TA,TI:VA;1) & \multicolumn{2}{l}{298.15   0; 6000.00  N !}\\
	PARAMETER G(BCC\_A2,MO,TA,TI:VA;2) & & \\
	\multicolumn{3}{l}{298.15   -1.5473118E+05; 6000.00  N !}\\
	PARAMETER G(BCC\_A2,NB,TA,TI:VA;0) & & \\
	\multicolumn{3}{l}{2.98150E+02  -1.3660332E+05; 6.00000E+03  N !}\\
	PARAMETER G(BCC\_A2,NB,TA,TI:VA;1) & &\\
	\multicolumn{3}{l}{2.98150E+02  -1.3660269E+05; 6.00000E+03  N  !}\\
	PARAMETER G(BCC\_A2,NB,TA,TI:VA;2) & & \\
	\multicolumn{3}{l}{2.98150E+02  0; 6.00000E+03 N !}\\
	\multicolumn{3}{l}{\$-----------------------------------------------------------------------------------------------}\\
	\multicolumn{3}{l}{PHASE HCP\_A3  \%  2 1   .5 !}\\
	\multicolumn{3}{l}{CONSTITUENT HCP\_A3  :NB,TI\%,ZR,TA,MO : VA\% :  !}\\
	& & \\
	PARAMETER G(HCP\_A3,TI:VA;0) & & \\
	\multicolumn{3}{l}{298.15 +GHSERTI; 4000.00 N REF:20 !}\\
	PARAMETER G(HCP\_A3,MO:VA;0) & & \\
	\multicolumn{3}{l}{298.15 +GHCPMO; 5000.00 N REF:20 !}\\	
	PARAMETER G(HCP\_A3,NB:VA;0) & & \\
	\multicolumn{3}{l}{298.15 +GHCPNB; 6000.00 N REF:20 !}\\
	PARAMETER G(HCP\_A3,TA:VA;0) & &\\
	\multicolumn{3}{l}{298.15 +GHCPTA; 6000.00 N REF:20 !}\\ 
	PARAMETER G(HCP\_A3,ZR:VA;0) & & \\
	\multicolumn{3}{l}{298.15 +GHSERZR; 6000.00 N REF:20 !}\\	
	PARAMETER G(HCP\_A3,MO,TI:VA;0) & & \\
	\multicolumn{3}{l}{298.15   22760-6*T; 6000.00 N REF:25 !}\\
	PARAMETER G(HCP\_A3,NB,TI:VA;0) & & \\
	\multicolumn{3}{l}{298.15  +11742.4; 6000.00 N REF:23   !}\\
	PARAMETER G(HCP\_A3,TA,TI:VA;0) & &\\
	\multicolumn{3}{l}{298.15  8500; 6000.00 N REF:25 !}\\	 
	PARAMETER G(HCP\_A3,TI,ZR:VA;0) & & \\
	\multicolumn{3}{l}{298.15  +5133.02; 6000.00 N  REF:22  !}\\
	PARAMETER G(HCP\_A3,MO,ZR:VA;0) & & \\
	\multicolumn{3}{l}{298.15  +26753.79+4.556*T; 6000.00 N REF:26 !}\\	 
	PARAMETER G(HCP\_A3,NB,ZR:VA;0) & & \\
	\multicolumn{3}{l}{298.15  24411; 6000.00 N REF:28 !}\\
	PARAMETER G(HCP\_A3,ZR,TA:VA;0) & & \\
	\multicolumn{3}{l}{298.15 +30051.7; 6000.00 N REF:27 !}\\
	\multicolumn{3}{l}{\$-----------------------------------------------------------------------------------------------}\\
	\multicolumn{3}{l}{TYPE\_DEFINITION * GES A\_P\_D FCC\_A1 MAGNETIC  -3.0    2.80000E-01 !}\\
	\multicolumn{3}{l}{PHASE FCC\_A1  \%*  2 1   1 !}\\
	\multicolumn{3}{l}{CONSTITUENT FCC\_A1  :MO,TI,TA,ZR,NB : VA\% :  !}\\
	& & \\
	PARAMETER G(FCC\_A1,TI:VA;0) & & \\
	\multicolumn{3}{l}{298.15 +GFCCTI; 4000.00 N REF:20 !}\\
	PARAMETER G(FCC\_A1,MO:VA;0) & & \\
	\multicolumn{3}{l}{298.15 +GFCCMO; 5000.00 N REF:20 !}\\
	PARAMETER G(FCC\_A1,NB:VA;0) & & \\
	\multicolumn{3}{l}{298.15 +GFCCNB; 6000.00 N REF:20 !}\\
	PARAMETER G(FCC\_A1,TA:VA;0) & & \\
	\multicolumn{3}{l}{298.15 +GFCCTA; 6000.00 N REF:20 !}\\
	PARAMETER G(FCC\_A1,MO,TI:VA;0) & & \\
	\multicolumn{3}{l}{298.15   16500.0; 6000.00 N REF:25 !}\\
	PARAMETER G(FCC\_A1,TA,TI:VA;0) & & \\
	\multicolumn{3}{l}{298.15  8500; 6000.00 N REF:25 !}\\
	\multicolumn{3}{l}{\$-----------------------------------------------------------------------------------------------}\\
	\multicolumn{3}{l}{PHASE AL3M\_D022  \%  2 3   1 !}\\
	\multicolumn{3}{l}{CONSTITUENT AL3M\_D022  :TI,MO : TA,TI,MO :  !}\\
	& & \\
	PARAMETER G(AL3M\_D022,TI:TI;0) & & \\
	\multicolumn{3}{l}{298.15 +4*GFCCTI; 6000.00 N REF:25 !}\\
	PARAMETER G(AL3M\_D022,MO:MO;0) & & \\
	\multicolumn{3}{l}{298.15 +4*GFCCMO; 6000.00 N REF:25 !}\\
	PARAMETER G(AL3M\_D022,TI:MO;0) & & \\
	\multicolumn{3}{l}{298.15 GFCCMO+3.0*GFCCTI; 6000.00 N REF:25 !}\\	
	PARAMETER G(AL3M\_D022,MO:TI;0) & &\\
	\multicolumn{3}{l}{298.15 3*GFCCMO+GFCCTI; 6000.00 N REF:25 !}\\	  
	PARAMETER G(AL3M\_D022,TI:TA;0) & & \\
	\multicolumn{3}{l}{298.15 +3*GFCCTI+GFCCTA; 6000.00 N REF:25 !}\\
	\multicolumn{3}{l}{\$-----------------------------------------------------------------------------------------------}\\
	\multicolumn{3}{l}{TYPE\_DEFINITION \& GES A\_P\_D ALM\_D019 MAGNETIC  -3.0    2.80000E-01 !}\\
	\multicolumn{3}{l}{PHASE ALM\_D019  \%\&  2 3   1 !}\\
	\multicolumn{3}{l}{CONSTITUENT ALM\_D019  :MO,TA,TI\% : MO,TA,TI :  !}\\
	& & \\
	PARAMETER G(ALM\_D019,TI:TI;0) & &\\
	\multicolumn{3}{l}{298.15 +4.0+4.0*GHSERTI; 6000.00 N REF:25 !}\\
	PARAMETER G(ALM\_D019,TA:MO;0) & \multicolumn{2}{l}{298.15 0; 6000 N!}\\
	PARAMETER G(ALM\_D019,MO:TA;0) & \multicolumn{2}{l}{298.15 0; 6000 N!}\\ 
	PARAMETER G(ALM\_D019,MO:MO;0) & & \\
	\multicolumn{3}{l}{298.15 +4.0*GHCPMO; 6000.00 N REF:25 !}\\
	PARAMETER G(ALM\_D019,TA:TA;0) & & \\ 
	\multicolumn{3}{l}{298.15 +4.0*GHCPTA; 6000.00 N REF:25 !}\\
	PARAMETER G(ALM\_D019,TI:MO;0) & & \\
	\multicolumn{3}{l}{298.15 +17072.0-4.5*T+GHCPMO+3.0*GHSERTI;    6000.00 N REF:25 !}\\
	PARAMETER G(ALM\_D019,MO:TI;0) & & \\
	\multicolumn{3}{l}{298.15 +17072.0-4.5*T+3.0*GHCPMO+GHSERTI;    6000.00 N REF:25 !}\\
	PARAMETER G(ALM\_D019,TI:TA;0) & & \\
	\multicolumn{3}{l}{298.15 +6376+3.0*GHSERTI+GHCPTA; 6000.00 N REF:25 !}\\
	PARAMETER G(ALM\_D019,TA:TI;0) & & \\
	\multicolumn{3}{l}{298.15 +6376+3.0*GHCPTA+GHSERTI; 6000.00 N REF:25 !}\\
	PARAMETER G(ALM\_D019,MO,TI:MO;0) & & \\
	\multicolumn{3}{l}{298.15 +51212-13.5*T; 6000.00 N REF:25 !}\\
	PARAMETER G(ALM\_D019,MO,TI:TI;0) & & \\
	\multicolumn{3}{l}{298.15 +51212-13.5*T; 6000.00 N REF:25 !}\\
	PARAMETER G(ALM\_D019,MO:MO,TI;0) & & \\
	\multicolumn{3}{l}{298.15 +5692-1.5*T; 6000.00 N REF:25 !}\\
	PARAMETER G(ALM\_D019,TI:MO,TI;0) & & \\
	\multicolumn{3}{l}{298.15 +5692-1.5*T; 6000.00 N REF:25 !}\\	 
	PARAMETER G(ALM\_D019,TA,TI:TA;0) & & \\
	\multicolumn{3}{l}{298.15 +19128; 6000.00 N REF:25 !}\\
	PARAMETER G(ALM\_D019,TA,TI:TI;0) & & \\
	\multicolumn{3}{l}{298.15 +19128; 6000.00 N REF:25 !}\\	 
	PARAMETER G(ALM\_D019,TA:TA,TI;0) & & \\
	\multicolumn{3}{l}{298.15 2128; 6000.00 N REF:25 !}\\
	PARAMETER G(ALM\_D019,TI:TA,TI;0) & & \\ 
	\multicolumn{3}{l}{298.15 +2128; 6000.00 N REF:25 !}\\
	\multicolumn{3}{l}{\$-----------------------------------------------------------------------------------------------}\\
	\multicolumn{3}{l}{TYPE\_DEFINITION ' GES A\_P\_D ALTI MAGNETIC  -1.0    4.00000E-01 !}\\
	\multicolumn{3}{l}{PHASE ALTI  \%'  2 1   1 !}\\
	\multicolumn{3}{l}{CONSTITUENT ALTI  :MO,TA,TI : MO,TA,TI\% :  !}\\
	& & \\
	PARAMETER G(ALTI,TI:TI;0) & & \\
	\multicolumn{3}{l}{298.15 +2*GFCCTI; 6000.00 N REF:25 !}\\	
	PARAMETER G(ALTI,MO:MO;0) & & \\
	\multicolumn{3}{l}{298.15 +2*GFCCMO; 6000.00 N REF:25 !}\\ 
	PARAMETER G(ALTI,TA:TA;0) & & \\
	\multicolumn{3}{l}{298.15 +2*GFCCTA; 6000.00 N REF:25 !}\\
	PARAMETER G(ALTI,TI:MO;0) & & \\
	\multicolumn{3}{l}{298.15 8250+GFCCMO+GFCCTI; 6000.00 N REF:25  !}\\	 
	PARAMETER G(ALTI,MO:TI;0) & & \\
	\multicolumn{3}{l}{298.15 8250+GFCCMO+GFCCTI; 6000.00 N REF:25  !}\\
	PARAMETER G(ALTI,TI:TA;0) & & \\
	\multicolumn{3}{l}{298.15 +4250+GFCCTI+GFCCTA; 6000.00 N REF:25 !}\\
	PARAMETER G(ALTI,TA:TI;0) & & \\
	\multicolumn{3}{l}{298.15 +4250+GFCCTA+GFCCTI; 6000.00 N REF:25 !}\\	 
	PARAMETER G(ALTI,MO,TI:MO;0) & & \\
	\multicolumn{3}{l}{298.15 8250; 6000.00 N REF:25 !}\\
	PARAMETER G(ALTI,MO,TI:TI;0) & & \\
	\multicolumn{3}{l}{298.15 8250; 6000.00 N REF:25 !}\\
	PARAMETER G(ALTI,MO:MO,TI;0) & & \\
	\multicolumn{3}{l}{298.15 8250; 6000.00 N REF:25 !}\\
	PARAMETER G(ALTI,TI:MO,TI;0) & & \\
	\multicolumn{3}{l}{298.15 8250; 6000.00 N REF:25 !}\\
	PARAMETER G(ALTI,TA,TI:TA;0) & & \\
	\multicolumn{3}{l}{298.15 4250; 6000.00 N REF:25 !}\\
	PARAMETER G(ALTI,TA,TI:TI;0) & & \\
	\multicolumn{3}{l}{298.15 4250; 6000.00 N REF:25 !}\\	 
	PARAMETER G(ALTI,TA:TA,TI;0) & & \\
	\multicolumn{3}{l}{298.15 4250; 6000.00 N REF:25 !}\\
	PARAMETER G(ALTI,TI:TA,TI;0) & & \\
	\multicolumn{3}{l}{298.15 4250; 6000.00 N REF:25 !}\\
	\multicolumn{3}{l}{\$-----------------------------------------------------------------------------------------------}\\
	\multicolumn{3}{l}{TYPE\_DEFINITION ) GES AMEND\_PHASE\_DESCRIPTION}\\
	\multicolumn{3}{l}{BCC\_B2 DIS\_PART BCC\_A2,,,!}\\
	\multicolumn{3}{l}{PHASE BCC\_B2  \%)  3 .5   .5   3 !}\\
	\multicolumn{3}{l}{CONSTITUENT BCC\_B2  :MO,NB,TA,TI,ZR : MO,NB,TA,TI\%,ZR : VA :  !}\\
	& & \\
	PARAMETER G(BCC\_B2,MO:MO:VA;0) & \multicolumn{2}{l}{298.15 0; 6000 N REF:25!}\\
	PARAMETER G(BCC\_B2,TA:TA:VA;0) & \multicolumn{2}{l}{298.15 0; 6000 N REF:25 !}\\
	PARAMETER G(BCC\_B2,TI:TI:VA;0) & \multicolumn{2}{l}{298.15 0; 6000 N REF:25 !}\\ 
	PARAMETER G(BCC\_B2,TI:MO:VA;0) & & \\
	\multicolumn{3}{l}{298.15 10000.0; 6000.00 N REF:25 !}\\
	PARAMETER G(BCC\_B2,MO:TI:VA;0) & & \\
	\multicolumn{3}{l}{298.15 10000.0; 6000.00 N REF:25 !}\\	 
	PARAMETER G(BCC\_B2,TI:TA:VA;0) & & \\
	\multicolumn{3}{l}{298.15 +5000.0; 6000.00 N REF:25 !}\\
	PARAMETER G(BCC\_B2,TA:TI:VA;0) & & \\
	\multicolumn{3}{l}{298.15 +5000.0; 6000.00 N REF:25 !}\\
	\multicolumn{3}{l}{\$-----------------------------------------------------------------------------------------------}\\
	\multicolumn{3}{l}{PHASE BCT\_A5  \%  1  1.0  !}\\
	\multicolumn{3}{l}{CONSTITUENT BCT\_A5  :TI :  !}\\
	& & \\
	PARAMETER G(BCT\_A5,TI;0) & & \\
	\multicolumn{3}{l}{298.15 +4602.2+GHSERTI; 3000.00 N REF:20 !}\\
	\multicolumn{3}{l}{\$-----------------------------------------------------------------------------------------------}\\
	\multicolumn{3}{l}{PHASE CBCC\_A12  \%  2 1   1 !}\\
	\multicolumn{3}{l}{CONSTITUENT CBCC\_A12  :TI : VA :  !}\\
	& & \\
	PARAMETER G(CBCC\_A12,TI:VA;0) & & \\
	\multicolumn{3}{l}{298.15 +4602.2+GHSERTI; 6000.00 N REF:20 !}\\
	\multicolumn{3}{l}{\$-----------------------------------------------------------------------------------------------}\\
	\multicolumn{3}{l}{PHASE CUB\_A13  \%  2 1   1 !}\\
	\multicolumn{3}{l}{CONSTITUENT CUB\_A13  :TI : VA :  !}\\
	& & \\
	PARAMETER G(CUB\_A13,TI:VA;0) & & \\
	\multicolumn{3}{l}{298.15 +7531.2+GHSERTI; 6000.00 N REF:20 !}\\
	\multicolumn{3}{l}{\$-----------------------------------------------------------------------------------------------}\\
	\multicolumn{3}{l}{PHASE DIAMOND\_A4  \%  1  1.0  !}\\
	\multicolumn{3}{l}{CONSTITUENT DIAMOND\_A4  :TI :  !}\\
	& & \\
	PARAMETER G(DIAMOND\_A4,TI;0) & & \\
	\multicolumn{3}{l}{298.15 +25000+GHSERTI; 6000.00 N REF:20 !}\\
	\multicolumn{3}{l}{\$-----------------------------------------------------------------------------------------------}\\
	\multicolumn{3}{l}{PHASE LAVES\_C14  \%  2 2   1 !}\\
	\multicolumn{3}{l}{CONSTITUENT LAVES\_C14  :TI : TI\% :  !}\\
	& & \\
	PARAMETER G(LAVES\_C14,TI:TI;0) & &\\
	\multicolumn{3}{l}{298.15 +15000+3*GHSERTI; 6000.00 N REF:20 !}\\
	\multicolumn{3}{l}{\$-----------------------------------------------------------------------------------------------}\\
	\multicolumn{3}{l}{PHASE LAVES\_C15  \%  2 2   1 !}\\
	\multicolumn{3}{l}{CONSTITUENT LAVES\_C15  :TI,MO,ZR : TI,MO,ZR :  !}\\
	& & \\
	PARAMETER G(LAVES\_C15,TI:TI;0) & & \\
	\multicolumn{3}{l}{298.15 +15000+3*GHSERTI; 6.00000E+03   N REF:24!}\\
	PARAMETER G(LAVES\_C15,MO:MO;0) & & \\
	\multicolumn{3}{l}{298.15 +3*GHSERMO+15000; 6000 N REF:26 !}\\
	PARAMETER G(LAVES\_C15,ZR:ZR;0) & & \\
	\multicolumn{3}{l}{298.15 +3*GHSERZR+15000; 6000 N  REF:26 !}\\
	PARAMETER G(LAVES\_C15,TI:MO;0) & & \\ 
	\multicolumn{3}{l}{298.15 +GHSERMO+2*GHSERTI+15000; 6000 N REF:24 !}\\
	PARAMETER G(LAVES\_C15,MO:TI;0) & & \\
	\multicolumn{3}{l}{298.15 +2*GHSERMO+GHSERTI+15000; 6000 N REF:24 !}\\
	PARAMETER G(LAVES\_C15,TI:ZR;0) & & \\
	\multicolumn{3}{l}{298.15 +2*GHSERTI+GHSERZR+9000; 6000 N REF:24 !}\\	 	 
	PARAMETER G(LAVES\_C15,ZR:TI;0) & & \\ 
	\multicolumn{3}{l}{298.15 +GHSERTI+2*GHSERZR+15000; 6000 N REF:24 !}\\
	PARAMETER G(LAVES\_C15,MO:ZR;0) & & \\
	\multicolumn{3}{l}{298.15 +2*GHSERMO+GHSERZR-21734.78+0.1441789*T; 6000 N REF:26 !}\\
	PARAMETER G(LAVES\_C15,ZR:MO;0) & &\\
	\multicolumn{3}{l}{298.15 +GHSERMO+2*GHSERZR+21734.78-0.1441789*T; 6000 N REF:26 !}\\
	PARAMETER G(LAVES\_C15,MO:MO,ZR;0) & \multicolumn{2}{l}{298.15 +60000; 6000 N REF:26 !}\\
	PARAMETER G(LAVES\_C15,ZR:MO,ZR;0) & \multicolumn{2}{l}{298.15 +60000; 6000 N REF:26 !}\\
	PARAMETER G(LAVES\_C15,MO,ZR:MO;0) & \multicolumn{2}{l}{298.15 +100000; 6000 N REF:26 !}\\
	PARAMETER G(LAVES\_C15,MO,ZR:ZR;0) & \multicolumn{2}{l}{298.15 +100000; 6000 N REF:26 !}\\
	PARAMETER G(LAVES\_C15,TI:MO,ZR;0) & \multicolumn{2}{l}{298.15 +60000; 6000 N REF:27 !}\\
	PARAMETER G(LAVES\_C15,MO,ZR:TI;0) & \multicolumn{2}{l}{298.15 +100000; 6000 N REF:27 !}\\
	\multicolumn{3}{l}{\$-----------------------------------------------------------------------------------------------}\\
	\multicolumn{3}{l}{PHASE OMEGA  \%  2 1   .5 !}\\
	\multicolumn{3}{l}{CONSTITUENT OMEGA  :NB,TI\% : VA\% :  !}\\
	& & \\
	PARAMETER G(OMEGA,ZR;0) & & \\
	\multicolumn{3}{l}{298.15 -8878.082+144.432234*T}\\
	\multicolumn{3}{l}{-26.8556*T*LN(T)-.002799446*T**2+38376*T**(-1); 2128 Y}\\
	\multicolumn{3}{l}{-29500.524+265.290858*T-42.144*T*LN(T)}\\
	\multicolumn{3}{l}{+7.17445E+31*T**(-9); 6000 N REF:20 !}\\
	PARAMETER G(OMEGA,TI:VA;0) & & \\
	\multicolumn{3}{l}{298.15 1886.7-0.15161*T+GHSERTI; 4.00000E+03  N REF:23 !}\\
	PARAMETER G(OMEGA,NB:VA;0) & &\\
	\multicolumn{3}{l}{2.98150E+02  15000+2.4*T++GHSERNB; 6.00000E+03   N REF:23 !}\\
	PARAMETER G(OMEGA,NB,TI:VA;0) & &\\
	\multicolumn{3}{l}{298.15 -3775.9; 6000.00 N REF:23   !}\\
	\multicolumn{3}{l}{\$-----------------------------------------------------------------------------------------------}\\
	\multicolumn{3}{l}{PHASE SI3TI5  \%  3 2   3   3 !}\\
	\multicolumn{3}{l}{CONSTITUENT SI3TI5  :TI : TI : TI :  !}\\
	& & \\
	PARAMETER G(SI3TI5,TI:TI:TI;0) & &\\
	\multicolumn{3}{l}{298.15 +40000+20*T+8*GHSERTI; 6000.00 N REF:20 !}\\
	\multicolumn{3}{l}{\$-----------------------------------------------------------------------------------------------}\\
	\multicolumn{3}{l}{PHASE SNTI3  \%  2 1   3 !}\\
	\multicolumn{3}{l}{CONSTITUENT SNTI3  :TI : TI\% :  !}\\
	& & \\
	PARAMETER G(SNTI3,TI:TI;0) &  &\\
	\multicolumn{3}{l}{298.15 +4*GHSERTI+4; 6000.00 N REF:20 !}\\
	\multicolumn{3}{l}{\$-----------------------------------------------------------------------------------------------}\\
	\multicolumn{3}{l}{PHASE ORTHORHOMBIC\_A20  \%  1  1.0  !}\\
	\multicolumn{3}{l}{CONSTITUENT ORTHORHOMBIC\_A20  :ZR :  !}\\
	& & \\
	PARAMETER G(ORTHORHOMBIC\_A20,ZR;0) & &\\
	\multicolumn{3}{l}{298.15 +4474.461+124.9457*T}\\
	\multicolumn{3}{l}{-25.607406*T*LN(T)-3.40084E-04*T**2-9.729E-09*T**3}\\
	\multicolumn{3}{l}{+25233*T**(-1)-7.6143E-11*T**4; 2128 Y}\\
	\multicolumn{3}{l}{-25705.955+264.284163*T-42.144*T*LN(T)}\\
	\multicolumn{3}{l}{+1.276058E+32*T**(-9); 6000 N REF:20 !}\\
	\multicolumn{3}{l}{\$-----------------------------------------------------------------------------------------------}\\
	\multicolumn{3}{l}{PHASE TETRAGONAL\_U  \%  1  1.0  !}\\
	\multicolumn{3}{l}{CONSTITUENT TETRAGONAL\_U  :ZR :  !}\\
	& & \\
	PARAMETER G(TETRAGONAL\_U,ZR;0)&  & \\
	\multicolumn{3}{l}{298.15 +4474.461+124.9457*T}\\
	\multicolumn{3}{l}{-25.607406*T*LN(T)-3.40084E-04*T**2-9.729E-09*T**3}\\
	\multicolumn{3}{l}{+25233*T**(-1)-7.6143E-11*T**4; 2128 Y}\\
	\multicolumn{3}{l}{-25705.955+264.284163*T-42.144*T*LN(T)}\\
	\multicolumn{3}{l}{+1.276058E+32*T**(-9); 6000 N REF:20 !}\\
	\multicolumn{3}{l}{\$-----------------------------------------------------------------------------------------------}\\
	\multicolumn{3}{l}{\$*************************************************************}\\
	\multicolumn{3}{l}{DATABASE\_INFO 'FOR THE TI\_MO\_NB\_TA\_ZR SYSTEM' !}\\
	\multicolumn{3}{l}{LIST\_OF\_REFERENCES}\\
	\multicolumn{3}{l}{NUMBER  SOURCE}\\
	\multicolumn{3}{l}{20     'A.T. Dinsdale, SGTE Data for Pure Elements, CALPHAD.}\\ 
	\multicolumn{3}{l}{15 (1991) 317-425. '}\\
	\multicolumn{3}{l}{21     'W. Xiong, Y. Du, Y. Liu, B.Y. Huang, H.H. Xu, H.L. Chen,}\\ 
	\multicolumn{3}{l}{et al., Thermodynamic assessment of the Mo-Nb-Ta system,}\\ 
	\multicolumn{3}{l}{Calphad-Computer Coupling Phase Diagrams Thermochem. 28}\\ 
	\multicolumn{3}{l}{(2004) 133-140.'}\\
	\multicolumn{3}{l}{22     'K.C.H. Kumar, P. Wollants, L. Delaey, Thermodynamic}\\ 
	\multicolumn{3}{l}{assessment of the Ti-Zr system and calculation of the}\\ 
	\multicolumn{3}{l}{Nb-Ti-Zr phase-diagram, J. Alloys Compd. 206 (1994) 121-127.'}\\
	\multicolumn{3}{l}{23	  'Y.L. Zhang, H.S. Liu, Z.P. Jin, Thermodynamic assessment}\\ 
	\multicolumn{3}{l}{of the Nb-Ti system, Calphad-Computer Coupling Phase}\\ 
	\multicolumn{3}{l}{Diagrams Thermochem. 25 (2001) 305-317.'}\\
	\multicolumn{3}{l}{24	  'S. Kar, D.M. Lipkin, A CALPHAD-based phase equilibrium model}\\ 
	\multicolumn{3}{l}{of Mo-Ti-Zr-C, in: TMS 2008 Annu. Meet. Suppl. Proceedings,}\\ 
	\multicolumn{3}{l}{Vol 2 Mater. Charact. Comput. Model., Minerals, Metals \&}\\ 
	\multicolumn{3}{l}{Materials Soc, Warrendale, 2008: pp. 237-243.'}\\ 
	\multicolumn{3}{l}{25	  'I. Ansara, A.T. Dinsdale, M.H. Rand, COST 507:}\\ 
	\multicolumn{3}{l}{thermochemical database for light metal alloys,}\\ 
	\multicolumn{3}{l}{European Communities, Brussels and Luxembourg, 1998.'}\\
	\multicolumn{3}{l}{26	  'R.J. Perez, B. Sundman, Thermodynamic assessment of the Mo-Zr}\\
	\multicolumn{3}{l}{binary phase diagram, Calphad-Computer Coupling Phase Diagrams}\\ 
	\multicolumn{3}{l}{Thermochem. 27 (2003) 253-262.'}\\ 
	\multicolumn{3}{l}{27	  'A.F. Guillermet, Phase-diagram and thermochemical properties}\\
	\multicolumn{3}{l}{of the Zr-Ta system - an assessment based on Gibbs energy}\\ 
	\multicolumn{3}{l}{modeling, J. Alloys Compd. 226 (1995) 174-184.'}\\
	\multicolumn{3}{l}{28	  'A.F. Guillermet, Thermodynamic analysis of the stable phases}\\ 
	\multicolumn{3}{l}{n the Zr-Nb system and calculation of the phase-diagram,}\\ 
	\multicolumn{3}{l}{Zeitschrift Fur Met. 82 (1991) 478-487.'}\\   
	\multicolumn{3}{l}{!}\\
\end{longtable}
\Appendix{Sn-Ta Database}

\begin{table}[H]
	\centering
	\begin{tabular}{ l l l }
		\hline
		\multicolumn{3}{l}{\$*************************************************************}\\
		\multicolumn{3}{l}{\$ The definition of the pure elements, vacancy and species}\\
		\multicolumn{3}{l}{\$-----------------------------------------------------------------------------------------------}\\
		ELEMENT /- & ELECTRON\_GAS & 0.0000E+00  0.0000E+00  0.0000E+00!\\
		ELEMENT VA & VACUUM & 0.0000E+00  0.0000E+00  0.0000E+00!\\
		ELEMENT SN & BCT\_A5 & 1.1871E+02  6.3220E+03  5.1195E+01!\\
		ELEMENT TA & BCC\_A2 & 1.8095E+02  5.6819E+03  4.1472E+01!\\
	\end{tabular}
	\label{ab-table:snta}
\end{table}
\begin{longtable}[H]{ l l l }
	\label{ab-table:snta1} \\
	\hline
	\endhead
	\hline
	\endfoot
	\multicolumn{3}{l}{\$*************************************************************}\\
	\multicolumn{3}{l}{\$ The Gibbs energies of the elements}\\
	\multicolumn{3}{l}{\$ in the stable and metastable forms from SGTE}\\
	\multicolumn{3}{l}{\$-----------------------------------------------------------------------------------------------}\\
	\multicolumn{3}{l}{\$-----------------------------------------------------------------------------------------------}\\
	\multicolumn{3}{c}{* TA *}\\
	\multicolumn{3}{l}{\$-----------------------------------------------------------------------------------------------}\\
	\multicolumn{3}{l}{\$-----------------------------------------------------------------------------------------------}\\
	FUNCTION GHSERTA & & \\
	\multicolumn{3}{l}{298.15 -7285.889+119.139857*T-23.7592624*T*LN(T)} \\ \multicolumn{3}{l}{-.002623033*T**2+1.70109E-07*T**3-3293*T**(-1); 1300 Y}\\
	\multicolumn{3}{l}{-22389.955+243.88676*T-41.137088*T*LN(T)+.006167572*T**2}\\
	\multicolumn{3}{l}{-6.55136E-07*T**3+2429586*T**(-1); 2500 Y}\\
	\multicolumn{3}{l}{+229382.886-722.59722*T+78.5244752*T*LN(T)-.017983376*T**2}\\
	\multicolumn{3}{l}{+1.95033E-07*T**3-93813648*T**(-1); 3290 Y}\\
	\multicolumn{3}{l}{-1042384.01+2985.49125*T-362.159132*T*LN(T)+.043117795*T**2}\\
	\multicolumn{3}{l}{-1.055148E-06*T**3+5.54714342E+08*T**(-1); 6000 N REF20 !}\\
	FUNCTION GFCCTA & & \\
	\multicolumn{3}{l}{298.15 +8714.111+120.839857*T-23.7592624*T*LN(T)}\\
	\multicolumn{3}{l}{-.002623033*T**2+1.70109E-07*T**3-3293*T**(-1); 1300 Y}\\
	\multicolumn{3}{l}{-6389.955+245.58676*T-41.137088*T*LN(T)+.006167572*T**2}\\
	\multicolumn{3}{l}{-6.55136E-07*T**3+2429586*T**(-1); 2500 Y}\\
	\multicolumn{3}{l}{+245382.886-720.89722*T+78.5244752*T*LN(T)-.017983376*T**2}\\
	\multicolumn{3}{l}{+1.95033E-07*T**3-93813648*T**(-1); 3290 Y}\\
	\multicolumn{3}{l}{-1026384.01+2987.19125*T-362.159132*T*LN(T)+.043117795*T**2}\\
	\multicolumn{3}{l}{-1.055148E-06*T**3+5.54714342E+08*T**(-1); 6000 N REF20 !}\\
	FUNCTION GHCPTA & & \\
	\multicolumn{3}{l}{298.15 +4714.111+121.539857*T-23.7592624*T*LN(T)}\\
	\multicolumn{3}{l}{-.002623033*T**2+1.70109E-07*T**3-3293*T**(-1); 1300 Y}\\
	\multicolumn{3}{l}{-10389.955+246.28676*T-41.137088*T*LN(T)+.006167572*T**2}\\
	\multicolumn{3}{l}{-6.55136E-07*T**3+2429586*T**(-1); 2500 Y}\\
	\multicolumn{3}{l}{+241382.886-720.19722*T+78.5244752*T*LN(T)-.017983376*T**2}\\
	\multicolumn{3}{l}{+1.95033E-07*T**3-93813648*T**(-1); 3290 Y}\\
	\multicolumn{3}{l}{-1030384.01+2987.89125*T-362.159132*T*LN(T)+.043117795*T**2}\\
	\multicolumn{3}{l}{-1.055148E-06*T**3+5.54714342E+08*T**(-1); 6000 N REF20 !}\\
	FUNCTION GLIQTA & & \\
	\multicolumn{3}{l}{298.15 +21875.086+111.561128*T-23.7592624*T*LN(T)}\\
	\multicolumn{3}{l}{-.002623033*T**2+1.70109E-07*T**3-3293*T**(-1); 1000 Y}\\
	\multicolumn{3}{l}{+43884.339-61.981795*T+.0279523*T*LN(T)-.012330066*T**2}\\
	\multicolumn{3}{l}{+6.14599E-07*T**3-3523338*T**(-1); 3290 Y}\\
	\multicolumn{3}{l}{-6314.543+258.110873*T-41.84*T*LN(T); 6000 N REF20 !}\\
	\multicolumn{3}{l}{\$-----------------------------------------------------------------------------------------------}\\
	\multicolumn{3}{l}{\$-----------------------------------------------------------------------------------------------}\\
	\multicolumn{3}{c}{* SN *}\\
	\multicolumn{3}{l}{\$-----------------------------------------------------------------------------------------------}\\
	\multicolumn{3}{l}{\$-----------------------------------------------------------------------------------------------}\\
	FUNCTION GHSERSN & & \\
	\multicolumn{3}{l}{100 -7958.517+122.765451*T-25.858*T*LN(T)}\\
	\multicolumn{3}{l}{+5.1185E-04*T**2-3.192767E-06*T**3+18440*T**(-1); 250 Y}\\
	\multicolumn{3}{l}{-5855.135+65.443315*T-15.961*T*LN(T)-.0188702*T**2+3.121167E-06*T**3}\\
	\multicolumn{3}{l}{-61960*T**(-1); 505.08 Y}\\
	\multicolumn{3}{l}{+2524.724+4.005269*T-8.2590486*T*LN(T)-.016814429*T**2}\\
	\multicolumn{3}{l}{+2.623131E-06*T**3-1081244*T**(-1)-1.2307E+25*T**(-9); 800 Y}\\
	\multicolumn{3}{l}{-8256.959+138.99688*T-28.4512*T*LN(T)-1.2307E+25*T**(-9); 3000 N REF20 !}\\
	FUNCTION GBCCSN & & \\
	\multicolumn{3}{l}{100 -3558.517+116.765451*T-25.858*T*LN(T)}\\
	\multicolumn{3}{l}{+5.1185E-04*T**2-3.192767E-06*T**3+18440*T**(-1); 250 Y}\\
	\multicolumn{3}{l}{-1455.135+59.443315*T-15.961*T*LN(T)-.0188702*T**2+3.121167E-06*T**3}\\
	\multicolumn{3}{l}{-61960*T**(-1); 505.08 Y}\\
	\multicolumn{3}{l}{+6924.724-1.994731*T-8.2590486*T*LN(T)-.016814429*T**2}\\
	\multicolumn{3}{l}{+2.623131E-06*T**3-1081244*T**(-1)-1.2307E+25*T**(-9); 800 Y}\\
	\multicolumn{3}{l}{-3856.959+132.99688*T-28.4512*T*LN(T)-1.2307E+25*T**(-9); 3000 N REF20 !}\\
	FUNCTION GA12SN & & \\
	\multicolumn{3}{l}{100 -5958.517+122.765451*T-25.858*T*LN(T)}\\
	\multicolumn{3}{l}{+5.1185E-04*T**2-3.192767E-06*T**3+18440*T**(-1); 250 Y}\\
	\multicolumn{3}{l}{-3855.135+65.443315*T-15.961*T*LN(T)-.0188702*T**2+3.121167E-06*T**3}\\
	\multicolumn{3}{l}{-61960*T**(-1); 505.08 Y+4524.724+4.005269*T-8.2590486*T*LN(T)}\\
	\multicolumn{3}{l}{-.016814429*T**2+2.623131E-06*T**3-1081244*T**(-1)-1.2307E+25*T**(-9); 800 Y}\\
	\multicolumn{3}{l}{-6256.959+138.99688*T-28.4512*T*LN(T)-1.2307E+25*T**(-9); 3000 N REF20 !}\\
	FUNCTION GA13SN & & \\
	\multicolumn{3}{l}{100 -5958.517+122.765451*T-25.858*T*LN(T)}\\
	\multicolumn{3}{l}{+5.1185E-04*T**2-3.192767E-06*T**3+18440*T**(-1); 250 Y}\\
	\multicolumn{3}{l}{-3855.135+65.443315*T-15.961*T*LN(T)-.0188702*T**2+3.121167E-06*T**3}\\
	\multicolumn{3}{l}{-61960*T**(-1); 505.08 Y}\\
	\multicolumn{3}{l}{+4524.724+4.005269*T-8.2590486*T*LN(T)-.016814429*T**2}\\
	\multicolumn{3}{l}{+2.623131E-06*T**3-1081244*T**(-1)-1.2307E+25*T**(-9); 800 Y}\\
	\multicolumn{3}{l}{-6256.959+138.99688*T-28.4512*T*LN(T)-1.2307E+25*T**(-9); 3000 N REF20 !}\\
	FUNCTION GDIAMOND & & \\
	\multicolumn{3}{l}{100 -9579.608+114.007785*T-22.972*T*LN(T)-.00813975*T**2}\\
	\multicolumn{3}{l}{+2.7288E-06*T**3+25615*T**(-1); 298.15 Y}\\
	\multicolumn{3}{l}{-9063.001+104.84654*T-21.5750771*T*LN(T)-.008575282*T**2}\\
	\multicolumn{3}{l}{+1.784447E-06*T**3-2544*T**(-1); 800 Y}\\
	\multicolumn{3}{l}{-10909.351+147.396535*T-28.4512*T*LN(T); 3000 N REF20 !}\\
	FUNCTION GFCCSN & & \\
	\multicolumn{3}{l}{298.15 -345.135+56.983315*T-15.961*T*LN(T)-.0188702*T**2}\\
	\multicolumn{3}{l}{+3.121167E-06*T**3-61960*T**(-1); 505.08 Y}\\
	\multicolumn{3}{l}{+8034.724-4.454731*T-8.2590486*T*LN(T)-.016814429*T**2}\\
	\multicolumn{3}{l}{+2.623131E-06*T**3-1081244*T**(-1)-1.2307E+25*T**(-9); 800 Y}\\
	\multicolumn{3}{l}{-2746.959+130.53688*T-28.4512*T*LN(T)-1.2307E+25*T**(-9); 3000 N REF20 !}\\
	FUNCTION GHCPSN & & \\
	\multicolumn{3}{l}{298.15 -1955.135+57.797315*T-15.961*T*LN(T)}\\
	\multicolumn{3}{l}{-.0188702*T**2+3.121167E-06*T**3-61960*T**(-1); 505.08 Y}\\
	\multicolumn{3}{l}{+6424.724-3.640731*T-8.2590486*T*LN(T)-.016814429*T**2}\\
	\multicolumn{3}{l}{+2.623131E-06*T**3-1081244*T**(-1)-1.2307E+25*T**(-9); 800 Y}\\
	\multicolumn{3}{l}{-4356.959+131.35088*T-28.4512*T*LN(T)-1.2307E+25*T**(-9); 3000 N REF20 !}\\
	FUNCTION GHCPZN\_S & & \\
	\multicolumn{3}{l}{298.15 -1950.135+57.797315*T-15.961*T*LN(T)}\\
	\multicolumn{3}{l}{-.0188702*T**2+3.121167E-06*T**3-61960*T**(-1); 505.08 Y}\\
	\multicolumn{3}{l}{+6429.724-3.640731*T-8.2590486*T*LN(T)-.016814429*T**2}\\
	\multicolumn{3}{l}{+2.623131E-06*T**3-1081244*T**(-1)-1.2307E+25*T**(-9); 800 Y}\\
	\multicolumn{3}{l}{-4351.959+131.35088*T-28.4512*T*LN(T)-1.2307E+25*T**(-9); 3000 N REF20 !}\\
	FUNCTION GLIQSN & & \\
	\multicolumn{3}{l}{100 -855.425+108.677684*T-25.858*T*LN(T)+5.1185E-04*T**2}\\
	\multicolumn{3}{l}{-3.192767E-06*T**3+18440*T**(-1)+1.47031E-18*T**7; 250 Y}\\
	\multicolumn{3}{l}{+1247.957+51.355548*T-15.961*T*LN(T)-.0188702*T**2+3.121167E-06*T**3}\\
	\multicolumn{3}{l}{-61960*T**(-1)+1.47031E-18*T**7; 505.08 Y}\\
	\multicolumn{3}{l}{+9496.31-9.809114*T-8.2590486*T*LN(T)-.016814429*T**2}\\
	\multicolumn{3}{l}{+2.623131E-06*T**3-1081244*T**(-1); 800 Y}\\
	\multicolumn{3}{l}{-1285.372+125.182498*T-28.4512*T*LN(T); 3000 N REF20 !}\\
	FUNCTION GA7SN & & \\
	\multicolumn{3}{l}{100 -5923.517+122.765451*T-25.858*T*LN(T)}\\
	\multicolumn{3}{l}{+5.1185E-04*T**2-3.192767E-06*T**3+18440*T**(-1); 250 Y}\\
	\multicolumn{3}{l}{-3820.135+65.443315*T-15.961*T*LN(T)-.0188702*T**2+3.121167E-06*T**3}\\
	\multicolumn{3}{l}{-61960*T**(-1); 505.08 Y}\\
	\multicolumn{3}{l}{+4559.724+4.005269*T-8.2590486*T*LN(T)-.016814429*T**2}\\
	\multicolumn{3}{l}{+2.623131E-06*T**3-1081244*T**(-1)-1.2307E+25*T**(-9); 800 Y}\\
	\multicolumn{3}{l}{-6221.959+138.99688*T-28.4512*T*LN(T)-1.2307E+25*T**(-9); 3000 N REF20 !}\\
	FUNCTION GA6SN & & \\
	\multicolumn{3}{l}{298.15 -468.135+57.181195*T-15.961*T*LN(T)-.0188702*T**2}\\
	\multicolumn{3}{l}{+3.121167E-06*T**3-61960*T**(-1); 505.08 Y}\\
	\multicolumn{3}{l}{+7911.724-4.256851*T-8.2590486*T*LN(T)-.016814429*T**2}\\
	\multicolumn{3}{l}{+2.623131E-06*T**3-1081244*T**(-1)-1.2307E+25*T**(-9); 800 Y}\\
	\multicolumn{3}{l}{-2869.959+130.73476*T-28.4512*T*LN(T)-1.2307E+25*T**(-9); 3000 N REF20 !}\\
	\multicolumn{3}{l}{\$-----------------------------------------------------------------------------------------------}\\
	\multicolumn{3}{l}{FUNCTION UN\_ASS    298.15 +0.0; 300 N !}\\
	\multicolumn{3}{l}{\$-----------------------------------------------------------------------------------------------}\\
	\multicolumn{3}{l}{\$*************************************************************}\\
	\multicolumn{3}{l}{TYPE\_DEFINITION \% SEQ *!}\\
	\multicolumn{3}{l}{DEFINE\_SYSTEM\_DEFAULT ELEMENT 2 !}\\
	\multicolumn{3}{l}{DEFAULT\_COMMAND DEF\_SYS\_ELEMENT VA /- !}\\
	\multicolumn{3}{l}{\$-----------------------------------------------------------------------------------------------}\\
	\multicolumn{3}{l}{\$*************************************************************}\\
	\multicolumn{3}{l}{\$-----------------------------------------------------------------------------------------------}\\
	\multicolumn{3}{l}{TYPE\_DEFINITION $\&$ GES A\_P\_D BCC\_A2 MAGNETIC  -1.0    4.00000E-01 !}\\
	\multicolumn{3}{l}{PHASE BCC\_A2  \%$\&$  2 1   3 !}\\
	\multicolumn{3}{l}{CONSTITUENT BCC\_A2  :SN,TA : VA :  !}\\
	& & \\
	PARAMETER G(BCC\_A2,SN:VA;0) & \multicolumn{2}{l}{100 +GBCCSN; 3000 N REF20 !}\\
	PARAMETER G(BCC\_A2,TA:VA;0) & \multicolumn{2}{l}{298.15 +GHSERTA; 6000 N REF20 !}\\
	PARAMETER G(BCC\_A2,SN,TA:VA;0) & \multicolumn{2}{l}{298.15  7.0451375E+04; 6000 N !}\\
	PARAMETER G(BCC\_A2,SN,TA:VA;1) & \multicolumn{2}{l}{298.15 1.1223739E+05; 6000 N !}\\
	\multicolumn{3}{l}{\$-----------------------------------------------------------------------------------------------}\\
	\multicolumn{3}{l}{PHASE BCT\_A5  \%  1  1.0  !}\\
	\multicolumn{3}{l}{CONSTITUENT BCT\_A5  :SN :  !}\\
	& & \\
	PARAMETER G(BCT\_A5,SN;0) & \multicolumn{2}{l}{100 +GHSERSN; 3000 N REF20 !}\\
	\multicolumn{3}{l}{\$-----------------------------------------------------------------------------------------------}\\
	\multicolumn{3}{l}{TYPE\_DEFINITION ' GES A\_P\_D CBCC\_A12 MAGNETIC  -3.0    2.80000E-01 !}\\
	\multicolumn{3}{l}{PHASE CBCC\_A12  \%'  2 1   1 !}\\
	\multicolumn{3}{l}{CONSTITUENT CBCC\_A12  :SN : VA :  !}\\
	& & \\
	PARAMETER G(CBCC\_A12,SN:VA;0) & \multicolumn{2}{l}{100 +GA12SN; 3000 N REF20 !}\\
	\multicolumn{3}{l}{\$-----------------------------------------------------------------------------------------------}\\
	\multicolumn{3}{l}{PHASE CUB\_A13  \%  2 1   1 !}\\
	\multicolumn{3}{l}{CONSTITUENT CUB\_A13  :SN : VA :  !}\\
	& & \\
	PARAMETER G(CUB\_A13,SN:VA;0) & \multicolumn{2}{l}{100 +GA13SN; 3000 N REF20 !}\\
	\multicolumn{3}{l}{\$-----------------------------------------------------------------------------------------------}\\
	\multicolumn{3}{l}{PHASE DIAMOND\_A4  \%  1  1.0  !}\\
	\multicolumn{3}{l}{CONSTITUENT DIAMOND\_A4  :SN :  !}\\
	& & \\
	PARAMETER G(DIAMOND\_A4,SN;0) & \multicolumn{2}{l}{100 +GDIAMOND; 3000 N REF20 !}\\
	\multicolumn{3}{l}{\$-----------------------------------------------------------------------------------------------}\\
	\multicolumn{3}{l}{TYPE\_DEFINITION ( GES A\_P\_D FCC\_A1 MAGNETIC  -3.0    2.80000E-01 !}\\
	\multicolumn{3}{l}{PHASE FCC\_A1  \%(  2 1   1 !}\\
	\multicolumn{3}{l}{CONSTITUENT FCC\_A1  :SN,TA : VA :  !}\\
	& & \\
	PARAMETER G(FCC\_A1,SN:VA;0) & \multicolumn{2}{l}{298.15 +GFCCSN; 3000 N REF20 !}\\
	PARAMETER G(FCC\_A1,TA:VA;0) & \multicolumn{2}{l}{298.15 +GFCCTA; 6000 N REF20 !}\\
	\multicolumn{3}{l}{\$-----------------------------------------------------------------------------------------------}\\
	\multicolumn{3}{l}{TYPE\_DEFINITION ) GES A\_P\_D HCP\_A3 MAGNETIC  -3.0    2.80000E-01 !}\\
	\multicolumn{3}{l}{PHASE HCP\_A3  \%)  2 1   .5 !}\\
	\multicolumn{3}{l}{CONSTITUENT HCP\_A3  :SN,TA : VA :  !}\\
	& & \\
	PARAMETER G(HCP\_A3,SN:VA;0) & \multicolumn{2}{l}{298.15 +GHCPSN; 3000 N REF20 !}\\
	PARAMETER G(HCP\_A3,TA:VA;0) & \multicolumn{2}{l}{298.15 +GHCPTA; 6000 N REF20 !}\\
	\multicolumn{3}{l}{\$-----------------------------------------------------------------------------------------------}\\
	\multicolumn{3}{l}{PHASE HCP\_ZN  \%  2 1   .5 !}\\
	\multicolumn{3}{l}{CONSTITUENT HCP\_ZN  :SN : VA :  !}\\
	& & \\
	PARAMETER G(HCP\_ZN,SN:VA;0) & \multicolumn{2}{l}{298.15 +GHCPZN\_S; 3000 N REF20 !}\\
	\multicolumn{3}{l}{\$-----------------------------------------------------------------------------------------------}\\
	\multicolumn{3}{l}{PHASE LIQUID  \%  1  1.0  !}\\
	\multicolumn{3}{l}{CONSTITUENT LIQUID  :SN,TA :  !}\\
	& & \\
	PARAMETER G(LIQUID,SN;0) & \multicolumn{2}{l}{100 +GLIQSN; 3000 N REF20 !}\\
	PARAMETER G(LIQUID,TA;0) & \multicolumn{2}{l}{298.15 +GLIQTA; 6000 N REF20 !}\\
	PARAMETER G(LIQUID,SN,TA;0) & \multicolumn{2}{l}{298.15 -1.7117919E+04; 6000 N !}\\
	\multicolumn{3}{l}{\$-----------------------------------------------------------------------------------------------}\\
	\multicolumn{3}{l}{PHASE RHOMBOHEDRAL\_A7  \%  1  1.0  !}\\
	\multicolumn{3}{l}{CONSTITUENT RHOMBOHEDRAL\_A7  :SN :  !}\\
	& & \\
	PARAMETER G(RHOMBOHEDRAL\_A7,SN;0) & \multicolumn{2}{l}{100 +GA7SN; 3000 N REF20 !}\\
	\multicolumn{3}{l}{\$-----------------------------------------------------------------------------------------------}\\
	\multicolumn{3}{l}{PHASE TA3SN  \%  2 3   1 !}\\
	\multicolumn{3}{l}{CONSTITUENT TA3SN  :TA : SN :  !}\\
	& & \\
	PARAMETER G(TA3SN,TA:SN;0) & & \\
	\multicolumn{3}{l}{298.15 -68843.951-6.00E+00*T+3*GHSERTA+GHSERSN; 3000 N !}\\
	\multicolumn{3}{l}{\$-----------------------------------------------------------------------------------------------}\\
	\multicolumn{3}{l}{PHASE TASN2  \%  2 1   2 !}\\
	\multicolumn{3}{l}{CONSTITUENT TASN2  :TA : SN :  !}\\
	& & \\
	PARAMETER G(TASN2,TA:SN;0) & &\\
	\multicolumn{3}{l}{298.15 -29678.180-4.202*T+GHSERTA+2*GHSERSN; 3000 N !}\\
	\multicolumn{3}{l}{\$-----------------------------------------------------------------------------------------------}\\
	\multicolumn{3}{l}{PHASE TETRAGONAL\_A6  \%  1  1.0  !}\\
	\multicolumn{3}{l}{CONSTITUENT TETRAGONAL\_A6  :SN :  !}\\
	& & \\
	PARAMETER G(TETRAGONAL\_A6,SN;0) & \multicolumn{2}{l}{298.15 +GA6SN; 3000 N REF20 !}\\
	\multicolumn{3}{l}{\$-----------------------------------------------------------------------------------------------}\\
	\multicolumn{3}{l}{\$*************************************************************}\\
	\multicolumn{3}{l}{LIST\_OF\_REFERENCES}\\
	\multicolumn{3}{l}{NUMBER  SOURCE}\\
	\multicolumn{3}{l}{20 'A.T. Dinsdale, SGTE Data for Pure Elements,}\\
	\multicolumn{3}{l}{CALPHAD.15 (1991) 317-425. '}\\
	\multicolumn{3}{l}{!}\\
\end{longtable}
\Appendix{Calculation and Fitting Details}

\section*{Calculation Details for Chapter 3, 5, and 6}
As discussed in each chapter the Monkhorst-Pack scheme is used for Brillouin zone sampling \cite{Kresse1996,Monkhorst1976a}. The k-points grid used for each calculation done in chapter 3, 5 and 6 are listed in the table. The k-points grids are listed as AxAxA. In some cases, the automated k-point mesh generator in VASP was used and the length of the subdivisions specified are specified as number such as 50. 

\begin{longtable}[H]{ c c c }
	\hline
	Structure & Type of Calc & relaxation k-points\\
	\hline
	\endhead
	\hline
	\endfoot
	hcp-Ti & Elemental & 30 \\
	bcc-Ti & Elemental & 50 \\
	bcc-Mo & Elemental & 50 \\
	bcc-Nb & Elemental & 50 \\
	bcc-Ta & Elemental & 50 \\
	hcp-Zr & Elemental & 30 \\
	bcc-Zr & Elemental & 4x4x4 \\
	bcc-Ti$_{0.94}$Mo$_{0.06}$ & Dilute & 5x5x5 \\
	bcc-Ti$_{0.88}$Mo$_{0.12}$ & Dilute & 4x4x4\\
	bcc-Ti$_{0.75}$Mo$_{0.25}$ & SQS & 4x4x4\\
	bcc-Ti$_{0.50}$Mo$_{0.50}$ & SQS & 4x4x4\\
	bcc-Ti$_{0.25}$Mo$_{0.75}$ & SQS & 4x4x4\\
	bcc-Ti$_{0.06}$Mo$_{0.94}$ & Dilute & 80\\
	bcc-Ti$_{0.02}$Mo$_{0.98}$ & Dilute & 50\\
	bcc-Ti$_{0.98}$Nb$_{0.02}$ & Dilute & 4x4x4\\
	bcc-Ti$_{0.88}$Nb$_{0.12}$ & Dilute & 4x4x4\\ 
	bcc-Ti$_{0.75}$Nb$_{0.25}$ & SQS & 4x4x4\\
	bcc-Ti$_{0.50}$Nb$_{0.50}$ & SQS & 4x4x4\\
	bcc-Ti$_{0.25}$Nb$_{0.75}$ & SQS & 4x4x4\\
	bcc-Ti$_{0.06}$Nb$_{0.94}$ & Dilute & 80\\
	bcc-Ti$_{0.02}$Nb$_{0.98}$ & Dilute & 80\\
	bcc-Ti$_{0.94}$Sn$_{0.06}$ & Dilute & 80\\
	bcc-Ti$_{0.75}$Sn$_{0.25}$ & SQS & 4x4x4\\
	bcc-Ti$_{0.50}$Sn$_{0.50}$ & SQS & 80\\
	bcc-Ti$_{0.25}$Sn$_{0.75}$ & SQS & 4x4x4\\
	bcc-Ti$_{0.98}$Ta$_{0.02}$ & Dilute & 50\\
	bcc-Ti$_{0.94}$Ta$_{0.06}$ & Dilute & 80\\
	bcc-Ti$_{0.88}$Ta$_{0.12}$ & Dilute & 80\\
	bcc-Ti$_{0.75}$Ta$_{0.25}$ & SQS & 80\\
	bcc-Ti$_{0.50}$Ta$_{0.50}$ & SQS & 80\\
	bcc-Ti$_{0.25}$Ta$_{0.75}$ & SQS & 4x4x4\\
	bcc-Ti$_{0.12}$Ta$_{0.88}$ & Dilute & 80\\
	bcc-Ti$_{0.02}$Ta$_{0.98}$ & Dilute & 80\\
	bcc-Ti$_{0.98}$Zr$_{0.02}$ & Dilute & 3x3x3\\
	bcc-Ti$_{0.75}$Zr$_{0.25}$ & SQS & 4x4x4\\
	bcc-Ti$_{0.50}$Zr$_{0.50}$ & SQS & 4x4x4\\
	bcc-Ti$_{0.25}$Zr$_{0.75}$ & SQS & 4x4x4\\
	bcc-Ti$_{0.06}$Zr$_{0.94}$ & Dilute & 80\\
	bcc-Mo$_{0.50}$Nb$_{0.50}$ & SQS & 3x3x3\\
	bcc-Mo$_{0.50}$Sn$_{0.50}$ & SQS & 80\\
	bcc-Mo$_{0.50}$Ta$_{0.50}$ & SQS & 80\\
	bcc-Mo$_{0.50}$Zr$_{0.50}$ & SQS & 80\\
	bcc-Nb$_{0.50}$Sn$_{0.50}$ & SQS & 80\\
	bcc-Nb$_{0.50}$Ta$_{0.50}$ & SQS & 80\\
	bcc-Nb$_{0.50}$Zr$_{0.50}$ & SQS & 80\\
	bcc-Sn$_{0.50}$Ta$_{0.50}$ & SQS & 80\\
	bcc-Sn$_{0.50}$Zr$_{0.50}$ & SQS & 80\\
	bcc-Ta$_{0.50}$Zr$_{0.50}$ & SQS & 80\\
	bcc-Ti$_{0.33}$Mo$_{0.33}$Nb$_{0.33}$ & SQS & 4x4x4\\
	bcc-Ti$_{0.50}$Mo$_{0.25}$Nb$_{0.25}$ & SQS & 4x4x4\\
	bcc-Ti$_{0.74}$Mo$_{0.13}$Nb$_{0.13}$ & SQS & 4x4x4\\
	bcc-Ti$_{0.33}$Mo$_{0.33}$Sn$_{0.33}$ & SQS & 4x4x4\\
	bcc-Ti$_{0.50}$Mo$_{0.25}$Sn$_{0.25}$ & SQS & 4x4x4\\
	bcc-Ti$_{0.74}$Mo$_{0.13}$Sn$_{0.13}$ & SQS & 4x4x4\\
	bcc-Ti$_{0.33}$Mo$_{0.33}$Ta$_{0.33}$ & SQS & 4x4x4\\
	bcc-Ti$_{0.50}$Mo$_{0.25}$Ta$_{0.25}$ & SQS & 4x4x4\\
	bcc-Ti$_{0.74}$Mo$_{0.13}$Ta$_{0.13}$ & SQS & 4x4x4\\
	bcc-Ti$_{0.33}$Mo$_{0.33}$Zr$_{0.33}$ & SQS & 4x4x4\\
	bcc-Ti$_{0.50}$Mo$_{0.25}$Zr$_{0.25}$ & SQS & 4x4x4\\
	bcc-Ti$_{0.74}$Mo$_{0.13}$Zr$_{0.13}$ & SQS & 4x4x4\\
	bcc-Ti$_{0.33}$Nb$_{0.33}$Sn$_{0.33}$ & SQS & 4x4x4\\
	bcc-Ti$_{0.50}$Nb$_{0.25}$Sn$_{0.25}$ & SQS & 4x4x4\\
	bcc-Ti$_{0.74}$Nb$_{0.13}$Sn$_{0.13}$ & SQS & 4x4x4\\
	bcc-Ti$_{0.33}$Nb$_{0.33}$Ta$_{0.33}$ & SQS & 4x4x4\\
	bcc-Ti$_{0.50}$Nb$_{0.25}$Ta$_{0.25}$ & SQS & 4x4x4\\
	bcc-Ti$_{0.74}$Nb$_{0.13}$Ta$_{0.13}$ & SQS & 4x4x4\\
	bcc-Ti$_{0.33}$Nb$_{0.33}$Zr$_{0.33}$ & SQS & 4x4x4\\
	bcc-Ti$_{0.50}$Nb$_{0.25}$Zr$_{0.25}$ & SQS & 4x4x4\\
	bcc-Ti$_{0.74}$Nb$_{0.13}$Zr$_{0.13}$ & SQS & 4x4x4\\
	bcc-Ti$_{0.33}$Sn$_{0.33}$Ta$_{0.33}$ & SQS & 4x4x4\\
	bcc-Ti$_{0.50}$Sn$_{0.25}$Ta$_{0.25}$ & SQS & 4x4x4\\
	bcc-Ti$_{0.74}$Sn$_{0.13}$Ta$_{0.13}$ & SQS & 4x4x4\\
	bcc-Ti$_{0.33}$Sn$_{0.33}$Zr$_{0.33}$ & SQS & 4x4x4\\
	bcc-Ti$_{0.50}$Sn$_{0.25}$Zr$_{0.25}$ & SQS & 4x4x4\\
	bcc-Ti$_{0.74}$Sn$_{0.13}$Zr$_{0.13}$ & SQS & 4x4x4\\
	bcc-Ti$_{0.33}$Ta$_{0.33}$Zr$_{0.33}$ & SQS & 4x4x4\\
	bcc-Ti$_{0.50}$Ta$_{0.25}$Zr$_{0.25}$ & SQS & 4x4x4\\
	bcc-Ti$_{0.74}$Ta$_{0.13}$Zr$_{0.13}$ & SQS & 4x4x4\\
	\hline
\end{longtable}
%%%
\clearpage
%%%

\section*{Fitting Code for Chapter 5 and 6}
The code used in Mathematica to fit the binary and ternary interaction parameters is listed below.

input: n = {{0, 0}, {0.019, -2.81}, {0.019, -2.81}, {0.019, -2.81}, {0.019, \
		-2.81}, {0.019, -2.81}, {0.019, -2.81}, {0.125, 4.00}, {0.125, 
		4.00}, {0.125, 4.00}, {0.125, 4.00}, {0.125, 4.00}, {0.125, 
		4.00}, {0.25, 8.67}, {0.25, 8.67}, {0.25, 8.67}, {0.25, 
		8.67}, {0.25, 8.67}, {0.25, 8.67}, {0.50, 12.00}, {0.75, 
		1.00}, {0.938, 6.42}, {0.981, -0.11}, {1, 0}}
\\
\noindent output: {{0, 0}, {0.019, -2.81}, {0.019, -2.81}, {0.019, -2.81}, {0.019, \
		-2.81}, {0.019, -2.81}, {0.019, -2.81}, {0.125, 4.}, {0.125, 
		4.}, {0.125, 4.}, {0.125, 4.}, {0.125, 4.}, {0.125, 4.}, {0.25, 
		8.67}, {0.25, 8.67}, {0.25, 8.67}, {0.25, 8.67}, {0.25, 
		8.67}, {0.25, 8.67}, {0.5, 12.}, {0.75, 1.}, {0.938, 
		6.42}, {0.981, -0.11}, {1, 0}}
\\
\noindent input: gp = ListPlot[n, PlotMarkers -> {$\square$, 10}, PlotStyle -> {Blue}]
\noindent input: Fit[n, {(x*(1 - x)), ((x*(1 - x))*(x - (1 - x)))}, x] (For a two parameter fit)
\noindent input: Fit[n, {(x*(1 - x))}, x] (For a one parameter fit)
\noindent input: Plot[\%, {x, 0, 1}, PlotRange -> {-100, 100}]
\noindent input: Show[\%, gp]

\begin{figure}[H]
	\centering
	\includegraphics[width=\textwidth]{Appendix-C/Figures/fitting.png}
\end{figure}
\Appendix{Ti Elastic Database}

\begin{table}[H]
	\centering
	\begin{tabular}{ l l l }
		\hline
		\multicolumn{3}{l}{\$*************************************************************}\\
		\multicolumn{3}{l}{\$ The definition of the pure elements, vacancy and species}\\
		\multicolumn{3}{l}{\$-----------------------------------------------------------------------------------------------}\\
		\multicolumn{3}{l}{TEMPERATURE\_LIMIT 0 6000.00 !}\\
		ELEMENT /- &ELECTRON\_GAS & 0.0000E+00  0.0000E+00  0.0000E+00!\\
		ELEMENT VA & VACUUM & 0.0         0.0         0.0 !\\
		ELEMENT TI & BCC\_A2 ! & \\                  
		ELEMENT MO & BCT\_A5 ! & \\                   
		ELEMENT NB & BCC\_A2 ! & \\       
		ELEMENT SN & BCT\_A5 ! & \\                   
		ELEMENT TA & BCC\_A2 ! & \\                   
		ELEMENT ZR & BCT\_A5 ! & \\
	\end{tabular}
	\label{ac-table:timonbsntazr}
\end{table}
\begin{longtable}[H]{ l l l }
	\label{ac-table:timonbtazr1} \\
	\hline
	\endhead
	\hline
	\endfoot
	\multicolumn{3}{l}{\$*************************************************************}\\
	\multicolumn{3}{l}{\$-----------------------------------------------------------------------------------------------}\\
	\multicolumn{3}{l}{\$-----------------------------------------------------------------------------------------------}\\
	FUNCTION C11BCCTI & 298.15 +93; & 6000 N ! \\
	FUNCTION C12BCCTI & 298.15 +115; & 6000 N ! 
\\
	FUNCTION C44BCCTI & 298.15 +41; & 6000 N ! 
\\
	FUNCTION C11BCCMO & 298.15 +475; & 6000 N ! \\
	FUNCTION C12BCCMO & 298.15 +164; & 6000 N ! 
\\
	FUNCTION C44BCCMO & 298.15 +108; & 6000 N !
\\
	FUNCTION C11BCCNB & 298.15 +245; & 6000 N !
\\
	FUNCTION C12BCCNB & 298.15 +144; & 6000 N ! 
\\
	FUNCTION C44BCCNB & 298.15 +27; & 6000 N !
\\
	FUNCTION C11BCCSN & 298.15 +50; & 6000 N !
\\
	FUNCTION C12BCCSN & 298.15 +52; & 6000 N ! 
\\
	FUNCTION C44BCCSN & 298.15 +29; & 6000 N ! 
\\
	FUNCTION C11BCCTA & 298.15 +278; & 6000 N !
\\
	FUNCTION C12BCCTA & 298.15 +164; & 6000 N ! 
\\
	FUNCTION C44BCCTA & 298.15 +81; & 6000 N ! 
\\
	FUNCTION C11BCCZR & 298.15 +86; & 6000 N !
\\
	FUNCTION C12BCCZR & 298.15 +91; & 6000 N ! 
\\
	FUNCTION C44BCCZR & 298.15 +32; & 6000 N !  \\
	FUNCTION UN\_ASS & 298.15 +0; & 300 N !\\
	\multicolumn{3}{l}{\$-----------------------------------------------------------------------------------------------}\\
	\multicolumn{3}{l}{FUNCTION UN\_ASS 298.15 0; 300 N !}\\
	\multicolumn{3}{l}{\$-----------------------------------------------------------------------------------------------}\\
	\multicolumn{3}{l}{\$*************************************************************}\\
	\multicolumn{3}{l}{TYPE\_DEFINITION \% SEQ * !}\\
	\multicolumn{3}{l}{TYPE\_DEFINITION G SEQ * !}\\
	\multicolumn{3}{l}{DEFINE\_SYSTEM\_DEFAULT SPECIE 5 !}\\
	\multicolumn{3}{l}{DEFAULT\_COMMAND DEF\_SYS\_ELEMENT VA !}\\
	\multicolumn{3}{l}{\$-----------------------------------------------------------------------------------------------}\\
	\multicolumn{3}{l}{\$*************************************************************}\\
	\multicolumn{3}{l}{\$*************************************************************}\\
	\multicolumn{3}{l}{\$-----------------------------------------------------------------------------------------------}\\
	\multicolumn{3}{l}{TYPE\_DEFINITION \& GES A\_P\_D BCC\_A2 MAGNETIC  -1.0    4.00000E-01 !}\\
	\multicolumn{3}{l}{ PHASE BCC\_A2  \%\&  2 1   3 !}\\
	\multicolumn{3}{l}{CONSTITUENT BCC\_A2  :TI,MO,NB,SN,TA,ZR : VA :  !}\\
	& & \\
	PARAMETER C11(BCC\_A2,TI:VA;0) & & \\
	\multicolumn{3}{l}{298.15 +C11BCCTI; 6000 N !}\\
	PARAMETER C11(BCC\_A2,MO:VA;0) & & \\
	\multicolumn{3}{l}{298.15 +C11BCCMO; 6000 N !}\\
	PARAMETER C11(BCC\_A2,NB:VA;0) & & \\
	\multicolumn{3}{l}{298.15 +C11BCCNB; 6000 N !}\\
	PARAMETER C11(BCC\_A2,SN:VA;0) & & \\
	\multicolumn{3}{l}{298.15 +C11BCCSN; 6000 N !}\\
	PARAMETER C11(BCC\_A2,TA:VA;0) & & \\
	\multicolumn{3}{l}{298.15 +C11BCCTA; 6000 N !}\\
	PARAMETER C11(BCC\_A2,ZR:VA;0) & & \\
	\multicolumn{3}{l}{298.15 +C11BCCZR; 6000 N !}\\
	PARAMETER C11(BCC\_A2,TI,MO:VA;0) & & \\
	\multicolumn{3}{l}{298.15 -22.16;     6000 N !}\\
	PARAMETER C11(BCC\_A2,TI,NB:VA;0) & & \\
	\multicolumn{3}{l}{298.15 +40.46;     6000 N !}\\
	PARAMETER C11(BCC\_A2,TI,SN:VA;0) & & \\
	\multicolumn{3}{l}{298.15 +119.46;    6000 N !}\\
	PARAMETER C11(BCC\_A2,TI,TA:VA;0) & & \\
	\multicolumn{3}{l}{298.15 +83.65;     6000 N !}\\
	PARAMETER C11(BCC\_A2,TI,TA:VA;1) & & \\
	\multicolumn{3}{l}{298.15 -67.76;     6000 N !}\\
	PARAMETER C11(BCC\_A2,TI,ZR:VA;0) & & \\
	\multicolumn{3}{l}{298.15 +246.97;    6000 N !}\\
	PARAMETER C11(BCC\_A2,TI,ZR:VA;1) & & \\
	\multicolumn{3}{l}{298.15 -135.95;    6000 N !}\\
	PARAMETER C11(BCC\_A2,TI,MO,NB:VA;0) & & \\
	\multicolumn{3}{l}{298.15 -29.97;     6000 N !}\\   
	PARAMETER C11(BCC\_A2,TI,MO,SN:VA;0) & & \\
	\multicolumn{3}{l}{298.15 -83.85;     6000 N !}\\    
	PARAMETER C11(BCC\_A2,TI,MO,TA:VA;0) & & \\
	\multicolumn{3}{l}{298.15 -106.53;    6000 N !}\\    
	PARAMETER C11(BCC\_A2,TI,MO,ZR:VA;0) & & \\
	\multicolumn{3}{l}{298.15 -245.27;    6000 N !}\\    
	PARAMETER C11(BCC\_A2,TI,NB,SN:VA;0) & & \\
	\multicolumn{3}{l}{298.15 -41.52;     6000 N !}\\    
	PARAMETER C11(BCC\_A2,TI,NB,TA:VA;0) & & \\
	\multicolumn{3}{l}{298.15 -93.77;     6000 N !}\\   
	PARAMETER C11(BCC\_A2,TI,NB,ZR:VA;0) & & \\
	\multicolumn{3}{l}{298.15 -220.35;    6000 N !}\\
	PARAMETER C11(BCC\_A2,TI,SN,TA:VA;0) & & \\
	\multicolumn{3}{l}{298.15 -95.39;     6000 N !}\\ 
	PARAMETER C11(BCC\_A2,TI,SN,ZR:VA;0) & & \\
	\multicolumn{3}{l}{298.15 -155.34;    6000 N !}\\
	PARAMETER C11(BCC\_A2,TI,TA,ZR:VA;0) & & \\
	\multicolumn{3}{l}{298.15 -149.67;    6000 N !}\\     
	\multicolumn{3}{l}{\$-----------------------------------------------------------------------------------------------}\\
	PARAMETER C12(BCC\_A2,TI:VA;0) & & \\
	\multicolumn{3}{l}{298.15 +C12BCCTI; 6000 N !}\\
	PARAMETER C12(BCC\_A2,MO:VA;0) & & \\
	\multicolumn{3}{l}{298.15 +C12BCCMO; 6000 N !}\\
	PARAMETER C12(BCC\_A2,NB:VA;0) & & \\
	\multicolumn{3}{l}{298.15 +C12BCCNB; 6000 N !}\\
	PARAMETER C12(BCC\_A2,SN:VA;0) & & \\
	\multicolumn{3}{l}{298.15 +C12BCCSN; 6000 N !}\\
	PARAMETER C12(BCC\_A2,TA:VA;0) & & \\
	\multicolumn{3}{l}{298.15 +C12BCCTA; 6000 N !}\\
	PARAMETER C12(BCC\_A2,ZR:VA;0) & & \\
	\multicolumn{3}{l}{298.15 +C12BCCZR; 6000 N !}\\
	PARAMETER C12(BCC\_A2,TI,MO:VA;0) & & \\
	\multicolumn{3}{l}{298.15 -36.40;     6000 N !}\\
	PARAMETER C12(BCC\_A2,TI,NB:VA;0) & & \\
	\multicolumn{3}{l}{298.15 -32.39;     6000 N !}\\
	PARAMETER C12(BCC\_A2,TI,SN:VA;0) & & \\
	\multicolumn{3}{l}{298.15 +15.90;     6000 N !}\\
	PARAMETER C12(BCC\_A2,TI,SN:VA;1) & & \\
	\multicolumn{3}{l}{298.15 -146.80;    6000 N !}\\
	PARAMETER C12(BCC\_A2,TI,TA:VA;0) & & \\
	\multicolumn{3}{l}{298.15 +38.05;     6000 N !}\\
	PARAMETER C12(BCC\_A2,TI,ZR:VA;0) & & \\
	\multicolumn{3}{l}{298.15 -110.53;    6000 N !}\\
	PARAMETER C12(BCC\_A2,TI,ZR:VA;1) & & \\
	\multicolumn{3}{l}{298.15 +78.00;     6000 N !}\\
	PARAMETER C12(BCC\_A2,TI,MO,NB:VA;0) & & \\
	\multicolumn{3}{l}{298.15 +13.97;     6000 N !}\\   
	PARAMETER C12(BCC\_A2,TI,MO,SN:VA;0) & & \\
	\multicolumn{3}{l}{298.15 +31.80;     6000 N !}\\    
	PARAMETER C12(BCC\_A2,TI,MO,TA:VA;0) & & \\
	\multicolumn{3}{l}{298.15 -12.35;     6000 N !}\\    
	PARAMETER C12(BCC\_A2,TI,MO,ZR:VA;0) & & \\
	\multicolumn{3}{l}{298.15 +50.43;     6000 N !}\\    
	PARAMETER C12(BCC\_A2,TI,NB,SN:VA;0) & & \\
	\multicolumn{3}{l}{298.15 +25.52;     6000 N !}\\    
	PARAMETER C12(BCC\_A2,TI,NB,TA:VA;0) & & \\
	\multicolumn{3}{l}{298.15 -15.80;     6000 N !}\\    
	PARAMETER C12(BCC\_A2,TI,NB,ZR:VA;0) & & \\
	\multicolumn{3}{l}{298.15 +72.10;     6000 N !}\\
	PARAMETER C12(BCC\_A2,TI,SN,TA:VA;0) & & \\
	\multicolumn{3}{l}{298.15 -10.94;     6000 N !}\\ 
	PARAMETER C12(BCC\_A2,TI,SN,ZR:VA;0) & & \\
	\multicolumn{3}{l}{298.15 +68.86;     6000 N !}\\
	PARAMETER C12(BCC\_A2,TI,TA,ZR:VA;0) & & \\
	\multicolumn{3}{l}{298.15 -8.91;      6000 N !}\\
	\multicolumn{3}{l}{\$-----------------------------------------------------------------------------------------------}\\
	PARAMETER C44(BCC\_A2,TI:VA;0) & & \\
	\multicolumn{3}{l}{298.15 +C44BCCTI; 6000 N !}\\
	PARAMETER C44(BCC\_A2,MO:VA;0) & & \\
	\multicolumn{3}{l}{298.15 +C44BCCMO; 6000 N !}\\
	PARAMETER C44(BCC\_A2,NB:VA;0) & & \\
	\multicolumn{3}{l}{298.15 +C44BCCNB; 6000 N !}\\
	PARAMETER C44(BCC\_A2,SN:VA;0) & & \\
	\multicolumn{3}{l}{298.15 +C44BCCSN; 6000 N !}\\
	PARAMETER C44(BCC\_A2,TA:VA;0) & & \\
	\multicolumn{3}{l}{298.15 +C44BCCTA; 6000 N !}\\
	PARAMETER C44(BCC\_A2,ZR:VA;0) & & \\
	\multicolumn{3}{l}{298.15 +C44BCCZR; 6000 N !}\\
	PARAMETER C44(BCC\_A2,TI,MO:VA;0) & & \\
	\multicolumn{3}{l}{298.15 -142.90;    6000 N !}\\
	PARAMETER C44(BCC\_A2,TI,NB:VA;0) & & \\
	\multicolumn{3}{l}{298.15 -41.54;     6000 N !}\\
	PARAMETER C44(BCC\_A2,TI,NB:VA;1) & & \\
	\multicolumn{3}{l}{298.15 -41.95;     6000 N !}\\
	PARAMETER C44(BCC\_A2,TI,SN:VA;0) & & \\
	\multicolumn{3}{l}{298.15 +59.75;     6000 N !}\\
	PARAMETER C44(BCC\_A2,TI,SN:VA;1) & & \\
	\multicolumn{3}{l}{298.15 -94.38;     6000 N !}\\
	PARAMETER C44(BCC\_A2,TI,TA:VA;0) & & \\
	\multicolumn{3}{l}{298.15 -51.96;     6000 N !}\\
	PARAMETER C44(BCC\_A2,TI,ZR:VA;0) & & \\
	\multicolumn{3}{l}{298.15 +70.06;     6000 N !}\\
	PARAMETER C44(BCC\_A2,TI,MO,NB:VA;0) & & \\
	\multicolumn{3}{l}{298.15 +9.72;      6000 N !}\\   
	PARAMETER C44(BCC\_A2,TI,MO,SN:VA;0) & & \\
	\multicolumn{3}{l}{298.15 +74.73;     6000 N !}\\    
	PARAMETER C44(BCC\_A2,TI,MO,TA:VA;0) & & \\
	\multicolumn{3}{l}{298.15 +5.27;      6000 N !}\\    
	PARAMETER C44(BCC\_A2,TI,MO,ZR:VA;0) & & \\
	\multicolumn{3}{l}{298.15 -44.96;     6000 N !}\\    
	PARAMETER C44(BCC\_A2,TI,NB,SN:VA;0) & & \\
	\multicolumn{3}{l}{298.15 +67.85;     6000 N !}\\    
	PARAMETER C44(BCC\_A2,TI,NB,TA:VA;0) & & \\
	\multicolumn{3}{l}{298.15 +4.25;      6000 N !}\\    
	PARAMETER C44(BCC\_A2,TI,NB,ZR:VA;0) & & \\
	\multicolumn{3}{l}{298.15 -55.29;     6000 N !}\\
	PARAMETER C44(BCC\_A2,TI,SN,TA:VA;0) & & \\
	\multicolumn{3}{l}{298.15 +67.85;     6000 N !}\\ 
	PARAMETER C44(BCC\_A2,TI,SN,ZR:VA;0) & & \\
	\multicolumn{3}{l}{298.15 +3.85;      6000 N !}\\
	PARAMETER C44(BCC\_A2,TI,TA,ZR:VA;0) & & \\
	\multicolumn{3}{l}{298.15 -23.70;     6000 N !}\\ 
	\multicolumn{3}{l}{\$-----------------------------------------------------------------------------------------------}\\
	\multicolumn{3}{l}{\$*************************************************************}\\
	\multicolumn{3}{l}{LIST\_OF\_REFERENCES}\\
	\multicolumn{3}{l}{NUMBER  SOURCE}\\
	\multicolumn{3}{l}{!}\\
\end{longtable}
\Appendix{Pycalphad script}

\noindent This code is used in pycalphad to plot the elastic moduli and elastic properites as a function of composition. A tdb file must be loaded into the script. In the present work the tdb file in Appendix D is used.
{\lstset{language=Python}
\footnotesize
\begin{lstlisting}
import matplotlib
from matplotlib.axes import Axes
from matplotlib.patches import Polygon
from matplotlib.path import Path
from matplotlib.ticker import NullLocator, Formatter, FixedLocator
from matplotlib.transforms import Affine2D, BboxTransformTo, IdentityTransform
from matplotlib.projections import register_projection
import matplotlib.spines as mspines
import matplotlib.axis as maxis
import matplotlib.pyplot as plt

import numpy as np

class TriangularAxes(Axes):
    """
    A custom class for triangular projections.
    """

    name = 'triangular'

    def __init__(self, *args, **kwargs):
         Axes.__init__(self, *args, **kwargs)
         self.set_aspect(1, adjustable='box', anchor='SW')
         self.cla()

    def _init_axis(self):
         self.xaxis = maxis.XAxis(self)
         self.yaxis = maxis.YAxis(self)
         self._update_transScale()

    def cla(self):
         """
         Override to set up some reasonable defaults.
         """
         # Don't forget to call the base class
         Axes.cla(self)

         x_min = 0
         y_min = 0
         x_max = 1
         y_max = 1
         x_spacing = 0.1
         y_spacing = 0.1
         self.xaxis.set_minor_locator(NullLocator())
         self.yaxis.set_minor_locator(NullLocator())
         self.xaxis.set_ticks_position('bottom')
         self.yaxis.set_ticks_position('left')
         Axes.set_xlim(self, x_min, x_max)
         Axes.set_ylim(self, y_min, y_max)
         self.xaxis.set\_ticks(np.arange(x\_min, x\_max+x\_spacing, x\_spacing))
         self.yaxis.set\_ticks(np.arange(y\_min, y\_max+y\_spacing, y\_spacing))

    def \_set\_lim\_and\_transforms(self):
         """
         This is called once when the plot is created to set up all the
         transforms for the data, text and grids.
         """
         # There are three important coordinate spaces going on here:
         #
         #    1. Data space: The space of the data itself
         #
         #    2. Axes space: The unit rectangle (0, 0) to (1, 1)
         #       covering the entire plot area.
         #
         #    3. Display space: The coordinates of the resulting image,
         #       often in pixels or dpi/inch.

         # This function makes heavy use of the Transform classes in
         # ``lib/matplotlib/transforms.py.`` For more information, see
         # the inline documentation there.

         # The goal of the first two transformations is to get from the
         # data space (in this case longitude and latitude) to axes
         # space.  It is separated into a non-affine and affine part so
         # that the non-affine part does not have to be recomputed when
         # a simple affine change to the figure has been made (such as
         # resizing the window or changing the dpi).

         # 1) The core transformation from data space into
         # rectilinear space defined in the HammerTransform class.
         self.transProjection = IdentityTransform()
         # 2) The above has an output range that is not in the unit
         # rectangle, so scale and translate it so it fits correctly
         # within the axes.  The peculiar calculations of xscale and
         # yscale are specific to a Aitoff-Hammer projection, so don't
         # worry about them too much.
         self.transAffine = Affine2D.from_values(
              1., 0, 0.5, np.sqrt(3)/2., 0, 0)
         self.transAffinedep = Affine2D.from_values(
              1., 0, -0.5, np.sqrt(3)/2., 0, 0)
         #self.transAffine = IdentityTransform()

         # 3) This is the transformation from axes space to display
         # space.
         self.transAxes = BboxTransformTo(self.bbox)

         # Now put these 3 transforms together -- from data all the way
         # to display coordinates.  Using the '+' operator, these
         # transforms will be applied "in order".  The transforms are
         # automatically simplified, if possible, by the underlying
         # transformation framework.
         self.transData = \
              self.transProjection + \
              self.transAffine + \
              self.transAxes

         # The main data transformation is set up.  Now deal with
         # gridlines and tick labels.

         # Longitude gridlines and ticklabels.  The input to these
         # transforms are in display space in x and axes space in y.
         # Therefore, the input values will be in range (-xmin, 0),
         # (xmax, 1).  The goal of these transforms is to go from that
         # space to display space.  The tick labels will be offset 4
         # pixels from the equator.

         self.\_xaxis\_pretransform = IdentityTransform()
         self.\_xaxis\_transform = \
              self.\_xaxis\_pretransform + \
              self.transData
         self.\_xaxis\_text1\_transform = \
              Affine2D().scale(1.0, 0.0) + \
              self.transData + \
              Affine2D().translate(0.0, -20.0)
         self.\_xaxis\_text2\_transform = \
              Affine2D().scale(1.0, 0.0) + \
              self.transData + \
              Affine2D().translate(0.0, -4.0)

         # Now set up the transforms for the latitude ticks.  The input to
         # these transforms are in axes space in x and display space in
         # y.  Therefore, the input values will be in range (0, -ymin),
         # (1, ymax).  The goal of these transforms is to go from that
         # space to display space.  The tick labels will be offset 4
         # pixels from the edge of the axes ellipse.

         self._yaxis_transform = self.transData
         yaxis_text_base = \
              self.transProjection + \
              (self.transAffine + \
              self.transAxes)
         self._yaxis_text1_transform = \
              yaxis_text_base + \
              Affine2D().translate(-8.0, 0.0)
         self._yaxis_text2_transform = \
              yaxis_text_base + \
              Affine2D().translate(8.0, 0.0)

    def get_xaxis_transform(self,which='grid'):
         assert which in ['tick1','tick2','grid']
         return self._xaxis_transform

    def get_xaxis_text1_transform(self, pad):
         return self._xaxis_text1_transform, 'bottom', 'center'

    def get_xaxis_text2_transform(self, pad):
         return self._xaxis_text2_transform, 'top', 'center'

    def get_yaxis_transform(self,which='grid'):
         assert which in ['tick1','tick2','grid']
         return self._yaxis_transform

    def get\_yaxis_text1_transform(self, pad):
         return self._yaxis_text1_transform, 'center', 'right'

    def get\_yaxis_text2_transform(self, pad):
         return self._yaxis_text2_transform, 'center', 'left'

    def _gen_axes_spines(self):
         dep_spine = mspines.Spine.linear_spine(self,
                                                    'right')
         # Fix dependent axis to be transformed the correct way
         dep_spine.set_transform(self.transAffinedep + self.transAxes)
         return {'left':mspines.Spine.linear_spine(self,
                                                        'left'),
                     'bottom':mspines.Spine.linear_spine(self,
                                                            'bottom'),
            'right':dep_spine}

    def _gen_axes_patch(self):
         """
         Override this method to define the shape that is used for the
         background of the plot.  It should be a subclass of Patch.
         Any data and gridlines will be clipped to this shape.
         """

         return Polygon([[0,0], [0.5,np.sqrt(3)/2], [1,0]], closed=True)

    # Interactive panning and zooming is not supported with this projection,
    # so we override all of the following methods to disable it.
    def can_zoom(self):
         """
         Return True if this axes support the zoom box
         """
         return False
    def start_pan(self, x, y, button):
         pass
    def end_pan(self):
         pass
    def drag_pan(self, button, key, x, y):
         pass


# Now register the projection with matplotlib so the user can select
# it.
register_projection(TriangularAxes)





import pycalphad.io.tdb_keywords
pycalphad.io.tdb_keywords.TDB_PARAM_TYPES.extend\
(['EM', 'BULK', 'SHEAR', 'C11', 'C12', 'C44'])
from pycalphad import Database, Model, equilibrium, calculate
import numpy as np
import pycalphad.variables as v
import sympy
from tinydb import where

class ElasticModel(Model):
    def build_phase(self, dbe):
         phase = dbe.phases[self.phase_name]
         param_search = dbe.search
         # EM, BULK, SHEAR, C11, C12, C44
         for prop in ['EM', 'BULK', 'SHEAR', 'C11', 'C12', 'C44']:
              prop_param_query = (
              (where('phase_name') == phase.name) & \
              (where('parameter_type') == prop) & \
              (where('constituent_array').test(self._array_validity))
              )
              prop_val = self.redlich_kister_sum \
              (phase, param_search, prop_param_query).subs(dbe.symbols)
              setattr(self, prop, prop_val)





dbf = Database('ElasticTi.tdb')
mod = ElasticModel(dbf, ['TI', 'SN', 'ZR', 'VA'], 'BCC_A2')
symbols = dict([(sympy.Symbol(s), val) for s, val in dbf.symbols.items()])
mod.C11 = mod.C11.xreplace(symbols)
mod.C12 = mod.C12.xreplace(symbols)
mod.C44 = mod.C44.xreplace(symbols)
x1 = np.linspace(0,1, num=100)
x2 = np.linspace(0,1, num=100)
mesh = np.meshgrid(x1, x2)
X = mesh[0]
Y = mesh[1]
mesh_arr = np.array(mesh)
mesh_arr = np.moveaxis(mesh_arr, 0, 2)
dep_col = 1 - np.sum(mesh_arr, axis=-1, keepdims=True)
mesh_arr = np.concatenate((mesh_arr, dep_col), axis=-1)
mesh_arr = np.concatenate((mesh_arr, np.ones(mesh_arr.shape[:-1] + (1,))), axis=-1)
orig_shape = tuple(mesh_arr.shape[:-1])
mesh_arr = mesh_arr.reshape(-1, mesh_arr.shape[-1])
mesh_arr[np.any(mesh_arr < 0, axis=-1), :] = np.nan
res_c11 = calculate(dbf, ['TI', 'SN', 'TA', 'VA'], 'BCC_A2', T=300, P=101325,
      model=mod, output='C11', points=mesh_arr)
res_c11 = res_c11.C11.values.reshape(orig_shape)
res_c12 = calculate(dbf, ['TI', 'SN', 'TA', 'VA'], 'BCC_A2', T=300, P=101325,
      model=mod, output='C12', points=mesh_arr)
res_c12 = res_c12.C12.values.reshape(orig_shape)
res_c44 = calculate(dbf, ['TI', 'SN', 'TA', 'VA'], 'BCC_A2', T=300, P=101325,
      model=mod, output='C44', points=mesh_arr)
res_c44 = res_c44.C44.values.reshape(orig_shape)





import numpy as np
def compute_moduli(c11, c12, c44):
    """Consume elastic stiffness constants and, under symmetry assumptions, compute
    bulk modulus, shear modulus, and Young's modulus.

    Parameters
    ----------
    c11: float64 array-like
    c12: float64 array-like
    c44: float64 array-like

    Returns
    -------
    B, G, Y : tuple of float64 array-likes"""
    # Ported from a matlab code
    c11 = np.array(c11)
    c12 = np.array(c12)
    c44 = np.array(c44)
    cij = np.zeros(c11.shape + (6,6))
    cij[..., 0, 0] = cij[..., 1, 1] = cij[..., 2, 2] = c11
    cij[..., 0, 1] = cij[..., 1, 0] = cij[..., 0, 2] = \
    cij[..., 2, 0] = cij[..., 1, 2] = cij[..., 2, 1] = c12
    cij[..., 3, 3] = cij[..., 4, 4] = cij[..., 5, 5] = c44
    sij = np.linalg.inv(cij)
    A_c = (cij[..., 0, 0] + cij[..., 1, 1] + cij[..., 2, 2]) / 3.
    B_c = (cij[..., 0, 1] + cij[..., 0, 2] + cij[..., 1, 2]) / 3.
    C_c = (cij[..., 3, 3] + cij[..., 4, 4] + cij[..., 5, 5]) / 3.
    A_s = (sij[..., 0, 0] + sij[..., 1, 1] + sij[..., 2, 2]) / 3.
    B_s = (sij[..., 0, 1] + sij[..., 0, 2] + sij[..., 1, 2]) / 3.
    C_s = (sij[..., 3, 3] + sij[..., 4, 4] + sij[..., 5, 5]) / 3.
    Bv = (A_c + 2*B_c) / 3.
    Gv = (A_c - B_c + 3*C_c) / 5.
    Br = 1. / (3*A_s + 6*B_s)
    Gr = 5. / (4*A_s - 4*B_s + 3*C_s)
    Bvrh = (Br + Bv) / 2.
    Gvrh = (Gr + Gv) / 2.
    Yvrh = (9*Bvrh*Gvrh) / (Gvrh + 3*Bvrh)
    return Bvrh, Gvrh, Yvrh

bulk_modulus, shear_modulus, young_modulus = \
compute_moduli(res_c11, res_c12, res_c44)





%matplotlib inline
import matplotlib.pyplot as plt

fig = plt.figure(figsize=(12,12))
ax = fig.gca(projection='triangular')
CS = ax.contour(X, Y, bulk_modulus, linewidths=4, \
levels=list(range(100, 300, 10)), cmap='cool')
ax.clabel(CS, inline=1, fontsize=13, fmt='%1.0f')
#PCM=ax.get_children()[0] #get the mappable, 
#the 1st and the 2nd are the x and y axes
#plt.colorbar(PCM, ax=ax)
ax.set_xlabel('x(Mo)', fontsize=18)
ax.set_ylabel('x(Nb)', fontsize=18, rotation=60, labelpad=-180)
ax.tick_params(axis='both', which='major', labelsize=18)
ax.tick_params(axis='both', which='minor', labelsize=18)
ax.set_title('Bulk modulus')
#fig.savefig('TiMoNb-Bulk.pdf')





%matplotlib inline
import matplotlib.pyplot as plt

fig = plt.figure(figsize=(12,12))
ax = fig.gca(projection='triangular')
CS = ax.contour(X, Y, shear_modulus, linewidths=4, \
levels=list(range(0, 150, 5)), cmap='cool')
ax.clabel(CS, inline=1, fontsize=13, fmt='%1.0f')
#PCM=ax.get_children()[0] #get the mappable, 
#the 1st and the 2nd are the x and y axes
#plt.colorbar(PCM, ax=ax)
ax.set_xlabel('x(Mo)', fontsize=18)
ax.set_ylabel('x(Nb)', fontsize=18, rotation=60, labelpad=-180)
ax.tick_params(axis='both', which='major', labelsize=18)
ax.tick_params(axis='both', which='minor', labelsize=18)
ax.set_title('Shear modulus')
#fig.savefig('TiMoNb-Shear.pdf')





%matplotlib inline
import matplotlib.pyplot as plt

fig = plt.figure(figsize=(12,12))
ax = fig.gca(projection='triangular')
CS = ax.contour(X, Y, young_modulus, linewidths=4, \
levels=list(range(0, 350, 10)), cmap='cool')
ax.clabel(CS, inline=1, fontsize=13, fmt='%1.0f')
#PCM=ax.get_children()[0] #get the mappable, 
#the 1st and the 2nd are the x and y axes
#plt.colorbar(PCM, ax=ax)
ax.set_xlabel('x(Mo)', fontsize=18)
ax.set_ylabel('x(Nb)', fontsize=18, rotation=60, labelpad=-180)
ax.tick_params(axis='both', which='major', labelsize=18)
ax.tick_params(axis='both', which='minor', labelsize=18)
ax.set_title('Young\'s modulus')
#fig.savefig('TiMoNb-Young.pdf')

\end{lstlisting}
}

\Appendix{Ti-Nb Experimental Elastic Data}
The experimentally determined $E$ values for the Ti-Nb system reviewed and averaged for chapter 7 are listed here \cite{Timoshevskii2011,Friak2012,Karre2015,Ozaki2004}.
\noindent 
\begin{longtable}[H]{ c c c c }
	\hline
	x(Nb) & x(Ti) & E (GPa) & Reference \\
	\hline
	\endhead
	\hline
	\endfoot
	 0.00 & 1.00 & 117.16 & \cite{Timoshevskii2011}\\
	 0.00 & 1.00 & 132.00 & \cite{Friak2012}\\
	 0.00 & 1.00 & 115.77 & \cite{Ozaki2004}\\
	 0.00 & 1.00 & 108.30 & \cite{Karre2015}\\
	 0.01 & 0.99 & 112.30 & \cite{Timoshevskii2011}\\
	 0.01 & 0.99 & 112.74 & \cite{Ozaki2004}\\
	 0.02 & 0.98 & 109.05 & \cite{Timoshevskii2011}\\
	 0.02 & 0.98 & 107.61 & \cite{Ozaki2004}\\
	 0.05 & 0.95 & 79.46 & \cite{Timoshevskii2011}\\
	 0.05 & 0.95 & 78.69 & \cite{Ozaki2004}\\
	 0.05 & 0.95 & 80.27 & \cite{Timoshevskii2011}\\
	 0.08 & 0.92 & 66.49 & \cite{Timoshevskii2011}\\
	 0.08 & 0.92 & 66.33 & \cite{Ozaki2004}\\
	 0.09 & 0.91 & 66.89 & \cite{Timoshevskii2011}\\
	 0.09 & 0.91 & 70.58 & \cite{Karre2015}\\
	 0.10 & 0.90 & 66.49 & \cite{Timoshevskii2011}\\
	 0.10 & 0.90 & 115.00 & \cite{Friak2012}\\
	 0.10 & 0.90 & 91.00 & \cite{Friak2012}\\
	 0.10 & 0.90 & 66.09 & \cite{Ozaki2004}\\
	 0.11 & 0.89 & 77.99 & \cite{Ozaki2004}\\
	 0.11 & 0.89 & 79.46 & \cite{Timoshevskii2011}\\
	 0.18 & 0.82 & 92.43 & \cite{Timoshevskii2011}\\
	 0.18 & 0.82 & 93.62 & \cite{Ozaki2004}\\
	 0.19 & 0.81 & 65.27 & \cite{Timoshevskii2011}\\
	 0.19 & 0.81 & 63.24 & \cite{Timoshevskii2011}\\
	 0.20 & 0.80 & 75.00 & \cite{Friak2012}\\
	 0.20 & 0.80 & 89.00 & \cite{Friak2012}\\
	 0.20 & 0.80 & 68.85 & \cite{Karre2015}\\
	 0.22 & 0.78 & 71.46 & \cite{Ozaki2004}\\
	 0.22 & 0.78 & 72.63 & \cite{Karre2015}\\
	 0.22 & 0.78 & 68.62 & \cite{Karre2015}\\
	 0.23 & 0.77 & 72.16 & \cite{Timoshevskii2011}\\
	 0.23 & 0.77 & 103.64 & \cite{Ozaki2004}\\
	 0.23 & 0.77 & 60.34 & \cite{Karre2015}\\
	 0.23 & 0.77 & 67.24 & \cite{Karre2015}\\
	 0.24 & 0.76 & 65.16 & \cite{Karre2015}\\
	 0.24 & 0.76 & 57.07 & \cite{Karre2015}\\
	 0.25 & 0.75 & 71.76 & \cite{Timoshevskii2011}\\
	 0.25 & 0.75 & 74.00 & \cite{Friak2012}\\
	 0.25 & 0.75 & 78.00 & \cite{Friak2012}\\
	 0.25 & 0.75 & 66.33 & \cite{Ozaki2004}\\
	 0.26 & 0.74 & 61.85 & \cite{Karre2015}\\
	 0.26 & 0.74 & 73.09 & \cite{Karre2015}\\
	 0.26 & 0.74 & 82.19 & \cite{Ozaki2004}\\
	 0.26 & 0.74 & 60.50 & \cite{Ozaki2004}\\
	 0.26 & 0.74 & 67.70 & \cite{Timoshevskii2011}\\
	 0.26 & 0.74 & 54.31 & \cite{Karre2015}\\
	 0.27 & 0.73 & 56.46 & \cite{Karre2015}\\
	 0.27 & 0.73 & 52.76 & \cite{Karre2015}\\
	 0.29 & 0.71 & 62.83 & \cite{Ozaki2004}\\
	 0.29 & 0.71 & 61.43 & \cite{Ozaki2004}\\
	 0.30 & 0.70 & 67.70 & \cite{Timoshevskii2011}\\
	 0.30 & 0.70 & 62.69 & \cite{Karre2015}\\
	 0.30 & 0.70 & 72.00 & \cite{Friak2012}\\
	 0.30 & 0.70 & 69.00 & \cite{Friak2012}\\
	 0.30 & 0.70 & 69.31 & \cite{Karre2015}\\
	 0.30 & 0.70 & 67.96 & \cite{Ozaki2004}\\
	 0.34 & 0.66 & 76.21 & \cite{Karre2015}\\
	 0.34 & 0.66 & 86.02 & \cite{Karre2015}\\
	 0.34 & 0.66 & 74.26 & \cite{Ozaki2004}\\
	 0.34 & 0.66 & 75.84 & \cite{Karre2015}\\
	 0.34 & 0.66 & 75.00 & \cite{Timoshevskii2011}\\
	 0.36 & 0.64 & 73.78 & \cite{Timoshevskii2011}\\
	 0.39 & 0.61 & 76.62 & \cite{Timoshevskii2011}\\
	 0.43 & 0.57 & 84.00 & \cite{Ozaki2004}\\
	\hline
\end{longtable}
%%%
%%%%%%%%%%%%%%%%%%%%%%%%%%%%%%%%%%%%%%%%%%%%%%%%%%%%%%%%%%%%%%%
% ESM students need to include a Nontechnical Abstract as the %
% last appendix.                                              %
%%%%%%%%%%%%%%%%%%%%%%%%%%%%%%%%%%%%%%%%%%%%%%%%%%%%%%%%%%%%%%%
% This \include command should point to the file containing
% that abstract.
%\include{nontechnical-abstract}
%%%%%%%%%%%%%%%%%%%%%%%%%%%%%%%%%%%%%%%%%%%
} % End of the \allowdisplaybreak command %
%%%%%%%%%%%%%%%%%%%%%%%%%%%%%%%%%%%%%%%%%%%

%%%%%%%%%%%%%%%%
% BIBLIOGRAPHY %
%%%%%%%%%%%%%%%%
% You can use BibTeX or other bibliography facility for your
% bibliography. LaTeX's standard stuff is shown below. If you
% bibtex, then this section should look something like:
	\begin{singlespace}
	\bibliographystyle{GLG-bibstyle}
	\addcontentsline{toc}{chapter}{Bibliography}
	\bibliography{Thesis}
	\end{singlespace}

%\begin{singlespace}
%\begin{thebibliography}{99}
%\addcontentsline{toc}{chapter}{Bibliography}
%\frenchspacing

%\bibitem{Wisdom87} J. Wisdom, ``Rotational Dynamics of Irregularly Shaped Natural Satellites,'' \emph{The Astronomical Journal}, Vol.~94, No.~5, 1987  pp. 1350--1360.

%\bibitem{G&H83} J. Guckenheimer and P. Holmes, \emph{Nonlinear Oscillations, Dynamical Systems, and Bifurcations of Vector Fields}, Springer-Verlag, New York, 1983.

%\end{thebibliography}
%\end{singlespace}

\backmatter

% Vita
\vita{SupplementaryMaterial/Vita}

\end{document}

