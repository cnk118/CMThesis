%SourceDoc ../YourName-Dissertation.tex
\vspace*{-80mm}
\chapter{Introduction} \label{chapter1:introduction}

\section{\sloppy Motivation}
Titanium (Ti) and its alloys have been used in biomedical applications for many years because of their biocompatibility and corrosion resistance properties \cite{Long1998a}. In recent years there has been an increasing interest in developing better materials for load-bearing implants, due to the increase in total knee and hip replacements. Krutz et al. predicted that the total number of hip and knee replacements would increase by 174\% and 673\%, respectively, from 2005 to 2030, leading to 572,000 hip and 3.48 million knee procedures in 2030 \cite{Kurtz2007}. Two of the driving factors for this situation are an increasing number of younger individuals requiring these replacements and the fact that the average life of total hip and knee replacements is only about 7-12 years before another replacement must be done \cite{Krishna2007a}. These factors all contribute significantly to the necessity for the development of better implant materials. The primary considerations for biomedical implants, such as load-bearing knee and hip implants, are biocompatibility, corrosion resistance, fatigue strength, and Young’s modulus (E) \cite{Long1998a}. In previous years the most common implants for these applications have been Ti-6Al-4V, stainless steels, and MoCoCr alloys \cite{Niinomi2003,Niinomi2012}. However, there have been issues with these materials, such as, cytotoxicity that has been observed with aluminum and vanadium \cite{Ito1995a}. Another important impediment concerning the common implant materials is stress shielding, which leads to implant failure. Stress shielding occurs when the E of the implant is higher than that of bone. Due to the difference in E, load applications to the joint result in the implant material absorbing all of the stress and causing the bone surrounding the implant to atrophy, which leads to a loss in bone density, implant loosening and failure \cite{Long1998a}.  \ref{table:commonEM} summarizes the comparison of the E of common implant materials (> 100 GPa) to bone (10-40 GPa) \cite{Long1998a}. This data indicates the extreme elasticity mismatch between the various materials. Computational thermodynamics and development of a knowledge base of Ti and its alloys will be useful tools in overcoming these challenges.

This work will focus on investigating the thermodynamics and elastic properties of the biocompatible Ti-Mo-Nb-Sn-Ta-Zr system. The thermodynamic and elastic properties will be calculated using first-principles based on Density Functional Theory (DFT). The paramterization of the properties will be completed with the CALculation of PHAse Diagram (CALPHAD) method.  will be used to paramterize these properties and give predictions  using an integrated first-principles based on Density Functional Theory (DFT) calculations and CALPHAD modeling (CALculation PHAse Diagram) approach. The combination of these two methodologies has been shown to eliminate the need for trial-and-error metallurgy, thus saving time, money and other resources. New computational methodology to predict the metastable phase formation will be presented and verified by neutron scattering experiments. The culmination of this work will provide a fundamental understanding of the thermodynamics and elastic properties for the Ti-Mo-Nb-Sn-Ta-Zr system. 


\section{Overview}

The phase stability of Ti-alloys have seen to greatly influence the mechanical properties of the alloys and thus understanding the phase of a Ti-alloy will greatly impact it's effectiveness as a biomedical implant.


\subsubsection{Equilibrium Phases}

Titanium is stable in the $\alpha$ (hexagonal close packed, hcp) phase under the standard temperature and pressure. However, $\beta$ Ti alloys have received much attention because of their low \textit{E} and high specific strength and is thus targeted for the current application \cite{Mei2011,Brailovski2011b}. Mo, Nb and Ta are all biocompatible elements and considered strong -stabilizers, while Zr is a bio-compatible weak -stabilizer individually but strong stabilizer when in combination with other elements \cite{Long1998a}.  In conjunction with their bcc phase stability, Mo, Nb, Ta and Zr, in the presence of many non-toxic elements, studies have shown excellent corrosion resistance and no allergy problems \cite{Tane2008a}. Recently, tin (Sn) has also been studied for use in Ti-alloys,  due to its biocompatibility and low cost \cite{Niinomi2012}. Due to these reason, the effects of alloying Ti with Mo, Nb, Sn, Ta and Zr were studied in the present work. The CALPHAD method allows for the parameterization of the Gibbs energy equations to predict the equilibrium phase stabilities and thermodynamic properties of multi-component systems as a function of temperature and composition. The parameterization is completed using a combination of DFT and experimental results and build from the pure elements, binary and ternary interactions. 

\subsection{Elastic Properties}

WRITE

\subsubsection{Metastable Phases}

While the thermodynamic database will predict the formation of the equilibrium phases, Ti and it's alloys form two metastable phases, $\alpha"$ and $\omega$. $\alpha"$ is a orthorhombic martensitic phase (space group Cmcm). The martensitic tranformation ADD is displacive and closely related to the system critical phenomenon CITe [9], [10]. $\omega$ is a metastable hexagonal phase (space group P6/mmm) of Ti that has lattice parameters closely matching that of bcc Ti. The formation of the $\omega$ and $\alpha"$ phases are thought to be a transition state for the bcc to hcp formation that never GOES AWAY LOOK UP. The formation of $\omega$ and $\alpha"$ has been observed in Ti-Ta and Ti-Nb alloys. It is seen that different cooling of certain compositions of these binary alloys from a temperature in the single-phase bcc region to room temperature causes either the $\alpha"$ phase or the $\omega$ phase but never both. Quenching the samples leads to the formation of $\alpha"$ while slow cooling the sample leads to the $\omega$ phase. The formation of theses phases causes variations to the predicted elastic properites as seen in Figure PUT IN FIGURE where BLANK AND BLANK. While the formation of the metastable phases greatly effects the properties of the Ti-alloys, the properties of unstable structures cannot be efficiently predicted by DFT-based first-principles calculations.



\pagebreak
\begin{table}
	\caption{Young's modulus of common implant materials compared with the Young's modulus of bone \cite{Long1998a}.}
	\centering
	\begin{tabular}{ c c }
		\hline
		Alloy & Young's Modulus (GPa) \\
		\hline
		Bone & 10-40\\
		cp-Ti* & 105\\
		Ti-6Al-4V & 110\\
		Stainless Steel & 200\\
		CoCrMo & 200-230\\
		\hline
		*cp-commercially pure titanium 
	\end{tabular}
\label{table:commonEM}
\end{table}
%%%